\section{Basic Group Definitions And Properties}

\begin{remark}
  These notes follow Dummit and Foote's Abstract Algebra textbook.
\end{remark}

\subsection{Definitions and Propositions}

\begin{definition}
  A \define{group} is a set with a binary operation. The binary
  operation must be \define{associative}. There must be an
  \define{identity element}. (The identity must be both a left and right
  identity. That is, the identity must commute.) And there must be
  \define{inverses}.

  An \define{abelian group} is \define{commutative}. That is: $ab = ba$.
  If not abelian, then every non-identity element might not commute.
\end{definition}

\begin{proposition} Each of the following is true:
  \begin{enumerate}
    \item The identity $1$ element is unique.
    \item Each inverse $\inv{a}$ is unique.
    \item $\parensinv{ab} = \inv{b} \inv{a}$.
  \end{enumerate}
\end{proposition}

\begin{proof}
  Say that $a$ was also an identity element. Then what would be $a1$? Is
  it $1$ or is it $a$?

  Say that both $x$ and $y$ were inverses of $a$. Then what is $xay$? Is
  it $x$ or is it $y$?

  It is clear that $\inv{b}\inv{a}$ will invert $ab$.
\end{proof}

\begin{proposition}
  Both left and right cancelation hold. That is $ab = ac$ implies $b =
  c$.
\end{proposition}

\begin{proof}
  Say not. Then multiply on the right by $\inv{b}$ and the left by
  $\inv{a}$. We get $1 = c \inv{b}$. But we've already shown that
  because inverses are unique we must have $b = c$.
\end{proof}

\subsection{Permutations, Homomorphisms, and Isomorphisms}

\begin{definition}
  The \define{symmetric group} on $A$ $S_A$ is the group of all
  bijections $\sigma: A \to A$. The operation for $S_A$ is function
  composition. Elements of the symmetric group are also called
  \define{permutations}.
\end{definition}

\begin{proposition}
  We can always decompose an element of $S_A$ into disjoint
  \define{cycles}.
\end{proposition}

\begin{definition}
  A \define{homomorphism} is a map $\varphi: A \to B$ such that
  $\varphi(xy) = \varphi(x) \varphi(y)$.

  An \define{isomorphism} is a homomorphism that is bijective.
\end{definition}

\begin{proposition}
  The inverse of an isomorphism $\inv{\varphi}$ is itself an
  isomorphism.
\end{proposition}

\begin{proof}
  I prove only that it is a homomorphism. It is clearly bijective.

  \begin{nedqn}
    \varphi(ab)
  \eqcol
    \varphi(a) \varphi(b)
  \\
    \inv{\varphi}\parens{\varphi(ab)}
  \eqcol
    \inv{\varphi}\parens{\varphi(a) \varphi(b)}
  \\
    \alpha \beta
  \eqcol
    \inv{\varphi}\parens{\alpha \beta}
  \end{nedqn}

  \noindent
  Here I've introduced $\alpha, \beta$ defined as $\inv{\varphi}(a),
  \inv{\varphi}(b)$. The new notation demonstrates that $\inv\varphi$ is
  a homomorphism.
\end{proof}

\begin{remark}
  All symmetric groups over the same number of elements can be put into
  isomorphism in the obvious way. There are $n!$ such isomorphisms
  (corresponding to $n!$ ways to match off corresponding elements).
\end{remark}

\subsection{Group Actions}

\begin{definition}
  A \define{group action} $\cdot$ of a group $G$ on a set $A$ satisfies:

  \begin{enumerate}
    \item $g_1 \cdot (g_2 \cdot a) = (g_1 g_2) \cdot a$,
    \item $1 \cdot a = a$.
  \end{enumerate}
\end{definition}

\begin{remark}
  Each element $g$ implies a permutation $\sigma_g$ on $A$. Why must
  $\sigma_g$ be bijective? Because it must have an inverse
  $\inv\sigma = \sigma_{\inv{g}}$!

  Thus any group action implicitly defines a \emph{homomorphism} from
  $G$ into $S_A$.
\end{remark}
