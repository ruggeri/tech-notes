\section{Fourier Densities}

We've started to gesture at \define{spectral density}. This allows us to
explore different measures on the space $\expf{i\omega t}$. The measures
we'll explore will be described by densities, so our goal is to find
$\hat{f}(\omega)$ such that:

\begin{nedqn}
  f(t)
\eqcol
  \int
  \hat{f}(\omega)
  \expf{i\omega t}
  \domega
\end{nedqn}

Let's consider. Consider a windowed version of $f$ that clips it to zero
outside the range $(-T/2, T/2)$. Then note that we can get the
\emph{exact} Fourier masses (for the restricted-domain $f$) by
calculating:

\begin{nedqn}
  \int_{-T/2}^{T/2}
  \frac{
    f(t)
    \expf{-i\omega t}
  }{T}
  \dt
\end{nedqn}

This works for any $\omega = \frac{2\pi}{T}k$.

As we increase the window size by scaling $T$, we will get a more and
more accurate mass. Unfortunately, we know that mass will go to zero if
$f$ has finite $L^2$ norm.

But, what if we only care about the \emph{density}? Let's try shmearing
the calculated masses through the $\omega$ space. For instance, we
shmear the mass for $\omega$ uniformly through the interval
$\parens{\omega - \frac{\pi}{T}, \omega + \frac{\pi}{T}}$.

If we divide the mass by the interval width $\frac{2\pi}{T}$, we'll find
that the density is:

\begin{nedqn}
  \frac{1}{2\pi}
  \int_{-T/2}^{T/2}
    f(t)
    \expf{-i\omega t}
  \dt
\end{nedqn}

So what we do next is simple. We simply consider the limit of this
quantity.

We now have a notational choice. We may write:

\begin{nedqn}
  \hat{f}(\omega)
\eqcol
  \frac{1}{2\pi}
  \int_{-\infty}^\infty
    f(t) \expf{-i\omega t}
  \dt
\\
  f(t)
\eqcol
  \int_{-\infty}^\infty
  \hat{f}(t) \expf{i\omega t}
  \domega
\end{nedqn}

Alternatively and equivalently, we could write:

\begin{nedqn}
  \hat{f}(\omega)
\eqcol
  \frac{1}{\sqrt{2\pi}}
  \int_{-\infty}^\infty
    f(t) \expf{-i\omega t}
  \dt
\\
  f(t)
\eqcol
  \frac{1}{\sqrt{2\pi}}
  \int_{-\infty}^\infty
  \hat{f}(t) \expf{i\omega t}
  \domega
\end{nedqn}

Last, we could avoid the problem by not using angular frequencies:

\begin{nedqn}
  \hat{f}(\xi)
\eqcol
  \int_{-\infty}^\infty
    f(t) \expf{-2\pi i \xi t}
  \dt
\\
  f(t)
\eqcol
  \int_{-\infty}^\infty
  \hat{f}(t) \expf{2\pi i \xi t}
  \domega
\end{nedqn}

Physicists seem to like the second convention for symmetry. Engineers
like the last because they'd rather work with normal frequencies.
