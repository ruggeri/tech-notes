\section{Mass vs Density}

As we've just been saying, all the basis vectors have infinite $L^2$
norm. Thus all functions that are spanned by the Fourier basis must also
have infinite $L^2$ norm.

But what about aperiodic functions with \emph{finite} norm? It's obvious
that using our current version of the Fourier transform, we would assign
zero weight to every frequency $\omega$. This follows because our inner
product has a denominator that is ever increasing, versus a numerator
that has a finite upper bound.

Up to now, we've really been talking about \emph{mass}: how much mass to
assign each frequency. Instead, we could talk about \emph{density}. We
could say that indeed there is zero mass contributed by any one
frequency $\omega$. But we could also say that \emph{intervals} of
frequency can contibute mass in aggregate. Here we're talking density.

Here's how we can use the density:

\begin{nedqn}
  f(t)
\eqcol
  \int_{-\infty}^{\infty}
  \hat{f}(\omega)
  \expf{i\omega t}
  \domega
\end{nedqn}

We could speak even more generally still. We could do a Lebesgue
integration relative to any measure on the real space of $\omega$
values. That would allow us to put mass in some places (via Dirac
deltas), but have density elsewhere.
