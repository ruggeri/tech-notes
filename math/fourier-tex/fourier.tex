\documentclass[11pt, oneside]{amsart}

\usepackage{geometry}
\geometry{letterpaper}

\usepackage{ned-common}
\usepackage{ned-calculus}
\usepackage{ned-linear-algebra}
\usepackage{ned-stats}

\begin{document}

\title{The Fourier Transform}
\maketitle

\section{Definitions}

A sinusoidal function has the form:

\begin{nedqn}
  \sinf{\omega t + \phi}
\end{nedqn}

We call $\omega$ the \define{angular frequency}. This is how many
radians we rotate through for each second. We call $\phi$ the
\define{phase shift}. It is how many radians we have already rotated at
time $t = 0$.

We can define the frequency in terms of \define{rotations} (also called
\define{cycles}) rather than radians. For instance, the
\define{(ordinary) frequency} $f$ is the number of rotations per second
(cycles per second, hertz). We of course have:

\begin{nedqn}
  f
\eqcol
  \frac{\omega}{2\pi}
\end{nedqn}

\noindent
We may substitute this into our prior formula for a sinusoidal:

\begin{nedqn}
  \sinf{2\pi ft + \phi}
\end{nedqn}

\noindent
This is a pleasing form if we consider that $ft$ is the (fractional)
number of cycles we have rotated through.

Instead of frequency, we might prefer to talk about \define{period}. The
\define{angular period} is the time needed to rotate through one radian.
Much more commonly, we care about the \define{(ordinary) period}, which
is the time needed to rotate through one cycle. In either case:

\begin{nedqn}
  T_\text{angular}
\eqcol
  \frac{1}{\omega}
\\
  T_\text{ordinary}
\eqcol
  \frac{1}{f}
\end{nedqn}

\noindent
Because ordinary period is what we almost always want, we just denote
this $T$. And note that $\omega = \frac{2\pi}{T}$.

Note that we can write our sinusoidal in terms of period:

\begin{nedqn}
  \sinf{2\pi \frac{t}{T} + \phi}
\end{nedqn}

\noindent
This is pleasing because $\frac{t}{T}$ is again the number of cycles
that have been rotated through at time $t$.

There is a corresponding concept of \define{wavelength} denoted
$\lambda$. This means roughly the same thing as period: the period is a
measurement relative to time, while the wavelength is a measurement
relative to distance. I think wavelength is relevant for waves that are
\emph{traveling} through space. If the wave velocity is $v$, I think
that $\lambda = vT$.

We won't be talking about propagation of waves through space, so we will
only speak in terms of periods here.

\section{Purpose}

Many phenomena can be described using sinusoids. For instance, when a
note on a guitar is plucked, the pressure on your eardrum due to the
vibration of the string varies through time proportionally to a
sinusoidal function of a given frequency. For instance, the ``concert
A'' pitch is one which vibrates at a frequency of 440 times per second.

That means that when a violinist plays an open A string, the sound
causes a oscillating change in pressure on your eardrum. 440 oscilations
happen every second.

How great is the peak change in pressure? This is called the
\define{amplitude}, which we will denote $a$. The magnitude of $a$ will
correspond to the \emph{volume} you hear. The change in pressure on your
eardrum from a concert A is $a \sin\parens{2\pi \cdot 440 \cdot t}$.

What if \emph{multiple} strings are plucked simultaneously? Say each of
$k$ strings has frequency $f_i$ and is plucked with amplitude $a_i$.
Then the \emph{aggregate} change in your eardrum pressure is denoted
$f$. And we know:

\begin{nedqn}
  f(t)
\eqcol
  \sum_{i = 1}^k
  a_i \sin\parens{2\pi f_i t}
\end{nedqn}

Experimentally, we could measure $f$ with a pressure sensor, and record
our readings. But from the measurements of $f$, how could we find out
which strings were plucked (and at what amplitude)? Basically, we want
to \emph{decompose} $f$ into the constituent sinusoidal waves, but how
do we do that? The answer is the \define{Fourier transform}.

\section{Linear Algebra Review}

Breaking something apart into its constituents is something we do in
\define{linear algebra}. The space of waveforms is a vector space. We
want to decompose a vector $\vv$ into a linear combination of basis
vectors. That's linear algebra.

The function that does decomposing is called a \define{inner product}.
Let's explore what kind of inner product we'll need.

\subsection{Defining Length, Angle, and Projection}

Consider a vector space and a basis for the space $\mathcal{B}$. We will
declare the basis vectors to have ``unit length.'' We will declare that
any rotation between unit vectors is also a unit vector. Recall that a
``rotation'' between two vectors can simply be defined algebraically:

\begin{nedqn}
  \cos\theta \vu + \sin\theta \vv
\end{nedqn}

Since every vector $\vv$ is a scalar multiple of a unit vector (easy to
prove), we can extend our notion of ``length'' to non-unit vectors. Let
$\norm{\vv} = \alpha$ if $\vv = \alpha \vw$ where $\vw$ is a unit-vector
(rotation of basis vectors).

We'll declare the basis vectors to be ``orthogonal.'' We can go further
and define the ``angle'' between two arbitrary unit vectors $\vu, \vv$.
We can start defining this notion by saying the angle between $\ve_i$
and $\vv$. I say that there exists some unique pair $\vw$ and $\theta$
such that: (1) $\vw$ is in the subspace spanned by $\setof{\ve_j | j \ne
i}$, and (2):

\begin{nedqn}
  \vv
\eqcol
  \cos\theta \ve_i + \sin\theta \vw
\end{nedqn}

\noindent
I call $\theta$ the angle between $\ve_i$ and $\vv$. It's the angle
needed to rotate $\ve_i$ in the direction of some (orthogonal) $\vw$
such that the result is $\vv$.

Why do we know that $\vw$ exists? First, we know that $\vv$ lies in
\emph{some} plane defined by $\ve_i$ and some $\vw$, since the basis
spans the vector space. We might as well assume $\vw$ is unit-length.
But then all unit-length vectors in this plane are rotations of $\vu$
toward $\vw$. This (almost) uniquely specifies $\theta$, up to two
choices: $\theta$ vs $2\pi - \theta$.

We can further extend the notion of angle-between to any two unit
vectors by saying that angle-between should be rotation-invariant.

Once we have this notion of angle-between nailed down, we can
unambiguously define the \define{inner product} $\innerprod{\vu}{\vv} =
\cos\theta$, where $\theta$ is the angle in between two unit vectors. We
can then further extend this to vectors of arbitrary length by defining
$\innerprod{\vu}{\vv} = \norm{\vu} \norm{\vv} \cos\theta$. We call
$\frac{\innerprod{\vu}{\vv}}{\norm{\vv}} \vv$ the \define{projection} of
$\vu$ onto $\vv$.

Notice that since we are using $\cos\theta$, it doesn't matter the sign
of $\theta$. We could iron out the sign wrinkles of our
``angle-between'' concept, but we don't need to.

So, if we have a basis for a vector space, this suggests a notion of
``angle-between,'' ``length,'' and even ``projection.'' Alternatively,
given a pre-existing notion of projection defined by
$\innerprod{\cdot}{\cdot}$, any orthonormal basis $\mathcal{B}$ would be
equally ``compatible'' with this inner-product. That is: the standard
construction of angle/length/projection from $\mathcal{B}$ would exactly
correspond to the inner product $\innerprod{\cdot}{\cdot}$ we started
with.

\subsection{Projection Properly Projects/Decomposes}

We have called $\innerprod{\vv}{\vu}$ the ``projection'' of $\vv$ onto
$\vu$. This terminology is compatible with our usual geometric
understanding of ``projection.'' If, in $n$-dimensional Euclidean space,
we rotate $\vu$ by $\theta$ degrees and get $\vv$, then it's true that
the length of the projection of $\vv$ onto $\vu$ is:

\begin{nedqn}
  \norm{\vv} \cos\theta
\eqcol
  \frac{\innerprod{\vv}{\vu}}{\norm{\vu}}
\end{nedqn}

Also, the inner-product gives us the way to decompose a vector $\vu$ into
a linear combination of $\ve_i$ in an orthonormal basis $\mathcal{B}$:

\begin{nedqn}
  \innerprod{\vv}{\ve_i}
\eqcol
  \innerprod{\sum_{j=1}^n \alpha_j \ve_j}{\ve_i}
\\
\eqcol
  \sum_{j=1}^n \alpha_j \innerprod{\ve_j}{\ve_i}
  \nedcomment{by linearity}
\\
\eqcol
  \alpha_i
  \nedcomment{other $\alpha_j$ cancel by orthonormality}
\end{nedqn}

\noindent
This proof relied on (1) the basis is orthonormal, (2) the inner product
is \define{linear} in the first argument. We will show later why the
inner-product must be linear in the first argument.

But we don't need to adopt this proof that comes from linearity. We
could instead rely on our earlier proof that (1) $\vv$ is always the
result of rotating $\ve_i$ by some $\theta$ in the direction of some
$\vw$ that lives in the subspace spanned by the other basis vectors, and
(2) $\innerprod{\vv}{\ve_i} = \cos\theta$. In that case, it's clear that
$\alpha_i = \cos\theta$. To find the other $\setof{\alpha_j}$ to
complete the decomposition into a linear combination, we decompose
$\sin\theta \vw$ in the subspace orthogonal to $\ve_i$. This proves that
$\innerprod{\cdot}{\cdot}$ properly decomposes unit-vectors; the proof
extends trivially to scaling.

To tie things up: consider an orthonormal basis $\mathcal{B}$. Then:

\begin{nedqn}
  \vv
\eqcol
  \sum_{i=1}^n \innerprod{\vv}{\vb_i} \vb_i
\end{nedqn}

\subsection{Inner products are always dot products}

Let's finally show that an inner-product needs to act like a \define{dot
product}.

One way to show this is simply to appeal to $n$-dimensional Euclidean
space. Any $n$-dimensional vector space $V$ has the same notions of
angle and length as an $n$-dimensional Euclidean space (where the
coordinate vectors correspond to a set of orthonormal basis vectors for
$V$). If you already know that the dot-product does the decomposing in
an $n$-dimensional Euclidean space, then we are done.

But let's not make this appeal. We'll show that an inner-product must
always be a dot product. To do this, it suffices to show: (1) linearity
in first argument, and (2) symmetry of the inner product. Then we have:

\begin{nedqn}
  \innerprod{\vu}{\vv}
\eqcol
  \innerprod{
    \sum \alpha_i \ve_i
  }{
    \sum \beta_j \ve_j
  }
\\
\eqcol
  \sum
  \alpha_i
  \innerprod{\ve_i}{\sum \beta_j \ve_j}
  \nedcomment{linearity in first argument}
\\
\eqcol
  \sum
  \alpha_i
  \innerprod{\sum \beta_j \ve_j}{\ve_i}
  \nedcomment{symmetry}
\\
\eqcol
  \sum_{i=1}^n
  \sum_{j=1}^n
  \alpha_i
  \beta_j
  \innerprod{\ve_j}{\ve_i}
  \nedcomment{linearity}
\\
\eqcol
  \sum \alpha_i \beta_i
  \nedcomment{orthonormality}
\end{nedqn}

\noindent
That is: an inner-product can always be calculated by doing a dot
product where the vectors are decomposed into tuple-representation with
respect to a common orthonormal basis. Or in short: inner products are
always dot products.

We can start by proving symmetry. Consider units $\vu, \vv$. Then
$\innerprod{\vu}{\vv} = \cos\theta$ by definition. But note that the
angle between $\vu, \vv$ is irrespective of the ordering of $\vu, \vv$.
Or perhaps it changes from $\theta$ to $2\pi - \theta$. But who cares:
$\cos\theta$ stays the same. So the inner product must be symmetric.

That means we need to show linearity. We can consider $\innerprod{\vu +
\vv}{\vw}$. We could geometrically prove this by adding $\vu$ and $\vv$
head-to-tail. We would see that the length of the projection of the sum
is equal to the sum of the lengths of the projections.

We'll do exactly that, with just a little more algebra/rigor. The
projection operation doesn't care about the components of $\vu, \vv$
orthogonal to $\vw$. So we might as well throw them out and just
consider $\vu' = \norm{\vu} \cos\theta_u \vw$ and $\vv' = \norm{\vv}
\cos\theta_v \vw$. But then it is clear that the projection of $\vu' +
\vv'$ onto $\vw$ is the sum of their projections.

This establishes linearity in the first argument. Putting these two
facts (linearity, symmetry) together, we have that every inner-product
is simply a dot-product with respect to any orthonormal basis.

\section{Relationship to Linear Algebra}

So let's return to our problem of breaking up $f$ into a weighted sum of
sinusoidals. That is: we want to find the amplitude $a_i$ for each
frequency $\omega_i$.

$\omega_i$ could be anything, but let's keep things simple for now.
Let's only consider $\omega_i$ that are integers. We will only consider
functions $f$ where we can break it up like so:

\begin{nedqn}
  f(t)
\eqcol
  \sum_{k = 0}^\infty
  a_k \sin k t
\end{nedqn}

We will call the collection of sinusoidals $\sin kt$ our \define{basis}.
That is, we'll treat the sinusoidal functions like vectors. Likewise,
we'll call $f$ a vector. So we're returning to the problem of breaking a
vector into a weighted sum of basis vectors.

To do the breaking up, we'll need to define a new \define{inner product}
for our ``function-vectors.'' We will need the following properties to
hold:

\begin{nedqn}
  \innerprod{\sin k t}{\sin k' t}
\eqcol
  0
  \nedcomment{whenever $k \ne k'$}
\\
  \innerprod{\sin k t}{\sin k t}
\eqcol
  1
  \nedcomment{for every $k$}
\\
  \innerprod{\va + \vb}{\sin k t}
\eqcol
  \innerprod{\va}{\sin k t} + \innerprod{\vb}{\sin k t}
\end{nedqn}

Keep in mind that $\va, \vb$ are ``function-vectors.''

\subsection{Notes about our function space}

First, please note that every $f$ we are decomposing will be periodic
with an angular frequency that is an exact multiple of the base angular
frequency $\omega = 1$. That's because every \emph{constituent} of $f$
has that same property.

Second, please note that $f$ does not consist of any weighted part $\cos
k t$. In particular, please note that $f(t) = 0$ at every $t = k 2\pi$.
We will fix this later and consider ``phase-shifted'' functions $f$.
This will not be too hard.

Third, note that the basis $\sin kt$ is \emph{infinite}. To start, let
us assume that $a_k = 0$ for all but a finite number of $k$. That is:
$f$ is made up of a finite number of sinusoidals. Later we can
generalize this property, too.

\section{Our Inner Product}

So now we must find an appropriate inner product for our function space.
An ``obvious'' analogue to the dot product is:

\begin{nedqn}
  \innerprod{f}{g}
\eqcol
  \int_{t \in \reals}
  f(t)g(t)
  \dt
\end{nedqn}

This won't exactly work. First of all, integration over the entire real
domain is a little tricky. It must involve a limit. Setting this aside,
there is another problem. Consider $\innerprod{\sin t}{\sin t}$. We want
this to be 1.0. But is that true? Consider just:

\begin{nedqn}
  \int_0^{2\pi} \sin^2 t \dt
\end{nedqn}

Note that $\sin t$ is periodic with period $2\pi$. Thus:

\begin{nedqn}
  \int_{k2\pi}^{(k+1)2\pi} \sin^2 t \dt
\eqcol
  \int_{k'2\pi}^{(k'+1)2\pi} \sin^2 t \dt
\end{nedqn}

no matter which choice of $k, k'$. But this is trouble, since if

\begin{nedqn}
  \int_0^{2\pi} \sin^2 t \dt
& > &
  0
\end{nedqn}

then we would have that the integral over the entire real domain is
infinite.

But let's look more closely on the integration from just 0 to $2\pi$.
Integrating over this domain gives us all the information we need.
Integrating over any other period would simply be the same and is thus
redundant. But does the integral come out correctly? Note:

\begin{nedqn}
  \int_0^{2\pi} \sin^2 t + \cos^2 t \dt
\eqcol
  \int_0^{2\pi} 1 \dt
=
  2\pi
\end{nedqn}

But note that:

\begin{nedqn}
  \int_0^{2\pi} \sin^2 t \dt
\eqcol
  \int_0^{2\pi} \cos^2 t \dt
\end{nedqn}

since one is merely a phase-shifted version of the other. Thus we see
that:

\begin{nedqn}
  \int_0^{2\pi} \sin^2 t \dt
\eqcol
  \pi
\end{nedqn}

That isn't exactly what we want, but it's a start. Another way to see
this is to remember that $\parens{\cos t, \sin t}$ is the position on
the circle at angle $t$. The square of the length of that vector is
always $1.0$, because the radius is always $1.0$. If you integrate all
the way around the circle, you should get $2\pi$. Of that, $\sin^2 t$ is
only half the story.

Note that the same argument holds for any $\sin kt$, so we know that
$\norm{\sin kt} = \pi$ for all $k$.

Anyway, to ``fix'' this problem, we should \emph{keep} our inner
product, but simply scale down our basis vectors to be
$\frac{1}{\sqrt{\pi}} \sin kt$.

The next important question: are two sinusoidals $\sin kt$ and $\sin k'
t$ orthogonal?

\subsection{Orthogonality of $\sin kt$ and $\sin k' t$}

Here's how I see it. Say the first of these waves is traveling faster
than the second (one has to be if $k1 \ne k2$). Then for every rotation
that the first makes, the second falls $\theta$ radians behind (for
whatever theta).

So consider each time when the first wave is at phase $\phi$. The
question is: what is the sum of the second wave's values at these times?

We know that these angles are all $\theta$ radians apart, and that they
create an entire circuit. Let $\theta$ correspond to some complex number
$c$. We're basically asking: what is

\begin{nedqn}
  1 + c + c^2 .. + c^{n - 1}
\end{nedqn}

when $c^n$ would again equal $1.0$.

The answer is zero. Why? Well, you can see by multiplying everything by
$c$. The original $1.0$ becomes $c$, and note that the final $c^{n-1}$
becomes 1. So you get the same thing. What does that mean? It means,
whatever the sum is, when rotated, it stays the same. And unless $c =
1$, that means that the sum must be zero.

If the sum is zero, both the real and the imaginary parts are zero. Thus
$\sin(kt)$ is orthonormal to $\sin(k't)$.

\subsection{Linearity}

The projection operation is clearly linear, because:

\begin{nedqn}
  \int_0^{2\pi}
  \parens{f(t) + g(t)}
  \cdot
  h(t)
  \dt
\eqcol
  \int_0^{2\pi}
  f(t) \cdot h(t) \dt
  +
  \int_0^{2\pi}
  g(t) \cdot h(t) \dt
\end{nedqn}

So we now see that we have a nice inner product. It is linear in the
correct way, and it declares our chosen basis to be orthonormal. Thus we
can use it for decomposing a sinusoidal into the basis.

% Inspiration: https://math.stackexchange.com/questions/891875/

\section{Adding Cosines In}

Let's start considering

\begin{nedqn}
  f(t)
\eqcol
  \sum_{k = 1}^{\infty}
  a_k \sin kt + b_k \cos kt
\end{nedqn}

\noindent
That is: we now want to extend our basis from just $\sin k t$ to also
include $\cos k t$. This is fine, so long as our inner product still
ensures that any two different basis vectors are orthogonal. By
analogous arguments to what has preceded, we know:

\begin{nedqn}
  \innerprod{\cos k t}{\cos k t}
\eqcol
  1
\\
  \innerprod{\cos k t}{\cos k' t}
\eqcol
  0
\end{nedqn}

This is simple because $\cos kt$ is merely a phase shift of $\sin kt$.
But we must consider one new possibility:

\begin{nedqn}
  \innerprod{\cos kt}{\sin k't}
&&
  \nedcomment{$k$ may equal $k'$}
\end{nedqn}

Note that $\cos kt$ is an ``odd'' function where $\cos -kt = -\cos kt$.
Also note that $\sin k't$ is an ``even'' function where $\sin -k't =
\sin k't$. That means that, overall, $\cos kt \sin k't$ is an odd
function:

\begin{nedqn}
  \cosf{kt} \sinf{k't}
\eqcol
  -\cosf{-kt} \sinf{-k't}
\end{nedqn}

Next, note that the integral over $\parens{0, 2\pi}$ is the same as
$\parens{-\pi, \pi}$. This is true for any function $f$ with a period
dividing $2\pi$. But integrating any odd function over $-T, +T$ should
always give zero. In the specific case of $\cos kt \sin k't$:

\begin{nedqn}
  \int_0^{\pi}
  \cos kt \sin k' t \dt
\eqcol
  -\int_0^{\pi}
  \cos -kt \sin -k' t \dt
  \nedcomment{because odd}
\\
\eqcol
  -\int_{-\pi}^{0}
    \cos kt \sin k' t \dt
\\
&&
  \nedcomment{substitution of $k$ for $-k$}
\end{nedqn}

\noindent
We therefore see that there is no harm in adding the $\cos kt$ in. We
always have

\begin{nedqn}
  \innerprod{\cos kt}{\cos kt} \eqcol 1
\\
\innerprod{\cos kt}{\cos k't} \eqcol 0
  \nedcomment{whenever $k \ne k'$}
\\
  \innerprod{\cos kt}{\sin k't} \eqcol 0
  \nedcomment{regardless whether $k = k'$ or not}
\end{nedqn}

Thus we our inner-product will continue to work as a good decomposer
function even over a space containing sinusoidals offset by
$\frac{\pi}{2}$ radians (that is, $\cosf{kt} = \sinf{kt +
\frac{pi}{2}}$).

\subsection{Rotations between $\sin t$ and $\cos t$ give phase-shifts}

Consider $f = \cos, g = \sin$. Consider the linear combination
$h = a \cos + b \sin$. Then:

\begin{nedqn}
  h(t)
\eqcol
  a \cos t + b \sin t
\end{nedqn}

\noindent
Note that we can always write $(a, b)$ in polar coordinates. That is: as
$r(\cos \varphi, \sin \varphi)$. Let's simplify and assume that $r = 1$
for the moment. Thus, we are working with:

\begin{nedqn}
  h(t)
\eqcol
  \cos\varphi \cos t + \sin\varphi \sin t
\end{nedqn}

\noindent
We can see that this is the projection of a vector rotated by $t$
radians onto a vector rotated by $\varphi$ radians (or vice-versa).
That is the same as:

\begin{nedqn}
  h(t)
\eqcol
  \cos\parens{t - \varphi}
=
  \cos\parens{\varphi - t}
\end{nedqn}

\noindent
That is: a linear combination of $\sin t$ and $\cos t$ is a
\define{phase shift} of $\cos t$ (plus maybe some scaling). Equivalently
we could see this as a phase shift of $\sin t$ (by $\varphi -
\frac{\pi}{2}$ or whatever), since $\cos$ and $\sin$ are phase shifts of
each other.

I say this makes sense. The linear combination of two orthogonal vectors
is always (1) some homogenous scaling (via $r$), and (2) some mixing
(via $\cos\varphi, \sin\varphi$). The mixing can be seen as a \emph{pure
rotation} of one vector toward (or away) from the other.

Here we started with the vector $f = \cos$. We rotated it ``toward'' $g
= \sin$ by $\varphi$ radians. It turns out that the result of a rotation
by $\varphi$ radians is $\cosf{t - \varphi}$, which is exactly equal to
$\cos$ when $\varphi = 0$ and exactly equal to $\cosf{t - \frac{\pi}{2}}
= \sin t$ when $\varphi = \frac{\pi}{2}$.

This is convenient. How did it happen? Certainly it has something to do
with the fact that the very vectors we are rotating between themselves
represent the result of a rotation at different times $t$.

Maybe I wish that the result of the rotation were $\cosf{t + \varphi}$
rather than $\cosf{t - \varphi}$. But if that were the case, a rotation
by $\frac{\pi}{2}$ rad would not give us $\sin t$. Maybe things will
come out cleaner when we work with complex sinusoidal functions.

Anyway, if we include both $\sin kt$ and $\cos kt$ functions in our
basis, the space spanned will include all sinusoidals with (1) period
dividing $2\pi$ and, (2) any phase-shift. Thus our space is the
(algebraic) closure of \emph{all} sinusoidal functions.


\end{document}

% The basis of sine functions is sufficient to span cosines, as any cosine
% function may be written as a phase-shifted sine function.

% We want not only finite linear combinations of basis vectors, but we
% also want *infinite linear combinations*. We want to consider the basis
% as a *topological basis*, rather than merely an *algebraic basis*. But
% we'll consider that later...

% In conclusion, the basis is orthonormal. The projection operation
% continues to be linear. Therefore, we have that the basis is
% algebraically independent. (Even better, if we consider the topology
% induced by the inner product, the basis is topologically independent,
% too).

% ## L2 Inner Product Intuition

% Let me try to give some intuitive pictures for why the integral inner
% product is natural.

% Recall that the standard inner product in `R^n` (the dot product) is
% correct if you want to decompose a vector `u` into (1) a component along
% `v`, and (2) a component in the rank `n-1` space "orthogonal" to `v`.
% This "orthogonal" notion presupposes the concept of rotation of basis
% vectors.

% We can see functions on `R` as an extension to this. Especially if we
% restrict to functions that are the (pointwise?) convergence of step
% functions.

% One could say: in `R^n` we had the notion that we could always "keep
% rotating" and eventually end up back at the starting vector. So that
% suggests that our basis vectors ought to be all "translates" of each
% other, with respect to rotation. But that is also a little odd: in `R^n`
% there are `n-1` angles that define a unit vector, whereas if we choose
% the Fourier basis for functions on `R` there is only one continuous
% variable of rotation...

% A "statistics" view is that you are asking for the "correlation" between
% `u` and `v`. In that setting, the inner product makes sense.

% **TODO**: clarify these random, still un-unified and incomplete thoughts
% into a more coherent story.
