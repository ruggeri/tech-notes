\documentclass[11pt, oneside]{amsart}

\usepackage{geometry}
\geometry{letterpaper}

\usepackage{ned-common}
\usepackage{ned-calculus}
\usepackage{ned-linear-algebra}
\usepackage{ned-stats}

\begin{document}

\title{The Fourier Transform}
\maketitle

\section{Definitions}

First, let's make some definitions. Consider the sinusoidal function
$\sin t$. The \define{period} $T$ of this function is equal to $2\pi$.
The period is the time required for the sinusoidal function to go
through exactly one cycle.

There is a corresponding concept of \define{wavelength} denoted
$\lambda$. This means roughly the same thing as period: the period is a
measurement relative to time, while the wavelength is a measurement
relative to distance. I think wavelength is relevant for waves that are
\emph{travelling} through space. We won't be talking about propagation
of waves through space, so we will only speak in terms of periods here.

The \define{frequency} $f$ is how many cycles occur in one unit of time.
Since $f$ and $T$ are in the same units, we have $f = \frac{1}{T}$.

We may note that:

\begin{nedqn}
  \sin\parens{\frac{2\pi}{T} t}
\eqcol
  \sin\parens{2\pi f t}
\end{nedqn}

For convenience, we also talk in terms of \define{angular frequency}
$\omega = 2\pi f = \frac{2\pi}{T}$. The angular frequency is the number
of cycles that pass in $2\pi$ seconds of time. Thus the function $\sin
t$ has angular frequency $\omega = 1$.

\section{Purpose}

Many phenomena can be described using sinusoids. For instance, when a
note on a guitar is plucked, the pressure on your eardrum due to the
vibration of the string varies through time proportionally to a
sinusoidal function of a given frequency. For instance, the ``concert
A'' pitch is one which vibrates at a frequency of 440 times per second.

That means that when a violinist plays an open A string, the sound
causes a oscillating change in pressure on your eardrum. 440 oscilations
happen every second.

How great is the peak change in pressure? This is called the
\define{amplitude}, which we will denote $a$. The magnitude of $a$ will
correspond to the \emph{volume} you hear. The change in pressure on your
eardrum from a concert A is $a \sin\parens{2\pi \cdot 440 \cdot t}$.

What if \emph{multiple} strings are plucked simultaneously? Say each of
$k$ strings has frequency $f_i$ and is plucked with amplitude $a_i$.
Then the \emph{aggregate} change in your eardrum pressure is denoted
$f$. And we know:

\begin{nedqn}
  f(t)
\eqcol
  \sum_{i = 1}^k
  a_i \sin\parens{2\pi f_i t}
\end{nedqn}

Experimentally, we could measure $f$ with a pressure sensor, and record
our readings. But from the measurements of $f$, how could we find out
which strings were plucked (and at what amplitude)? Basically, we want
to \emph{decompose} $f$ into the constituent sinusoidal waves, but how
do we do that? The answer is the \define{Fourier transform}.

\section{Linear Algebra Review}

Breaking something apart into its constituents is something we do in
\define{linear algebra}. The space of waveforms is a vector space. We
want to decompose a vector $\vv$ into a linear combination of basis
vectors. That's linear algebra.

The function that does decomposing is called a \define{inner product}.
Let's explore what kind of inner product we'll need.

Consider a vector space and a basis for the space $\mathcal{B}$. We will
declare the basis vectors to have ``unit length.'' We will declare that
any rotation between unit vectors is also a unit vector. Recall that a
``rotation'' between two vectors can simply be defined algebraically:

\begin{nedqn}
  \cos\theta \vu + \sin\theta \vv
\end{nedqn}

Since every vector $\vv$ is a scalar multiple of a unit vector (easy to
prove), we can extend our notion of ``length'' to non-unit vectors. Let
$\norm{\vv} = \alpha$ if $\vv = \alpha \vw$ where $\vw$ is a unit-vector
(rotation of basis vectors).

We'll declare the basis vectors to be ``orthogonal.'' We can go further
and define the ``angle'' between two arbitrary unit vectors $\vu, \vv$.
We can start defining this notion by saying the angle between $\ve_i$
and $\vv$. I say that there exists some unique pair $\vw$ and $\theta$
such that: (1) $\vw$ is in the subspace spanned by $\setof{\ve_j | j \ne
i}$, and (2):

\begin{nedqn}
  \vv
\eqcol
  \cos\theta \ve_i + \sin\theta \vw
\end{nedqn}

\noindent
I call $\theta$ the angle between $\ve_i$ and $\vv$. It's the angle
needed to rotate $\ve_i$ in the direction of some (orthogonal) $\vw$
such that the result is $\vv$.

Why do we know that $\vw$ exists? First, we know that $\vv$ lies in
\emph{some} plane defined by $\ve_i$ and some $\vw$, since the basis
spans the vector space. We might as well assume $\vw$ is unit-length.
But then all unit-length vectors in this plane are rotations of $\vu$
toward $\vw$. This (almost) uniquely specifies $\theta$, up to two
choices: $\theta$ vs $2\pi - \theta$.

We can further extend the notion of angle-between to any two unit
vectors by saying that angle-between should be rotation-invariant.

Once we have this notion of angle-between nailed down, we can
unambiguously define the \define{inner product} $\innerprod{\vu}{\vv} =
\cos\theta$, where $\theta$ is the angle in between two unit vectors. We
can then further extend this to vectors of arbitrary length by defining
$\innerprod{\vu}{\vv} = \norm{\vu} \norm{\vv} \cos\theta$. We call
$\frac{\innerprod{\vu}{\vv}}{\norm{\vv}} \vv$ the \define{projection} of
$\vu$ onto $\vv$.

Notice that since we are using $\cos\theta$, it doesn't matter the sign
of $\theta$. We could iron out the sign wrinkles of our
``angle-between'' concept, but we don't need to.

So, if we have a basis for a vector space, this suggests a notion of
``angle-between,'' ``length,'' and even ``projection.'' Alternatively,
given a pre-existing notion of projection defined by
$\innerprod{\cdot}{\cdot}$, any orthonormal basis $\mathcal{B}$ would be
equally ``compatible'' with this inner-product. That is: the standard
construction of angle/length/projection from $\mathcal{B}$ would exactly
correspond to the inner product $\innerprod{\cdot}{\cdot}$ we started
with.

We have called $\innerprod{\vv}{\vu}$ the ``projection'' of $\vv$ onto
$\vu$. This terminology is compatible with our usual geometric
understanding of ``projection.'' If, in $n$-dimensional Euclidean space,
we rotate $\vu$ by $\theta$ degrees and get $\vv$, then it's true that
the projection of $\vv$ onto $\vu$ is:

\begin{nedqn}
  \norm{\vv} \cos\theta
\eqcol
  \frac{\innerprod{\vv}{\vu}}{\norm{\vu}}
\end{nedqn}

Also, the inner-product gives us the way to decompose a vector $\vu$ into
a linear combination of $\ve_i$ in an orthonormal basis $\mathcal{B}$:

\begin{nedqn}
  \innerprod{\vv}{\ve_i}
\eqcol
  \innerprod{\sum_{j=1}^n \alpha_j \ve_j}{\ve_i}
\\
\eqcol
  \sum_{j=1}^n \alpha_j \innerprod{\ve_j}{\ve_i}
  \nedcomment{by linearity}
\\
\eqcol
  \alpha_i
  \nedcomment{other $\alpha_j$ cancel by orthonormality}
\end{nedqn}

\noindent
This proof relied on (1) the basis is orthonormal, (2) the inner product
is \define{linear} in the first argument. Linearity is implied by
$\innerprod{\cdot}{\cdot}$ being rotation-invariant, but I won't prove
that right now.

But we don't need to adopt this proof that comes from linearity. We
could instead rely on our earlier proof that $\innerprod{\vv}{\ve_i}$ is
the cosine of the angle between $\ve_i$ and some $\vw$ that lives in the
subspace spanned by the other basis vectors. In that case, it's clear
that $\alpha_i = \norm{\vv} \cos\theta$. To find the other
$\setof{\alpha_j}$ to complete the decomposition into a linear
combination, we decompose $\norm{\vv} \sin\theta \vw$ in the subspace
orthogonal to $\ve_i$.

To tie things up: consider an orthonormal basis $\mathcal{B}$. Then:

\begin{nedqn}
  \vv
\eqcol
  \sum_{i=1}^n \innerprod{\vv}{\vb_i} \vb_i
\end{nedqn}

Let's finally show that an inner-product needs to act like a \define{dot
product}. The easiest way to show this \emph{is} by proving (1)
linearity in the first argument and (2) (conjugate) symmetry.

To prove (1), we could argue that the geometric projection operation is
clearly linear in the first argument.

By definition we have that $\innerprod{\alpha \vu}{\vv} = \alpha
\innerprod{\vu}{\vv}$. That's because we started by defining
$\innerprod{\cdot}{\cdot}$ as the cosine of the angle between two
\emph{unit} vectors, and only later extended this to non-unit vectors by
saying that $\innerprod{\alpha \vu}{\beta \vv} = \alpha \beta
\innerprod{\vu}{\vv}$.

Next, we consider $\innerprod{\vu + \vv}{\vw}$. We can assume that $\vu,
\vv$ are unit vectors. Now, we know that there exists $\vz_u, \vz_v$
(each orthogonal to $\vw$) and $\theta_u, \theta_v$ such that:

\begin{nedqn}
  \vu
\eqcol
  \cos\theta_u \vw + \sin\theta_u \vz_u
\\
  \vv
\eqcol
  \cos\theta_v \vw + \sin\theta_v \vz_v
\intertext{thus}
  \vu + \vv
\eqcol
  \parens{\cos\theta_u + \cos\theta_v} \vw
  +
  \sin\theta_u \vz_u
  +
  \sin\theta_v \vz_v
\end{nedqn}

\noindent
Since any linear combination of $\vz_u, \vz_v$ is orthogonal to $\vw$,
it's clear that $\innerprod{\vu + \vv}{\vw} = \cos\theta_u +
\cos\theta_v$.

In summary: this shows that projection is linear in the first argument.

Last, we should show that inner-product is symmetric. But this is clear
because $\innerprod{\vu}{\vv} = \norm{\vu}\norm{\vv} \cos\theta$. But
the norms stay the same if we flip the arguments. And the $\theta$
either stays the same (or maybe becomes $2\pi - \theta$), which leaves
$\cos\theta$ the same.

Another way to see this: if $\vu, \vv$ are unit vectors, then the length
of the projection of $\vu$ onto $\vv$ is the same whether $\vv$ is a
rotation by $\theta$ or $-\theta$ degrees.

Once we have linearity in the first argument and symmetry, then it shows
that $\innerprod{\cdot}{\cdot}$ \emph{has} to act like a dot product. It
is equally the dot-product for any basis $\mathcal{B}$ with which you
might want to represent vectors $\vv$ in the vector space.

Last, you could say it has to be the dot-product, because the
dot-product is geometrically correct for $n$-dimensional Euclidean
space, and every finite dimensional vector space is geometrically
equivalent to an $n$-dimensional Euclidean space.

\section{Relationship to Linear Algebra}

So let's return to our problem of breaking up $f$ into a weighted sum of
sinusoidals. That is: we want to find the amplitude $a_i$ for each
frequency $\omega_i$.

To keep things simple, we will only consider angular frequencies $k$
that are an integer multiple of $\omega = 1$. That is, we will want to
break up $f$ into:

\begin{nedqn}
  f(t)
\eqcol
  \sum_{k = 0}^\infty
  a_k \sin k t
\end{nedqn}

We will call the collection of sinusoidals $\sin kt$ our \define{basis}.
That is, we'll treat the sinusoidal functions like vectors. Likewise,
we'll call $f$ a vector. So we're returning to the problem of breaking a
vector into a weighted sum of basis vectors.

To do the breaking up, we'll need to define a new \define{inner product}
for our ``function-vectors.'' We will need the following properties to
hold:

\begin{nedqn}
  \innerprod{\sin k t}{\sin k' t}
\eqcol
  0
  \nedcomment{whenever $k \ne k'$}
\\
  \innerprod{\sin k t}{\sin k t}
\eqcol
  1
  \nedcomment{for every $k$}
\\
  \innerprod{\va + \vb}{\sin k t}
\eqcol
  \innerprod{\va}{\sin k t} + \innerprod{\vb}{\sin k t}
\end{nedqn}

Keep in mind that $\va, \vb$ are ``function-vectors.''

\subsection{Notes about our function space}

First, please note that every $f$ we are decomposing will be periodic
with an angular frequency that is an exact multiple of the base angular
frequency $\omega = 1$. That's because every \emph{constituent} of $f$
has that same property.

Second, please note that $f$ does not consist of any weighted part $\cos
k t$. In particular, please note that $f(t) = 0$ at every $t = k 2\pi$.
We will fix this later and consider ``phase-shifted'' functions $f$.
This will not be too hard.

Third, note that the basis $\sin kt$ is \emph{infinite}. To start, let
us assume that $a_k = 0$ for all but a finite number of $k$. That is:
$f$ is made up of a finite number of sinusoidals. Later we can
generalize this property, too.

\section{Our Inner Product}

So now we must find an appropriate inner product for our function space.
An ``obvious'' analogue to the dot product is:

\begin{nedqn}
  \innerprod{f}{g}
\eqcol
  \int_{t \in \reals}
  f(t)g(t)
  \dt
\end{nedqn}

This won't exactly work. First of all, integration over the entire real
domain is a little tricky. It must involve a limit. Setting this aside,
there is another problem. Consider $\innerprod{\sin t}{\sin t}$. We want
this to be 1.0. But is that true? Consider just:

\begin{nedqn}
  \int_0^{2\pi} \sin^2 t \dt
\end{nedqn}

Note that $\sin t$ is periodic with period $2\pi$. Thus:

\begin{nedqn}
  \int_{k2\pi}^{(k+1)2\pi} \sin^2 t \dt
\eqcol
  \int_{k'2\pi}^{(k'+1)2\pi} \sin^2 t \dt
\end{nedqn}

no matter which choice of $k, k'$. But this is trouble, since if

\begin{nedqn}
  \int_0^{2\pi} \sin^2 t \dt
& > &
  0
\end{nedqn}

then we would have that the integral over the entire real domain is
infinite.

But let's look more closely on the integration from just 0 to $2\pi$.
Integrating over this domain gives us all the information we need.
Integrating over any other period would simply be the same and is thus
redundant. But does the integral come out correctly? Note:

\begin{nedqn}
  \int_0^{2\pi} \sin^2 t + \cos^2 t \dt
\eqcol
  \int_0^{2\pi} 1 \dt
=
  2\pi
\end{nedqn}

But note that:

\begin{nedqn}
  \int_0^{2\pi} \sin^2 t \dt
\eqcol
  \int_0^{2\pi} \cos^2 t \dt
\end{nedqn}

since one is merely a phase-shifted version of the other. Thus we see
that:

\begin{nedqn}
  \int_0^{2\pi} \sin^2 t \dt
\eqcol
  \pi
\end{nedqn}

That isn't exactly what we want, but it's a start. Another way to see
this is to remember that $\parens{\cos t, \sin t}$ is the position on
the circle at angle $t$. The square of the length of that vector is
always $1.0$, because the radius is always $1.0$. If you integrate all
the way around the circle, you should get $2\pi$. Of that, $\sin^2 t$ is
only half the story.

Note that the same argument holds for any $\sin kt$, so we know that
$\norm{\sin kt} = \pi$ for all $k$.

Anyway, to ``fix'' this problem, we should \emph{keep} our inner
product, but simply scale down our basis vectors to be
$\frac{1}{\sqrt{\pi}} \sin kt$.

The next important question: are two sinusoidals $\sin kt$ and $\sin k'
t$ orthogonal?

\subsection{Orthogonality of $\sin kt$ and $\sin k' t$}

Here's how I see it. Say the first of these waves is traveling faster
than the second (one has to be if $k \ne k'$). Then for every rotation
that the first sine wave makes, the second sine wave falls $\theta$
radians behind (for some $\theta$).

Let's start at $t_0 = 0$. Let's consider each the set of times $t_i$ at
which the first wave crosses zero again. In fact, let's consider
$\parens{\cos kt_i, \sin kt_i}$ as a pair. We'll consider the points
$t_i$ where $\parens{\cos kt_i, \sin kt_i} = \parens{1, 0}$.

Now, let's consider the corresponding $\parens{\cos k't_i, \sin k't_i}$.
We know this starts at $\parens{1, 0}$ for $t_0$, but subsequent $t_i$
will differ. As we said: each $\parens{\cos k't_i, \sin k't_i}$ falls
behind by a fixed angle $\theta$. Specifically, we know:

\begin{nedqn}
  \parens{\cos k't_m, \sin k't_m}
\eqcol
  \parens{\cos m\theta, \sin m\theta}
\end{nedqn}

Now, I claim the following:

\begin{nedqn}
  \sum_{m = 0}^n
  \parens{\cos m\theta, \sin m\theta}
\eqcol
  \parens{0, 0}
\end{nedqn}

\subsection{An Aside About Complex Numbers}

To see this, let's move to complex numbers. Complex numbers of norm 1.0
represent positions on the circle. The real part is along the $x$-axis,
and the imaginary part is along the $y$-axis.

Multiplication by a complex number corresponds to \emph{rotation}.
Consider $c = \parens{\cos \varphi, \sin \varphi}$ It's easy to see that
multiplying $1$ by $c$ maps $1$ to $c$. $1$ originally lived at
zero-degrees CCW from the $x$-axis. Now it corresponds to $\varphi$
degrees CCW from the $x$-axis.

Likewise one can consider $i$. $ci$ maps the $x$-coordinate of $c$ to
the $y$-coordinate. And the $y$-coordinate is mapped to the
$x$-coordinate, but in the opposite direction. This is exactly
corresponding to a CCW rotation of $i$.

\subsection{Completing Our Proof}

So repeated rotation of $\parens{1, 0}$ by $\theta$ can be expressed as:

\begin{nedqn}
  1, c, c^2 + \ldots, c^{n - 1}
\end{nedqn}

Now consider what would happen if we summed all these terms. We would
get zero. Why? Consider multiplying everything by $c$. The original $1$
becomes $c$, and note that the final $c^{n-1}$ becomes $1$. So you get
the same thing as you started with.

When does a $cx = x$? Either when $c = 1$ (no rotation), or when $x = 0$
(rotation does nothing to the zero vector).

We thus know that the sum is zero (since $c$ corresponds to a rotation
by $\varphi > 0$).

Thus the sum for of $\sin k't \cos k't$ at points $t_0, \ldots, t_n$ is
zero. And in fact this holds regardless of starting point $t_0 \ne 0$.
Thus:

\begin{nedqn}
  \int \sin kt \sin k't \dt
\eqcol
  0.0
\end{nedqn}

Thus the two are orthogonal.

\subsection{Linearity}

The projection operation is clearly linear, because:

\begin{nedqn}
  \int_0^{2\pi}
  \parens{f(t) + g(t)}
  \cdot
  h(t)
  \dt
\eqcol
  \int_0^{2\pi}
  f(t) \cdot h(t) \dt
  +
  \int_0^{2\pi}
  g(t) \cdot h(t) \dt
\end{nedqn}

So we now see that we have a nice inner product. It is linear in the
correct way, and it declares our chosen basis to be orthonormal. Thus we
can use it for decomposing a sinusoidal into the basis.

% Inspiration: https://math.stackexchange.com/questions/891875/

\section{Adding Cosines In}

Let's start considering

\begin{nedqn}
  f(t)
\eqcol
  \sum_{k = 1}^{\infty}
  a_k \sin kt + b_k \cos kt
\end{nedqn}

\noindent
That is: we now want to extend our basis from just $\sin k t$ to also
include $\cos k t$. This is fine, so long as our inner product still
ensures that any two different basis vectors are orthogonal. By
analogous arguments to what has preceded, we know:

\begin{nedqn}
  \innerprod{\cos k t}{\cos k t}
\eqcol
  1
\\
  \innerprod{\cos k t}{\cos k' t}
\eqcol
  0
\end{nedqn}

This is simple because $\cos kt$ is merely a phase shift of $\sin kt$.
But we must consider one new possibility:

\begin{nedqn}
  \innerprod{\cos kt}{\sin k't}
&&
  \nedcomment{$k$ may equal $k'$}
\end{nedqn}

Note that $\cos kt$ is an ``odd'' function where $\cos -kt = -\cos kt$.
Also note that $\sin k't$ is an ``even'' function where $\sin -k't =
\sin k't$. That means that, overall, $\cos kt \sin k't$ is an odd
function:

\begin{nedqn}
  \cosf{kt} \sinf{k't}
\eqcol
  -\cosf{-kt} \sinf{-k't}
\end{nedqn}

Next, note that the integral over $\parens{0, 2\pi}$ is the same as
$\parens{-\pi, \pi}$. This is true for any function $f$ with a period
dividing $2\pi$. But integrating any odd function over $-T, +T$ should
always give zero. In the specific case of $\cos kt \sin k't$:

\begin{nedqn}
  \int_0^{\pi}
  \cos kt \sin k' t \dt
\eqcol
  -\int_0^{\pi}
  \cos -kt \sin -k' t \dt
  \nedcomment{because odd}
\\
\eqcol
  -\int_{-\pi}^{0}
    \cos kt \sin k' t \dt
\\
&&
  \nedcomment{substitution of $k$ for $-k$}
\end{nedqn}

\noindent
We therefore see that there is no harm in adding the $\cos kt$ in. We
always have

\begin{nedqn}
  \innerprod{\cos kt}{\cos kt} \eqcol 1
\\
\innerprod{\cos kt}{\cos k't} \eqcol 0
  \nedcomment{whenever $k \ne k'$}
\\
  \innerprod{\cos kt}{\sin k't} \eqcol 0
  \nedcomment{regardless whether $k = k'$ or not}
\end{nedqn}

Thus we our inner-product will continue to work as a good decomposer
function even over a space containing sinusoidals offset by
$\frac{\pi}{2}$ radians (that is, $\cosf{kt} = \sinf{kt +
\frac{pi}{2}}$).

\subsection{Rotations between $\sin t$ and $\cos t$ give phase-shifts}

Consider $f = \cos, g = \sin$. Consider the linear combination
$h = a \cos + b \sin$. Then:

\begin{nedqn}
  h(t)
\eqcol
  a \cos t + b \sin t
\end{nedqn}

\noindent
Note that we can always write $(a, b)$ in polar coordinates. That is: as
$r(\cos \varphi, \sin \varphi)$. Let's simplify and assume that $r = 1$
for the moment. Thus, we are working with:

\begin{nedqn}
  h(t)
\eqcol
  \cos\varphi \cos t + \sin\varphi \sin t
\end{nedqn}

\noindent
We can see that this is the projection of a vector rotated by $t$
radians onto a vector rotated by $\varphi$ radians (or vice-versa).
That is the same as:

\begin{nedqn}
  h(t)
\eqcol
  \cos\parens{t - \varphi}
=
  \cos\parens{\varphi - t}
\end{nedqn}

\noindent
That is: a linear combination of $\sin t$ and $\cos t$ is a
\define{phase shift} of $\cos t$ (plus maybe some scaling). Equivalently
we could see this as a phase shift of $\sin t$ (by $\varphi -
\frac{\pi}{2}$ or whatever), since $\cos$ and $\sin$ are phase shifts of
each other.

I say this makes sense. The linear combination of two orthogonal vectors
is always (1) some homogenous scaling (via $r$), and (2) some mixing
(via $\cos\varphi, \sin\varphi$). The mixing can be seen as a \emph{pure
rotation} of one vector toward (or away) from the other.

Here we started with the vector $f = \cos$. We rotated it ``toward'' $g
= \sin$ by $\varphi$ radians. It turns out that the result of a rotation
by $\varphi$ radians is $\cosf{t - \varphi}$, which is exactly equal to
$\cos$ when $\varphi = 0$ and exactly equal to $\cosf{t - \frac{\pi}{2}}
= \sin t$ when $\varphi = \frac{\pi}{2}$.

This is convenient. How did it happen? Certainly it has something to do
with the fact that the very vectors we are rotating between themselves
represent the result of a rotation at different times $t$.

Maybe I wish that the result of the rotation were $\cosf{t + \varphi}$
rather than $\cosf{t - \varphi}$. But if that were the case, a rotation
by $\frac{\pi}{2}$ rad would not give us $\sin t$. Maybe things will
come out cleaner when we work with complex sinusoidal functions.

Anyway, if we include both $\sin kt$ and $\cos kt$ functions in our
basis, the space spanned will include all sinusoidals with (1) period
dividing $2\pi$ and, (2) any phase-shift. Thus our space is the
(algebraic) closure of \emph{all} sinusoidal functions.

\section{Complex Sinusoidals}

We know that $e^x$ is defined so that this function has itself as its
own derivative. More generally: $\expf{\alpha x}$ has derivative
$\alpha \expf{\alpha x}$. If you interpret $\alpha$ as an
interest rate, that's how we talk about continuously compounding
interest.

$\alpha$ is telling you how to scale your current position to calculate
your velocity. When $\alpha = 1$, your position is equal to your
velocity.

What about $\expf{it}$? How will we \emph{choose} to define that?
Well, we want our same property to hold. We want the velocity to equal
$i\expf{it}$.

That says: we want velocity always to be perpendicular to the position,
with equal magnitude. We start with $t = 0$ and $\expf{i0} = 1$.

We know what happens when velocity is always exactly orthogonal to
angle. That describes a counter-clockwise rotation around the unit
circle with constant speed equal to $1.0$. A rotation takes $2\pi$
seconds. And the position at any time $t$ is given by:

\begin{nedqn}
  \cos(t) + i\sin(t)
\end{nedqn}

And that's what $\expf{it}$ will be defined as. This is
\define{Euler's formula}.

Note that therefore:

\begin{nedqn}
  \expf{i \pi}
\eqcol
  \cos(\pi) + i \sin(\pi)
\\\eqcol
  -1
\end{nedqn}

This is \define{Euler's identity}.

\subsection{Complex Sinusoidal Function Space}

We've been considering a space with basis vectors such as $\sin \omega
t, \cos \omega' t$. I want to begin to explore the space spanned by
$\expf{i \omega t}$. These are complex valued functions, but
restricted to just their real component they are simply $\cos \omega t$.

Our life would be less rich if we only considered linear combinations of
the $\expf{i \omega t}$ with real weights. In particular: we'd lose
functions where the real part corresponded to $\sin \omega t$. That is:
no two basis vectors would be phase-shifts of each other.

Well, there's an easy way to fix that! A phase shift is simply a
rotation by a complex number $c$. So we should allow linear combinations
with \emph{complex weights}.

Note that even though we are now allowing complex amplitudes that
phase-shift the basis sinusoidal, we do \emph{not} allow complex
$\omega$. Our $\omega$ will continue to be real, else the velocity would
not always be tangent to the position, and the motion described would
not be circular.

Note though that $\omega$ may be \emph{negative}. A negative $\omega$
corresponds to a clockwise rotation, whereas a positive $\omega$
corresponds to a counter-clockwise rotation. We previously did not need
to consider negative frequencies, because these (1) only would have
affected $\sin$ waves, and (2) were already handled by negative
amplitudes. Basically: a real valued wave $\sin -t$ looked like an out
of phase version of $\sin t$. With appropriate amplitudes on $\cos t$
and $\sin t$ we could construct $\sin -t$. But that is no longer the
case with complex-valued $\expf{it}$. No two counter-clockwise-rotating
sinusoidals can add up to a clockwise-rotating sinusoidal. A complex
clockwise-rotating is not merely an out-of-phase version of a complex
counter-clockwise-rotating sinusoidal.

\subsection{Inner Product In Complex Vector Space}

Consider if we are given $c \expf{\omega t}$ and want to recover $c$. If
$\omega, t$ are known, then we can multiply by $\expf{-\omega t}$.

There is a way to see this. Multiplying $c$ by $\expf{\omega t}$ rotates
$c$ counter-clockwise by $\omega t$ radians. To return to $c$, we can
rotate \emph{clockwise} by $\omega t$ radians. That corresponds to
multiplication by $\expf{-\omega t}$.

Consider if we don't know $\omega$. For most choices of $\omega'$
$\expf{\parens{\omega - \omega'} t} \ne 1$. But the solution is to
consider $\expf{\omega t}$ not at one point in time, but across its
period:

\begin{nedqn}
  \int_0^{2\pi} c \expf{\omega t} \expf{-\omega t} \dt
\eqcol
  c \int_0^{2\pi} \expf{\parens{\omega - \omega'} t} \dt
\end{nedqn}

Again, assuming that $\omega, \omega'$ are integers, then all our old
arguments hold when $\omega \ne \omega'$. In fact, we can see them more
easily. We're saying that there is a non-zero $k = \omega - \omega'$,
and thus:

\begin{nedqn}
  \int_0^{2\pi} \expf{\parens{\omega - \omega'} t} \dt
\eqcol
  \int_0^{2\pi} \expf{k t} \dt
\end{nedqn}

This is clearly zero. Through the period the function $\expf{k t}$ will
make one (or more) rotations counter-clockwise (or clockwise) through
the unit circle. But it will end where it started. So the total
displacement is zero.

Naturally things are different when $\omega = \omega'$. Then:

\begin{nedqn}
  \int_0^{2\pi} \expf{\parens{\omega - \omega'} t} \dt
\eqcol
  \int_0^{2\pi} 1 \dt
\\\eqcol
  2\pi
\end{nedqn}

Ah, notice the slight difference from before! The ``correction'' factor
we should apply is not $\frac{1}{\sqrt\pi}$, but now
$\frac{1}{\sqrt{2\pi}}$! The reason is that when we worked with
real-valued sinusoidals, we were ignoring half the story: the imaginary
part!

\subsection{Conjugation}

In general, to do a projection of $f$ onto $g$, we write:

\begin{nedqn}
  \innerprod{f}{g}
\eqcol
  \int f(t) \conj{g(t)} \dt
\end{nedqn}

Why use the conjugate? Isn't that different from when we worked with
projections with a real scalar field? Not exactly: we wouldn't have
``noticed'' conjugation in our original dot-product projection because
reals never had any imaginary part.

Of course, we can see the conjugate as a clockwise rotation vs a
counter-clockwise rotation. We can also see it as an ``unnormalized''
inverse of $g(t)$ that removes any imaginary/rotational component.

\subsection{And Real-Valued Sinusoidals?}

What of real-valued sinusoidals like $\cos t$? Are they spanned by our
basis of complex sinusoidals $\expf{i\omega t}$? Yes!

\begin{nedqn}
  \cos t
\eqcol
  \half \expf{i\omega t}
  +
  \half \expf{-i\omega t}
\\\eqcol
  \half
  \parens{\cosf{\omega t} + i\sinf{\omega t}}
  +
  \half
  \parens{\cosf{\omega t} - i\sinf{\omega t}}
\\\eqcol
  \cosf{\omega t}
\end{nedqn}

This shows that our basis of complex sinusoidals is strictly more
expressive than our original basis of real-valued sinusoidals!

\subsection{Some Last Intuitions}

Let's think a little more about $f(t)\conj{g(t)}$. This is now not only
collecting amplitude information: it is collecting \emph{phase}
information. It is recording how many \emph{radians} $f(t)$ is ahead of
$g(t)$.

The inner product, which integrates this, is giving us the
\emph{average} phase difference between $f(t)$ and $g(t)$. As we've
mentioned before: if $f(t)$ and $g(t)$ have different periods, then for
every time that $f(t)$ is $\varphi$ radians ahead of $g(t)$, there is a
corresponding time when it is $\varphi$ radians \emph{behind} $g(t)$.

If you treat the $\varphi$ as complex numbers $c$ and average, you will
get zero. There is a net-zero correlation.

The only time the average phase-difference will be \emph{non-zero} is if
the two waves move in lock-step. Then $c$ is \emph{always the same}.

Last, consider two sinusoidals $\expf{i\omega t}, \expf{-i\omega t}$.
These should have zero correlation, because they are rotating in
opposite directions.


\end{document}

% We want not only finite linear combinations of basis vectors, but we
% also want *infinite linear combinations*. We want to consider the basis
% as a *topological basis*, rather than merely an *algebraic basis*. But
% we'll consider that later...

% In conclusion, the basis is orthonormal. The projection operation
% continues to be linear. Therefore, we have that the basis is
% algebraically independent. (Even better, if we consider the topology
% induced by the inner product, the basis is topologically independent,
% too).

% ## L2 Inner Product Intuition

% Let me try to give some intuitive pictures for why the integral inner
% product is natural.

% Recall that the standard inner product in `R^n` (the dot product) is
% correct if you want to decompose a vector `u` into (1) a component along
% `v`, and (2) a component in the rank `n-1` space "orthogonal" to `v`.
% This "orthogonal" notion presupposes the concept of rotation of basis
% vectors.

% We can see functions on `R` as an extension to this. Especially if we
% restrict to functions that are the (pointwise?) convergence of step
% functions.

% One could say: in `R^n` we had the notion that we could always "keep
% rotating" and eventually end up back at the starting vector. So that
% suggests that our basis vectors ought to be all "translates" of each
% other, with respect to rotation. But that is also a little odd: in `R^n`
% there are `n-1` angles that define a unit vector, whereas if we choose
% the Fourier basis for functions on `R` there is only one continuous
% variable of rotation...

% A "statistics" view is that you are asking for the "correlation" between
% `u` and `v`. In that setting, the inner product makes sense.

% **TODO**: clarify these random, still un-unified and incomplete thoughts
% into a more coherent story.

% Note that now we have the possibility of infinite sums. This is an
% *infinite dimensional vector space*. This brings in a *topological*
% question. As in: what does it mean for an infinite series to converge?
% But let's just stick to finite linear combinations of the basis vectors
% to start. We'll get back to topology later.
