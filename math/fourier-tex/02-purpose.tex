\section{Purpose}

Many phenomena can be described using sinusoids. For instance, when a
note on a guitar is plucked, the pressure on your eardrum due to the
vibration of the string varies through time proportionally to a
sinusoidal function of a given frequency. For instance, the ``concert
A'' pitch is one which vibrates at a frequency of 440 times per second.

That means that when a violinist plays an open A string, the sound
causes a oscillating change in pressure on your eardrum. 440 oscilations
happen every second.

How great is the peak change in pressure? This is called the
\define{amplitude}, which we will denote $a$. The magnitude of $a$ will
correspond to the \emph{volume} you hear. The change in pressure on your
eardrum from a concert A is $a \sin\parens{2\pi \cdot 440 \cdot t}$.

What if \emph{multiple} strings are plucked simultaneously? Say each of
$k$ strings has frequency $f_i$ and is plucked with amplitude $a_i$.
Then the \emph{aggregate} change in your eardrum pressure is denoted
$f$. And we know:

\begin{nedqn}
  f(t)
\eqcol
  \sum_{i = 1}^k
  a_i \sin\parens{2\pi f_i t}
\end{nedqn}

Experimentally, we could measure $f$ with a pressure sensor, and record
our readings. But from the measurements of $f$, how could we find out
which strings were plucked (and at what amplitude)? Basically, we want
to \emph{decompose} $f$ into the constituent sinusoidal waves, but how
do we do that? The answer is the \define{Fourier transform}.
