\section{Complex Sinusoidals}

We know that $e^x$ is defined so that this function has itself as its
own derivative. More generally: $\expf{\alpha x}$ has derivative
$\alpha \expf{\alpha x}$. If you interpret $\alpha$ as an
interest rate, that's how we talk about continuously compounding
interest.

$\alpha$ is telling you how to scale your current position to calculate
your velocity. When $\alpha = 1$, your position is equal to your
velocity.

What about $\expf{it}$? How will we \emph{choose} to define that?
Well, we want our same property to hold. We want the velocity to equal
$i\expf{it}$.

That says: we want velocity always to be perpendicular to the position,
with equal magnitude. We start with $t = 0$ and $\expf{i0} = 1$.

We know what happens when velocity is always exactly orthogonal to
angle. That describes a counter-clockwise rotation around the unit
circle with constant speed equal to $1.0$. A rotation takes $2\pi$
seconds. And the position at any time $t$ is given by:

\begin{nedqn}
  \cos(t) + i\sin(t)
\end{nedqn}

\noindent
And that's what $\expf{it}$ will be defined as. This is \define{Euler's
formula}. Note that therefore:

\begin{nedqn}
  \expf{i \pi}
\eqcol
  \cos(\pi) + i \sin(\pi)
\\\eqcol
  -1
\end{nedqn}

\noindent
This is \define{Euler's identity}.

\subsection{Complex Sinusoidal Function Space}

We've been considering a space with basis vectors such as $\sin \omega
t, \cos \omega' t$. I want to begin to explore the space spanned by
$\expf{i \omega t}$. These are complex valued functions $\cosf{\omega t}
+ i \sinf{\omega t}$, but restricted to just their real component they
are simply $\cos \omega t$.

Our life would be less rich if we only considered linear combinations of
the $\expf{i \omega t}$ with real weights. In particular: we'd lose
functions where the real part corresponded to $\sin \omega t$. That is:
no two basis vectors would be phase-shifts of each other.

Well, there's an easy way to fix that! A phase shift is simply a
rotation by a complex number $c$. So we should allow linear combinations
with \emph{complex weights}. Let's show how to recover a function whose
real part is $\sinf{\omega t}$:

\begin{nedqn}
  -i \expf{i \omega t}
\eqcol
  -i \parens{
    \cosf{\omega t} + i \sinf{\omega t}
  }
\\
\eqcol
  \sinf{\omega t} - i \cosf{\omega t}
\end{nedqn}

Note that even though we are now allowing complex amplitudes that
phase-shift the basis sinusoidal, we do \emph{not} allow complex
$\omega$. Our $\omega$ will continue to be real, else the velocity would
not always be tangent to the position, and the motion described would
not be circular.

Note though that $\omega$ may be \emph{negative}. A negative $\omega$
corresponds to a clockwise rotation, whereas a positive $\omega$
corresponds to a counter-clockwise rotation. We previously did not need
to consider negative frequencies, because these (1) only would have
affected $\sin$ waves, and (2) were already handled by negative
amplitudes.

Now the direction of rotation matters. We showed that an amplitude of
$i$ allows us to construct $\sinf{\omega t} - i \cosf{\omega t}$. But if
we want to construct $\sinf{\omega t} + i \cosf{\omega t}$, this
necessitates a completely opposite angular frequency. No two
counter-clockwise-rotating sinusoidals can add up to a
clockwise-rotating sinusoidal. A complex clockwise-rotating is not
merely an out-of-phase version of a complex counter-clockwise-rotating
sinusoidal.

\subsection{Inner Product In Complex Vector Space}

Consider if we are given $c \expf{\omega t}$ and want to recover $c$. If
$\omega, t$ are known, then we can multiply by $\expf{-\omega t}$.

There is a way to see this. Multiplying $c$ by $\expf{\omega t}$ rotates
$c$ counter-clockwise by $\omega t$ radians. To return to $c$, we can
rotate \emph{clockwise} by $\omega t$ radians. That corresponds to
multiplication by $\expf{-\omega t}$.

Consider if we don't know $\omega$. For most choices of $\omega'$
$\expf{\parens{\omega - \omega'} t} \ne 1$. But the solution is to
consider $\expf{\omega t}$ not at one point in time, but across its
period:

\begin{nedqn}
  \int_0^{2\pi} c \expf{\omega t} \expf{-\omega t} \dt
\eqcol
  c \int_0^{2\pi} \expf{\parens{\omega - \omega'} t} \dt
\end{nedqn}

Again, assuming that $\omega, \omega'$ are integers, then all our old
arguments hold when $\omega \ne \omega'$. In fact, we can see them more
easily. We're saying that there is a non-zero $k = \omega - \omega'$,
and thus:

\begin{nedqn}
  \int_0^{2\pi} \expf{\parens{\omega - \omega'} t} \dt
\eqcol
  \int_0^{2\pi} \expf{k t} \dt
\end{nedqn}

This is clearly zero. Through the period the function $\expf{k t}$ will
make one (or more) rotations counter-clockwise (or clockwise) through
the unit circle. But it will end where it started. So the total
displacement is zero.

Naturally things are different when $\omega = \omega'$. Then:

\begin{nedqn}
  \int_0^{2\pi} \expf{\parens{\omega - \omega'} t} \dt
\eqcol
  \int_0^{2\pi} 1 \dt
\\\eqcol
  2\pi
\end{nedqn}

Ah, notice the slight difference from before! The ``correction'' factor
we should apply is not $\frac{1}{\sqrt\pi}$, but now
$\frac{1}{\sqrt{2\pi}}$! The reason is that when we worked with
real-valued sinusoidals, we were ignoring half the story: the imaginary
part!

\subsection{Conjugation}

In general, to do a projection of $f$ onto $g$, we write:

\begin{nedqn}
  \innerprod{f}{g}
\eqcol
  \int f(t) \conj{g(t)} \dt
\end{nedqn}

Why use the conjugate? Isn't that different from when we worked with
projections with a real scalar field? Not exactly: we wouldn't have
``noticed'' conjugation in our original dot-product projection because
reals never had any imaginary part.

Of course, we can see the conjugate as a clockwise rotation vs a
counter-clockwise rotation. We can also see it as an ``unnormalized''
inverse of $g(t)$ that removes any imaginary/rotational component.

\subsection{And Real-Valued Sinusoidals?}

What of real-valued sinusoidals like $\cos t$? Are they spanned by our
basis of complex sinusoidals $\expf{i\omega t}$? Yes!

\begin{nedqn}
  \cos \omega t
\eqcol
  \half \expf{i\omega t}
  +
  \half \expf{-i\omega t}
\\\eqcol
  \half
  \parens{\cosf{\omega t} + i\sinf{\omega t}}
  +
  \half
  \parens{\cosf{\omega t} - i\sinf{\omega t}}
\\\eqcol
  \cosf{\omega t}
\end{nedqn}

This shows that our basis of complex sinusoidals is strictly more
expressive than our original basis of real-valued sinusoidals!

\subsection{Some Last Intuitions}

Let's think a little more about $f(t)\conj{g(t)}$. This is now not only
collecting amplitude information: it is collecting \emph{phase}
information. It is recording how many \emph{radians} $f(t)$ is ahead of
$g(t)$.

The inner product, which integrates this, is giving us the
\emph{average} phase difference between $f(t)$ and $g(t)$. As we've
mentioned before: if $f(t)$ and $g(t)$ have different periods, then for
every time that $f(t)$ is $\varphi$ radians ahead of $g(t)$, there is a
corresponding time when it is $\varphi$ radians \emph{behind} $g(t)$.

If you treat the $\varphi$ as complex numbers $c$ and average, you will
get zero. There is a net-zero correlation.

The only time the average phase-difference will be \emph{non-zero} is if
the two waves move in lock-step. Then $c$ is \emph{always the same}.

Last, consider two sinusoidals $\expf{i\omega t}, \expf{-i\omega t}$.
These again will have zero correlation, because they are rotating in
opposite directions.
