\section{Relationship to Linear Algebra}

So let's return to our problem of breaking up $f$ into a weighted sum of
sinusoidals. That is: we want to find the amplitude $a_i$ for each
frequency $\omega_i$.

$\omega_i$ could be anything, but let's keep things simple for now.
Let's only consider $\omega_i$ that are integers. We will only consider
functions $f$ where we can break it up like so:

\begin{nedqn}
  f(t)
\eqcol
  \sum_{k = 0}^\infty
  a_k \sin k t
\end{nedqn}

We will call the collection of sinusoidals $\sin kt$ our \define{basis}.
That is, we'll treat the sinusoidal functions like vectors. Likewise,
we'll call $f$ a vector. So we're returning to the problem of breaking a
vector into a weighted sum of basis vectors.

To do the breaking up, we'll need to define a new \define{inner product}
for our ``function-vectors.'' We will need the following properties to
hold:

\begin{nedqn}
  \innerprod{\sin k t}{\sin k' t}
\eqcol
  0
  \nedcomment{whenever $k \ne k'$}
\\
  \innerprod{\sin k t}{\sin k t}
\eqcol
  1
  \nedcomment{for every $k$}
\\
  \innerprod{\va + \vb}{\sin k t}
\eqcol
  \innerprod{\va}{\sin k t} + \innerprod{\vb}{\sin k t}
\end{nedqn}

Keep in mind that $\va, \vb$ are ``function-vectors.''

\subsection{Notes about our function space}

First, please note that every $f$ we are decomposing will be periodic
with an angular frequency that is an exact multiple of the base angular
frequency $\omega = 1$. That's because every \emph{constituent} of $f$
has that same property.

Second, please note that $f$ does not consist of any weighted part $\cos
k t$. In particular, please note that $f(t) = 0$ at every $t = k 2\pi$.
We will fix this later and consider ``phase-shifted'' functions $f$.
This will not be too hard.

Third, note that the basis $\sin kt$ is \emph{infinite}. To start, let
us assume that $a_k = 0$ for all but a finite number of $k$. That is:
$f$ is made up of a finite number of sinusoidals. Later we can
generalize this property, too.
