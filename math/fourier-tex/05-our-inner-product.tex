\section{Our Inner Product}

So now we must find an appropriate inner product for our function space.
An ``obvious'' analogue to the dot product is:

\begin{nedqn}
  \innerprod{f}{g}
\eqcol
  \int_{t \in \reals}
  f(t)g(t)
  \dt
\end{nedqn}

This won't exactly work. First of all, integration over the entire real
domain is a little tricky. It must involve a limit. Setting this aside,
there is another problem. Consider $\innerprod{\sin t}{\sin t}$. We want
this to be 1.0. But is that true? Consider just:

\begin{nedqn}
  \int_0^{2\pi} \sin^2 t \dt
\end{nedqn}

Note that $\sin t$ is periodic with period $2\pi$. Thus:

\begin{nedqn}
  \int_{k2\pi}^{(k+1)2\pi} \sin^2 t \dt
\eqcol
  \int_{k'2\pi}^{(k'+1)2\pi} \sin^2 t \dt
\end{nedqn}

no matter which choice of $k, k'$. But this is trouble, since if

\begin{nedqn}
  \int_0^{2\pi} \sin^2 t \dt
& > &
  0
\end{nedqn}

then we would have that the integral over the entire real domain is
infinite.

But let's look more closely on the integration from just 0 to $2\pi$.
Integrating over this domain gives us all the information we need.
Integrating over any other period would simply be the same and is thus
redundant. But does the integral come out correctly? Note:

\begin{nedqn}
  \int_0^{2\pi} \sin^2 t + \cos^2 t \dt
\eqcol
  \int_0^{2\pi} 1 \dt
=
  2\pi
\end{nedqn}

But note that:

\begin{nedqn}
  \int_0^{2\pi} \sin^2 t \dt
\eqcol
  \int_0^{2\pi} \cos^2 t \dt
\end{nedqn}

since one is merely a phase-shifted version of the other. Thus we see
that:

\begin{nedqn}
  \int_0^{2\pi} \sin^2 t \dt
\eqcol
  \pi
\end{nedqn}

That isn't exactly what we want, but it's a start. Another way to see
this is to remember that $\parens{\cos t, \sin t}$ is the position on
the circle at angle $t$. The square of the length of that vector is
always $1.0$, because the radius is always $1.0$. If you integrate all
the way around the circle, you should get $2\pi$. Of that, $\sin^2 t$ is
only half the story.

Note that the same argument holds for any $\sin kt$, so we know that
$\norm{\sin kt} = \pi$ for all $k$.

Anyway, to ``fix'' this problem, we should \emph{keep} our inner
product, but simply scale down our basis vectors to be
$\frac{1}{\sqrt{\pi}} \sin kt$.

The next important question: are two sinusoidals $\sin kt$ and $\sin k'
t$ orthogonal?

\subsection{Orthogonality of $\sin kt$ and $\sin k' t$}

Here's how I see it. Say the first of these waves is traveling faster
than the second (one has to be if $k1 \ne k2$). Then for every rotation
that the first makes, the second falls $\theta$ radians behind (for
whatever theta).

So consider each time when the first wave is at phase $\phi$. The
question is: what is the sum of the second wave's values at these times?

We know that these angles are all $\theta$ radians apart, and that they
create an entire circuit. Let $\theta$ correspond to some complex number
$c$. We're basically asking: what is

\begin{nedqn}
  1 + c + c^2 .. + c^{n - 1}
\end{nedqn}

when $c^n$ would again equal $1.0$.

The answer is zero. Why? Well, you can see by multiplying everything by
$c$. The original $1.0$ becomes $c$, and note that the final $c^{n-1}$
becomes 1. So you get the same thing. What does that mean? It means,
whatever the sum is, when rotated, it stays the same. And unless $c =
1$, that means that the sum must be zero.

If the sum is zero, both the real and the imaginary parts are zero. Thus
$\sin(kt)$ is orthonormal to $\sin(k't)$.

\subsection{Linearity}

The projection operation is clearly linear, because:

\begin{nedqn}
  \int_0^{2\pi}
  \parens{f(t) + g(t)}
  \cdot
  h(t)
  \dt
\eqcol
  \int_0^{2\pi}
  f(t) \cdot h(t) \dt
  +
  \int_0^{2\pi}
  g(t) \cdot h(t) \dt
\end{nedqn}

So we now see that we have a nice inner product. It is linear in the
correct way, and it declares our chosen basis to be orthonormal. Thus we
can use it for decomposing a sinusoidal into the basis.

% Inspiration: https://math.stackexchange.com/questions/891875/
