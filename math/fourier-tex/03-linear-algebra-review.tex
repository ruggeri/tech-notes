\section{Linear Algebra Review}

Breaking something apart into its constituents is something we do in
\define{linear algebra}.

Consider a basis of the space $\mathcal{B}$. We will declare the basis
vectors to have unit length. We will declare that any rotation between
unit vectors is also a unit vector.

We'll declare the basis vectors to be orthogonal. We can go further and
define the angle between two arbitrary unit vectors $\vu, \vv$. We can
start defining this notion by saying the angle between $\ve_i$ and $\vv$
corresponds to the angle needed to rotate $\ve_i$ toward a vector $\vw$
in the subspace spanned by the other basis vectors.

Why do we know that $\vw$ exists? First, we know that $\vv$ lies in
\emph{some} plane defined by $\ve_i$ and some $\vw$, since the basis
spans the vector space. We might as well assume $\vw$ is unit-length.
But then all unit-length vectors in this plane are rotations of $\vu$
toward $\vw$.

This (almost) uniquely specifies $\theta$, up to two choices: $\theta$
vs $2\pi - \theta$.

We can further extend the notion of angle-between to any two unit
vectors by saying that angle-between should be rotation-invariant.

Once we have this notion of angle-between nailed down, we can
unambiguously define $\innerprod{\vu}{\vv} = \cos\theta$, where $\theta$
is the angle in between two unit vectors. We can then further extend
this to vectors of arbitrary length by defining $\innerprod{\vu}{\vv} =
\norm{\vu} \norm{\vv} \cos\theta$.

Notice that since we are using $\cos\theta$, it doesn't matter the sign
of $\theta$. We could iron out the sign wrinkles, but we don't need to.

So, if we have a basis for a vector space, this suggests a notion of
``angle-between,'' and even ``projection.'' Equivalently, a notion of
``angle-between'' (``projection'') would suggest a set of privileged
bases which consist of orthogonal (orthonormal) vectors.

Note that if $\norm{\vu} = 1$, then $\innerprod{\vv}{\vu}$ is exactly
the length of the ``projection'' of $\vv$ onto $\vu$.

Thus, the inner-product will work as a decomposer for any orthonormal
basis. As I said before, a notion of ``angle-between'' is enough to
define orthogonality, and a notion of ``projection'' is enough to define
orthonormality. And the projection will work on any orthonormal basis.
The only thing we need for ``projection'' to work is (1) non-zero
vectors have non-zero projection onto themselves, (2) scaling a vector
scales the projection, (3) rotation invariance.

To tie things up: consider an orthonormal basis $\mathcal{B}$. Then:

\begin{nedqn}
  \vv
\eqcol
  \sum_{i=1}^n \innerprod{\vv}{\vb_i} \vb_i
\end{nedqn}

