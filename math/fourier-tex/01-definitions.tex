\section{Definitions}

A sinusoidal function has the form:

\begin{nedqn}
  \sinf{\omega t + \phi}
\end{nedqn}

We call $\omega$ the \define{angular frequency}. This is how many
radians we rotate through for each second. We call $\phi$ the
\define{phase shift}. It is how many radians we have already rotated at
time $t = 0$.

We can define the frequency in terms of \define{rotations} (also called
\define{cycles}) rather than radians. For instance, the
\define{(ordinary) frequency} $f$ is the number of rotations per second
(cycles per second, hertz). We of course have:

\begin{nedqn}
  f
\eqcol
  \frac{\omega}{2\pi}
\end{nedqn}

\noindent
We may substitute this into our prior formula for a sinusoidal:

\begin{nedqn}
  \sinf{2\pi ft + \phi}
\end{nedqn}

\noindent
This is a pleasing form if we consider that $ft$ is the (fractional)
number of cycles we have rotated through.

Instead of frequency, we might prefer to talk about \define{period}. The
\define{angular period} is the time needed to rotate through one radian.
Much more commonly, we care about the \define{(ordinary) period}, which
is the time needed to rotate through one cycle. In either case:

\begin{nedqn}
  T_\text{angular}
\eqcol
  \frac{1}{\omega}
\\
  T_\text{ordinary}
\eqcol
  \frac{1}{f}
\end{nedqn}

\noindent
Because ordinary period is what we almost always want, we just denote
this $T$. And note that $\omega = \frac{2\pi}{T}$.

Note that we can write our sinusoidal in terms of period:

\begin{nedqn}
  \sinf{2\pi \frac{t}{T} + \phi}
\end{nedqn}

\noindent
This is pleasing because $\frac{t}{T}$ is again the number of cycles
that have been rotated through at time $t$.

There is a corresponding concept of \define{wavelength} denoted
$\lambda$. This means roughly the same thing as period: the period is a
measurement relative to time, while the wavelength is a measurement
relative to distance. I think wavelength is relevant for waves that are
\emph{traveling} through space. If the wave velocity is $v$, I think
that $\lambda = vT$.

We won't be talking about propagation of waves through space, so we will
only speak in terms of periods here.
