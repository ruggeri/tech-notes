\section{Definitions}

First, let's make some definitions. Consider the sinusoidal function
$\sin t$. The \define{period} $T$ of this function is equal to $2\pi$.
The period is the time required for the sinusoidal function to go
through exactly one cycle.

There is a corresponding concept of \define{wavelength} denoted
$\lambda$. This means roughly the same thing as period: the period is a
measurement relative to time, while the wavelength is a measurement
relative to distance. I think wavelength is relevant for waves that are
\emph{travelling} through space. We won't be talking about propagation
of waves through space, so we will only speak in terms of periods here.

The \define{frequency} $f$ is how many cycles occur in one unit of time.
Since $f$ and $T$ are in the same units, we have $f = \frac{1}{T}$.

We may note that:

\begin{nedqn}
  \sin\parens{\frac{2\pi}{T} t}
\eqcol
  \sin\parens{2\pi f t}
\end{nedqn}

For convenience, we also talk in terms of \define{angular frequency}
$\omega = 2\pi f = \frac{2\pi}{T}$. The angular frequency is the number
of cycles that pass in $2\pi$ seconds of time. Thus the function $\sin
t$ has angular frequency $\omega = 1$.
