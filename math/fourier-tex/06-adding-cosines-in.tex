\section{Adding Cosines In}

Let's start considering

\begin{nedqn}
  f(t)
\eqcol
  \sum_{k = 1}^{\infty}
  a_k \sin kt + b_k \cos kt
\end{nedqn}

\noindent
That is: we now want to extend our basis from just $\sin k t$ to also
include $\cos k t$. This is fine, so long as our inner product still
ensures that any two different basis vectors are orthogonal. By
analogous arguments to what has preceded, we know:

\begin{nedqn}
  \innerprod{\cos k t}{\cos k t}
\eqcol
  1
\\
  \innerprod{\cos k t}{\cos k' t}
\eqcol
  0
\end{nedqn}

This is simple because $\cos kt$ is merely a phase shift of $\sin kt$.
But we must consider one new possibility:

\begin{nedqn}
  \innerprod{\cos kt}{\sin k't}
&&
  \nedcomment{$k$ may equal $k'$}
\end{nedqn}

Note that $\cos kt$ is an ``odd'' function where $\cos -kt = -\cos kt$.
Also note that $\sin k't$ is an ``even'' function where $\sin -k't =
\sin k't$. That means that, overall, $\cos kt \sin k't$ is an odd
function:

\begin{nedqn}
  \cosf{kt} \sinf{k't}
\eqcol
  -\cosf{-kt} \sinf{-k't}
\end{nedqn}

Next, note that the integral over $\parens{0, 2\pi}$ is the same as
$\parens{-\pi, \pi}$. This is true for any function $f$ with a period
dividing $2\pi$. But integrating any odd function over $-T, +T$ should
always give zero. In the specific case of $\cos kt \sin k't$:

\begin{nedqn}
  \int_0^{\pi}
  \cos kt \sin k' t \dt
\eqcol
  -\int_0^{\pi}
  \cos -kt \sin -k' t \dt
  \nedcomment{because odd}
\\
\eqcol
  -\int_{-\pi}^{0}
    \cos kt \sin k' t \dt
\\
&&
  \nedcomment{substitution of $k$ for $-k$}
\end{nedqn}

\noindent
We therefore see that there is no harm in adding the $\cos kt$ in. We
always have

\begin{nedqn}
  \innerprod{\cos kt}{\cos kt} \eqcol 1
\\
\innerprod{\cos kt}{\cos k't} \eqcol 0
  \nedcomment{whenever $k \ne k'$}
\\
  \innerprod{\cos kt}{\sin k't} \eqcol 0
  \nedcomment{regardless whether $k = k'$ or not}
\end{nedqn}

Thus we our inner-product will continue to work as a good decomposer
function even over a space containing sinusoidals offset by
$\frac{\pi}{2}$ radians (that is, $\cosf{kt} = \sinf{kt +
\frac{pi}{2}}$).

\subsection{Rotations between $\sin t$ and $\cos t$ give phase-shifts}

Consider $f = \cos, g = \sin$. Consider the linear combination
$h = a \cos + b \sin$. Then:

\begin{nedqn}
  h(t)
\eqcol
  a \cos t + b \sin t
\end{nedqn}

\noindent
Note that we can always write $(a, b)$ in polar coordinates. That is: as
$r(\cos \varphi, \sin \varphi)$. Let's simplify and assume that $r = 1$
for the moment. Thus, we are working with:

\begin{nedqn}
  h(t)
\eqcol
  \cos\varphi \cos t + \sin\varphi \sin t
\end{nedqn}

\noindent
We can see that this is the projection of a vector rotated by $t$
radians onto a vector rotated by $\varphi$ radians (or vice-versa).
That is the same as:

\begin{nedqn}
  h(t)
\eqcol
  \cos\parens{t - \varphi}
=
  \cos\parens{\varphi - t}
\end{nedqn}

\noindent
That is: a linear combination of $\sin t$ and $\cos t$ is a
\define{phase shift} of $\cos t$ (plus maybe some scaling). Equivalently
we could see this as a phase shift of $\sin t$ (by $\varphi -
\frac{\pi}{2}$ or whatever), since $\cos$ and $\sin$ are phase shifts of
each other.

I say this makes sense. The linear combination of two orthogonal vectors
is always (1) some homogenous scaling (via $r$), and (2) some mixing
(via $\cos\varphi, \sin\varphi$). The mixing can be seen as a \emph{pure
rotation} of one vector toward (or away) from the other.

Here we started with the vector $f = \cos$. We rotated it ``toward'' $g
= \sin$ by $\varphi$ radians. It turns out that the result of a rotation
by $\varphi$ radians is $\cosf{t - \varphi}$, which is exactly equal to
$\cos$ when $\varphi = 0$ and exactly equal to $\cosf{t - \frac{\pi}{2}}
= \sin t$ when $\varphi = \frac{\pi}{2}$.

This is convenient. How did it happen? Certainly it has something to do
with the fact that the very vectors we are rotating between themselves
represent the result of a rotation at different times $t$.

Maybe I wish that the result of the rotation were $\cosf{t + \varphi}$
rather than $\cosf{t - \varphi}$. But if that were the case, a rotation
by $\frac{\pi}{2}$ rad would not give us $\sin t$. Maybe things will
come out cleaner when we work with complex sinusoidal functions.

Anyway, if we include both $\sin kt$ and $\cos kt$ functions in our
basis, the space spanned will include all sinusoidals with (1) period
dividing $2\pi$ and, (2) any phase-shift. Thus our space is the
(algebraic) closure of \emph{all} sinusoidal functions.
