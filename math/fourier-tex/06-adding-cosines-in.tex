\section{Adding Cosines In}

Let's start considering

\begin{nedqn}
  f(t)
\eqcol
  \sum_{k = 1}^{\infty}
  a_k \sin kt + b_k \cos kt
\end{nedqn}

That is: we now want to extend our basis from just $\frac{1}{\sqrt{\pi}}
\sin k t$ to also include $\frac{1}{\sqrt{\pi}} \cos k t$.

This is fine, so long as our inner product still ensures that any two
different basis vectors are orthogonal. By analagous arguments to what
has preceeded, we know:

\begin{nedqn}
  \innerprod{\cos k t}{\cos k t}
\eqcol
  \pi
\\
  \innerprod{\cos k t}{\cos k' t}
\eqcol
  0
\end{nedqn}

This is simple because $\cos kt$ is merely a phase shift of $\sin kt$.
But we must consider one new possibility:

\begin{nedqn}
  \innerprod{\cos kt}{\sin k't}
&&
  \nedcomment{$k$ may equal $k'$}
\end{nedqn}

Note that $\cos kt$ is an ``odd'' function where $\cos -kt = -\cos kt$.
Also note that $\sin k't$ is an ``even'' function where $\sin -k't =
\sin k't$. That means that, overall, $\cos kt \sin k't$ is an odd
function.

Note that the integral over $\parens{0, 2\pi}$ is the same as
$\parens{-\pi, \pi}$. This is true for any function $f$ with a period
dividing $2\pi$. In the specific case of $\cos kt \sin k't$, we are
integrating an odd function. This integrates to zero, since:

\begin{nedqn}
  \int_0^{\pi}
  \cos kt \sin k' t \dt
\eqcol
  -\int_0^{\pi}
  \cos -kt \sin -k' t \dt
\\\eqcol
  -\int_{-\pi}^{0}
    \cos kt \sin k' t \dt
\end{nedqn}

We therefore see that there is no harm in adding the $\cos t$ in.

\subsection{How $\sin t$ and $\cos t$ Interact}

Consider a weighted combination: $b \cos t + a \sin t$. Note that we can
always write $(b, a)$ in polar coordinates. That is: as $r(\cos \phi,
\sin \phi)$. Let's simplify and assume that $r = 1$ for the moment.
Thus, we are working with:

\begin{nedqn}
  \cos \varphi \cos t + \sin \varphi \sin t
\end{nedqn}

We can see that this is the projection of a vector rotated by $t$
radians onto a vector rotated by $\varphi$ radians. That is the same as:

\begin{nedqn}
  \cos\parens{t - \varphi}
\end{nedqn}

That is: a linear combination of $\sin t$ and $\cos t$ is a
\define{phase shift} of $\cos t$. (Equivalently we could see this as a
phase shift of $\sin t$, since $\cos$ and $\sin$ are phase shifts of
each other.)

I say this makes sense. The linear combination of two orthogonal vectors
is always (1) some homogenous scaling (via $r$), and (2) some mixing
(via $\cos\varphi, \sin\varphi$). The mixing can be seen as a \emph{pure
rotation} of one vector toward (or away) from the other.

Here there is a happy match-up because the original basis vectors are
themselves sinusoids. At every time $t$, you are rotating an $x$ length
$\cos t$ by $\varphi$ radians toward the $y$ length $\sin t$. This
corresponds exactly to the $x$ length of the vector $\cos t - \varphi$.

Thus by adding in just $\cos t$, our space of $f$ functions are the sum
of sinusoidals with (1) period dividing $2\pi$ and, (2) any phase-shift.
Thus our space is the (algebraic) closure of \emph{all} sinusoidal
functions.
