\documentclass[11pt, oneside]{amsart}

\newcommand{\calC}{\mathcal{C}}

\usepackage{geometry}
\geometry{letterpaper}

\usepackage{ned-common}
\usepackage{ned-calculus}
\usepackage{ned-linear-algebra}
\usepackage{ned-stats}

\begin{document}

\title{Div, Grad, Curl}
\maketitle

\begin{enumerate}

\item With a function of a single variable, we have already defined:

\begin{nedqn}
  \fderiv{x} f(x_0)
\eqcol
  \lim_{x \to x_0}
  \frac{
    f(x) - f(x_0)
  }{
    x - x_0
  }
\end{nedqn}

\item We can generalize in two directions. First, we can consider a
function of many variables, but mapping to a single real variable. Then
we define:

\begin{nedqn}
  \fpartial{x_i} f(\vx)
\eqcol
  \lim_{t \to \vx_i}
  \frac{
    \ff{\tilde\vx} - \ff{\vx}
  }{
    t - \vx_i
  }
\intertext{where}
  \tilde{\vx}
\eqcol
  \parens{\vx_0, \ldots, t, \ldots, \vx_n}
\end{nedqn}

\noindent
Simply, we're restricting $f$ to a function of just $x_i$, where all
other parameter values $x_j$ are held constant as $\vx_j$. Then, we
calculate the typical univariate derivative at $x_i$. Put another way,
this partial derivative is the slope of the function in the direction of
$x_i$.

\item We can further define the \define{gradient}. This is:

\begin{nedqn}
  \gradient f(\vx)
\eqcol
  \begin{pmatrix}
    \fpartial{x_0} \ff{\vx}
    \\
    \vdots
    \\
    \fpartial{x_n} \ff{\vx}
  \end{pmatrix}
\end{nedqn}

\noindent
What does this mean? Well, $\gradient f(\vx) \cdot \vu$ is the
\define{directional derivative} of the function at $\vx$ along the
direction defined by $\vu$. If $\vu$ is a vector along one of the
coordinate axes, then this is of course just a regular partial
derivative.

\item However, we should be careful to note that this only true for
somewhat well-behaved functions! It's not always true that you can
calculate any directional derivatives simply from knowing the gradient
along some axes. Take a silly example:

\begin{nedqn}
  f(x, y)
\eqcol
  \begin{cases}
    x \text{ if } y \ne x \\
    0 \text{ if } y = x
  \end{cases}
\end{nedqn}

\noindent
Here, $\fpartial{x} f(0, 0) = 1$, $\fpartial{y} f(0, 0) = 0$. But the
derivative along the direction defined $y = x$ is 0, and not
$\frac{\sqrt{2}}{2}$. You can even create smooth examples!

\item It might seem nicer if $\grad \ff{\vx}$ were a \emph{row} vector
rather than a \emph{column} vector. It sure seems that way to me!
However, it appears that it is typical to define the gradient as a
column vector for reasons that are somewhat above my pay-grade at this
time.

\item The other direction of generalization is to consider
\define{vector-valued functions}. Our prior definition of a partial
derivative (and of a directional derivative) will still apply. We just
need to note that the numerator in the definition is a vector.

\item We should be a little careful when generalizing the gradient. This
was a vector, but the analogue should be a \emph{matrix}. We will define
the \define{Jacobian} to be the analogue of the gradient:

\begin{nedqn}
  \mJ(\vx)
\eqcol
  \begin{bmatrix}
    \grad\tran f(\vx)_0 \\
    \grad\tran f(\vx)_1 \\
    \vdots \\
    \grad\tran f(\vx)_n \\
  \end{bmatrix}
\end{nedqn}

\noindent
It's a little unsatisfying that the \emph{rows} are the (transposed)
gradients, rather than the \emph{columns}. Again, it seems like it would
have been more natural if the gradient had always been a row
vector...

On the plus side, we have that $\mJ \vu$ is the directional derivative
along $\vu$, which is what we always wanted. (Again, this depends
on the function being ``differentiable'').

\item Take $f$ is a function from $\R^m \to \R^n$. We say that a
function $df_a$ is the \define{(total) derivative} or \define{(total)
differential} of $f$ at $a$ if:

\begin{nedqn}
  \lim_{\vx \to \va}
  \frac{
    \norm{
      \ff{\vx}
      -
      \parens{
        \ff{\va}
        +
        df_a\parens{\vx - \va}
      }
    }
  }{
    \norm{
      \vx - \va
    }
  }
\eqcol
  0
\end{nedqn}

\noindent
If $m=1$, then of course $df_a$ is just the ordinary derivative
$\fderiv{x} \ff{a}$. If $m > 1$, the word ``total'' emphasizes that we
are not talking about a partial derivative, but something that describes
rate of change in an arbitrary direction.

\item In order to be considered a total derivative, mathematicians want
$df_a$ to be a \emph{linear transformation}. Note that this implies that
the differential is

\begin{nedqn}
  df_a
\eqcol
  \mJ\parens{a}
\end{nedqn}

\noindent
except when the differential doesn't exist, because the function is not
differentiable. It feels weird that we have different names for Jacobian
and differential, but I suppose the point is that the differential does
not necessarily exist everywhere the Jacobian does.

\item A vector-valued function $f: \R^n \to \R^n$ is also sometimes
called a \define{vector field}. This use of \emph{field} is at odds with
the algebra definition. This use is typical of physics.

\item We can next define a \define{line integral} (also called a path
integral). First we have to define a \define{curve}. A curve
$\calC$ is a subset of $\R^m$ with the following requirement:
there is a continuous map $r: [a, b] \to \R^m$, and the image of $r$ is
$\calC$ ($[a, b]$ is a closed interval in $\R$, of course). In
this case, we'll want $r$ to be piecewise differentiable.

\item Now we can define the line integral of $f$ along a curve
$\calC$ defined by $r$. It is defined to be:

\begin{nedqn}
  \int_\calC
  f \diff{s}
\eqcol
  \int_a^b
  \ff{\vr(t)}
  \norm{\vr'(t)}
  \dt
\end{nedqn}

\noindent
Here the notation $\vr(t)$ emphasizes that this is a vector-valued
function of $t$. Note that the ``rate'' at which we move through the
curve (which is arbitrary) is corrected by $\norm{\vr'(t)}$. This makes
the line integral independent on the parameterization of $\calC$.

\item For now, I will consider $\int_\calC f \diff{s}$ as pure
notation. It feels meaningless and arbitrary at present.

\item Note that the path-integral concept trivially generalizes to a
vector-valued $f$.

\item It feels a little unfortunate that $\vr$ is required to be
piecewise differentiable. It would be nicer if the only requirement were
that $r$ be continuous.

\item Maybe we can consider this a moment. The \define{length} of a path
can be defined as $\int_\calC 1 \diff{s}$. Is there any other way
to define a length? Could we not talk about line-segments in $\R^n$: the
interpolation between two points? A line-segment has a natural
definition of length; it's the norm of the difference between the two
points.

Can we then break up $\calC$ into a sequence of line-segments? Of course
not if $\calC$ is curved. Could we have a sequence of progressively
better approximations for $\calC$? It \emph{seems} intuitive, but it's
not exactly clear how we could do this. We can't appeal to dividing up
the length of $\calC$ into $k$-evenly spaced points on $\calC$ - that
would put the cart before the horse!

\item If it makes one feel happier, there is only one $\vr$ that traces
$\mathcal{C}$ at a constant velocity $\vr' = 1$. So the appearance of
$\norm{\vr'(t)}$ in the integral is just to correct for an arbitrarily
stretched choice of $\vr$. Also, the requirement that $\vr'$ exists
(almost everywhere) simply demands that there be a well-defined notion
of tangency everywhere on the path.

\item Once we can break up the path into equal-length segments, I
believe we can do the typical thing of using the convergence of upper
and lower sums to calculate the path interval of a function $f$ along
$\calC$.

\item We can define a corresponding \define{surface integral}. We must
again first define what we mean by a \define{surface}. A surface $S$ is
a subset of $R^n$ with the following property: there exists a continuous
mapping $r: U \to \R^n$ ($U$ should be an open set in $\R^n$) where
$S$ is the image of $r$.

\item We should next define the notion of \define{surface area} (just as
we defined curve length). We could try to define this as:

\begin{nedqn}
  \iint_S
  1 \diff{s}
& \stackrel{?}{=} &
  \iint_U
  \det \mtxJ_{\vr}(\vu)
  \diff{\vu}
\end{nedqn}

\noindent
Here we use that $\det \mtxJ_{\vr}(\vu)$ is the amount by which area is
stretched by $\vr$ in the vicinity of $\vu$. However, since $\det$ is
only defined for \emph{square} matrices, this won't work when we are
mapping from two-dimensions to three!

\item Of course, the coordinate axes in $\R^n$ are arbitrary, so we
could rotate them so that $\mtxJ$ had all zeros outside the first two
rows. Then, to find the stretching of an input area, we would just need
to calculate the determinate of a (square) 2x2 matrix.

\item Let's consider the specific case of two dimensions. I will note
first that Wikipedia defines surface area as:

\begin{nedqn}
  \iint_S
  1 \diff{s}
\eqcol
  \iint_T
  \norm{
    \frac{\partial \vr}{\partial x}(\vu)
    \cross
    \frac{\partial \vr}{\partial y}(\vu)
  }
  \diff{x} \diff{y}
\end{nedqn}

\noindent
Already, the norm of this cross-product \emph{looks} like it should be
equal to $\det \mtxJ_{\vr}(\vu)$. We can verify this quickly. First,
note:

\begin{nedqn}
  \det A
\eqcol
  ad - bc
\intertext{and}
  \colvec{a \\ c \\ 0} \cross \colvec{b \\ d \\ 0}
\eqcol
  \parens{
    ad - bc
  } \vk
\end{nedqn}

\item Anyway, we can see that my $\det \mtxJ$ idea is not totally silly.
However, it isn't quite right, and the cross-product way expresses what
we want to do simpler, and accurately.

I would note that area is defined by height times width, where those are
measured orthogonally. We're assuming that $x$ and $y$ are orthogonal
directions in the space. So the local stretching of a space depends on
(1) how much differences in $x$ and $y$ get stretched, and (2) how much
they remain orthogonal in the target space.

\item We can finally define the \define{divergence} of a vector field
$\mtxF$ at a point $\vu$:

\begin{nedqn}
  \operatorname{div} \mtxF(\vu)
\defeqcol
  \lim_{V \to 0}
  \frac{1}{\abs{V}}
  \iint_{S(V)}
  \mtxF(\vx) \cdot \vn(\vx)
  \diff{\vx}
\end{nedqn}

\noindent
What the hell does this mean? First, we will $V$ is a volume which
contains $\vu$. The limit is expressing that we take a series of volumes
where the volume of $V$ (denoted $|V|$) is approaching zero. For
instance, $V$ could be a series of $\epsilon$-balls, where $\epsilon \to
0$.

Next, we are going to do a surface integral. The surface we integrate
over will be the surface of the volume. This could, for instance, be the
$\epsilon$-shell.

What will we integrate? $\mtxF(\vx) \cdot \vn(\vx)$. What is $\vn(\vx)$?
It will be the unit-vector that is \define{normal} to the surface $S(V)$
at position $\vx$.

\item For a field $\mtxF$, we define the \define{flux} $\Phi_{\mtxF}(S)$
to because

\begin{nedqn}
  \Phi_{\mtxF}(S)
\defeqcol
  \iint_S
  \mtxF(\vu) \cdot \vn(\vu)
  \diff{\vu}
\end{nedqn}

\noindent
Thus, with this definition, we can equivalently define:

\begin{nedqn}
  \operatorname{div} \mtxF(\vu)
\defeqcol
  \lim_{\abs{V_i} \to 0}
  \frac{
    \Phi_{\mtxF}(S_i)
  }{
    \abs{V_i}
  }
\end{nedqn}

\item Thus, for a small volume about $\vu$, multiplying the volume by
the the divergence of $\mtxF$ will give you the flux out of the volume.

\item \TODO{Define nabla differential operator}.
\item \TODO{Define curl}.

\end{enumerate}

\end{document}
