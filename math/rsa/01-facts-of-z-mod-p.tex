\section{Facts of $\Zmodp$}

Here we will prove some basic facts of the finite field $\Zmodp$.

\begin{lemma}
  No two non-zero elements of $\Zmodp$ multiply to zero.
\end{lemma}

\begin{proof}
  Take any $a, b \ne 0$. Consider their product $ab \bmod p$.

  It is not possible that $ab$ could be a product of $p$, because that
  would mean $p$ is a divisor of $ab$. That cannot be the case because
  $a, b$ are both chosen less than $p$.
\end{proof}

\begin{lemma}
  You may ``cancel'' non-zero elements. That is, for $a \ne 0$, $ab =
  ac$ implies $b = c$.
\end{lemma}

\begin{proof}
  Say that $ab = ac$. Then $a (b - c) = 0$.

  From before, we know that at least one of $a, b$ is equal to zero.
  Since we've assumed $a \ne 0$, that means $b - c = 0$, which implies
  $b = c$.

  Note that both left and right cancelation are allowed.
\end{proof}

\begin{lemma}
  Every non-zero element is invertible.
\end{lemma}

\begin{proof}
  Consider the series $1a, 2a, 3a, \ldots, (p - 1)a$. I claim that this
  is a permutation of $1, 2, 3, \ldots, p - 1$.

  Why? Because no two $b_0 a = b_1 a$ for distinct $b_0, b_1$. This
  follows from being able to cancel $a$.

  But then, by pigeonhole principle, there is some $b$ such that $ba
  \equiv 1 \pmod{p}$. In which case $b$ is an inverse for $a$.
\end{proof}

\begin{definition}
  Since every non-zero element is invertible, then the set of all
  non-zero elements forms a subgroup of the finite field $\Zmodp$. This
  is called the \define{multiplicative subgroup}. We denote this group
  $\Zmodpx$. $\Zmodpx$ contains all other subgroups of $\Zmodp$
  (trivially, since it already contains all invertible elements).
  $\Zmodpx$ has exactly $p - 1$ elements: $1, 2, ..., p - 1$.
\end{definition}

\begin{definition}
  The subgroup \define{generated} by $a \ne 0$ consists of powers of
  $a$:

  \begin{nedqn}
    1, a^1, a^2, a^3, \ldots
  \end{nedqn}

  How do we know this is a subgroup? We will need to prove that! To
  start, it is clearly closed under multiplication (by its very
  definition).

  We will next work toward showing that if $x$ is in the series, then
  its inverse $\inv{x}$ is also in the series.
\end{definition}

\begin{lemma}
  The series $1, a, a^2, \ldots$ eventually loops back to unity.
\end{lemma}

\begin{proof}
  Because $\Zmodp$ is finite, it must eventually be the case that:

  \begin{nedqn}
    a^{k_0} \equiv a^{k_1} \pmod{p}
  \end{nedqn}

  By cancelation (proven above), we have:

  \begin{nedqn}
    a^{k_0 - k_1} \equiv 1 \pmod{p}
  \end{nedqn}

  Since this is true for some $k_0 - k_1$, we could choose a smallest
  $|a|$ such that: $a^{|a|} \equiv 1 \mod{p}$. We call $|a|$ the
  \define{order} of $a$.
\end{proof}

\begin{lemma}
  If the series $1, a, a^2$ contains $x$, it also contains $\inv{x}$.
\end{lemma}

\begin{proof}
  Consider any element $a^i$ (for $0 \leq i < |a|$). Then its inverse is
  simply:

  \begin{nedqn}
    a^{|a| - i} = a^{-i}
  \end{nedqn}
\end{proof}

We've now proven that $1, a, a^2, \ldots$ forms a subgroup: it is closed
under multiplication and multiplicative inverse.

\begin{lemma}
  The order of the subgroup generated by any $a \ne 0$ divides $p - 1$.
  That is, $\order{a}$ divides $p - 1$.
\end{lemma}

\begin{proof}
  We already said that $\Zmodpx$ contains every other subgroup of the
  finite field $\Zmodp$. This is trivial because $\Zmodpx$ contains
  every element with an inverse in $\Zmodp$.

  By Lagrange's theorem, the order of any subgroup generated by $a$ must
  therefore divide $\order{\Zmodpx} = p - 1$.

  Very important note, the subgroup generated by $a$ may be
  \emph{strictly contained} contained within the multiplicative
  subgroup. That is, we may have $\order{a} \ne p - 1$.
\end{proof}

\begin{example}
  For example, consider $\Zmod{5}$. Consider the subgroup generated by
  $a = 4$. Note that $4$ is its own inverse, as:

  \begin{nedqn}
    4 \cdot 4
  & \equiv &
    16
  \\
  & \equiv &
    1 \pmod{5}
  \end{nedqn}

  Therefore we have $\order{4} = 2$; the subgroup generated by $4$ has
  exactly two elements. Even still, we have that $\order{4} = 2$ divides
  $p - 1= 4$, as expected from Lagrange's Theorem.
\end{example}
