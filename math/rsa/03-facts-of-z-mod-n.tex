\section{Factors of $\Zmodn$}

Let's now consider the finite \emph{ring} $\Zmodn$, where $n$ need not
be prime. We will contrast with the scenario of $\Zmodp$.

\begin{lemma}
  Some non-zero elements will multiply to zero.
\end{lemma}

\begin{proof}
  Since $n$ is not prime, there exist $a, b$ (possibly equal) such that
  $n = ab$. Then we have that $ab \equiv 0 \pmod{n}$.
\end{proof}

\begin{lemma}
  Some non-zero elements won't cancel.
\end{lemma}

\begin{proof}
  From the above, we have that:

  \begin{nedqn}
    a 1
  \equivcol
    a (b + 1) \pmod{n}
  \end{nedqn}

  This happens even though $1 \ne b + 1$. Therefore it is not safe to
  cancel $a$.
\end{proof}

\begin{lemma}
  Some non-zero elements aren't invertible.
\end{lemma}

\begin{proof}
  Since two products $ak_0 = ak_1$, it is not possible for an inverse of
  $a$ to exist.
\end{proof}

\begin{lemma}
  $a$ is invertible if $\gcd(a, n) = 1$.
\end{lemma}

\begin{proof}
  Consider an element $a$ such that $\gcd(a, n) = 1$. Then for any $b
  \ne 0$ in $\Zmodn$, we know that:

    \begin{nedqn}
      ba
    & \not\equiv &
      0 \pmod{n}
    \end{nedqn}

  The reason is: how could $ba$ possibly be a multiple of $n$? For that
  to be so, $ba$ would have to contain all prime factors of $n$ (with
  the correct multiplicity). If $a$ is relatively prime to $n$, then we
  would need that $b$ contains all the factors (with the correct
  multiplicity) of $n$. But in that case $b$ would be a multiple of $n$,
  in which case $b \equiv 0 \pmod{n}$.

  Therefore we see that $ba \not\equiv 0 \pmod{n}$ whenever $\gcd(a, n)
  = 1$ and $b \ne 0$.

  Note that we have already previously explained (when talking about
  $\Zmodp$) that when $ba$ never is equal to zero, it must be that:

  \begin{nedqn}
    1a, 2a, ... (n - 1)a
  \end{nedqn}

  is a permutation of

  \begin{nedqn}
    1, 2, ..., n - 1
  \end{nedqn}

  And we previously explained that this means that $a$ invertible.
\end{proof}

\begin{lemma}
  $a$ is \emph{not} invertible whenever $\gcd(a, n) > 1$.
\end{lemma}

\begin{proof}
  Consider $b = \frac{n}{\gcd(a, n)}$. Note that $0 < b < n$. We now
  have:

  \begin{nedqn}
    ba
  \equivcol
    n
  \\
  \equivcol
    0 \pmod{n}
  \end{nedqn}

  Of course this immediately implies that $a$ is not invertible.
\end{proof}

\begin{definition}
  The \define{multiplicative subgroup} of $\Zmodn$ consists of all
  invertible elements of the ring. We denote this $\Zmodnx$.

  We've just said that this consists of exactly those elements $a$ where
  $\gcd(a, n) = 1$. That is: exactly those $a$ that are relatively prime
  to $n$.

  This generalizes our previous definition of $\Zmodpx$, since all
  non-zero elements of $\Zmodp$ are relatively prime to $p$.
\end{definition}
