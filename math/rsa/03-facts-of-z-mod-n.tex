\section{Facts of $\Zmodn$}

Let's now consider the finite \emph{ring} $\Zmodn$, where $n$ need not
be prime. We will contrast with the scenario of $\Zmodp$.

\begin{lemma}
  Some non-zero elements will multiply to zero.
\end{lemma}

\begin{proof}
  Since $n$ is not prime, there exist $a, b$ (possibly equal) such that
  $n = ab$. Then we have that $ab \equiv 0 \pmod{n}$.
\end{proof}

\begin{lemma}
  Some non-zero elements won't cancel.
\end{lemma}

\begin{proof}
  From the above, we have that:

  \begin{nedqn}
    a 1
  \equivcol
    a (b + 1) \pmod{n}
  \end{nedqn}

  This happens even though $1 \ne b + 1$. Therefore it is not safe to
  cancel $a$.
\end{proof}

\begin{lemma}
  Some non-zero elements aren't invertible.
\end{lemma}

\begin{proof}
  Since two products $ak_0 = ak_1$, it is not possible for an inverse of
  $a$ to exist.
\end{proof}

\begin{lemma}
  If $\gcd(x, n) = 1$, then $xy \not\equiv 0 \pmod{n}$ for any
  $y\not\equiv 0 \pmod{n}$.
\end{lemma}

\begin{proof}
  How could $xy$ possibly be a multiple of $n$? For that to be so, $xy$
  would have to contain all prime factors of $n$ (with the correct
  multiplicity). If $x$ is relatively prime to $n$, then we would need
  that $y$ contains all the factors (with the correct multiplicity) of
  $n$. But in that case $y$ would be a multiple of $n$, in which case $y
  \equiv 0 \pmod{n}$.
\end{proof}

\begin{theorem}
  $x$ is invertible if $\gcd(x, n) = 1$.
\end{theorem}

\begin{proof}
  Note that we have already previously explained (when talking about
  $\Zmodp$) that when $xy$ never is equal to zero, it must be that:

  \begin{nedqn}
    1x, 2x, ... (n - 1)x
  \end{nedqn}

  \noindent
  is a permutation of

  \begin{nedqn}
    1, 2, ..., n - 1
  \end{nedqn}

  \noindent
  And we previously explained that this means that $x$ invertible.
  Reason: some $yx$ must have multiplied to 1.
\end{proof}

\begin{theorem}
  $x$ is \emph{not} invertible whenever $\gcd(x, n) > 1$.
\end{theorem}

\begin{proof}
  Consider $y = \frac{n}{\gcd(x, n)}$. Note that $0 < x < n$. We now
  have:

  \begin{nedqn}
    yx
  \equivcol
    n
  \\
  \equivcol
    0 \pmod{n}
  \end{nedqn}

  Of course this immediately implies that $x$ is not invertible.
\end{proof}

\begin{definition}
  The \define{multiplicative subgroup} of $\Zmodn$ consists of all
  invertible elements of the ring. We denote this $\Zmodnx$.
\end{definition}

\begin{remark}
  We've proven that $\Zmodnx$ consists of exactly those elements $x$
  where $\gcd(x, n) = 1$. That is: exactly those $x$ that are relatively
  prime to $n$.

  This generalizes our previous definition of $\Zmodpx$, since all
  non-zero elements of $\Zmodp$ are relatively prime to $p$.
\end{remark}
