\section{Characterization of $\Zmodn$}

\subsection{Group Decomposition of $\Zmodn$}

\begin{theorem}
  When $n$ factorizes as $\prod p_i^{k_i}$ (for distinct $p_i$), then

  \begin{nedqn}
    \Zmodn
  \congcol
    \Zmod{p_1^{k_1}}
    \times
    \cdots
    \times
    \Zmod{p_n^{k_n}}
  \end{nedqn}
\end{theorem}

\begin{proof}
  We already know that for any abelian group $G$ with order $n$
  factorizes into subgroups $G_i$ of order $p_i^{k_i}$. Those subgroups
  are not necessarily cyclic though!

  Of course, if any $G_i$ is acyclic, then $G_i$ can be further
  factorized into a product of $\Zmod{p_i^{m_j}}$, with non-distinct
  $p_i$. This is the Fundamental Theorem of Finite Abelian Groups.

  In the case of $\Zmodn$, every subgroup must already be cyclic, since
  $\Zmodn$ itself is. Thus no further factorization of the $G_i$ is
  possible.
\end{proof}

\begin{proposition}
  A cyclic group $G$ of $n$ elements contains \emph{exactly one}
  subgroup of each order $d$ dividing $\order{G}$.
\end{proposition}

\begin{proof}
  In my Abstract Algebra notes, I proved that a cyclic group $\Zmodn$
  contains exactly one subgroup of order $d$ for each divisor of $n$. I
  will re-prove this, though.

  Consider a generator $g$ and a divisor $d$ of $n$. $g^{n/d}$ is a
  generator for a subgroup of order $d$. Now consider any other $k$ such
  that $g^k$ generates a subgroup of order $d$. I claim that $g^{n/d}$
  is generated by $g^k$. Note that $kd \equiv 1 \pmod{n}$; in fact, we
  know that $d = \frac{n}{\gcdf{k, n}}$. But note this means $\gcdf{k,
  n} = \gcdf{n/d, n} = n/d$. We may now apply Bézout's lemma to show
  that there is some $m$ such that $km \equiv n/d \pmod{n}$.
\end{proof}

\begin{proposition}
  If a cyclic group $G$ contains \emph{exactly one} subgroup of each
  order $d$ dividing $\order{G}$, then $G$ is cyclic.
\end{proposition}

\begin{proof}
  We'll prove this via the contrapositive. We know that if $G$ is
  \emph{acyclic}, then it has factors $\Zmod{p^k}$ and $\Zmod{p^m}$ with
  the same base $p$. But we know that both these factors contain
  subgroups of order $p$. But that means there are two distinct
  subgroups of order $p$. Done.
\end{proof}

\begin{theorem}
  $G$ is cyclic iff it contains a cyclic subgroup of order $d$ for each
  $d$ dividing $\order{G}$.
\end{theorem}

\begin{proof}
  This is the synthesis of the prior two proofs.
\end{proof}

\begin{theorem}
  \begin{nedqn}
    \ff{x}
  & \mapsto &
    \parens{x \bmod a} \times \parens{x \bmod b}
  \end{nedqn}

  \noindent
  is a ring isomorphism between $\Zmod{ab}$ and $\Zmoda \times \Zmodb$.
\end{theorem}

\begin{proof}
  First we show this is a homomorphism:

  \begin{nedqn}
    \ff{x + y}
  & \mapsto &
    \parens{x + y \bmod a} \times \parens{x + y \bmod b}
  \\
  \eqcol
    \parens{x \bmod a} \times \parens{x \bmod b}
    +
    \parens{y \bmod a} \times \parens{y \bmod b}
  \\
  \eqcol
    \ff{x} + \ff{y}
  \end{nedqn}

  Next we should show that $f$ is bijective. Note that both $\Zmod{ab}$
  and $\Zmoda \times \Zmodb$ have exactly $ab$ elements. So it would
  suffice to show either that $f$ is injective or surjective.

  I will show that $f$ is injective. Note that, since $f$ is a
  homomorphism, it is injective exactly when $\ker f = \setof{0}$. That
  is: the homomorphism $f$ is injective iff $0$ is the only value such
  that $f(0) = 0 \times 0$.

  But this is clearly the case, as there is no common multiple of $a, b$
  greater than $0$ and less than $ab$.
\end{proof}

\subsection{Ring Decomposition of $\Zmodn$}

\begin{remark}
  We've shown how to factor the \emph{group} $\Zmodn$ into a direct
  product of cyclic group factors $\Zmod{p_i^{k_i}}$. But what about the
  \emph{ring} structure of $\Zmodn$?
\end{remark}

\begin{theorem}
  For any coprime $a, b$, then $\Zmod{ab}$ is \define{ring isomorphic}
  to $\Zmoda \times \Zmodb$. That means there exists an bijective map
  $f$ such that both:

  \begin{nedqn}
    f(x + y)
  \eqcol
    f(x) + f(y)
  \intertext{and}
    f(xy)
  \eqcol
    f(x) f(y)
  \end{nedqn}
\end{theorem}

\begin{proof}
  We previously gave a group isomorphism:

  \begin{nedqn}
    f(x)
  & \mapsto &
    \parens{x \bmod a} \times \parens{y \bmod b}
  \end{nedqn}

  \noindent
  We will prove that $f$ is a ring isomorphism. Consider $f(xy) =
  \parens{xy \bmod a} \times \parens{xy \bmod b}$. To prove $f$
  preserves multiplication, we will show that for any $k$:

  \begin{nedqn}
    xy \bmod k
  & \equiv &
    \parens{x \bmod k}\parens{y \bmod k}
    \pmod{k}
  \end{nedqn}

  \noindent
  Perhaps we ought to have already established this basic fact of
  $\Zmod{k}$. We do so now. Denote $x \bmod k, y \bmod k$ as $b_x, b_y$.
  Then there exist $a_x, b_y$ such that:

  \begin{nedqn}
    x
  \eqcol
    a_xk + b_x
  \\
    y
  \eqcol
    a_yk + b_y
  \end{nedqn}

  Thus the product $xy$ is:

  \begin{nedqn}
    xy
  \eqcol
    \parens{a_xk + b_x} \parens{a_yk + b_y}
  \\
  \eqcol
    a_x a_y k^2
    + a_x b_y k
    + b_x a_y k
    + b_x b_y
  \\
  \equivcol
    b_x b_y
    \pmod{k}
  \\
  \equivcol
    \parens{x \bmod k}\parens{y \bmod k}
    \pmod{k}
  \end{nedqn}
\end{proof}

\begin{remark}
  The $f$ provided is the \emph{only} ring isomorphism between $\Zmodn$
  and the direct product of its factors. Why? Because it must fix

  \begin{nedqn}
    f(1)
  \mapstocol
    1 \times \cdots \times 1
  \end{nedqn}

  But the multiplicative identity 1 is a generator of the ring $\Zmodn$.
  Thus every

  \begin{nedqn}
    f(k) = f(1 + \cdots + 1) = f(1) + \cdots + f(1)
  \end{nedqn}

  \noindent
  is determined. $f$ is not merely the \emph{natural} ring isomorphism:
  it is the only one!
\end{remark}
