\section{Carmichael's Totient Function}

\subsection{Exploring $\lambdaf{n}$ for factorable $n$}

\begin{definition}
  The \define{exponent} of a group $G$ is the smallest $n$ such that
  $g^n = 1$ for all $g \in G$.
\end{definition}

\begin{definition}
  \define{Carmichael's totient function} $\lambda(n)$ maps $n$ to the
  exponent of $\Zmodnx$. That is: the smallest number such that
  $x^{\lambda(n)} \equiv 1 \pmod{n}$ for all units $x \in \Zmodnx$.
\end{definition}

\begin{proposition}
  $\lambdaf{n} = \vphif{n}$ iff $\Zmodnx$ is cyclic. Else $\lambdaf{n}$
  must divide $\vphif{n}$.
\end{proposition}

\begin{theorem}
  \begin{nedqn}
    \lambdaf{\prod p_i^{k_i}}
  \eqcol
    \lcm\parens{\lambdaf{p_1^{k_1}}, \ldots, \lambdaf{p_n^{k_n}}}
  \end{nedqn}
\end{theorem}

\begin{proof}
  We've previously shown that $\Zmod{ab}$ is ring isomorphic to
  $\Zmod{a} \times \Zmod{b}$. We've shown this implies an isomorphism
  between the multiplicative group $\Zmodx{ab}$ and $\Zmodax \times
  \Zmodbx$.

  So take any $u_a, u_b$ in $\Zmodax, \Zmodbx$ such that $\order{u_a},
  \order{u_b}$ are maximum. That is: $\order{u_a} = \lambdaf{a},
  \order{u_b} = \lambdaf{b}$.

  Now consider powers $\parens{u_a \times u_b}^i$. These equal $u_a^i
  \times u_b^i$. We know this equals $1 \times 1$ iff $i$ is a common
  multiple of $\lambdaf{a}, \lambdaf{b}$. That implies:

  \begin{nedqn}
    \order{u_a \times u_b}
  \eqcol
    \lcmf{\lambdaf{a}, \lambdaf{b}}
  \end{nedqn}

  Also: since every other $u_a', u_b'$ have orders dividing
  $\lambdaf{a}, \lambdaf{b}$, $u_a \times u_b$ has maximum order in
  $\Zmodax \times \Zmodbx$.

  By isomorphism we know that $\order{u} = \lcmf{\lambdaf{a},
  \lambdaf{b}}$. Thus we conclude:

  \begin{nedqn}
    \lambdaf{ab}
  \eqcol
    \lcmf{\lambdaf{a}, \lambdaf{b}}
  \end{nedqn}

  It's easy to extend our proof to more prime power factors.
\end{proof}

\begin{remark}
  We've given a recursive definition of $\lambda$. We must now fill in
  the base cases. When we have characterized $\Zmod{2^k}$ and
  $\Zmod{p^k}$ as direct products of cyclic groups, we will immediately
  know $\lambdaf{2^k}$ and $\lambdaf{p^k}$. The recursive definition
  above will apply to any factorization of $\Zmodx{2^k}, \Zmodx{p^k}$.
\end{remark}

\begin{proposition}
  Note that $\lambdaf{2} = 1$ and $\lambdaf{4} = 2$. $\lambdaf{2^k} <
  \vphif{2^k}$ for $k > 2$.
\end{proposition}

\begin{proof}
  We've previously shown that $\Zmodx{2}, \Zmodx{4}$ are cyclic groups.
  So $\lambdaf{2} = \vphif{2} = 1$, $\lambdaf{4} = \vphif{4} = 2$.

  We've shown that $\Zmodx{2^k}$ is \emph{acyclic} for $k > 2$. Thus we
  know that $\lambdaf{2^k} < \vphif{2^k}$.
\end{proof}

\subsection{Proving $\Zmodpx$ is Cyclic}

\begin{lemma}
  We've shown in our Abstract Algebra notes that the subgroups of a
  cyclic group on $n$ elements are in bijection with divisors of $n$.

  The converse is true: if all cyclic subgroups of an abelian group $G$
  have distinct order, then $G$ is cyclic.
\end{lemma}

\begin{proof}
  We will prove the contrapositive: that if $G$ is acyclic, there must
  be more than one subgroup with the same order.

  We can factor $G$ into cyclic groups $G_i$ by the Fundamental Theorem
  of Finite Abelian Groups. $G$ is acyclic precisely if two factors
  $G_i, G_j$ have cocomposite order.

  Let $k = \gcdf{\order{G_i}, \order{G_j}}$. We've said that since $G_i,
  G_j$ are themselves cyclic, they contain distinct subgroups of order
  $k$. Done.
\end{proof}

\begin{lemma}
  $\Zmodpx$ never has two distinct cyclic subgroups with the same order.
\end{lemma}

\begin{proof}
  Consider a cyclic subgroup of order $k$. It is generated by $a, a^2,
  a^3, \ldots, a^k \equiv 1$. Thus these are $k$ distinct roots to the
  equation:

  \begin{nedqn}
    a^k - 1
  \equivcol
    0
    \pmod{p}
  \end{nedqn}

  I argue that there can be no more than $k$ distinct roots to this
  polynomial. This is a weak version of the Fundamental Theorem of
  Algebra applied to general finite fields. Thus there can be no second
  subgroup of order $k$.

  Note that when we are working in $\Zmodnx$, then this is not a finite
  field, and thus we cannot apply (the weak version of) the Fundamental
  Theorem of Algebra.
\end{proof}

\begin{lemma}
  For any field $\mathbb{F}$, a polynomial of degree $n$ has \emph{at
  most} $n$ roots.
\end{lemma}

\begin{proof}
  First, note that 0 is a root of a polynomial $p$ if and only if the
  constant term of $p$ is itself zero. In that case, we can factor out a
  factor of $x$ from $p$ to reduce its degree.

  Say that $p$ does have a root $r$. Then consider the polynomial $p'(x)
  = p(x + r)$. Then $p'$ has a root at 0, and so it can be factored into
  the product of $x$ and a degree $n-1$ polynomial $q'(x)$.

  But then we can recall that $p(x) = p'(x - r)$ to substitute in $x -
  r$. This shows that $p$ is a product of $(x - r)$ and a degree $n-1$
  polynomial $q(x) = q'(x - r)$.

  This process of extracting a factor of $x - r_i$ can only happen at
  most $n$ times until we are left with a degree zero polynomial.

  Now that we have factored $p(x) = \prod (x - r_i)$, we may ask: can
  $p$ have any other roots? The answer is no, because for any other $r'$
  we know that $r' - r_i$ is not zero. And because we are working over a
  field, no product of non-zero elements can itself be zero.

  Note: we've considered \emph{distinct} roots $r_i$. It's possible that
  $p$ lacks $n$ distinct roots because some have \define{multiplicity}
  greater than zero. It's also possible that, even accounting for
  multiplicity, there are less than $n$ roots. This happens when the
  field $\mathbb{F}$ is not algebraically closed (see below).
\end{proof}

\begin{remark}
  We say the field $\mathbb{F}$ is \define{algebraically closed} if
  every polynomial of degree $n$ has \emph{exactly} $n$ roots
  (accounting for multiplicity). The (strong) Fundamental Theorem says
  that $\mathbb{C}$ is algebraically closed.

  An equivalent definition of algebraically closed: every polynomial $p$
  has \emph{at least} one root $\mathbb{F}$. This looks weaker, but it
  implies the stronger form. The reason? Extract the root, then re-apply
  the theorem to the ``remainder polynomial.''

  It turns out no \emph{finite} field (such as $\Zmodpx$) is
  algebraically closed. For consider the polynomial:

  \begin{nedqn}
    p(x)
  \eqcol
    \parens{x - a_1}
    \parens{x - a_2}
    \cdots
    \parens{x - a_n}
    + 1
  \end{nedqn}

  \noindent
  where $a_1, \ldots, a_n$ enumerate the field. Then $p$ has no root!
\end{remark}

\subsection{Random Facts}

\begin{remark}
  In a resource I was reading, it did not use the Fundamental Theorem of
  Finite Abelian groups to prove that $G$ is cyclic iff every subgroup
  has distinct order. It approached this a slightly different way.
\end{remark}

\begin{lemma}
  Consider $A$ a cyclic subgroup of $B$ with order $k$. Then there are
  exactly $\vphif{k}$ distinct generators of $A$.
\end{lemma}

\begin{proof}
  Since $A$ is cyclic, take any $g$ that generates it. Consider those
  $g^m$ where $m$ is coprime to $k$. It is clear that $g^m$ generates at
  least $k$ elements, because $\parens{g^m}^k$ is the first power of
  $g^m$ where the exponent is divisible by $k$.
\end{proof}

\begin{lemma}
  \begin{nedqn}
    \sum_{d | n} \vphif{d}
  \eqcol
    n
  \end{nedqn}
\end{lemma}

\begin{proof}
  For a given $d$ I ask: how many elements $x \leq n$ satisfy $\gcdf{x,
  n} = d$? For each such $x$, there is a corresponding $x' =
  \frac{x}{d}$ which must be coprime with $\frac{n}{d}$. Thus there are
  $\vphif{n/d}$ of these.

  Thus I can partition $1, \ldots, n$ into $S_d = \setof{x\ |\ \gcdf{x,
  n} = d}$. Each $S_d$ has exactly $\vphif{n/d}$ elements. So the sum of
  all $\vphif{n/d}$ must be $n$.
\end{proof}

\begin{theorem}
  $\Zmodpx$ is cyclic.
\end{theorem}

\begin{proof}
  Consider all the cyclic subgroups of a finite abelian group $G$. Each
  cyclic subgroup has some degree $d$ and thus some $\vphif{d}$
  generators for the subgroup.

  Since every element of $G$ is a generator of \emph{some} subgroup,
  then we saw from the above that it should be possible that there is
  exactly one of each subgroup of degree $d$ dividing $\order{G}$. The
  various $\vphif{d}$ add up to $\order{G}$.

  If for some $d$ there is \emph{no} cyclic subgroup of that order, this
  implies that there must be more than one cyclic subgroup of some
  \emph{other} order $d'$. Otherwise when we consider the generators of
  all cyclic subgroups, we wouldn't have our $\order{G}$ elements.

  But if $G$ is a field, then there \emph{can't} be more than one cyclic
  subgroup of order $d$ (weak Finite Theorem of Algebra). In which case
  there must be a cyclic subgroup of every order $d$. Even of order
  $\order{G}$. Which means $G$ is cyclic.
\end{proof}

\begin{remark}
  I feel like this way is less clear, since the Fundamental Theorem of
  Finite Abelian Groups is always at the forefront of my mind.
\end{remark}

% TODO:
%
% Fundamental theorem says G is isomorphic to some direct sum of cyclic
% groups with order p_i^k_i. However, not all p_i need be distinct.
%
% With Zmodn, because it is cyclic, we do need all p_i to be distinct.
% Else the direct sum would not be cyclic because some factor would be
% "aligned."
%
% Multiplication is effectively a notation for repeated addition. Thus
% it is unsurprising that the group isomorphism between Zmodn and its
% factors is in fact a *ring* isomorphism.
%
% But this leaves a question. If we care only about the multiplication
% operation on the units, can we *further* factor the factors of Zmodn?
%
% So we need to explore \Zmodx{p^k}. But even before doing so, we can
% note that the various $\Zmodx{p_i^k_i}$ no longer sum to a cyclic
% group. The reason is that we already know the *size* of these groups.
% They are $\phi(p_i^k_i)$. But for odd primes $p_i$ we know that
% $\phi(p_i^k_i)$ is always even. And thus we know that there are some
% cyclic factors that must be aligned.
%
% Note that the same proof holds whenever p and 4 are factors. But *not*
% when p and 2 are factors!
%
% That synthetic direction isn't my focus. I'd rather focus on breaking
% down $\Zmodx{p_k}$ further.
%
% An excellent start would be to prove that \Zmodpx is cyclic. I believe
% there is a proof here:
%
% https://pi.math.cornell.edu/~mathclub/Media/mult-grp-cyclic-az.pdf
