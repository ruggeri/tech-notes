\section{$\Zmodpx$ Is Cyclic}

\begin{definition}
  The \define{exponent} of a group $G$ is the smallest $n$ such that
  $g^n = 1$ for all $g \in G$.
\end{definition}

\begin{proposition}
  The exponent of an abelian group $G$ is equal to $\lcmf{\order{g_1},
  \ldots, \order{g_n}}$.
\end{proposition}

\begin{proof}
  Factor $G$ into cyclic subgroups. Every element $x \in G$ can be
  written as a product of $g_i^{k_i}$  where $g_i$ is a generator of a
  factor cyclic subgroup. Each $g_i^{k_i}$ has order dividing $g_i$. And
  the order of $g_i g_j$ is equal to $\lcmf{\order{g_i}, \order{g_j}}$.
  Thus the order of $x$ divides $\lcmf{\order{g_1}, \ldots,
  \order{g_n}}$.
\end{proof}

\begin{corollary}
  If the exponent of a group $G$ is $k$, then there exists some $x\in G$
  such that $\order{k} = k$.
\end{corollary}

\begin{theorem}
  $\Zmodpx$ is cyclic.
\end{theorem}

\begin{proof}
  We know by Lagrange that every element $x \in \Zmodpx$ has order
  dividing $p-1$. So there are $p-1$ roots to the equation:

  \begin{nedqn}
    x^{p-1} - 1
  \equivcol
    0
    \pmod{p}
  \end{nedqn}

  The question is: is there any proper divisor $d$ of $p-1$ such that
  $x^d - 1 = 0$? But if $d < p-1$, then the (weak) Fundamental Theorem
  of Algebra says that there are at most $d < p-1$ roots to any such
  polynomial.

  Note how the Fundamental Theorem only applies because $\Zmodpx$ is a
  field! If $G$ is not a field, then $x^d - 1 = 0$ may have more than
  $d$ roots!

  Thus no proper divisor $d$ can be the exponent $\Zmodpx$. Therefore
  the exponent of $\Zmodpx$ is $p-1$. Therefore there is an element $g$
  with order $p-1$. Therefore $g$ generates all of $\Zmodpx$. Therefore
  $\Zmodpx \cong \Zmod{p-1}$.
\end{proof}

\begin{proof}
  Here is a second proof that is mostly the same idea. We appeal to the
  Fundamental Theorem of Algebra to argue that there exists \emph{at
  most} one cyclic subgroup of order $d$. That's because $x^d - 1$ has
  at most $d$ solutions.

  We've previously proven that this implies that the group is cyclic.
\end{proof}

\subsection{The (Weak) Fundamental Theorem Of Algebra}

\begin{lemma}
  For any field $\Field$, if a degree $n$ polynomial $p$ has a root $r$,
  then there exists a degree $n-1$ polynomial $q$ such that:

  \begin{nedqn}
    p(x)
  \eqcol
    \parens{x - r}
    q(x)
  \end{nedqn}
\end{lemma}

\begin{proof}
  First, consider if $r = 0$. Then the constant term of $p$ must also be
  zero. In that case, we can extract a factor of $x$ from $p$ to reduce
  its degree. This is justified by the requirement that all rings (and
  thus all fields) respect the distributive law.

  Say that root $r \ne 0$. Then consider the polynomial $p'(x) = p(x +
  r)$. We know that $p'$ has a root at 0, and so it can be factored into
  the product of $x$ and a degree $n-1$ polynomial $q'(x)$.

  But then we can recall that $p(x) = p'(x - r)$. Basically we just
  substitute $(x-r)$ for $x$ in $p'$. This shows that $p$ is a product
  of $(x - r)$ and a degree $n-1$ polynomial $q(x) = q'(x - r)$.
\end{proof}

\begin{lemma}
  For any field $\Field$, a (non-zero) polynomial $p$ of degree $n$ has
  \emph{at most} $n$ roots.
\end{lemma}

\begin{proof}
  We've shown that if $r_1$ is a root of $p$, then we can write $p$ as a
  product of $(x-r_1)$ and $q_1(x)$, where $q_1$ has degree $n-1$. We
  simply repeat this process by factoring any root $r_2$ of $q_1$. This
  tells us that $p$ is a product of $(x-r_1)(x-r_2)q_2(x)$.

  We may continue factoring $q$ until it is \define{irreducible}. That
  is: until $q$ has no root. We can repeat the process \emph{up to} $n$
  times. After $n$ roots are extracted, $q_n$ is a degree 0 polynomial.
  This has no root, so long as $q$ is non-zero. But $p$ was assumed to
  be non-zero, so $q$ is non-zero also.

  After extraction, we have:

  \begin{nedqn}
    p(x)
  \eqcol
    q(x)
    \prod_{i = 1}^k
    (x - r_i)
  \end{nedqn}

  \noindent
  Note: some $r_i$ may be repeated with \emph{multiplicity}. Also note:
  $q$ may be non-constant.

  At this point we know that we have found at most $n$ factors of $p$
  with the form $(x - r_i)$. This identifies at most $n$ roots. But can
  $p$ have \emph{more} roots?

  The answer is no, because for any other $r'$ we know that all $(r' -
  r_i)$ are nonzero. We also know that $q(r') \ne 0$ (since $q$ is
  presumed irreducible). Because we are working over a field, no product
  of non-zero elements can itself be zero.
\end{proof}

\begin{definition}
  We say that a field $\Field$ is \define{algebraically closed} if every
  polynomial over $\Field$ of degree $n$ has \emph{exactly} $n$ roots in
  $\Field$ (accounting for multiplicity). Equivalently: we can factor
  $p$ until the remainder $q$ is a degree zero (constant) polynomial.

  An equivalent definition of algebraically closed: every non-constant
  polynomial $p$ has \emph{at least} one root in $\Field$. This looks
  weaker, but it isn't. This ``weaker'' version implies that no
  non-constant remainder polynomial $q$ is irreducible, and thus no
  polynomial factorization will end before a total of $n$ roots are
  extracted.
\end{definition}

\begin{theorem}[The Fundamental Theorem Of Algebra]
  $\C$ is algebraically closed.
\end{theorem}

\begin{proposition}
  If $\Field$ is a finite field, then it cannot be algebraically closed.
\end{proposition}

\begin{proof}
  Consider the polynomial:

  \begin{nedqn}
    p(x)
  \eqcol
    \parens{x - a_1}
    \parens{x - a_2}
    \cdots
    \parens{x - a_n}
    + 1
  \end{nedqn}

  \noindent
  where $a_1, \ldots, a_n$ enumerate the field $\Field$. Then $p$ has no
  root!
\end{proof}

% https://pi.math.cornell.edu/~mathclub/Media/mult-grp-cyclic-az.pdf
