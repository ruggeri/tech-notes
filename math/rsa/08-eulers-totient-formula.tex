\section{Euler's Totient Function Formula}

\subsection{Multiplicative Formula For $\varphi$}

\begin{proposition}
  $\vphif{p} = p - 1$
\end{proposition}

\begin{proof}
  This is the definition of being prime.
\end{proof}

\begin{proposition}
  \begin{nedqn}
    \vphif{p^k}
  \eqcol
    p^{k-1} \parens{p - 1}
  \end{nedqn}
\end{proposition}

\begin{proof}
  A number is coprime to $p^k$ iff it does not contain $p$ as a factor.
  There are $p^{k-1}$ multiples of $p$ up to and including $p^k$. Thus:

  \begin{nedqn}
    \vphif{p^k}
  \eqcol
    p^k - p^{k-1}
  \\
  \eqcol
    p^{k-1} \parens{p - 1}
  \end{nedqn}
\end{proof}

\begin{proposition}
  If $a, b$ are coprime, then

  \begin{nedqn}
    \vphif{ab}
  \eqcol
    \vphif{a} \vphif{b}
  \end{nedqn}
\end{proposition}

\begin{proof}
  Consider $\Zmod{ab}$ ($a, b$ coprime). This is isomorphic to $\Zmod{a}
  \times \Zmod{b}$ by the Fundamental Theorem of Finite Abelian Groups.

  Consider any $u\in\Zmod{ab}$ and corresponding $u_a \times u_b$ in
  $\Zmoda \times \Zmodb$. Consider the sequence of sums

  \begin{nedqn}
    & u, u + u, u + u + u, \ldots &
  %
  \intertext{and the corresponding sequence of sums}
  %
  &
  u_a \times u_b,
  (u_a + u_a) \times (u_b + u_b),
  (u_a + u_a + u_a) \times (u_b + u_b + u_b)
  \ldots
  &
  \end{nedqn}

  Denote the $k$-th term of the first sequence as $ku$, and the $k$-th
  term of the second sequence as $k\parens{u_a \times u_b}$. Note that
  this is \emph{notation}.

  I claim that a group isomorphism always must fix the identity, since
  $f(x + 0) = f(x) + f(0)$. Therefore 0 always corresponds with $0
  \times 0$ and $ku = 0$ exactly when $k\parens{u_a \times u_b} = 0
  \times 0$.

  Define $\order{u_a \times u_b}$ as the minimum $k > 0$ such that
  $k\parens{u_a \times u_b} = 0 \times 0$. Likewise define $\order{u}$.
  And note that $\order{u} = \order{u_a \times u_b}$.

  Consider the orders $\order{u_a}, \order{u_b}$ in $\Zmoda, \Zmodb$. I
  claim that $\order{u_a \times u_b} = \lcmf{\order{u_a}, \order{u_b}}$.

  Because $\order{u_a}, \order{u_b}$ always divide $a, b$ (Lagrange's
  theorem), we know that the only way $\order{u_a \times u_b} = ab$ is
  if $\order{u_a} = a, \order{u_b} = b$. We also know $\order{u_a} = a$
  iff $u_a$ is coprime with $a$. Likewise $\order{u_b} = b$ iff $u_b$ is
  coprime with $b$. So there are exactly $\vphif{a}\vphif{b}$ distinct
  $u_a \times u_b$ with order equal to $ab$.

  Thus there are exactly $\vphif{a}\vphif{b}$ corresponding $u$ with
  order equal to $ab$. Which are exactly those $u$ which are coprime
  with $ab$.
\end{proof}

\begin{theorem}
  \begin{nedqn}
    \vphif{\prod p_i^{k_i}}
  \eqcol
    \prod p_i^{k_i-1} \parens{p_i - 1}
  \end{nedqn}
\end{theorem}

\begin{proof}
  Note that any two $p_i^{k_i}$ are coprime for distinct primes $p_i$.
  So we know that:

  \begin{nedqn}
    \vphif{\prod p_i^{k_i}}
  \eqcol
    \prod \vphif{p_i^{k_i}}
  %
  \intertext{and by our earlier formula we can then conclude}
  %
  \eqcol
    \prod p_i^{k_i-1} \parens{p_i - 1}
  \end{nedqn}
\end{proof}

\subsection{$\Zmod{ab}$ and $\Zmoda \times \Zmodb$ are Ring Isomorphic}

\begin{remark}
  We appealed to the Fundamental Theorem of Finite Abelian Groups to
  insist a group isomorphism exists between the additive groups
  $\Zmod{ab}$ and $\Zmoda \times \Zmodb$. Let me be more concrete by
  supplying such an isomorphism:

  \begin{nedqn}
    \ff{x}
  & \mapsto &
    \parens{x \bmod a} \times \parens{x \bmod b}
  \end{nedqn}

  First we show this is a homomorphism:

  \begin{nedqn}
    \ff{x + y}
  & \mapsto &
    \parens{x + y \bmod a} \times \parens{x + y \bmod b}
  \\
  \eqcol
    \parens{x \bmod a} \times \parens{x \bmod b}
    +
    \parens{y \bmod a} \times \parens{y \bmod b}
  \\
  \eqcol
    \ff{x} + \ff{y}
  \end{nedqn}

  Next we should show that $f$ is bijective. Note that both $\Zmod{ab}$
  and $\Zmoda \times \Zmodb$ have exactly $ab$ elements. So it would
  suffice to show either that $f$ is injective or surjective.

  I will show that $f$ is injective. Note that, since $f$ is a
  homomorphism, it is injective exactly when $\ker f = \setof{0}$. That
  is: the homomorphism $f$ is injective iff $0$ is the only value such
  that $f(0) = 0 \times 0$.

  But this is clearly the case, as there is no common multiple of $a, b$
  greater than $0$ and less than $ab$.
\end{remark}

\begin{theorem}
  $\Zmod{ab}$ is \define{ring isomorphic} to $\Zmoda \times \Zmodb$.

  That is, we have both:

  \begin{nedqn}
    f(x + y)
  \eqcol
    f(x) + f(y)
  \intertext{and}
    f(xy)
  \eqcol
    f(x) f(y)
  \end{nedqn}
\end{theorem}

\begin{proof}
  We previously gave a group isomorphism:

  \begin{nedqn}
    f(x)
  & \mapsto &
    \parens{x \bmod a} \times \parens{y \bmod b}
  \end{nedqn}

  \noindent
  Consider $f(xy) = \parens{xy \bmod a} \times \parens{xy \bmod b}$. To
  prove $f$ preserves multiplication, we will show that for any $k$:

  \begin{nedqn}
    xy \bmod k
  & \equiv &
    \parens{x \bmod k}\parens{y \bmod k}
    \pmod{k}
  \end{nedqn}

  \noindent
  Perhaps we ought to have already established this basic fact of
  $\Zmod{k}$. We do so now. Denote $x \bmod k, y \bmod k$ as $b_x, b_y$.
  Then there exist $a_x, b_y$ such that:

  \begin{nedqn}
    x
  \eqcol
    a_xk + b_x
  \\
    y
  \eqcol
    a_yk + b_y
  \end{nedqn}

  Thus the product $xy$ is:

  \begin{nedqn}
    xy
  \eqcol
    \parens{a_xk + b_x} \parens{a_yk + b_y}
  \\
  \eqcol
    a_x a_y k^2
    + a_x b_y k
    + b_x a_y k
    + b_x b_y
  \\
  \equivcol
    b_x b_y
    \pmod{k}
  \\
  \equivcol
    \parens{x \bmod k}\parens{y \bmod k}
    \pmod{k}
  \end{nedqn}
\end{proof}

\begin{remark}
  Question: is it true that \emph{every} group isomorphism $f$ of
  $\Zmodn$ must also be a ring isomorphism? No. A ring isomorphism must
  fix the multiplicative identity. But that is \emph{not} necessary for
  a group isomorphism. For instance, consider that if $f$ is a group
  isomorphism, then (for any unit $a$) so is

  \begin{nedqn}
    f'(x)
  \mapstocol
    f(ax)
  \end{nedqn}

  \noindent
  The multiplication by $a$ merely permutes the elements of $\Zmodn$
  before $f$ does its usual thing.

  \TODO{Can we have clarity on why the $f$ I've provided is the natural
  ring isomorphism?} Is it the \emph{only} ring isomorphism into
  $\Zmodn$? Yes if $\Zmodn$ is cyclically generated? No otherwise?
\end{remark}

\subsection{Characterization of $\Zmodnx$}

\begin{theorem}
  For any coprime $a, b$:

  \begin{nedqn}
    \Zmodx{ab}
  & \cong &
    \Zmodx{a} \times \Zmodx{b}
  \end{nedqn}

  More generally, if $n = \prod p_i^{k_i}$, then:

  \begin{nedqn}
    \Zmodx{n}
  & \cong &
    \Zmodx{p_1^{k_1}} \times \cdots \times \Zmodx{p_n^{k_n}}
  \end{nedqn}
\end{theorem}

\begin{proof}
  Our theorem about ring isomorphism has this exact implication for the
  multiplicative subgroup.
\end{proof}

\begin{remark}
  To understand the structure of $\Zmodnx$ it be useful to understand
  $\Zmodx{p^k}$. $\Zmodnx$ is just a direct product of these.

  We can explore a little those $\Zmodnx$ that \emph{can} be factorized,
  though.
\end{remark}

\begin{proposition}
  For odd primes $p, q$, $\Zmodx{pq}$ is never cyclic.
\end{proposition}

\begin{proof}
  Note that $\Zmodx{p}, \Zmodx{q}$ have $p-1, q-1$ elements
  respectively. Both $p-1, q-1$ are even, so both are divisible by 2.
  We've previously shown that if a group $G$ can be factored into
  subgroups with cocomposite order, then the group $G$ cannot be cyclic.

  Note that it really doesn't matter whether $\Zmodx{p}, \Zmodx{q}$ are
  cyclic (they are).
\end{proof}

\begin{proposition}
  If $n = 2^k p^m$ with $k_2 > 1$ and $p$ an odd prime, then $\Zmodnx$
  cannot be cyclic.
\end{proposition}

\begin{proof}
  It suffices to show that $\Zmodx{2^k}$ has even order. Knowing that,
  the prior proof will apply. Note that $\vphif{2^k} = 2^{k-1}(2 - 1) =
  2^{k-1}$, so $\Zmodx{2^k}$ has $2^{k-1}$ elements. Since $k>1$, this
  is divisible by 2.
\end{proof}

\begin{remark}
  These propositions can be trivially extended to any $n$ which is
  divisible by two distinct odd prime powers, or by one odd prime power
  and $2^k$ (with $k > 1$).
\end{remark}

\begin{proposition}
  If $n = 2 p^k$, then $\Zmodnx$ is isomorphic to $\Zmodx{p^k}$.
\end{proposition}

\begin{proof}
  This is trivial since $\Zmodx{2}$ has just the identity element.
\end{proof}

\begin{corollary}
  $\Zmodx{2 p^k}$ is cyclic iff $\Zmodx{p^k}$ is cyclic.
\end{corollary}

\begin{proposition}
  $\Zmodx{2}$ and $\Zmodx{4}$ are both cyclic.
\end{proposition}

\begin{proof}
  $\Zmodx{2}$ contains just a single element: 1. It is trivially cyclic.
  $\Zmodx{4}$ contains just two elements: $1, 3$. $3$ must be its own
  inverse. So this is also cyclic.
\end{proof}

\begin{proposition}
  $\Zmodx{2^k}$ is never cyclic for $k>2$.
\end{proposition}

\begin{proof}
  It suffices to show that $\Zmodx{8}$ is not cyclic. It contains the
  elements $\setof{1, 3, 5, 7}$. Note that:

  \begin{nedqn}
    & 3^2 \equiv 5^2 \equiv 7^2 \equiv 1 \pmod{8} &
  \end{nedqn}

  Now consider $\Zmodx{2^k}$. Assume it were cyclic. Then it would have
  a generator $g$ of order $2^{k-1}$. But then note that $g \mod 2^3$ is
  a generator for $\Zmodx{8}$, which we just saw is acyclic.
\end{proof}

\begin{remark}
  To conclude for now, we've seen how to factor $\Zmodnx$ into a direct
  product of $\Zmodx{p_i^{k_i}}$. We've seen that many $\Zmodnx$ are
  acyclic - whenever $\Zmodnx$ can be factored into two disjoint
  subgroups.

  We've seen that $\Zmodx{2^k}$ is cyclic iff $k \in \setof{1, 2}$. We
  haven't yet seen how to factor $\Zmodx{2^k}$ when $k > 2$!

  We haven't examined $\Zmodx{p^k}$ for odd primes $p$. We shall soon
  see that these are always cyclic and thus isomorphic to
  $\Zmod{\vphif{p^k}}$.
\end{remark}

%   We will see that often $\Zmodx{p^k}$ is cyclic. But even when these
%   factors are are cyclic, their direct
% \end{remark}

% \begin{remark}
%   Note that $\Zmodax, \Zmodbx$ may not be cyclic. Even when these are
%   cyclic, note that $\Zmodx{ab}$ may not be. $\Zmodx{ab}$ will never be
%   cyclic when $\order{\Zmodax}$ and $\order{\Zmodbx}$ are not coprime.
% \end{remark}
