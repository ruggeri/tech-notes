\section{Euler's Totient Function Formula}

\begin{proposition}
  $\vphif{p} = p - 1$
\end{proposition}

\begin{proof}
  This is the definition of being prime.
\end{proof}

\begin{proposition}
  \begin{nedqn}
    \vphif{p^k}
  \eqcol
    p^{k-1} \parens{p - 1}
  \end{nedqn}
\end{proposition}

\begin{proof}
  A number is coprime to $p^k$ iff it does not contain $p$ as a factor.
  There are $p^{k-1}$ multiples of $p$ up to and including $p^k$. Thus:

  \begin{nedqn}
    \vphif{p^k}
  \eqcol
    p^k - p^{k-1}
  \\
  \eqcol
    p^{k-1} \parens{p - 1}
  \end{nedqn}
\end{proof}

\begin{proposition}
  If $a, b$ are coprime, then

  \begin{nedqn}
    \vphif{ab}
  \eqcol
    \vphif{a} \vphif{b}
  \end{nedqn}
\end{proposition}

\begin{proof}
  Consider all values of the form: $ak_a + bk_b$. $k_a, k_b$ range over
  $0,\ldots, b-1$ and $0,\ldots, a-1$. I claim that distinct sums of
  $ak_a + bk_b$ are in bijection with pairs $k_a, k_b$.

  I won't belabor why, but you can recall the Fundamental Theorem of
  Finite Abelian Groups that $\Zmod{ab} \cong \Zmod{a} \oplus \Zmod{b}$.
  More concretely: consider sums of elements from the disjoint cyclic
  subgroups $\anglef{b}, \anglef{a}$.

  Great. Now, consider any $k_a$ coprime to $b$. I claim that $ak_a +
  bk_b$ is coprime to $b$. Why? First note that $bk_b$ is just a
  multiple of $b$. So $\gcd\parens{ak_a + bk_b, b} = \gcd\parens{ak_a,
  b}$. But both $a, k_a$ are coprime with $b$ so this is one.

  If I choose $k_a$ \emph{not} coprime with $b$, then $ak_a + bk_b$ will
  \emph{not} be coprime with $b$ either. So I have exactly $\vphif{b}$
  choices for $k_a$ if I want the sum $ak_a + bk_b$ coprime with $b$.

  Likewise I have exactly $\vphif{a}$ choices for $k_b$ if I want the
  sum $ak_a + bk_b$ to be coprime with $a$. Thus there are
  $\vphif{a}\vphif{b}$ choices for $k_a, k_b$ such that $ak_a + bk_b$ is
  a unit.
\end{proof}

Here is an alternate, more group theoretic proof. It amounts to the same
thing.

\begin{proof}

  Consider $\Zmod{ab}$ ($a, b$ coprime). This is isomorphic to $\Zmod{a}
  \oplus \Zmod{b}$ by the Fundamental Theorem of Finite Abelian Groups.

  Take a \emph{pair} of multiplicative units $u_a, u_b$ in $\Zmod{a},
  \Zmod{b}$. We know this means there exist $k_a, k_b$ such that
  $u_a^{k_a} \equiv 1 \pmod{a}, u_b^{k_b} \equiv 1 \pmod{b}$.

  Note that the concept of ``multiplication'' is precisely repeated
  addition. So we can talk about multiplication (and exponentiation) on
  \emph{any} additive group. That is: multiplication and exponentiation
  are \emph{notational}.

  We know that $(u_a^{k_a}) \oplus (u_b^{k_b}) = 1 \oplus 1$, which is
  good. But that doesn't directly relate to repeated addition of $u_a
  \oplus u_b$. So consider $k = k_a k_b$. We now know that:

  \begin{nedqn}
    \parens{u_a \oplus u_b}^{k}
  \eqcol
    (u_a^{k_a k_b}) \oplus (u_b^{k_a k_b})
  \\
  \eqcol
    (1^{k_b}) \oplus (1^{k_a})
  \\
  \eqcol
    1 \oplus 1
  \end{nedqn}

  What element of $\Zmod{ab}$ corresponds to $1 \oplus 1$? It must be a
  generator of $\Zmod{ab}$. In fact, we might as well assume that $1$
  maps to $1 \oplus 1$.  Because any isomorphism that maps $x$ unit to
  $1 \oplus 1$ can be ``fixed'' by first multiplying everything by
  $\inv{x}$.

  Thus, we've shown that for $u$ corresponding to $u_a \oplus u_b$, $u^k
  = 1$. Thus $u$ is a unit.

  Since $\Zmod{a}, \Zmod{b}$ have $\vphif{a}, \vphif{b}$ multiplicative
  units respectively, there are at least $\vphif{a}\vphif{b}$
  multiplicative units of $\Zmod{ab}$. It's simple to finish things by
  showing that if $u$ in $\Zmod{ab}$ is a mulitpicative unit, the
  corresponding $u_a, u_b$ are multiplicative units of $\Zmod{a},
  \Zmod{b}$.

  The multiplicative units of $\Zmod{ab}$ are thus in bijection with
  pairs of multiplicative units of $\Zmod{a}, \Zmod{b}$. In fact, we
  have:

  \begin{nedqn}
    \Zmodx{ab}
  & \cong &
    \Zmodx{a} \otimes \Zmodx{b}
  \end{nedqn}

  Thus there are $\vphif{a}\vphif{b}$ units of $\Zmodx{ab}$.
\end{proof}

\begin{remark}
  Don't be fooled by the isomorphism into thinking that if $\Zmodx{a},
  \Zmodx{b}$ are cyclic, then this implies that $\Zmodx{a} \otimes
  \Zmodx{b}$ is cyclic! We've seen when characterizing finite abelian
  groups that this is not necessarily true!

  We know that this is only true when $|\Zmodx{a}|, |\Zmodx{b}|$ are
  coprime!
\end{remark}

\begin{theorem}
  Since we can break any integer $n$ into a prime power factorization,
  we know that:

  \begin{nedqn}
    \vphif{\prod p_i^{k_i}}
  \eqcol
    \prod p_i^{k_i-1} \parens{p_i - 1}
  \end{nedqn}
\end{theorem}

\begin{proof}
  Note that any two $p_i^{k_i}$ are coprime for distinct primes $p_i$.
  So we know that:

  \begin{nedqn}
    \vphif{\prod p_i^{k_i}}
  \eqcol
    \prod \vphif{p_i^{k_i}}
  %
  \intertext{and by our earlier formula we can then conclude}
  %
  \eqcol
    \prod p_i^{k_i-1} \parens{p_i - 1}
  \end{nedqn}
\end{proof}
