\section{Euler's Totient Function Formula}

\begin{proposition}
  $\vphif{p} = p - 1$
\end{proposition}

\begin{proof}
  This is the definition of being prime.
\end{proof}

\begin{proposition}
  \begin{nedqn}
    \vphif{p^k}
  \eqcol
    p^{k-1} \parens{p - 1}
  \end{nedqn}
\end{proposition}

\begin{proof}
  A number is coprime to $p^k$ iff it does not contain $p$ as a factor.
  There are $p^{k-1}$ multiples of $p$ up to and including $p^k$. Thus:

  \begin{nedqn}
    \vphif{p^k}
  \eqcol
    p^k - p^{k-1}
  \\
  \eqcol
    p^{k-1} \parens{p - 1}
  \end{nedqn}
\end{proof}

\begin{proposition}
  If $a, b$ are coprime, then

  \begin{nedqn}
    \vphif{ab}
  \eqcol
    \vphif{a} \vphif{b}
  \end{nedqn}
\end{proposition}

\begin{proof}
  Consider $\Zmod{ab}$ ($a, b$ coprime). This is isomorphic to $\Zmod{a}
  \oplus \Zmod{b}$ by the Fundamental Theorem of Finite Abelian Groups.

  Consider any $u\in\Zmod{ab}$ and corresponding $u_a \oplus u_b$ in
  $\Zmoda \oplus \Zmodb$. Consider the sequence of sums

  \begin{nedqn}
    & u, u + u, u + u + u, \ldots &
  %
  \intertext{and the corresponding sequence of sums}
  %
  &
  u_a \oplus u_b,
  (u_a + u_a) \oplus (u_b + u_b),
  (u_a + u_a + u_a) \oplus (u_b + u_b + u_b)
  \ldots
  &
  \end{nedqn}

  Denote the $k$-th term of the first sequence as $ku$, and the $k$-th
  term of the second sequence as $k\parens{u_a \oplus u_b}$. Note that
  this is \emph{notation}.

  I claim that a group isomorphism always fixes the identity, so 0
  always corresponds with $0 \oplus 0$. Therefore $ku = 0$ exactly when
  $k\parens{u_a \oplus u_b} = 0 \oplus 0$.

  Define $\order{u_a \oplus u_b}$ as the minimum $k > 0$ such that
  $k\parens{u_a \oplus u_b} = 0 \oplus 0$. Likewise define $\order{u}$.
  And note that $\order{u} = \order{u_a \oplus u_b}$.

  Consider the orders $\order{u_a}, \order{u_b}$ in $\Zmoda, \Zmodb$. I
  claim that $\order{u_a \oplus u_b} = \lcmf{\order{u_a}, \order{u_b}}$.

  Because $\order{u_a}, \order{u_b}$ always divide $a, b$ (Lagrange's
  theorem), we know that the only way $\order{u_a \oplus u_b} = ab$ is
  if $\order{u_a} = a, \order{u_b} = b$. We also know $\order{u_a} = a$
  iff $u_a$ is coprime with $a$. Likewise $\order{u_b} = b$ iff $u_b$ is
  coprime with $b$. So there are exactly $\vphif{a}\vphif{b}$ distinct
  $u_a \oplus u_b$ with order equal to $ab$.

  Thus there are exactly $\vphif{a}\vphif{b}$ corresponding $u$ with
  order equal to $ab$. Which are exactly those $u$ which are coprime
  with $ab$.
\end{proof}

\begin{theorem}
  \begin{nedqn}
    \vphif{\prod p_i^{k_i}}
  \eqcol
    \prod p_i^{k_i-1} \parens{p_i - 1}
  \end{nedqn}
\end{theorem}

\begin{proof}
  Note that any two $p_i^{k_i}$ are coprime for distinct primes $p_i$.
  So we know that:

  \begin{nedqn}
    \vphif{\prod p_i^{k_i}}
  \eqcol
    \prod \vphif{p_i^{k_i}}
  %
  \intertext{and by our earlier formula we can then conclude}
  %
  \eqcol
    \prod p_i^{k_i-1} \parens{p_i - 1}
  \end{nedqn}
\end{proof}

\begin{remark}
  We appealed to the Fundamental Theorem of Finite Abelian Groups to
  insist a group isomorphism exists between the additive groups
  $\Zmod{ab}$ and $\Zmoda \oplus \Zmodb$. Let me be more concrete by
  supplying such an isomorphism:

  \begin{nedqn}
    \ff{x}
  & \mapsto &
    \parens{x \bmod a} \oplus \parens{x \bmod b}
  \end{nedqn}

  We verify this is a group isomorphism:

  \begin{nedqn}
    \ff{x + y}
  & \mapsto &
    \parens{x + y \bmod a} \oplus \parens{x + y \bmod b}
  \\
  \eqcol
    \parens{x \bmod a} \oplus \parens{x \bmod b}
    +
    \parens{y \bmod a} \oplus \parens{y \bmod b}
  \\
  \eqcol
    \ff{x} + \ff{y}
  \end{nedqn}

  I should show that $f$ is bijective. Consider the (additive) subgroups
  $\anglef{a}, \anglef{b}$ of $\Zmod{ab}$. These map to the subgroups
  $\setof{0} \oplus \Zmodb$ and $\Zmoda \oplus \setof{0}$ of $\Zmoda
  \oplus \Zmodb$. By closure, all of $\Zmoda \oplus \Zmodb$ is spanned
  by $f$. Since both $\Zmod{ab}$ and $\Zmoda \oplus \Zmodb$ have $ab$
  elements, they are in bijection.
\end{remark}

\begin{theorem}
  $\Zmod{ab}$ is \define{ring isomorphic} to $\Zmoda \oplus \Zmodb$.

  That is, we have both:

  \begin{nedqn}
    f(x + y)
  \eqcol
    f(x) + f(y)
  \intertext{and}
    f(xy)
  \eqcol
    f(x) f(y)
  \end{nedqn}
\end{theorem}

\begin{proof}
  Consider the previously provided group isomorphism

  \begin{nedqn}
    f(x)
  & \mapsto &
    \parens{x \bmod a} \oplus \parens{y \bmod b}
  \end{nedqn}

  Consider $f(xy) = \parens{xy \bmod a} \oplus \parens{xy \bmod b}$. It
  will suffice to show that for any $k$

  \begin{nedqn}
    xy \bmod k
  & \equiv &
    \parens{x \bmod k}\parens{y \bmod k}
    \pmod{k}
  \end{nedqn}

  Denote $x \bmod k, y \bmod k$ as $b_x, b_y$. There exists $a_x, b_y$
  such that:

  \begin{nedqn}
    x
  \eqcol
    a_xk + b_x
  \\
    y
  \eqcol
    a_yk + b_y
  \end{nedqn}

  Thus the product $xy$ is:

  \begin{nedqn}
    xy
  \eqcol
    \parens{a_xk + b_x} \parens{a_yk + b_y}
  \\
  \eqcol
    a_x a_y k^2
    + a_x b_y k
    + b_x a_y k
    + b_x b_y
  \\
  \equivcol
    b_x b_y
    \pmod{k}
  \\
  \equivcol
    \parens{x \bmod k}\parens{y \bmod k}
    \pmod{k}
  \end{nedqn}
\end{proof}

\begin{theorem}
  We may thus conclude that for any coprime $a, b$:

  \begin{nedqn}
    \Zmodx{ab}
  & \cong &
    \Zmodx{a} \oplus \Zmodx{b}
  \end{nedqn}

  More generally, if $n = \prod p_i^{k_i}$, then:

  \begin{nedqn}
    \Zmod{n}
  & \cong &
    \bigoplus \Zmod{p_i^{k_i}}
  \end{nedqn}
\end{theorem}

\begin{proof}
  Our theorem about ring isomorphism has this exact implication for the
  multiplicative subgroup.
\end{proof}

\begin{remark}
  Note that $\Zmodax, \Zmodbx$ may not be cyclic. Even when these are
  cyclic, note that $\Zmodx{ab}$ may not be. $\Zmodx{ab}$ will never be
  cyclic when $\order{\Zmodax}$ and $\order{\Zmodbx}$ are not coprime.
\end{remark}
