\section{Lagrange's Theorem}

\begin{definition}
  The \define{order} of a group $A$ (denoted $\order{A}$) is the number
  of elements in $A$.
\end{definition}

\begin{theorem}[Lagrange's Theorem]

  In any group $G$ (not just $\Zmodp$), the order of every subgroup $A$
  divides the order of the group $G$.
\end{theorem}

\begin{definition}
  Consider a subgroup $A$ of a group $G$. The \define{coset} $bA$ is
  the set of all elements $ba$.

  Note that most cosets are \emph{not} subgroups. For instance, given
  any $b \not\in A$, the coset $bA$ does not contain the identity
  element $1$ ($\inv{b} \not\in A$ because a group is closed under
  inverses).
\end{definition}

\begin{lemma}
  All cosets $bA$ of a group $G$ have the same size.
\end{lemma}

\begin{proof}
  I say that the size of $bA$ is equal to $A$. Why?

  Since $G$ is a group, the inverse $\inv{b}$ exists. Therefore, we know
  that:

  \begin{nedqn}
    \inv{b} bA
  \eqcol
    A
  \end{nedqn}

  If two $b a_0, b a_1$ mapped to the same value, they could not invert
  back to the separate values $a_0, a_1$.
\end{proof}

\begin{lemma}
  Cosets are either disjoint or identical.
\end{lemma}

\begin{proof}
  Consider two distinct $b_0 A$ and $b_1 A$. I say that they
  must either be identical, or completely disjoint. If they are not
  disjoint, then there are $a_0, a_1$ such that:

  \begin{nedqn}
    b_0 a_0
  \eqcol
    b_1 a_1
  \\
    b_1
  \eqcol
    b_0 a_0 \inv{a_1}
  \end{nedqn}

  Note that $a_0 \inv{a_1}$ is guaranteed to be in the subgroup $A$
  (subgroups have inverses, are closed under multiplication).

  Now, for any $b_1 a$ in $b_1 A$, we have:

  \begin{nedqn}
    b_1 a
  \eqcol
    (b_0 a_0 a_1^{-1}) a = b_0 (a_0 a_1^{-1} a)
  \end{nedqn}

  Thus every element of $b_1 A$ is also in $b_0 A$. And vice-versa.
  Therefore, if $b_0 A, b_1 A$ have any elements in common, they are
  identical.
\end{proof}

We may now complete our proof of Lagrange's Theorem.

\begin{proof}
  The union of the set of cosets $b A$ is $G$ (since $1 \in A$, and $b$
  can be any element of $G$). Each coset has the same size, and they are
  non-overlapping. Therefore, the \emph{distinct} cosets
  \emph{partition} the group $G$.

  Say there are $k$ distinct cosets. Then the order of the subgroup $A$
  must be $\frac{\order{G}}{k}$. Thus $\order{A}$ divides $\order{G}$.
\end{proof}
