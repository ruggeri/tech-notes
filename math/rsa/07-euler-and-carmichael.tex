\section{Carmichael's Totient Functions}

\subsection{Euler's Totient Function Formula}

\begin{remark}
  We know that $\varphi(p) = p - 1$.
\end{remark}

\begin{remark}
  We know that

  \begin{nedqn}
    \varphi\parens{p^k}
  \eqcol
    p^k - p^k/p
  \\
  \eqcol
    p^{k-1} \parens{p - 1}
  \end{nedqn}

  \noindent
  Why? Because a number is coprime to $p^k$ iff it does not contain $p$
  as a factor.
\end{remark}

\begin{remark}
  Next consider $\varphi(ab)$ for $a, b$ coprime. We know by Fundamental
  Theorem of Finite Abelian Groups that $\Zmod{ab}$ factors into a
  direct product of two subgroups $G_a, G_b$ of sizes $a, b$.

  Moreover, because $\Zmod{ab}$ is cyclic, we know that both $G_a, G_b$
  are cyclic. Thus they are isomorphic to $\Zmod{a}, \Zmod{b}$. There
  are $\varphi(a), \varphi(b)$ units in each of these.

  Any product of a unit of $G_a, G_b$ is a unit of $\Zmod{ab}$. Thus
  there are $\varphi(a)\varphi(b)$ units for $\Zmod{ab}$.
\end{remark}

\begin{theorem}
  In conclusion:

  \begin{nedqn}
    \varphi\parens{\prod p_i^{k_i}}
  \eqcol
    \prod p_i^{k_i-1} \parens{p_i - 1}
  \end{nedqn}
\end{theorem}

\subsection{Carmichael's Totient Function}

\begin{definition}
  \define{Carmichael's totient function} $\lambda(n)$ maps $n$ to the
  smallest number such that $x^{\lambda(n)} = x \pmod{n}$ for all units
  $x \in \Zmodnx$.
\end{definition}

\begin{remark}
  When $\varphi(n)$ is prime, $\Zmodnx$ is a finite abelian group of
  prime order. That means that it is cyclic. That means that it is
  generated by some $x$ with $|x| = \varphi(n)$. So when $\varphi(n)$ is
  prime, $\lambda(n) = \varphi(n)$.
\end{remark}

\begin{remark}
  Next, consider when $\varphi(n)$ factors into $\prod p_i$. We can
  again split $\Zmodnx$ into a direct product of cyclic groups, each of
  size $p_i$. That is: $\Zmodnx \cong \bigotimes_i \Zmod{p_i}$.

  Take generators $g_i$ in each subgroup. Multiply these all together to
  get $g = \prod g_i$. It should be clear that $g$ generates $\Zmodnx$.
  It thus has order $\varphi(n)$.

  So again we have $\lambda(n) = \varphi(n)$.
\end{remark}

\begin{remark}
  We may consider what happens when $\varphi(n) = \prod p_i^{k_i}$. Note
  that we can again break this into a direct product of subgroups $A_i$
  of size $p_i^{k_i}$.

  Here is where things are different! We \emph{do not know} whether a
  subgroup $A_i$ is isomorphic to $\Zmod{p_i^{k_i}}$. If this is indeed
  true for each $A_i$ we would again have $\lambda(n) = \varphi(n)$.

  But we \emph{could} have, for instance, $A_i$ is a direct product of
  $k_i$ copies of $\Zmod{p_i}$. In this case, $A_i$ is \emph{acyclic}.
  More to the point, for all $a \in A_i$, $a^{|p_i|} = a$.

  In summary: the question of whether $\lambda(n) = \varphi(n)$ comes
  down to whether every subgroup of $\Zmodnx$ is cyclic. Equivalently:
  it comes down to whether $\Zmodnx$ is itself cyclic. Equivalently: it
  comes down to whether $\Zmodnx$ contains a \emph{primitive root}.
\end{remark}

\begin{definition}
  An element $g$ is a \define{root} of $1$ if $g^n = 1$ for some $n$. In
  a finite group, every element is a root of 1.

  An element $g$ is a \define{primitive root} of $1$ if it has order
  $|G|$. A group has a primitive root of unity iff it is cyclic.
\end{definition}
