\section{Carmichael's Totient Function}

\subsection{Exploring $\lambdaf{n}$ for factorable $n$}

\begin{definition}
  The \define{exponent} of a group $G$ is the smallest $n$ such that
  $g^n = 1$ for all $g \in G$.
\end{definition}

\begin{definition}
  \define{Carmichael's totient function} $\lambda(n)$ maps $n$ to the
  exponent of $\Zmodnx$. That is: the smallest number such that
  $x^{\lambda(n)} \equiv 1 \pmod{n}$ for all units $x \in \Zmodnx$.
\end{definition}

\begin{proposition}
  $\lambdaf{n} = \vphif{n}$ iff $\Zmodnx$ is cyclic. Else $\lambdaf{n}$
  must divide $\vphif{n}$.
\end{proposition}

\begin{theorem}
  \begin{nedqn}
    \lambdaf{\prod p_i^{k_i}}
  \eqcol
    \lcm\parens{\lambdaf{p_1^{k_1}}, \ldots, \lambdaf{p_n^{k_n}}}
  \end{nedqn}
\end{theorem}

\begin{proof}
  We've previously shown that $\Zmod{ab}$ is ring isomorphic to
  $\Zmod{a} \times \Zmod{b}$. We've shown this implies an isomorphism
  between the multiplicative group $\Zmodx{ab}$ and $\Zmodax \times
  \Zmodbx$.

  So take any $u_a, u_b$ in $\Zmodax, \Zmodbx$ such that $\order{u_a},
  \order{u_b}$ are maximum. That is: $\order{u_a} = \lambdaf{a},
  \order{u_b} = \lambdaf{b}$.

  Now consider powers $\parens{u_a \times u_b}^i$. These equal $u_a^i
  \times u_b^i$. We know this equals $1 \times 1$ iff $i$ is a common
  multiple of $\lambdaf{a}, \lambdaf{b}$. That implies:

  \begin{nedqn}
    \order{u_a \times u_b}
  \eqcol
    \lcmf{\lambdaf{a}, \lambdaf{b}}
  \end{nedqn}

  Also: since every other $u_a', u_b'$ have orders dividing
  $\lambdaf{a}, \lambdaf{b}$, $u_a \times u_b$ has maximum order in
  $\Zmodax \times \Zmodbx$.

  By isomorphism we know that $\order{u} = \lcmf{\lambdaf{a},
  \lambdaf{b}}$. Thus we conclude:

  \begin{nedqn}
    \lambdaf{ab}
  \eqcol
    \lcmf{\lambdaf{a}, \lambdaf{b}}
  \end{nedqn}

  It's easy to extend our proof to more prime power factors.
\end{proof}

\begin{remark}
  We've given a recursive definition of $\lambda$. We must now fill in
  the base cases. When we have characterized $\Zmod{2^k}$ and
  $\Zmod{p^k}$ as direct products of cyclic groups, we will immediately
  know $\lambdaf{2^k}$ and $\lambdaf{p^k}$. The recursive definition
  above will apply to any factorization of $\Zmodx{2^k}, \Zmodx{p^k}$.
\end{remark}

\begin{proposition}
  Note that $\lambdaf{2} = 1$ and $\lambdaf{4} = 2$. $\lambdaf{2^k} <
  \vphif{2^k}$ for $k > 2$.
\end{proposition}

\begin{proof}
  We've previously shown that $\Zmodx{2}, \Zmodx{4}$ are cyclic groups.
  So $\lambdaf{2} = \vphif{2} = 1$, $\lambdaf{4} = \vphif{4} = 2$.

  We've shown that $\Zmodx{2^k}$ is \emph{acyclic} for $k > 2$. Thus we
  know that $\lambdaf{2^k} < \vphif{2^k}$.
\end{proof}
