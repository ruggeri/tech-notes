\section{Carmichael's Totient Function}

\begin{definition}
  \define{Carmichael's totient function} $\lambda(n)$ maps $n$ to the
  smallest number such that $x^{\lambda(n)} = 1 \pmod{n}$ for all units
  $x \in \Zmodnx$.
\end{definition}

\begin{theorem}
  \begin{nedqn}
    \lambdaf{\prod p_i^{k_i}}
  \eqcol
    \lcm\parens{\lambdaf{p_1^{k_1}}, \ldots, \lambdaf{p_n^{k_n}}}
  \end{nedqn}
\end{theorem}

\begin{proof}
  Recall our previous argument that multiplication and exponentiation
  are really notational conveniences for repeated addition. Thus
  isomorphism with respect to addition entails corresponding isomorphism
  with respect to multiplication/exponentiation. We showed that:

  \begin{nedqn}
    \Zmodx{ab}
  & \cong &
    \Zmodx{a} \otimes \Zmodx{b}
  \end{nedqn}

  \TODO{I don't think you need this isomorphism to make this argument?}

  So consider any $u$ in $\Zmodx{ab}$ and a pair of corresponding $u_a,
  u_b$ in $\Zmodx{a}, \Zmodx{b}$. We previously showed that $|u|$ must
  divide $|u_a||u_b|$. But we also discussed in a remark that $|u|$ can
  properly divide $|u_a||u_b|$. We identified that this happens exactly
  when $|\Zmodx{a}|$ and $|\Zmodx{b}|$ fail to be coprime.

  We can get more concrete. Corresponding to powers of $u$ are the
  powers of $u_a \otimes u_b$:

  \begin{nedqn}
    \parens{
      u_a \otimes u_b
    }^i
  \eqcol
    u_a^i \otimes u_b^i
  \end{nedqn}

  \noindent
  How long until this is equal to $1 \otimes 1$? It is exactly:

  \begin{nedqn}
    \lcm\parens{
      \lambdaf{a},
      \lambdaf{b}
    }
  \end{nedqn}

  \noindent
  This proves that $\lambdaf{ab} = \lcmf{\lambdaf{a}, \lambdaf{b}}$. But
  this can be trivially extended by successive decomposition to the full
  result:

  \begin{nedqn}
    \lambdaf{\prod p_i^{k_i}}
  \eqcol
    \lcm\parens{\lambdaf{p_1^{k_1}}, \ldots, \lambdaf{p_n^{k_n}}}
  \end{nedqn}
\end{proof}

\begin{remark}
  We now must examine $\lambdaf{p^k}$. We know that if $\Zmodx{p^k}$ can
  be factored into disjoint subgroups, then $\lambdaf{p^k} <
  \vphif{p^k}$. Exactly when $\Zmodx{p^k}$ cannot be factored (when it
  is already cyclic), we know that $\lambdaf{p^k} = \vphif{p^k}$.

  A generator for $\Zmodx{n}$ is called a \define{primitive root modulo
  $n$}.
\end{remark}

% TODO: Want to prove Carmichael's Theorem.
