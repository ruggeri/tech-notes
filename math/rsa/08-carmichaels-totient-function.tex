\section{Carmichael's Totient Function}

\begin{definition}
  \define{Carmichael's totient function} $\lambda(n)$ maps $n$ to the
  smallest number such that $x^{\lambda(n)} = 1 \pmod{n}$ for all units
  $x \in \Zmodnx$.
\end{definition}

\begin{theorem}
  \begin{nedqn}
    \lambdaf{\prod p_i^{k_i}}
  \eqcol
    \lcm\parens{\lambdaf{p_1^{k_1}}, \ldots}
  \end{nedqn}
\end{theorem}

\begin{proof}
  First recall that $\Zmodn$ is isomorphic to $\bigoplus
  \Zmod{p_i^{k_i}}$. WLOG, assume that $1$ corresponds to $1
  \oplus \cdots \oplus 1$.

  Next, note that properties about exponentiation are really properties
  about repeated multiplication. And properties of multiplication are
  really just properties of repeated addition.

  So when you talk about $|u|$, you're really asking how long this
  sequence is:

  \begin{nedqn}
    &u&
  \\
    &u + u&
  \\
    &(u + u) + (u + u)&
  \\
    &\ldots&
  \\
    &1&
  \end{nedqn}

  \noindent
  How long is that sequence? Well, it exactly the length of the
  corresponding sequence:

  \begin{nedqn}
    & (u_1 \oplus \cdots \oplus u_n)&
  \\
    &(u_1 \oplus \cdots \oplus u_n) + (u_1 \oplus \cdots \oplus u_n)&
  \\
    &\ldots&
  \\
    &(1 \oplus \cdots \oplus 1)&
  \end{nedqn}

  \noindent
  But we know this! It is:

  \begin{nedqn}
    \lcm\parens{
      \lambdaf{p_1^{k_1}},
      \ldots,
      \lambdaf{p_n^{k_n}}
    }
  \end{nedqn}

  \noindent
  Thus we are done!
\end{proof}

\begin{remark}
  More succinctly: we've previously proven that for any coprime $a, b$,
  $\Zmodx{ab} \cong \Zmodx{a} \otimes \Zmodx{b}$. Take any $u \in
  \Zmodx{ab}$ and corresponding $u_a, u_b$ in $\Zmodx{a}, \Zmodx{b}$.

  Clearly $|u|$ must divide $|u_a||u_b|$. In fact, clearly $|u|$ is
  equal to $\lcm{|u_a|, |u_b|}$.

  So $\lambdaf{ab} = \lcmf{\lambdaf{a}, \lambdaf{b}}$.
\end{remark}

\begin{remark}
  We now must examine $\lambdaf{p^k}$.
\end{remark}

\begin{remark}
  When $\vphif{n}$ is prime, $\Zmodnx$ is a finite abelian group of
  prime order. That means that it is cyclic. That means that it is
  generated by some $x$ with $|x| = \vphif{n}$. So when $\vphif{n}$ is
  prime, $\lambda(n) = \vphif{n}$.
\end{remark}

\begin{remark}
  Next, consider when $\vphif{n}$ factors into $\prod p_i$. We can
  again split $\Zmodnx$ into a direct product of cyclic groups, each of
  size $p_i$. That is: $\Zmodnx \cong \bigotimes_i \Zmod{p_i}$.

  Take generators $g_i$ in each subgroup. Multiply these all together to
  get $g = \prod g_i$. It should be clear that $g$ generates $\Zmodnx$.
  It thus has order $\vphif{n}$.

  So again we have $\lambda(n) = \vphif{n}$.
\end{remark}

\begin{remark}
  We may consider what happens when $\vphif{n} = \prod p_i^{k_i}$. Note
  that we can again break this into a direct product of subgroups $A_i$
  of size $p_i^{k_i}$.

  Here is where things are different! We \emph{do not know} whether a
  subgroup $A_i$ is isomorphic to $\Zmod{p_i^{k_i}}$. If this is indeed
  true for each $A_i$ we would again have $\lambda(n) = \vphif{n}$.

  But we \emph{could} have, for instance, $A_i$ is a direct product of
  $k_i$ copies of $\Zmod{p_i}$. In this case, $A_i$ is \emph{acyclic}.
  More to the point, for all $a \in A_i$, $a^{|p_i|} = a$.

  In summary: the question of whether $\lambda(n) = \vphif{n}$ comes
  down to whether every subgroup of $\Zmodnx$ is cyclic. Equivalently:
  it comes down to whether $\Zmodnx$ is itself cyclic. Equivalently: it
  comes down to whether $\Zmodnx$ contains a \emph{primitive root}.
\end{remark}

\begin{definition}
  An element $g$ is a \define{root} of $1$ if $g^n = 1$ for some $n$. In
  a finite group, every element is a root of 1.

  An element $g$ is a \define{primitive root} of $1$ if it has order
  $|G|$. A group has a primitive root of unity iff it is cyclic.
\end{definition}

\begin{proposition}
  Because $\Zmodnx$ into a direct product of $\Zmod{p_i^{k_i}}$, we
  have:

  \begin{nedqn}
    \lambdaf{\prod p_i^{k_i}}
  \eqcol
    \lcm\parens{
      \lambdaf{p_1^{k_1}},
      \ldots
      \lambdaf{p_n^{k_n}}
    }
  \end{nedqn}
\end{proposition}
