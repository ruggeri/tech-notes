\section{Carmichael's Totient Function}

\begin{definition}
  \define{Carmichael's totient function} $\lambda(n)$ maps $n$ to the
  smallest number such that $x^{\lambda(n)} = 1 \pmod{n}$ for all units
  $x \in \Zmodnx$.
\end{definition}

\begin{theorem}
  \begin{nedqn}
    \lambdaf{\prod p_i^{k_i}}
  \eqcol
    \lcm\parens{\lambdaf{p_1^{k_1}}, \ldots, \lambdaf{p_n^{k_n}}}
  \end{nedqn}
\end{theorem}

\begin{proof}
  We've previously shown that $\Zmod{ab}$ is ring isomorphic to
  $\Zmod{a} \oplus \Zmod{b}$. We've showed this implies an isomorphism
  between the groups $\Zmodx{ab}$ and $\Zmodax \oplus \Zmodbx$.

  So take any $u_a, u_b$ in $\Zmodax, \Zmodbx$ such that $\order{u_a},
  \order{u_b}$ are maximum. That is: $\order{u_a} = \lambdaf{a},
  \order{u_b} = \lambdaf{b}$.

  Now consider powers $\parens{u_a \oplus u_b}^i$. These equal $u_a^i
  \oplus u_b^i$. We know this equals $1 \oplus 1$ iff $i$ is a common
  multiple of $\lambdaf{a}, \lambdaf{b}$. That implies:

  \begin{nedqn}
    \order{u_a \oplus u_b}
  \eqcol
    \lcmf{\lambdaf{a}, \lambdaf{b}}
  \end{nedqn}

  Also: since every other $u_a', u_b'$ have orders dividing
  $\lambdaf{a}, \lambdaf{b}$, $u_a \oplus u_b$ has maximum order in
  $\Zmodax \oplus \Zmodbx$.

  By isomorphism we know that $\order{u} = \lcmf{\lambdaf{a},
  \lambdaf{b}}$. Thus we conclude:

  \begin{nedqn}
    \lambdaf{ab}
  \eqcol
    \lcmf{\lambdaf{a}, \lambdaf{b}}
  \end{nedqn}
\end{proof}

\begin{remark}
  We now must examine $\lambdaf{p^k}$. We know that if $\Zmodx{p^k}$ can
  be factored into disjoint subgroups, then $\lambdaf{p^k} <
  \vphif{p^k}$. Exactly when $\Zmodx{p^k}$ cannot be factored (when it
  is already cyclic), we know that $\lambdaf{p^k} = \vphif{p^k}$.

  A generator for $\Zmodx{n}$ is called a \define{primitive root modulo
  $n$}.
\end{remark}

% TODO:
%
% Fundamental theorem says G is isomorphic to some direct sum of cyclic
% groups with order p_i^k_i. However, not all p_i need be distinct.
%
% With Zmodn, because it is cyclic, we do need all p_i to be distinct.
% Else the direct sum would not be cyclic because some factor would be
% "aligned."
%
% Multiplication is effectively a notation for repeated addition. Thus
% it is unsurprising that the group isomorphism between Zmodn and its
% factors is in fact a *ring* isomorphism.
%
% But this leaves a question. If we care only about the multiplication
% operation on the units, can we *further* factor the factors of Zmodn?
%
% So we need to explore \Zmodx{p^k}. But even before doing so, we can
% note that the various $\Zmodx{p_i^k_i}$ no longer sum to a cyclic
% group. The reason is that we already know the *size* of these groups.
% They are $\phi(p_i^k_i)$. But for odd primes $p_i$ we know that
% $\phi(p_i^k_i)$ is always even. And thus we know that there are some
% cyclic factors that must be aligned.
%
% That synthetic direction isn't my focus. I'd rather focus on breaking
% down $\Zmodx{p_k}$ further.
%
% An excellent start would be to prove that \Zmodpx is cyclic. I believe
% there is a proof here:
%
% https://pi.math.cornell.edu/~mathclub/Media/mult-grp-cyclic-az.pdf
