\section{Characterization of $\Zmodpx$}

\subsection{$\Zmodpx$ is Cyclic}

\begin{lemma}
  $\Zmodpx$ never has two distinct cyclic subgroups with the same order.
\end{lemma}

\begin{proof}
  Consider a cyclic subgroup of order $k$. It is generated by $a, a^2,
  a^3, \ldots, a^k \equiv 1$. Thus these are $k$ distinct roots to the
  equation:

  \begin{nedqn}
    x^k - 1
  \equivcol
    0
    \pmod{p}
  \end{nedqn}

  \noindent
  I assert that there can be no more than $k$ distinct roots to this
  polynomial. This is a weak version of the Fundamental Theorem of
  Algebra applied to general finite \emph{fields}. Thus there can be no
  second subgroup of order $k$.

  Note that when we are working in $\Zmodnx$, then this is not a finite
  field, and thus we cannot apply (the weak version of) the Fundamental
  Theorem of Algebra.
\end{proof}

\begin{corollary}
  $\Zmodpx$ is cyclic.
\end{corollary}

\begin{proof}
  We've already shown previously that if a group $G$ has a unique
  subgroup for each $d$ dividing $\order{G}$, then $G$ is cyclic.
\end{proof}

\subsection{The (Weak) Fundamental Theorem Of Algebra}

\begin{lemma}
  For any field $\Field$, a (non-zero) polynomial $p$ of degree $n$ has
  \emph{at most} $n$ roots.
\end{lemma}

\begin{proof}
  First, note that 0 is a root of a polynomial $p$ if and only if the
  constant term of $p$ is itself zero. In that case, we can extract a
  factor of $x$ from $p$ to reduce its degree. Note that this is
  justified by the requirement that all rings (and thus all fields)
  respect the distributive law.

  Say that $p$ has a non-zero root $r_1$. Then consider the polynomial
  $p'(x) = p(x + r_1)$. We know that $p'$ has a root at 0, and so it can
  be factored into the product of $x$ and a degree $n-1$ polynomial
  $q'(x)$.

  But then we can recall that $p(x) = p'(x - r_1)$ to substitute in $x -
  r_1$. This shows that $p$ is a product of $(x - r_1)$ and a degree
  $n-1$ polynomial $q(x) = q'(x - r_1)$.

  To extract more factors, we can continue factoring the remainder
  polynomial $q$. The extraction of roots from $q$ cannot go on
  indefinitely: eventually $q$ would become a (non-zero) constant
  polynomial, which has no roots. In fact, a \emph{maximum} of $n$ roots
  can be extracted.

  After extraction, we have:

  \begin{nedqn}
    p(x)
  \eqcol
    q(x)
    \prod_{i = 1}^k
    (x - r_i)
  \end{nedqn}

  \noindent
  Note: some $r_i$ may be repeated with \emph{multiplicity}. Also note:
  $q$ may be non-constant. Regardless, we are not done until $q$ has no
  roots.

  At this point we know that we have found at most $n$ factors of $p$
  with the form $(x - r_i)$. This identifies at most $n$ roots. But can
  $p$ have \emph{more} roots?

  The answer is no, because for any other $r'$ we know that all $(r' -
  r_i)$ are nonzero. We also know that $q(r') \ne 0$ for all $r'$ (since
  $q$ is presumed irreducible). Because we are working over a field, no
  product of non-zero elements can itself be zero.
\end{proof}

\begin{definition}
  We say that a field $\Field$ is \define{algebraically closed} if every
  polynomial over $\Field$ of degree $n$ has \emph{exactly} $n$ roots in
  $\Field$ (accounting for multiplicity). Equivalently: we can factor
  $p$ until the remainder $q$ is a degree zero (constant) polynomial.

  An equivalent definition of algebraically closed: every non-constant
  polynomial $p$ has \emph{at least} one root in $\Field$. This looks
  weaker, but it isn't. This ``weaker'' version implies that no
  non-constant remainder polynomial $q$ is irreducible, and thus no
  polynomial factorization will end before a total of $n$ roots are
  extracted.
\end{definition}

\begin{remark}
  The (strong) Fundamental Theorem says that $\mathbb{C}$ is
  algebraically closed.

  It turns out no \emph{finite} field (such as $\Zmodpx$) is
  algebraically closed. For consider the polynomial:

  \begin{nedqn}
    p(x)
  \eqcol
    \parens{x - a_1}
    \parens{x - a_2}
    \cdots
    \parens{x - a_n}
    + 1
  \end{nedqn}

  \noindent
  where $a_1, \ldots, a_n$ enumerate the field. Then $p$ has no root!
\end{remark}

\subsection{Random Facts}

<<< UP TO HERE.

\begin{remark}
  In a resource I was reading, it did not use the Fundamental Theorem of
  Finite Abelian groups to prove that $G$ is cyclic iff every subgroup
  has distinct order. It approached this a slightly different way.
\end{remark}

\begin{lemma}
  Consider $A$ a cyclic subgroup of $B$ with order $k$. Then there are
  exactly $\vphif{k}$ distinct generators of $A$.
\end{lemma}

\begin{proof}
  Since $A$ is cyclic, take any $g$ that generates it. Consider those
  $g^m$ where $m$ is coprime to $k$. It is clear that $g^m$ generates at
  least $k$ elements, because $\parens{g^m}^k$ is the first power of
  $g^m$ where the exponent is divisible by $k$.
\end{proof}

\begin{lemma}
  \begin{nedqn}
    \sum_{d | n} \vphif{d}
  \eqcol
    n
  \end{nedqn}
\end{lemma}

\begin{proof}
  For a given $d$ I ask: how many elements $x \leq n$ satisfy $\gcdf{x,
  n} = d$? For each such $x$, there is a corresponding $x' =
  \frac{x}{d}$ which must be coprime with $\frac{n}{d}$. Thus there are
  $\vphif{n/d}$ of these.

  Thus I can partition $1, \ldots, n$ into $S_d = \setof{x\ |\ \gcdf{x,
  n} = d}$. Each $S_d$ has exactly $\vphif{n/d}$ elements. So the sum of
  all $\vphif{n/d}$ must be $n$.
\end{proof}

\begin{theorem}
  $\Zmodpx$ is cyclic.
\end{theorem}

\begin{proof}
  Consider all the cyclic subgroups of a finite abelian group $G$. Each
  cyclic subgroup has some degree $d$ and thus some $\vphif{d}$
  generators for the subgroup.

  Since every element of $G$ is a generator of \emph{some} subgroup,
  then we saw from the above that it should be possible that there is
  exactly one of each subgroup of degree $d$ dividing $\order{G}$. The
  various $\vphif{d}$ add up to $\order{G}$.

  If for some $d$ there is \emph{no} cyclic subgroup of that order, this
  implies that there must be more than one cyclic subgroup of some
  \emph{other} order $d'$. Otherwise when we consider the generators of
  all cyclic subgroups, we wouldn't have our $\order{G}$ elements.

  But if $G$ is a field, then there \emph{can't} be more than one cyclic
  subgroup of order $d$ (weak Finite Theorem of Algebra). In which case
  there must be a cyclic subgroup of every order $d$. Even of order
  $\order{G}$. Which means $G$ is cyclic.
\end{proof}

\begin{remark}
  I feel like this way is less clear, since the Fundamental Theorem of
  Finite Abelian Groups is always at the forefront of my mind.
\end{remark}

% https://pi.math.cornell.edu/~mathclub/Media/mult-grp-cyclic-az.pdf
