\section{Factoring The Ring $\Zmodn$}

\subsection{Ring Decomposition of $\Zmodn$}

\begin{remark}
  We've shown how to factor the \emph{group} $\Zmodn$ into a direct
  product of cyclic group factors $\Zmod{p_i^{k_i}}$. But what about the
  \emph{ring} structure of $\Zmodn$?
\end{remark}

\begin{theorem}
  For any coprime $a, b$, then $\Zmod{ab}$ is \define{ring isomorphic}
  to $\Zmoda \times \Zmodb$. That means there exists an bijective map
  $f$ such that both:

  \begin{nedqn}
    f(x + y)
  \eqcol
    f(x) + f(y)
  \intertext{and}
    f(xy)
  \eqcol
    f(x) f(y)
  \end{nedqn}
\end{theorem}

\begin{proof}
  We previously gave a group isomorphism:

  \begin{nedqn}
    f(x)
  & \mapsto &
    \parens{x \bmod a} \times \parens{y \bmod b}
  \end{nedqn}

  \noindent
  We will prove that $f$ is a ring isomorphism. Consider $f(xy) =
  \parens{xy \bmod a} \times \parens{xy \bmod b}$. To prove $f$
  preserves multiplication, we will show that for any $k$:

  \begin{nedqn}
    xy \bmod k
  & \equiv &
    \parens{x \bmod k}\parens{y \bmod k}
    \pmod{k}
  \end{nedqn}

  \noindent
  Perhaps we ought to have already established this basic fact of
  $\Zmod{k}$. We do so now. Denote $x \bmod k, y \bmod k$ as $b_x, b_y$.
  Then there exist $a_x, b_y$ such that:

  \begin{nedqn}
    x
  \eqcol
    a_xk + b_x
  \\
    y
  \eqcol
    a_yk + b_y
  \end{nedqn}

  Thus the product $xy$ is:

  \begin{nedqn}
    xy
  \eqcol
    \parens{a_xk + b_x} \parens{a_yk + b_y}
  \\
  \eqcol
    a_x a_y k^2
    + a_x b_y k
    + b_x a_y k
    + b_x b_y
  \\
  \equivcol
    b_x b_y
    \pmod{k}
  \\
  \equivcol
    \parens{x \bmod k}\parens{y \bmod k}
    \pmod{k}
  \end{nedqn}
\end{proof}

\begin{remark}
  The $f$ provided is the \emph{only} ring isomorphism between $\Zmodn$
  and the direct product of its factors. Why? Because it must fix

  \begin{nedqn}
    f(1)
  \mapstocol
    1 \times \cdots \times 1
  \end{nedqn}

  \noindent
  But the multiplicative identity 1 is a generator of the ring $\Zmodn$.
  Thus every

  \begin{nedqn}
    f(k) = f(1 + \cdots + 1) = f(1) + \cdots + f(1)
  \end{nedqn}

  \noindent
  is determined. $f$ is not merely the \emph{natural} ring isomorphism:
  it is the only one!
\end{remark}

\begin{remark}
  A group isomorphism $g$ that fixes $g(1) = 1 \times\cdots\times 1$,
  must be a ring isomorphism. If not there exists some $k = \inv{g(1)}$
  such that $k g(x)$ is an isomorphism. Equivalently, there must exist
  some corresponding $k'$ such that $g(k'x)$ is an isomorphism.

  Thus we see that the group isomorphisms of $\Zmodn$ are really just
  the (natural) ring isomorphism where we've first permuted the
  elements.
\end{remark}

\begin{remark}
  Keep in mind that the ring $\Zmodp \times \Zmodq$ is \emph{not} a
  \define{field}! Obviously $pq \equiv 0$. This is similar to saying that
  $1 \times 0$ and $0 \times 1$ both don't have an inverse. Notice how
  powers of $1 \times 0$ never hits zero, but also never hits one.

  Can finite fields of size $p^k$ exist? We know that examining any such
  field strictly as an additive group will look like a product of copies
  of $\Zmod{p_i^{k_i}}$. But what about their multiplicative group? This
  will \emph{also} look like a product of $\Zmod{p_i^{k_i}}$. The
  question then is: how do the additive and multiplicative operations
  relate?

  One way to create a finite field of size $p^k$ is to take polynomials
  of degree $k$ over the finite field of $p$ elements. When we define
  multiplication, we will mod out by an irreducible polynomial of degree
  $k+1$.
\end{remark}
