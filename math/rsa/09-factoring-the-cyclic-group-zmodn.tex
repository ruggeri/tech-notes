\section{Factoring The Cyclic Group $\Zmodn$}

\subsection{Group Decomposition of $\Zmodn$}

\begin{theorem}
  When $n$ factorizes as $\prod p_i^{k_i}$ (for distinct $p_i$), then

  \begin{nedqn}
    \Zmodn
  \congcol
    \Zmod{p_1^{k_1}}
    \times
    \cdots
    \times
    \Zmod{p_n^{k_n}}
  \end{nedqn}
\end{theorem}

\begin{proof}
  We already know from the Fundamental Theorem of Abelian Groups that
  any abelian group $G$ with order $n$ factorizes into subgroups $G_i$
  of order $p_i^{k_i}$.  When $G \cong \Zmodn$, must these factors $G_i$
  be cyclic? Or can they be further factorized?

  If there were any $G_i$ in the factorization that were not cyclic,
  then the corresponding subgroup of $\Zmodn$ would be acyclic. But
  manifestly every subgroup of a cyclic group must be cyclic. So $G_i$
  must be cyclic, too. Thus $G_i \cong \Zmod{p_i^{k_i}}$.
\end{proof}

\begin{remark}
  Oops, I kind of already used this at the outset of my proof that
  $\varphi$ is multiplicative.
\end{remark}

\begin{theorem}
  The function

  \begin{nedqn}
    \ff{x}
  & \mapsto &
    \parens{x \bmod a} \times \parens{x \bmod b}
  \end{nedqn}

  \noindent
  is a group isomorphism between $\Zmod{ab}$ and $\Zmoda \times \Zmodb$.
\end{theorem}

\begin{proof}
  First we show this is a homomorphism:

  \begin{nedqn}
    \ff{x + y}
  & \mapsto &
    \parens{x + y \bmod a} \times \parens{x + y \bmod b}
  \\
  \eqcol
    \parens{x \bmod a} \times \parens{x \bmod b}
    +
    \parens{y \bmod a} \times \parens{y \bmod b}
  \\
  \eqcol
    \ff{x} + \ff{y}
  \end{nedqn}

  Next we should show that $f$ is bijective. Note that both $\Zmod{ab}$
  and $\Zmoda \times \Zmodb$ have exactly $ab$ elements. So it would
  suffice to show either that $f$ is injective or surjective.

  I will show that $f$ is injective. Note that, since $f$ is a
  homomorphism, it is injective exactly when $\ker f = \setof{0}$. That
  is: the homomorphism $f$ is injective iff $0$ is the only value such
  that $f(0) = 0 \times 0$.

  But this is clearly the case, as there is no common multiple of $a, b$
  greater than $0$ and less than $ab$.
\end{proof}

\subsection{Characterization of Cyclic Groups by Their Subgroups}

\begin{proposition}
  A cyclic group $G$ of $n$ elements contains \emph{exactly one}
  subgroup of each order $d$ dividing $\order{G}$.
\end{proposition}

\begin{proof}
  Consider a generator $g$ and a divisor $d$ of $n$. $g^{n/d}$ is a
  generator for a subgroup of order $d$. Now consider any other $k$ such
  that $g^k$ generates a subgroup of order $d$. I claim that $g^{n/d}$
  is generated by $g^k$. Note that $kd \equiv 1 \pmod{n}$; in fact, we
  know that $d = \frac{n}{\gcdf{k, n}}$. But note this means $\gcdf{k,
  n} = \gcdf{n/d, n} = n/d$. We may now apply Bézout's lemma to show
  that there is some $m$ such that $km \equiv n/d \pmod{n}$.
\end{proof}

\begin{proposition}
  If a cyclic group $G$ contains no more than one subgroup of each
  order $d$ dividing $\order{G}$, then $G$ is cyclic.
\end{proposition}

\begin{proof}
  We'll prove this via the contrapositive. We know that if $G$ is
  \emph{acyclic}, then it has factors $\Zmod{p^k}$ and $\Zmod{p^m}$ with
  the same base $p$. But we know that both these factors contain
  subgroups of order $p$. But that means there are two distinct
  subgroups of order $p$. Done.
\end{proof}

\begin{theorem}
  $G$ is cyclic if and only if it contains exactly one cyclic subgroup
  of order $d$ for each $d$ dividing $\order{G}$.
\end{theorem}

\begin{corollary}
  A group $G$ that has \emph{no more} than one subgroup of each order
  $d$ dividing $\order{G}$ must in fact have \emph{exactly one} subgroup
  of order $d$.
\end{corollary}

\begin{proof}
  We know that $G$ must be cyclic. And we showed that cyclic groups
  contain exactly one subgroup with order $d$.
\end{proof}

\subsection{Little Facts Of $\Zmodn$}

\begin{proposition}
  Let $G_d$ be an order-$d$ cyclic subgroup of a cyclic group $G$. Let
  $g$ be a generator of $G_d$. Then for each $k$ coprime with $d$, $g^k$
  also generates $G_d$. These are all the generators of $G_d$. There are
  exactly $\vphif{d}$ of them.
\end{proposition}

\begin{proof}
  We may trivially show (by using Bézout's lemma) that $g^k$ generates
  the subgroup. There are exactly $\vphif{d}$ such choices of $k$.

  Again by Bézout, if $\gcdf{k', d} > 1$, then $g^{k'}$ cannot generate
  all of $G_d$. Of course no element $g'$ from outside $G_d$ could
  generate $G_d$. So these are all the generators.
\end{proof}

\begin{corollary}
  Let $G$ be a cyclic group of order $n$. For each $d$ dividing $n$
  there are exactly $\vphif{d}$ elements $g$ such that $\order{g} = d$.
  Also:

  \begin{nedqn}
    \sum_{d | n} \vphif{d}
  \eqcol
    n
  \end{nedqn}
\end{corollary}

\begin{proof}
  Note that every element of $G$ is in \emph{some} subgroup (even if
  this is all $G$). Thus when counting generators of subgroups, you
  account for all elements in $G$.

  Since a cyclic group $G$ has exactly one subgroup of order $d$
  for each $d$ dividing $n$, the above formula works perfectly.
\end{proof}
