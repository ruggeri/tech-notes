\section{The Extended Euclidean Algorithm}

\subsection{Euclid's Algorithm}

Euclid's Algorithm gives a procedure to compute $\gcdf{a, b}$. We assume
that $a < b$. The key insight is that any number that divides $a, b$
must also divide $b - a$. Indeed, we have:

\begin{nedqn}
  \gcdf{a, b}
\eqcol
  \gcdf{a, b - a}
\end{nedqn}

\noindent
Better still, we know that a divisor of both $a, b$ must also divide $b
\bmod a$. So we can have:

\begin{nedqn}
  \gcdf{a, b}
\eqcol
  \gcdf{a, b \bmod a}
\end{nedqn}

\noindent
The process continues until $b \bmod a$ is zero. In that case, you know
that $a$ evenly divides $b$, and thus it $\gcdf{a, b} = a$.

\subsection{Bezout's Identity}

\begin{theorem}[Bezout's Identity]
  For any $a, b$, there exists $x, y$ such that:

  \begin{nedqn}
    xa + yb
  \eqcol
    \gcdf{a, b}
  \end{nedqn}
\end{theorem}

\begin{remark}
  If $\gcdf{a, b} = 1$, then Bezout is saying that:

  \begin{nedqn}
    \inv{a}
  & \equiv & x
    \pmod{b}
  \\
    \inv{b}
  & \equiv &
    -y
    \pmod{a}
  \end{nedqn}
\end{remark}

\begin{proof}
  Let $z = \gcdf{a, b}$. Then there exists $k_a, k_b$ such that $a = k_a
  z$ and $b = k_b z$. Thus we desire to find $x, y$ such that:

  \begin{nedqn}
    x k_a z + y k_b z
  \eqcol
    z
  %
  \intertext{
    Let us simplify the question. Do there exist $x, y$ such that:
  }
  %
    x k_a + y k_b
  \eqcol
    1
  \end{nedqn}

  Why do I prefer this form? Because $k_a, k_b$ are \emph{relatively
  prime}. They have to be relatively prime, else the common factor
  should have been included as part of $z = \gcdf{a, b}$.

  Anyway, if we can prove Bezout's identity for $k_a, k_b$ relatively
  prime, then we have proven it for all $a, b$.

  To prove this, first assume that $k_a < k_b$. Now, since $k_a$ is
  relatively prime to $k_b$, we know that $k_a$ has an inverse in
  $\Zmodx{k_b}$. Denote this inverse as $x$.

  Since $x k_a \equiv 1 \pmod{k_b}$, this implies that there exists $y$
  such that $x k_a = 1 + y k_b$. But (flipping the sign of $y$), this
  gives us exactly the $x, y$ we have been searching for!
\end{proof}

\subsection{Extended Euclidean Algorithm}

Let's see an algorithm for finding $x, y$ for any $a, b$. Note that $x$
is the inverse of $a$ in $\Zmod{b}$ and $y$ is the inverse of $b$ in
$\Zmod{a}$. (Assuming $a, b$ are relatively prime).

Let's assume that $a < b$. Then we know:

\begin{nedqn}
    b
  \eqcol
    0a + 1b
  \\
    a
  \eqcol
    1a + 0b
  %
  \intertext{
    Starting from these two initial equations, let's derive a third.
    First calculate $k \defeq \floor{b / a}, c \defeq b \bmod a$. Using
    this, subtract $k$ times $1a + 0b$ from $0a + 1b$. We obtain:
  }
  %
    c
  \eqcol
    -ka + 1b
\end{nedqn}

Note of course that $c$ is smaller than $b, a$. We may repeat this
process with our equations for $a, c$ (similar to how we did for the
Euclidean algorithm).

Note that a common divisor of $a, b$ will always divide the left hand
side of the equations. If $a, b$ are relatively prime, then eventually
we will obtain $1$ on the left side, at which point we will have
obtained the appropriate $x, y$ to satisfy $1 = xa + yb$.

\begin{example}
  Let's invert $a = 71, b = 139$, each with respect to the other.

  \begin{IEEEeqnarray*}{rCrClCrCl}
    139 & = &   0 & \cdot & 71 & + &   1 & \cdot & 139 \\
    71  & = &   1 & \cdot & 71 & + &   0 & \cdot & 139 \\
    68  & = &  -1 & \cdot & 71 & + &   1 & \cdot & 139 \\
     3  & = &   2 & \cdot & 71 & + &  -1 & \cdot & 139 \\
     2  & = & -45 & \cdot & 71 & + &  23 & \cdot & 139 \\
     1  & = &  47 & \cdot & 71 & + & -24 & \cdot & 139 \\
  \end{IEEEeqnarray*}

  \noindent
  This says that the inverse of $71$ in $\Z \bmod 139$ is $47$.
\end{example}
