\section{Euler's Totient Theorem}

\begin{definition}
  We define \define{Euler's totient function} $\vphif{n}$ to be the
  number of numbers less than $n$ that are relatively prime to $n$. Note
  that of course $\vphif{p} = p - 1$.
\end{definition}

\begin{proposition}
  $\Zmodx{n}$ consists of exactly $\vphif{n}$ integers.
\end{proposition}

\begin{proof}
  We already proved that $\Zmodx{n}$ consists of exactly those numbers
  coprime with $n$. By definition there are exactly $\vphif{n}$ of
  these.
\end{proof}

\begin{theorem}[Euler's Totient Theorem] For any $x$ that is relatively
  prime to $n$, we have:

  \begin{nedqn}
    x^{\vphif{n}}
  \equivcol
    1 \pmod{n}
  \end{nedqn}
\end{theorem}

\begin{remark}
  Note that Euler's totient theorem generalizes Fermat's little theorem.
  They are both corollarys of Lagrange's theorem.
\end{remark}

\begin{proof}
  The theorem is another trivial corrollary of Lagrange's theorem. Since
  $\order{\Zmodx{n}}$ is $\vphif{n}$, the order of any cyclic subgroup
  generated by $x$ must divide $\vphif{n}$. This precisely mimics what
  we did for Fermat's little theorem in $\Zmodx{p}$.
\end{proof}
