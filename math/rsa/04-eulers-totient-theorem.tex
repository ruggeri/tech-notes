\section{Euler's Totient Theorem}

\begin{definition}
  We define \define{Euler's totient function} $\vphif{n}$ to be the
  number of numbers less than $n$ that are relatively prime to $n$. Note
  that of course $\vphif{p} = p - 1$.
\end{definition}

\begin{theorem}[Euler's Totient Theorem] For any $a$ that is relatively
  prime to $n$, we have:

  \begin{nedqn}
    a^{\vphif{n}}
  \equivcol
    1 \pmod{n}
  \end{nedqn}
\end{theorem}

\begin{remark}
  Note that Euler's totient theorem generalizes Fermat's little theorem.
  They are both corollarys of Lagrange's theorem.
\end{remark}

\begin{proof}
  We already discussed that $\Zmodx{n}$ consists exactly of those
  elements $a$ that are relatively prime to $n$. It contains all
  $\vphif{n}$ integers relatively prime to $n$ (including $a$).

  But then the theorem is trivial. It is simply an application of
  Lagrange's theorem to the multiplicative subgroup $\Zmodx{n}$. It
  mimics what we did for Fermat's little theorem in $\Zmodx{p}$.
\end{proof}
