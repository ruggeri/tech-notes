\section{Euler's Totient Theorem}

\begin{definition}
  We define \define{Euler's totient function} $\varphi(n)$ to be the
  number of numbers less than $n$ that are relatively prime to $n$. Note
  that of course $\varphi(p) = p - 1$.
\end{definition}

\begin{theorem}[Euler's Totient Theorem]
  For any $a$ that is \emph{relatively prime} to $n$, we have:

  \begin{nedqn}
    a^{\varphi(n)}
  \eqcol
    1 \mod n
  \end{nedqn}
\end{theorem}

\begin{remark}
  I emphasize again that $a$ must be \emph{relatively prime} to $n$.

  Note that Euler's totient theorem generalizes Fermat's little theorem.
\end{remark}

\begin{proof}
  We already discussed that the multiplicative subgroup of $\Zmodn$
  consists exactly of those elements $a$ that are relatively prime to
  $n$. So of course the size of the multiplicative subgroup is
  $\varphi(n)$.

  But then the theorem is trivial. It is simply an application of
  Lagrange's theorem within the multiplicative subgroup $\Zmodnx$. It
  mimics what we did for Fermat's little theorem in $\Zmodpx$.
\end{proof}
