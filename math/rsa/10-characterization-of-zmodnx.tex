\section{Characterization of $\Zmodnx$}

\subsection{Factorizing $\Zmodnx$ Into Subgroups}

\begin{remark}
  When $\vphif{n}$ has distinct prime power factors $p_1^{k_1},
  p_2^{k_2}$, then $\Zmodnx$ can be factored because the Fundamental
  Theorem of Finite Abelian Groups. But into what factor subgroups?
\end{remark}

\begin{theorem}
  For any coprime $a, b$:

  \begin{nedqn}
    \Zmodx{ab}
  & \cong &
    \Zmodx{a} \times \Zmodx{b}
  \end{nedqn}

  More generally, if $n = \prod p_i^{k_i}$ (for distinct $p_i$), then:

  \begin{nedqn}
    \Zmodx{n}
  & \cong &
    \Zmodx{p_1^{k_1}} \times \cdots \times \Zmodx{p_n^{k_n}}
  \end{nedqn}
\end{theorem}

\begin{proof}
  Our theorem about ring isomorphism has this exact implication for the
  multiplicative subgroup.
\end{proof}

\begin{corollary}
  For odd primes $p, q$, $\Zmodx{pq}$ is never cyclic.
\end{corollary}

\begin{proof}
  Note that $\Zmodx{p}, \Zmodx{q}$ have $p-1, q-1$ elements
  respectively. Both $p-1, q-1$ are even, so both are divisible by 2.
  We've previously shown that if a group $G$ can be factored into
  subgroups with cocomposite order, then the group $G$ cannot be cyclic.

  Note that it really doesn't matter whether $\Zmodx{p}, \Zmodx{q}$ are
  cyclic (we'll later show that they are).
\end{proof}

\begin{corollary}
  If $n = 2^k p^m$ with $k > 1$ and $p$ an odd prime, then $\Zmodnx$
  cannot be cyclic.
\end{corollary}

\begin{proof}
  It suffices to show that $\Zmodx{2^k}$ has even order. Knowing that,
  the prior proof will apply. Note that $\vphif{2^k} = 2^{k-1}(2 - 1) =
  2^{k-1}$, so $\Zmodx{2^k}$ has $2^{k-1}$ elements. Since $k>1$, this
  is divisible by 2.
\end{proof}

\begin{remark}
  These propositions can be trivially extended to any $n$ which is
  divisible by two distinct odd prime powers, or by one odd prime power
  and $2^k$ (with $k > 1$). No such $\Zmodnx$ can be cyclic.
\end{remark}

\begin{corollary}
  If $n = 2 p^k$, then $\Zmodnx$ is isomorphic to $\Zmodx{p^k}$.
\end{corollary}

\begin{proof}
  This is trivial since $\Zmodx{2}$ has just the identity element.
\end{proof}

\begin{corollary}
  $\Zmodx{2 p^k}$ is cyclic iff $\Zmodx{p^k}$ is cyclic.
\end{corollary}

\subsection{Examining $\Zmodx{2^k}$}

\begin{remark}
  We've just examined $\Zmodnx$ when $n$ factors into distinct prime
  powers. This is exactly when $\Zmodnx$ factors into a direct product
  of $\Zmodx{p_i^{k_i}}$.

  Having examined a little what this product might look, we now focus on
  the factors. We want to know about the structure of $\Zmodx{p^k}$.
  We'll start when $p = 2$.
\end{remark}

\begin{proposition}
  $\Zmodx{2}$ and $\Zmodx{4}$ are both cyclic.
\end{proposition}

\begin{proof}
  $\Zmodx{2}$ contains just a single element: 1. It is trivially cyclic.
  $\Zmodx{4}$ contains just two elements: $1, 3$. $3$ must be its own
  inverse. So this is also cyclic.
\end{proof}

\begin{proposition}
  $\Zmodx{8}$ is acyclic.
\end{proposition}

\begin{proof}
  This contains the elements $\setof{1, 3, 5, 7}$. Note that:

  \begin{nedqn}
    & 3^2 \equiv 5^2 \equiv 7^2 \equiv 1 \pmod{8} &
  \end{nedqn}

  \noindent
  Thus all non-1 element has degree 2.
\end{proof}

\begin{proposition}
  $\Zmodx{2^k}$ is acyclic for all $k>3$.
\end{proposition}

\begin{proof}
  \TODO{Must embed $\Zmodx{8}$ but how?}
\end{proof}

\begin{remark}
  \TODO{We really ought to \emph{factor} the multiplicative group
  $\Zmodx{2^k}$ for $k > 2$.}
\end{remark}
