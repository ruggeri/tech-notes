\section{Characterization of $\Zmodnx$}

\begin{theorem}
  For any coprime $a, b$:

  \begin{nedqn}
    \Zmodx{ab}
  & \cong &
    \Zmodx{a} \times \Zmodx{b}
  \end{nedqn}

  More generally, if $n = \prod p_i^{k_i}$, then:

  \begin{nedqn}
    \Zmodx{n}
  & \cong &
    \Zmodx{p_1^{k_1}} \times \cdots \times \Zmodx{p_n^{k_n}}
  \end{nedqn}
\end{theorem}

\begin{proof}
  Our theorem about ring isomorphism has this exact implication for the
  multiplicative subgroup.
\end{proof}

\begin{remark}
  To understand the structure of $\Zmodnx$ it be useful to understand
  $\Zmodx{p^k}$. $\Zmodnx$ is just a direct product of these.

  We can explore a little those $\Zmodnx$ that \emph{can} be factorized,
  though.
\end{remark}

\begin{proposition}
  For odd primes $p, q$, $\Zmodx{pq}$ is never cyclic.
\end{proposition}

\begin{proof}
  Note that $\Zmodx{p}, \Zmodx{q}$ have $p-1, q-1$ elements
  respectively. Both $p-1, q-1$ are even, so both are divisible by 2.
  We've previously shown that if a group $G$ can be factored into
  subgroups with cocomposite order, then the group $G$ cannot be cyclic.

  Note that it really doesn't matter whether $\Zmodx{p}, \Zmodx{q}$ are
  cyclic (they are).
\end{proof}

\begin{proposition}
  If $n = 2^k p^m$ with $k_2 > 1$ and $p$ an odd prime, then $\Zmodnx$
  cannot be cyclic.
\end{proposition}

\begin{proof}
  It suffices to show that $\Zmodx{2^k}$ has even order. Knowing that,
  the prior proof will apply. Note that $\vphif{2^k} = 2^{k-1}(2 - 1) =
  2^{k-1}$, so $\Zmodx{2^k}$ has $2^{k-1}$ elements. Since $k>1$, this
  is divisible by 2.
\end{proof}

\begin{remark}
  These propositions can be trivially extended to any $n$ which is
  divisible by two distinct odd prime powers, or by one odd prime power
  and $2^k$ (with $k > 1$).
\end{remark}

\begin{proposition}
  If $n = 2 p^k$, then $\Zmodnx$ is isomorphic to $\Zmodx{p^k}$.
\end{proposition}

\begin{proof}
  This is trivial since $\Zmodx{2}$ has just the identity element.
\end{proof}

\begin{corollary}
  $\Zmodx{2 p^k}$ is cyclic iff $\Zmodx{p^k}$ is cyclic.
\end{corollary}

\begin{proposition}
  $\Zmodx{2}$ and $\Zmodx{4}$ are both cyclic.
\end{proposition}

\begin{proof}
  $\Zmodx{2}$ contains just a single element: 1. It is trivially cyclic.
  $\Zmodx{4}$ contains just two elements: $1, 3$. $3$ must be its own
  inverse. So this is also cyclic.
\end{proof}

\begin{proposition}
  $\Zmodx{2^k}$ is never cyclic for $k>2$.
\end{proposition}

\begin{proof}
  It suffices to show that $\Zmodx{8}$ is not cyclic. It contains the
  elements $\setof{1, 3, 5, 7}$. Note that:

  \begin{nedqn}
    & 3^2 \equiv 5^2 \equiv 7^2 \equiv 1 \pmod{8} &
  \end{nedqn}

  Now consider $\Zmodx{2^k}$. Assume it were cyclic. Then it would have
  a generator $g$ of order $2^{k-1}$. But then note that $g \mod 2^3$ is
  a generator for $\Zmodx{8}$, which we just saw is acyclic.
\end{proof}

\begin{remark}
  To conclude for now, we've seen how to factor $\Zmodnx$ into a direct
  product of $\Zmodx{p_i^{k_i}}$. We've seen that many $\Zmodnx$ are
  acyclic - whenever $\Zmodnx$ can be factored into two disjoint
  subgroups.

  We've seen that $\Zmodx{2^k}$ is cyclic iff $k \in \setof{1, 2}$. We
  haven't yet seen how to factor $\Zmodx{2^k}$ when $k > 2$!

  We haven't examined $\Zmodx{p^k}$ for odd primes $p$. We shall soon
  see that these are always cyclic and thus isomorphic to
  $\Zmod{\vphif{p^k}}$.
\end{remark}

%   We will see that often $\Zmodx{p^k}$ is cyclic. But even when these
%   factors are are cyclic, their direct
% \end{remark}

% \begin{remark}
%   Note that $\Zmodax, \Zmodbx$ may not be cyclic. Even when these are
%   cyclic, note that $\Zmodx{ab}$ may not be. $\Zmodx{ab}$ will never be
%   cyclic when $\order{\Zmodax}$ and $\order{\Zmodbx}$ are not coprime.
% \end{remark}
