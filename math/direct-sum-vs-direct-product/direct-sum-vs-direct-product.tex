\documentclass[11pt, oneside]{amsart}

\usepackage{geometry}
\geometry{letterpaper}

\usepackage{ned-common}
\usepackage{ned-abstract-algebra}

\begin{document}

\title{Direct Sum vs Direct Product}
\maketitle

\begin{definition}
  Given a set of groups $G_1, \ldots G_n$, their \define{direct product}
  is a group defined on the Cartesian product $G_1 \times \cdots \times
  G_n$. This \emph{set} needs a group operation. It is trivial:

  \begin{nedqn}
    \parens{g_1 \times\cdots\times g_n}
    \cdot
    \parens{g'_1 \times\cdots\times g'_n}
  \defeqcol
    \parens{g_1 g'_1 \times\cdots\times g_n g'_n}
  \end{nedqn}

  We may extend this to \emph{infinite} products of groups $G_i$.
\end{definition}

\begin{definition}
  The \define{direct sum} of a set of groups $G_i$ is a \emph{subgroup}
  of the direct product. It consists of only those elements $g$ which
  are a \emph{finite product} of the underlying factors. That is where
  all but \emph{finitely many} $g_i$ are 1.

  We write:

  \begin{nedqn}
    G
  \eqcol
    G_1 \oplus G_2 \oplus \cdots
  \end{nedqn}
\end{definition}

\begin{remark}
  When the set of groups are finite, the direct product and direct sum
  are identical. When the set of groups is countably infinite, the
  direct product consists of all \emph{sequences} of $g_i$, while the
  direct sum consists of all \emph{terminating sequences} of $g_i$.

  Note: it makes no difference whether the underlying groups $G_i$ are
  finite or infinite.
\end{remark}

\begin{remark}
  This is all very analogous to the distinction between the box and
  product topologies.
\end{remark}

\begin{remark}
  Just as you can talk about the direct sum of groups, you can talk
  about the direct sum of rings.
\end{remark}

\begin{remark}
  In the context of a finite number of groups, we might prefer the
  direct sum notation when talking about groups written with
  \emph{additive} notation. We might prefer the direct product notation
  when talking about groups written with \emph{multiplicative} notation.
  None of this matters, because it's only notation.
\end{remark}

\begin{definition}
  There is a (not totally precise) distinction between an
  \define{external direct sum/product} and an \define{internal direct
  sum/product}.

  An external direct sum/product takes two existing objects and builds a new
  one. For instance, we can take two copies of the real line $\R$ and
  form the two dimensional space $\R \times \R$.

  An internal direct sum/product breaks down an existing object into two
  substructures. For instance, consider $\Zmod{6}$. This can be written
  as a direct sum of two (internal) subgroups: $\setof{0, 2, 4},
  \setof{0, 3}$.
\end{definition}

\begin{remark}
  It is suggested that we always use the \define{direct product}
  terminology when talking about \emph{rings}. Why?

  Consider $X$ and $Y$ and a direct sum/product $Z$. If we say $Z = X
  \oplus Y$, then consider the ring homomorphisms determined by:

  \begin{nedqn}
    \varphi_X(1_X)
  \mapstocol
    1 \oplus 0
  \\
  \varphi_Y(1_Y)
  \mapstocol
    0 \oplus 1
  \end{nedqn}

  If these two embeddings mapped $X$ and $Y$ into disjoint subrings,
  then they would be compatible. Closing over summations of $x \oplus 0$
  and $0 \oplus y$ would give a representation for every element of $Z$.

  But these embeddings are \emph{incompatible}. Reason: any element $1
  \oplus 0$ that is a multiplicative identity for the embedding of $X$
  is \emph{also} a multiplicative identity for the embedding of $Y$.
  Thus we must have that $1 \oplus 0 = 0 \oplus 1$. Thus we see that the
  two embeddings do not agree on a representation for $1_Z$.

  Thus the \emph{multiplicative} terminology is preferred. We consider
  ring homomorphisms defined by:

  \begin{nedqn}
    \varphi_X(1_X)
  \mapstocol
    1 \times 1
  \\
  \varphi_Y(1_Y)
  \mapstocol
    1 \times 1
  \end{nedqn}

  Note that this makes it non-trivial to find what elements correspond
  to $x \times 0$ and $0 \times y$! Note that those are \emph{not}
  $0_Z$!
\end{remark}

\begin{remark}
  Ultimately, this terminology preference has something to do with
  category theory, and how a ``product'' of objects is supposed to imply
  something about what embeddings/isomorphisms/homomorphisms exist
  between the parts and the whole. Blah blah blah.
\end{remark}

% Useful sources:
%
% https://www.quora.com/What-are-the-differences-between-a-direct-sum-and-a-direct-product-of-two-groups
%
% https://en.wikipedia.org/wiki/Direct_sum
%
% TODO: Could consider looking into the \emph{tensor} product. Kind of
% weird that the direct sum is written \oplus while the direct sum is
% written merely as \times.

\end{document}
