\subsection{Calculation of $\varmuxp$}

From before, we know that $C = a_1^2 + a_2^2$. Above, when I extracted a
constant of $\half$ from the exponent, I ensured that $C = \rhomuxp$.

That means:

\begin{nedqn}
  \rhomuxp
& = &
  \rhomux
  +
  \rhox
\end{nedqn}

That's pretty cool. It says that the precision increases by $\rhox$.
That makes sense: the more ``precise'' we think samples of $x$ are, the
more information they are telling us. One interesting fact: no matter
how unexpected our sample $x$ is, our posterior precision will always go
up.

We may put this in variance terms:

\begin{nedqn}
  \rhomuxp
& = &
  \rhomux
  +
  \rhox
  \\
  \invf{\varmuxp}
& = &
  \invf{\varmux}
  +
  \invf{\varx}
  \\
  \varmux
  \varx
& = &
  \varmuxp
  \varx
  +
  \varmuxp
  \varmux
  \\
  \varmuxp
& = &
  \frac{
    \varmux
    \varx
  }{
    \varmux + \varx
  }
\end{nedqn}

It is hard for me to give a very clear interpretation of this formula.
But here it is\ldots
