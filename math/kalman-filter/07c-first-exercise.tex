\subsection{First Exercise}

Let's get away from all these symbols and just try to simplify something
easy:

\begin{nedqn}
  \parensq{x - b_1} + \parensq{x - b_2}
\end{nedqn}

It's intuitive that this function is minimized at $x = \otwo{b_1 +
b_2}$, because then:

\begin{nedqn}
  x - b_1
& = &
  \otwo{b_1 + b_2}
  -
  b_1
=
  \otwo{b_2 - b_1}
  \\
  x - b_2
& = &
  \otwo{b_1 + b_2}
  -
  b_2
=
  \otwo{b_1 - b_2}
\end{nedqn}

These have the same magnitude, so the competing objectives to minimize
$\parensq{x - b_1}$ and $\parensq{x - b_2}$ are balanced. Of course, you
may verify this yourself by taking the derivative.

Once we have found the minimum, we know that

\begin{nedqn}
  \parensq{x - b_1}
  +
  \parensq{x - b_2}
& = &
  C
  \parensq{x - \otwo{b_1 + b_2}}
  +
  D
\end{nedqn}

\noindent
for constants $C, D$. We know that $C = 2$, because the two $x^2$ in the
original sum together. What is the $D$ term? It is the ``error'' when we
choose the best value for $x$: $\otwo{b_1 + b_2}$. So of course it
makes sense that this is $D = 2\parensq{\otwo{b_2 - b_1}}$. That's
the sum of the ``errors'' from each side, and the errors are both
$\parensq{\otwo{b_2 - b_1}}$.
