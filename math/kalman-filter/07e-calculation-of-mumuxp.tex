\subsection{Calculation of $\mumuxp$}

With what we've learned, let's look again at:

\begin{nedqn}
  \tcpmuxx
& \sim &
  \nexp{
    -
    \invf{2\varmux}
    \parensq{\mux - \mumux}
    -
    \invf{2\varx}
    \parensq{x - \mux}
  }
\end{nedqn}

We wanted to focus on simplifying

\begin{nedqn}
  \invf{2\varmux}
  \parensq{\mux - \mumux}
  +
  \invf{2\varx}
  \parensq{x - \mux}
\end{nedqn}

Now we have have just the formula to do so! But first, let's replace
$\varmux, \varx$ with their inverses: $\rhomux, \rhox$. These are
sometimes called the \define{precision}. (Great, precision is not
conventionally written square-style like $\var$ is\dots)

\begin{nedqn}
  \invf{2\varmux}
  \parensq{\mux - \mumux}
  +
  \invf{2\varx}
  \parensq{x - \mux}
& = &
  \half
  \parens{
    \rhomux
    \parensq{\mux - \mumux}
    +
    \rhox
    \parensq{\mux - x}
  }
\end{nedqn}

(I've also taken the opportunity to swap $\mux, x$, and extract a factor
of $\half$). We now can simply plug into our previous formula
\ref{weightedsum} to first get:

\begin{nedqn}
  \mumuxp
& \defeq &
  \frac{
    \rhomux \mumux
    +
    \rhox x
  }{
    \rhomux + \rhox
  }
\end{nedqn}

This makes total sense. The greater the precision value $\rhox$, the
more important it is that $\mumuxp$ be closer to $x$. Equivalently, the
greater the precision $\rhox$, the more meaningful a new data point $x$
is.

On the other hand, the greater the precision value of $\rhomux$, the
more ``firm'' we are in our prior belief of $\mumux$. That means we
won't want to change our opinion in $\mumux$ very much for just a single
new datapoint. We'll want more ``evidence.''

We can also state it in terms of $\varmux, \varx$:

\begin{nedqn}
  \mumuxp
& \defeq &
  \frac{
    \rhomux \mumux
    +
    \rhox x
  }{
    \rhomux + \rhox
  }
  \\
& = &
  \frac{
    \invf{\varmux} \mumux
    +
    \invf{\varx} x
  }{
    \invf{\varmux} + \invf{\varx}
  }
  \\
& = &
  \frac{
    \varx \mumux
    +
    \varmux x
  }{
    \varx + \varmux
  }
\end{nedqn}

This last result just tells us the same thing in reverse terms, of
course.
