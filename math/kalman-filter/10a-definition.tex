\subsection{Definition}

Because of all our preparation, the final work on the Kalman filter is
quite simple. The Kalman filter is very similar to a hidden Markov
model, except the conditional probability distributions are linear
Gaussian, rather than discrete.

First, we have an unobserved variable $\vz_t$. The state evolves in the
following simple way:

\begin{nedqn}
  \vz_t
\eqcol
  \mA \vz_{t-1}
  +
  \vepsilon\subvz
  \\
\intertext{where}
  \vepsilon\subvz
\simcol
  \normal{\veczero}{\mQ}
\end{nedqn}

Our insight into the unobserved state is via $\vy_t$. The observation
model is:

\begin{nedqn}
  \vy_t
\eqcol
  \mC \vx_t
  +
  \vepsilon\subvy
\intertext{where}
  \vepsilon\subvy
\simcol
  \normal{\veczero}{\mR}
\end{nedqn}

