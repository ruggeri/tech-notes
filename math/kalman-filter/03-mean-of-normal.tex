\section{The Mean of $\nnormal$}

It is clear that if $\snormal$ has finite mean, it must be $\mu = 0$. We
see this because $\snormal$ is symmetric across the point $x = 0$.
Likewise we can make a translation argument that $\mu$ will always be
the mean even if $\mu \ne 0$.

But let's do the math. I got a lot of practice through this warm-up.

\begin{nedqn}
  \expectation{\nnormal}
& = &
  \int_\reals
    x
    \nnormaleq
    \dx
\end{nedqn}

Because we are integrating over the whole range, we may replace $x$ with
$x \mapsto \sigma y + \mu$.

\begin{nedqn}
  \sigma
  \int_\reals
    \parens{\sigma y + \mu}
    %
    \nnormalc
    %
    \normalexpsq{
      \parens{\sigma y + \mu}
      -
      \mu
    }{2\var}
    \dy
& = &
  \int_\reals
    \parens{\sigma y + \mu}
    %
    \snormalc
    %
    \snormalexp[y]
    \dy
\end{nedqn}

Notice the leading correction coefficient of $\sigma$ we added (and
canceled). When doing this replacement, we are moving $\sigma$ times
faster through the range, so the mean would be scaled by
$\frac{1}{\sigma}$ if we didn't do anything. Or think: $\sigma\dy =
\dx$.

Continuing,

\begin{nedqn}
  \int_\reals
    \parens{\sigma y + \mu}
    %
    \snormalc
    %
    \snormalexp[y]
    \dy
& = &
  \sigma
  \int_\reals
    y
    \snormalc
    \snormalexp[y]
    \dy
  +
  \mu
  \int_\reals
    \snormaleq[y]
    \dy
\end{nedqn}

The first term is $\sigma$ times the expectation of $\snormal$.
Hopefully $\expectation{\snormal} = 0$. The second term is $\mu$
times $\snormal$ integrated over the entire range. That's $\mu$ times
$1$. We proceed:

\begin{nedqn}
  \mu
  +
  \sigma
  \int_\reals
    y
    \snormaleq[y]
    \dy
& = &
  \mu
  +
  \frac{\sigma}{\sqrttwopi}
  \parens{
    -2
    \snormalexp[y]
  }
  \intevalbar{-\infty}{\infty}
  \\
& = &
  \mu
  +
  \frac{\sigma}{\sqrttwopi}
  \parens{
    -2
    \snormalexp[\infty]
    +
    -2
    \snormalexp[(-\infty)]
  }
  \\
& = &
  \mu
  +
  \frac{\sigma}{\sqrttwopi}
  \parens{-2\cdot0 + -2\cdot0}
  \\
& = &
  \mu
\end{nedqn}

Done!
