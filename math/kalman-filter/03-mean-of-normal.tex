\section{The Mean of $\nnormal$}

It is clear that if $\snormal$ has finite mean, it must be $\mu = 0$. We
see this because $\snormal$ is symmetric across the point $x = 0$.
Likewise we can make a translation argument that $\mu$ will always be
the mean.

But let's check the math as a warmup:

\begin{nedqn}
  \int_\reals
  x
  \nnormaleq
  \dx
\end{nedqn}

Because we are integrating over the whole range, we may replace $x$ with
$x \mapsto \sigma y + \mu$.

\begin{nedqn}
  \sigma
  \int_\reals
    \left(
      \sigma y + \mu
    \right)
    %
    \normalc{\var}
    %
    \normalexpsq{
          \sigma y + \mu - \mu
    }{2\var}
    \dy
& = &
  \int_\reals
    \left(
      \sigma y + \mu
    \right)
    %
    \snormalc
    %
    \snormalexp[y]
    \dy
\end{nedqn}

Notice the leading correction coefficient of $\sigma$ we added (and
canceled). When doing this replacement, we are moving $\sigma$ times
faster through the range, so the mean would be scaled by
$\frac{1}{\sigma}$ if we didn't do anything. Or think: $\sigma\dy =
\dx$.

\begin{nedqn}
  \int_\reals
    \left(
      \sigma y + \mu
    \right)
    %
    \snormalc
    %
    \snormalexp[y]
    \dy
& = &
  \int_\reals
    \sigma y
    \snormalc
    \snormalexp[y]
    \dy
  +
  \mu
  \int_\reals
    \snormaleq[y]
    \dy
\end{nedqn}

Note the second term is just $\mu$ times $\snormal$ integrated over the
entire range. That's $\mu$ times $1$. We proceed:

\begin{nedqn}
  \mu
  +
  \int_\reals
    \sigma y
    \snormaleq[y]
    \dy
& = &
  \mu
  +
  \frac{\sigma}{\sqrttwopi}
  \left(
    -\half
    \snormalexp[y]
  \right)
  \intevalbar{-\infty}{\infty}
  \\
& = &
  \mu
  +
  \frac{\sigma}{\sqrttwopi}
  \left(
    -\half
    \snormalexp[\infty]
    +
    -\half
    \snormalexp[(-\infty)]
  \right)
  \\
& = &
  \mu
  +
  \frac{\sigma}{\sqrttwopi}
  \left(0 + 0\right)
  \\
& = &
  \mu
\end{nedqn}

Done!
