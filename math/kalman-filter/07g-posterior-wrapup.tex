\subsection{Posterior Wrap-up}

Let's wrap-up and see where we're at. Maybe let's also remember how we
got here.

\begin{nedqn}
  \cpmuxx
& = &
  \frac{
    \pmux \cpxmux
  }{\px}
  \\
& \sim &
  \pmux \cpxmux
  \\
& = &
  \parens{
    \normalc{\varmux}
    \normalc{\varx}
  }
  \nexp{
    -
    \invf{2\varmux}
    \parensq{\mux - \mumux}
    -
    \invf{2\varx}
    \parensq{x - \mux}
  }
  \\
& \sim &
  \nexp{
    -
    \invf{2\varmux}
    \parensq{\mux - \mumux}
    -
    \invf{2\varx}
    \parensq{x - \mux}
  }
  \\
& = &
  \nexp{
    -\half
    \parens{
      \rhomux
      \parensq{\mux - \mumux}
      +
      \rhox
      \parensq{\mux - x}
    }
  }
  \\
& = &
  \nexp{
    -\half
    \parens{
      \rhomuxp
      \parensq{
        \mux
        -
        \mumuxp
      }
      +
      D
    }
  }
  \\
& \sim &
  \nexp{
    -\half
    \rhomuxp
    \parensq{
      \mux
      -
      \mumuxp
    }
  }
  \\
& = &
  \nexpf{
    \parensq{
      \mux
      -
      \mumuxp
    }
  }{
    2 \varmuxp
  }
\end{nedqn}

This final form is indeed the form of an improperly normalized gaussian
with mean $\mumuxp$ and variance $\varmuxp$. I argue that the constants
I drop must have been exactly the missing $\normalc{\varmuxp}$ I need to
normalize my normal distribution.

Here's why.

\begin{nedqn}
  \cpmuxx
& = &
  A
  \nexpf{
    \parensq{
      \mux
      -
      \mumuxp
    }
  }{
    2 \varmuxp
  }
  \nedcomment{for some $A$}
\\
  \cpmuxx
& = &
  A
  \sqrt{\twopi \varmuxp}
  \normalc{\varmuxp}
  \nexpf{
    \parensq{
      \mux
      -
      \mumuxp
    }
  }{
    2 \varmuxp
  }
\\
  \cpmuxx
& = &
  A
  \sqrt{\twopi \varmuxp}
  \normal{\mumuxp}{\varmuxp}
\\
  \intR
    \cpmuxx \diff{\mux}
& = &
  A
  \sqrt{\twopi \varmuxp}
  \intR
    \normal{\mumuxp}{\varmuxp}
    \diff{\mux}
\\
  1
& = &
  A
  \sqrt{\twopi \varmuxp}
  \cdot
  1
\\
  A
& = &
  \normalc{\varmuxp}
\end{nedqn}

Boom. Done and done. $A$ has to be just the right normalizing value.

In conclusion, we have calculated the posterior:
$\normal{\mumuxp}{\varmuxp}$. Because the posterior has the same form as
the prior distribution, we say they are \define{conjugate}. In the same
way we calculated $\mumuxp, \varmuxp$ from the old values when we
observed a datapoint $x$, we can do just the same with a next datapoint.
We'll just use our updated prior parameters.

We know the posterior has a maximum at $\mumuxp$, which means that
$\mumuxp$ is our \define{maximum a posteriori} estimate of the true
value of $\mu$.

\subsubsection{MAP estimate of $\varmux$?}

No. We have a prior probability distribution over the joint space of
$\mux, x$. That distribution is parameterized by $\mumux, \varmux,
\varx$.

When we calculate the posterior distribution $\cpmuxx$, we are
describing it in terms of new parameters: $\mumuxp, \varmuxp$. But those
parameters are exact values. They are exactly the values for the
posterior distribution $\cpmuxx$.

(Of course you can start to introduce meta-parameters into your prior
distribution\ldots But the point is that you'll always have top-level
parameters that you don't put a distribution on.)

We treat $\varx$ as fixed. Could we also treat it as unknown? Yes, but
then we'd have to consider a joint distribution of $\mux, \varx, x$. And
indeed that can absolutely be done. Murphy has a long paper where he
goes through the work of showing you the inverse normal gamma something
something.

For our purposes, we may assume that $\varx$ was established in the
laboratory. We now wish to use it in the field where $\mu$ is unknown.

\subsubsection{Final note: pseudocounts...}

I will leave with a thought that relates the MAP estimate and the
maximum likelihood estimate. We know that the maximum likelihood
estimate is just the sample mean of our observed data points $x_i$.

You can show that if you start with a prior of $\mumux = x_1$, $\varmux
= 1$, assume that $\varx = 1$, and then one-by-one updated with data
points $x_2, \ldots, x_n$, your MAP estimate will come out the same as
your maximum likelihood estimate. Similarly, even if you set the initial
$\mumux = 0$, you can see that in the limit as your initial $\varmux \to
0$, MAP estimation becomes equivalent to maximum likelihood estimation.
In fact, the posterior becomes the Dirac delta at the MLE estimate.

Alternatively, consider the method of \define{pseudocounts} where you
bias a maximum likelihood estimate by ``pretending'' you observed
$\alpha$ trials with mean result $\beta$. The ``principled'' version of
this hack in the Bayesian world would be to set the initial $\varmux =
\alpha\invx$ and the initial $\mumux = \beta$.
