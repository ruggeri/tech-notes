\section{The Gaussian Integral}

Before we attack the normal distribution, we will need to find an
important related integral: the \define{Gaussian Integral}.

\begin{nedqn}
  \gaussianint
\end{nedqn}

We will use a trick. Let's consider the \emph{square} of this function:

\begin{nedqn}
  \left(
    \gaussianint
  \right)^2
& = &
  \left(
    \gaussianint
  \right)
  \left(
    \gaussianint[y]
  \right)
  \\
& = &
  \iint_{\reals, \reals} \nedexp{-\left(x^2 + y^2\right)} \dx\dy
  \nedcomment{Fubini}\nednumspace\nednumber
\end{nedqn}

Some technical assumptions should be shown to justify the use of
Fubini's theorem, but we needn't go that deep.

We may now perform a change of variables:

\begin{nedqn}
  \iint_{\reals, \reals} \nedexp{-\left(x^2 + y^2\right)} \dx\dy
& = &
  \nediint{_0^{\infty}}{_0^{2\pi}}
    r
    \nedexp{
      -r^2
      \left(
        \cos^2 \theta
        +
        \sin^2 \theta
      \right)
    }
    \dtheta
    \dr
  \\
& = &
  \nediint{_0^{\infty}}{_0^{2\pi}}
    r
    \gaussianexps[r]
    \dtheta
    \dr
  \nednumber
\end{nedqn}

We're doing things in polar coordinates now. Notice that very important
factor of $r$. That accounts for the fact that the rate at which
$\theta$ swings through values of $(x, y)$ is exactly the length of the
radius it is sweeping through.

We again rely on Fubini to reverse the order of integration and
continue:

\begin{nedqn}
  \nediint{_0^{\infty}}{_0^{2\pi}}
    r
    \gaussianexps[r]
    \dtheta
    \dr
& = &
  \nediint{_0^{2\pi}}{_0^{\infty}}
    r
    \gaussianexps[r]
    \dr
    \dtheta
  \\
& = &
  2\pi
  \int_0^{\infty}
    r
    \gaussianexps[r]
    \dr
  \\
& = &
  2\pi
  \left(
    -\half
    \gaussianexps[r]
  \right)
  \intevalbar{0}{\infty}
  \nedcomment{chain rule}
  \\
& = &
  2\pi
  \left(
    -\half
    \gaussianexps[
      \left(-\infty\right)
    ]
    +
    \half
    \gaussianexps[0]
  \right)
  \intevalbar{0}{\infty}
  \\
& = &
  2\pi
  \left(
    0 + \half
  \right)
  \\
& = &
  \pi
  \nednumber
\end{nedqn}

We now know:

\begin{nedqn}
  \left(
    \gaussianint
  \right)^2
& = &
  \pi
  %
  \\
  %
\intertext{so we may conclude that}
  %
  \gaussianint
  %
& = &
  \sqrttwopi
  \nednumber
\end{nedqn}

Done!
