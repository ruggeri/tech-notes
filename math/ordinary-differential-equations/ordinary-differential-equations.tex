\documentclass[11pt, oneside]{amsart}

\usepackage{geometry}
\geometry{letterpaper}

\usepackage{ned-common}
\usepackage{ned-linear-algebra}

\begin{document}

\title{Ordinary Differential Equations}
\maketitle

These notes are based on the book by Tenenbaum and Pollard.

\section{Basic Concepts}

\begin{enumerate}
  \item They define \define{set} and \define{element}. They define
  \define{interval}

  \item They define what it means to be a function of \define{one
  independent variable}. In this case, the value of the the function is
  uniquely determined by the value of the independent variable. The
  independent variable can take on any value in its \define{domain}.

  \item Likewise a function of two independent variables means that
  there is a uniquely determined function value for each pair of input
  values. The inputs can take on values throughout their domains. The
  independent variables are independent in the sense that their value is
  in no way restricted by the selection of the other variable.

  \item They define a \define{region} as a set in the plane which is
  open and connected. That means that (1) for every point $x$ in a
  region $R$, there is a circle which contains $x$ and is a subset of
  $R$. Also, for any $x, y \in R$, there exists a (continuous) path from
  $x$ to $y$ which is contained in $R$.

  \item They define a region to be \define{bounded} if a circle contains
  it.

  \item They say that an equation of $x, y$ defines $y$ as an
  \define{implicit function} of $x$ if for each value of $x$ there is a
  value of $y$ which satisfies the equation. In particular, they intend
  the equation to have the form:

  \begin{nedqn}
    h(x, y) = 0
  \end{nedqn}

  \item It seems clear that they do not require that there be a
  \define{unique} function mapping $x$ values to $y$. For instance, they
  would say that $y^2 - x = 0$ implicitly defines $y$ as a function of
  $x$, even though there are many such functions of $x$ that would
  satisfy this equation.

  \item This seems odd because it doesn't really ``define'' a functional
  relationship between $y$ and $x$ in a unique way. I would rather say
  that a \define{set of solutions} $\setof{f \mid h(x, f(x)) = 0}$ is
  defined by $h(x, y) = 0$. Note that the only way for $h$ \emph{not} to
  define a set of solutions (or rather, to define a nullset of
  solutions) is if there exists some $x$ in the domain such that no $y$
  can satisfy the constraint.

  \item They define \define{elementary functions} to be powers of $x$,
  roots of $x$, $e^x$, $\log x$, trigonometric functions of $x$, inverse
  trigonometric functions. This is then augmented by composition (i.e.,
  chaining) of the preceding functions. Finally, they augment this via
  finite combination via addition, subtraction, multiplication and
  division.

  \item They probably forgot constant functions. And they probably meant
  to actually include functions like $\sqrt{x + x^2}$. They were a
  little sloppy here.
\end{enumerate}

\end{document}
