\documentclass[11pt, oneside]{amsart}

\usepackage{geometry}
\geometry{letterpaper}

\usepackage{ned-common}
\usepackage{ned-calculus}

\begin{document}

\title{Chaos}
\maketitle

\begin{enumerate}
  \item Basically, I think we just want to give a definition of what it
  means to be \define{chaotic}.

  \item People talk about ``sensitive dependence'' on ``initial
  conditions.'' But if the function that maps from initial condition to
  output is required to be continuous and even differentiable, does that
  just mean that the derivative should be large?

  \item I think an important idea is that every open set in the space of
  initial conditions (no matter how small) maps to \emph{the entire}
  output space. In this sense, an approximation of the input will never
  help approximate the output.

  \item One possible idea: for every $x_0$ there exists a $\delta_0$
  such that for any $\delta < \delta_0$ the image of the $\delta$ ball
  around $x_0$ is the same as the image of the $\delta_0$ ball.

  \item If a physical system is modeled by a chaotic function, then
  arbitrarily small measurement error in the input implies that nothing
  can be predicted about the output of a system.
\end{enumerate}

\end{document}
