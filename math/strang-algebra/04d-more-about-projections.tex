We talked about how to project onto an orthonormal basis $\va_i$. This
was just $\mA\mAt$.

It was important that every pair of vectors are orthogonal. Otherwise,
we know the decomposition into the basis vectors won't quite work,
because of double counting.

So, let's say you have a subspace of $\R^n$. If you have an orthonormal
basis of $n$ vectors, this must span the entire space. So the projection
into this space is the identity. Therefore, $\mA\mAt=\mI$. This harkens
back to our discussion of orthogonal matrices, where we already saw this
very same property that $\mAinv=\mAt$.

So I am next interested in cases where there are `k<n` dimensions to the
subspace. In that case, the subspace is spanned by a basis of $k$
vectors. That means that $\mA$ has $k$ columns, but still has $n$ rows,
because the subspace is embedded in $\R^n$.

Likewise, we know that $\mAt$ projects a vector onto each of $k$ basis
vectors for the subspace. That means there are $k$ rows, but $n$
columns, because this is a projection down from an $n$-dimensional space
to a $k$-dimensional space.

The important thing to note is that $\mA\mAt$ is \emph{square}. That's
because this maps vectors in $\R^n$ to vectors in the $k$-dimensional
subspace embedded in $\R^n$, which still consists of $n$-tuples.

Again, $\mA\mAt$ will not be invertible, because $\mAt$ projects down
into a lower dimensionality space (doesn't have full row rank), which
immediately makes this not injective. Likewise, $\mA$ is not surjective,
because it embeds a lower-dimensionality space into a higher one. That's
because $\mA$ doesn't have full column-rank.
