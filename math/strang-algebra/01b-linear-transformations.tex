\section{Linear Transformations/Matrices}

\define{Matrices} map linear combinations to linear combinations. Such a
map is called a \define{linear transformation}. The first column is what
the first basis vector maps to, the second basis vector maps to the
second column, etc.

The product of a matrix with a vector is the result of applying the
transformation to the vector. I like to see this as a weighted sum of
the columns.

We can \define{compose} transformations. This involves matrix
multiplication. It is quite simple. Given $\mB\mA$, take the first
column of $\mA$; that is what $\ve_1$ maps to under $\mA$. Then apply
$\mB$ to this column. This is now equal to what ought to be the first
column of the product $\mA\mB$. This suggestions a method of
calculation.

Note to self: matrix multiplication involves $\mathcal{O}(n^3)$ time.

Another common way to apply a matrix to a vector is this: take the dot
product of each row in the matrix with the vector. What is the intuition
behind this? I suppose you could say this: the row is vector which is
"most" transformed to the $i$-th basis vector. This is because you can
treat a single row in the matrix as a linear functional, and the
gradient defines the direction of steepest ascent.
