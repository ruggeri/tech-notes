\subsection{\texorpdfstring{$\mL\mU$}{LU} Decomposition}

Performing "half" of this elimination can be seen to build the $\mL\mU$
decomposition. Here, what we do is we start with $\mI\mA$. Then, when we
add a row, we do the row operation on $\mA$, but the opposite operation
on $\mI$. Here, we \emph{only} try to eliminate in $\mA$ below the
diagonal. This results in transforming $\mA$ to an upper triangular
matrix, while $\mI$ becomes lower triangular.

Note that $\mA = \mL\mU$ has $\mL$ with 1s along the diagonal, while
$\mU$ has non-one diagonal. Sometimes we therefore factor to
$\mL\mD\mU$. $\mL$ is the same as before, but rows of $\mU$ are scaled
so that diagonal is one. The scaling is performed by $\mD$, which is a
diagonal matrix with just the scaling values.

Notice that because we may have needed to do pivots, in order to
decompose any matrix we may have to decompose to $\mP\mA = \mL\mU$ or
$\mP\mA=\mL\mD\mU$.
