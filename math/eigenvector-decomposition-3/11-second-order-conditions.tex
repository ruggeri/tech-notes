\section{Second Order Conditions}

To check out what is happening we should investigate the second order
conditions. In particular: we know that $f$ has a maximum, for the same
reason it has a minimum: the extreme value theorem!

So let's go ahead and examine the second partial.

\begin{nedqn}
  % Second derivative
  \fpthetax \left[
    \left( \mtxA'_{i, j}(\theta) \right)^2
  \right]
& = &
  % Derivative of derivative
  \fptheta \biggl[
    \fptheta \left[
      \left( \mtxA'_{i, j}(\theta) \right)^2
    \right]
  \biggl]
  \\
& = &
  % Chain rule.
  \fptheta \biggl[
    2
    \mtxA'_{i, j}(\theta)
    \fptheta \left[
      \mtxA'_{i, j}(\theta)
    \right]
  \biggl]
  \nedcomment{chain rule}%
  \\
& = &
  % Product rule
  2
  \biggl[
    % Squared first derivative
    \left(
      \fptheta \left[
        \mtxA'_{i, j}(\theta)
        \right]
    \right)^2
    +
    % Present value times second derivative
    \mtxA'_{i, j}(\theta)
    \fpthetax \left[
      \mtxA'_{i, j}(\theta)
    \right]
  \biggl]
  \nedcommenthard{product rule}%
  \\
& = &
  2 \Big[
    \mtxA'_{i, j}(\theta)
    \fpthetax \left[
      \mtxA'_{i, j}(\theta)
    \right]
  \Big]
  \nednumber\label{second:derivative:squared}%
\end{nedqn}

The last step is justified because we wouldn't bother with the second
order test unless $\fptheta \left[ \mtxA'_{i, j}(\theta) \right] = 0$.
We've already calculated $\mtxA'_{i, j}(\theta)$ previously, so let's
focus on the second derivative: $\fpthetax \left[ \mtxA'_{i, j}(\theta)
\right]$.

\begin{nedqn}
  % Second derivative.
  \fpthetax \left[
    \mtxA'_{i, j}(\theta)
  \right]
& = &
  % Derivative of derivative.
  \fptheta \Big[
    \fptheta \left[
      \mtxA'_{i, j}(\theta)
    \right]
  \Big]
  \\
& = &
  \fptheta \Big[
    {
      (\cos^2\theta - \sin^2\theta)
      (\mtxA_{j, j} - \mtxA_{i, i})
    }
    -
    {
      4
      \cos\theta
      \sin\theta
      \mtxA_{i, j}
    }
  \Big]
  \nedcommenthard{formula \ref{partial:i:j}}%
  \\
& = &
  \fptheta \Big[
    -4
    \cos\theta
    \sin\theta
    \mtxA_{i, j}
  \Big]
  \nedcomment{by presumption $\mtxA_{i, i} = \mtxA_{j, j}$}%
  \nednumspace%
  \nednumber%
\end{nedqn}

Notice that I've used our presumption that $\mtxA_{i, i} = \mtxA_{j,
j}$, since otherwise the first-order condition would have already
ensured that an off-diagonal partial is zero. We continue:

\begin{nedqn}
  \fpthetax \left[
    \mtxA'_{i, j}(\theta)
  \right]
&=
  \fptheta \Big[
    -4
    \cos\theta
    \sin\theta
    \mtxA_{i, j}
  \Big]
  \\
&=
  -4
  \left(
    \cos^2\theta - \sin^2\theta
  \right)
  \mtxA_{i, j}
  \nednumber%
  \nednumber\label{second:derivative}%
\end{nedqn}

And we may now plug formula \ref{second:derivative} into formula
\ref{second:derivative:squared}.

\begin{nedqn}
  \fpthetax \left[
    \left(
      \mtxA'_{i, j}(\theta)
    \right)^2
  \right]
& = &
  2
  \Big[
    \mtxA'_{i, j}(\theta)
    \fpthetax \left[
      \mtxA'_{i, j}(\theta)
    \right]
  \Big]
  \\
& = &
  2
  \mtxA'_{i, j}(\theta)
  \Big(
    -4
    \left(
      \cos^2\theta - \sin^2\theta
    \right)
    \mtxA_{i, j}
  \Big)
  \\
& = &
  -8 \mtxA'_{i, j}(\theta)
  \left(
    \cos^2\theta - \sin^2\theta
  \right)
  \mtxA_{i, j}
  \nednumber%
\end{nedqn}

There is nothing left to do but evaluate at $\theta = 0$:

\begin{nedqn}
  \fpthetax \left[
    \left(
      \mtxA'_{i, j}(0)
    \right)^2
  \right]
& = &
  -8
  \left(
    \mtxA'_{i, j}(0)
  \right)
  \left(
    \cos^2 0 - \sin^2 0
  \right)
  \mtxA_{i, j}
  \\
& = &
  -8
  \mtxA_{i, j}
  \mtxA_{i, j}
  \\
& = &
  -8
  \left(
    \mtxA_{i, j}
  \right)^2
  \nednumber
\end{nedqn}

And there you have it. This second derivative must always be negative,
since it is $-8$ (always negative) times $\left( \mtxA_{i, j} \right)^2$
(always positive). Note that we assumed $\mtxA_{i, j} \ne 0$, otherwise
there was no reason to go down this road. That saves us from any further
derivative testing.
