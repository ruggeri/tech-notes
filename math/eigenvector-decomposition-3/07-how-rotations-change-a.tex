\section{How Rotations Change $\mtxA$}

We also have that $f$ is differentiable. Thus we know that if there is a
minimum for $f$ at $\mtxA'$, we must have that all derivatives here are
zero. Let's consider the simplest rotation between two basis vectors
$\bvec{i}, \bvec{j}$:

\begin{nedqn}
  \mtxQ
& \defeq &
  \bvec{1}\bvec{1}\tran
  %
  + \ldots
  %
  + \left(
    \cos\theta \bvec{i} + \sin\theta \vec{j}
  \right)
  \bvec{i}\tran
  \\
&&
  \phantom{\bvec{1}}
  + \ldots
  %
  + \left(
    -\sin\theta \bvec{i} + \cos\theta \vec{j}
  \right)
  \bvec{j}\tran
  %
  + \ldots
  %
  + \bvec{n}\bvec{n}\tran
\end{nedqn}

I would like to know what happens to $f(\mtxQtAQ)$ as we change $\theta$
near zero. Let's first study $\mtxA'_{k_1, k_2}(\theta)$ where $k_1, k_2
\not\in \setof{i, j}$:

\begin{nedqn}
  \mtxA'_{k_1, k_2}(\theta)
& = &
  \bvec{k_1}\tran (\mtxQtAQ) \bvec{k_2}
  \\
& = &
  \bvec{k_1}\tran \mtxA \bvec{k_2}
  \\
& = &
  \mtxA_{k_1, k_2}
  \nednumber\label{change:k1:k2}%
\end{nedqn}

As we see, the only changed elements should be in rows (and columns) $i$
and $j$. Let's next consider what happens to $\mtxA'_{k, i}(\theta)$
where $k \not\in \setof{i, j}$.

\begin{nedqn}
  \mtxA'_{k, i}(\theta)
& = &
  \bvec{k}\tran (\mtxQtAQ) \bvec{i}
  \\
& = &
  \bvec{k}\tran \mtxA \left(
    \cos\theta \bvec{i} + \sin\theta \bvec{j}
  \right)
  \\
& = &
  \cos\theta \mtxA_{k, i} + \sin\theta \mtxA_{k, j}
  \nednumber
  \label{change:k:i}
\end{nedqn}

\textbf{TODO}: we really ought to prove that rotation preserves symmetry?

Well, that makes sense! Naturally, because $\mtxA'(\theta)$ is
symmetric, we know that $\mtxA'_{i, k}(\theta) = \mtxA'_{k, i}(\theta)$.
Let's next ask about $\mtxA'_{k, j}(\theta)$. Here we go:

\begin{nedqn}
  \mtxA'_{k, j}(\theta)
& = &
  \bvec{k}\tran (\mtxQtAQ) \bvec{j}
  \\
& = &
  \bvec{k}\tran \mtxA \left(
    -\sin\theta \bvec{i} + \cos\theta \bvec{j}
  \right)
  \\
& = &
  -\sin\theta \mtxA_{k, i} + \cos\theta \mtxA_{k, j}
  \nednumber\label{change:k:j}%
\end{nedqn}

Naturally, this is just the ``opposite'' of what is done for $\mtxA'_{k,
i}$. Again, we know $\mtxA'_{k, j} = \mtxA'_{j, k}$. Last, we have for
$\mtxA'_{i, j}(\theta)$:

\begin{nedqn}
  \mtxA'_{i, j}(\theta)
& = &
  \bvec{i}\tran (\mtxQtDQ) \bvec{j}
  \\
& = &
  \left(
    \cos\theta\bvec{i} + \sin\theta\bvec{j}
  \right)\tran
  %
  \mtxA
  %
  \left(
    -\sin\theta\bvec{i} + \cos\theta\bvec{j}
  \right)
  \\
& = &
  - \sin\theta \cos\theta \mtxA_{i, i}
  + \cos^2\theta \mtxA_{i, j}
  - \sin^2\theta \mtxA_{j, i}
  + \sin\theta \cos\theta \mtxA_{j, j}
  \\
& = &
  \left(
    \sin\theta \cos\theta
  \right)
  \left(
    \mtxA_{j, j} - \mtxA_{i, i}
  \right)
  +
  \left(
    \cos^2\theta - \sin^2\theta
  \right)
  \mtxA_{i, j}
  \nednumber\label{change:i:j}%
\end{nedqn}

Note: we don't consider $\mtxA'_{i, i}$ or $\mtxA'_{j, j}$ because
those are on diagonal.
