\documentclass[11pt, oneside]{amsart}

\usepackage{geometry}
\geometry{letterpaper}

\usepackage{ruggeri_common}
\usepackage{ruggeri_linear_algebra}

\begin{document}

\title{Every Real Symmetric Matrix Is Diagonalizable}
\maketitle

\section{abstract}

This proof tries to correct my original ``proof.'' JRN pointed out the
error in my proof to me and gave me an alternative one. His proof uses
the extreme value theorem for a continuous function defined on a compact
domain. His proof takes a slightly different approach than my original,
though. This document is an attempt to import his idea (using the
extreme value theorem) to correct my original version, while staying
true to the intuition/idea I had been pursuing.

\section{Proposition}

\begin{theorem}[Finite Dimensional Spectral Theorem]
  For any full-rank real symmetric matrix $\mtxA$, there exists an
  orthogonal (aka ``rotation'') matrix $\mtxQ$ and a diagonal (aka
  ``stretching'') matrix $\mtxD$ such that:

  \begin{nedqn}
    \mtxA
  & = &
    \mtxQDQt
  \end{nedqn}

  Equivalently: any full-rank real symmetric matrix $\mtxA$ with
  dimension $n$ possesses $n$ orthogonal (unit) eigenvectors $\setof{
  \vecu[1], \ldots \vecu[n] }$. That is, for suitable $\lambda_i$:

  \begin{nedqn}
    \mtxA\vecu[i]
  & = &
    \lambda_i \vecu[i]
  \end{nedqn}

  Note that if we have $\mtxQ$, we can obtain a set $\vecu[i]$ by
  extracting the columns. Alternatively, given $\vecu[i]$ we can
  produce $\mtxQ$ by writing them in as columns.
\end{theorem}

\section{Original Attempt}

My original ``proof'' was inductive. I first proved a base case: that a
full-rank symmetric matrix in $\mtxspace{2}$ has two orthogonal
eigenvectors. I next showed that, if we find even a single eigenvector
for $\mtxA \in \mtxspace{n}$, we can reduce the problem to finding $n-1$
orthogonal eigenvectors for a corresponding full-rank symmetric of
$\mtxspace{(n-1)}$. So far, this is correct.

So the problem then becomes to find even a single eigenvector of
$\mtxA$. I suggested a process which doesn't quite work. I recall it
here.

Our goal will be to find a rotation of basis such that $\mtxA' \defeq
\mtxQtAQ$ has blank first row/column (excepting $\mtxA'_{1, 1}$, which
cannot also be zero, by assumption that $\mtxA$ is full-rank).

First, find a rotation of basis vectors $\bvec{1}, \bvec{2}$ such that
we have a zero at $\mtxA'_{2, 1}$. I tacitly assume here that $\mtxA_{2,
2} \ne 0$.

Having done this, I want to repeat. I want to find a rotation of
$\bvec{1}, \bvec{3}$ to zero out $\mtxA'_{3, 1}$. Again, assuming that
$\mtxA_{3, 3} \ne 0$, this can be done.

The mistake was here though. By zeroing out $\mtxA'_{3, 1}$, we may
disrupt the zeroing out that was previously performed for $\mtxA'_{2,
1}$. That happens exactly if $\mtxA'_{2, 3} \ne 0$.

Ideally this wouldn't happen. If it didn't, and we could find a rotated
basis in which $\mtxA'_{k, 1} = 0, \forall k\ne 1$, then we can invoke a
property that, in any rotated basis, a symmetric matrix stays symmetric.
That is, we'd also know that $\mtxA'_{1, k} = 0, \forall k\ne 1$.

That would tell us that (1) $\bvec{1}$ is an eigenvector of $\mtxA'$,
and (2) any eigenvector $\vecu'$ of $\mtxA'_{2:n, 2:n}$ is also an
eigenvector of $\mtxA'$ (when you extend $\vecu'$ by prefixing with a
zero for the first coordinate).

As near as I can tell, nothing is wrong with the proof, except that the
elimination procedure I describe doesn't work.

\section{Contrast With Gaussian Elimination}

Let's contrast with Gaussian elimination. In Gaussian elimination, our
first task is to eliminate the off-diagonal entries of the first row.
One-by-one we eliminate $\mtxA_{1, k}, \forall k \ne 1$. For each $k \ne
1$, we subtract an appropriate amount of the first column $\mtxA_{:, 1}$
from every other column $\mtxA_{:, k}$.

Each \define{elementary (column) operation} corresponds to a shearing
operation, and shearing preserves determinant. That's why Gaussian
elimination is useful for calculation of the determinant.

If you repeat this row-by-row, you can turn the matrix into an
upper-triangular matrix. Collecting the operations that undo this gives
you the $\mtxLU$ decomposition of $\mtxA$. If one scales the
rows/columns of $\mtxL, \mtxU$ appropriately so that the diagonal
entries are all $1$, collecting these factors as $\mtxD$, then we get
the $\mtxLDU$ decomposition. The determinant of $\mtxA$ is simply the
determinant of $\mtxD$. That is: the product of the diagonal entries of
$\mtxD$.

Of course, the $\mtxLDU$ decomposition doesn't give us information about
the eigenvectors, because none of these matrices corresponds to a
representation of $\mtxA$ in a rotated basis.

There is a very similar approach that gives a different decomposition.
Rather than trying to clear out above-diagonal entries in a row
one-by-one, let's try to ``orthonormalize'' the columns of $\mtxA$.
Here, we start with $\mtxA' \defeq \mtxA$ and one-by-one reassign:

\begin{nedqn}
  \mtxA'_{:, k}
& \minuseq &
  \innerprod{
    \mtxA'_{:, 1}
  }{
    \mtxA'_{:,k}
  }
  \mtxA'_{:, 1}
\end{nedqn}

Such an approach gives us the $\mtxQR$ decomposition of $\mtxA$. Note:
we should also normalize the columns of $\mtxQ$, combining the extracted
diagonal matrix with $\mtxR$. Again, the determinant may be calculated
by multiplying the diagonal entries of $\mtxR$.

As with the $\mtxLDU$ decomposition before, the $\mtxQR$ doesn't give us
access to the eigenvectors.

\section{Exact Algorithm For Eigenvectors?}

Each column operation in the $\mtxLDU$ algorithm ``fixes'' (that is,
clears out) exactly one position in $\mtxA$. Once fixed, no subsequent
step disturbs that position.

Likewise, each column operation in the $\mtxQR$ algorithm fixes one
column in $\mtxA$. That is, we make a column $\mtxA_{:, i}$ orthogonal
to another column $\mtxA_{:, j}$. Again, no subsequent step disturbs
this.

Because each step moves us one ``unit'' of the way toward the final
result, the runtime of these algorithms is a function of the number of
``units'': that is, the number of elements (or columns) in $\mtxA$.

Our problem for zeroing-out the first column of $\mtxA$ with rotations
is this. If we try to use rotating operations instead of shearing
operations, subsequent operations will always disturb prior operations.

Let's look at a rotated basis which rotates only $\bvec{i}, \bvec{j}$;
other basis vectors are left alone. But the problem is that $\mtxA'$,
the new representation of $\mtxA$, will see
\emph{both} the columns $\mtxA'_{:, i}, \mtxA'_{:, j}$ \emph{and}
the corresponding rows $\mtxA'_{i, :}, \mtxA'_{j, :}$ changed.

How can we proceed then? One possibility is to explore an iterative
procedure that \emph{converges} to the solution, even if it never
terminates. In practice we'll need something like that. Because
compactness, an iterative procedure should also imply the existence of
an exact solution.

A second possibility (which I will pursue here), is to merely prove the
\emph{existence} of a solution. But, rather than be too abstract, I
will do this by showing that there always exists a series of steps that
monotonically ``improve'' your matrix $\mtxA$. I won't show exactly to
find those steps, but my approach is suggestive of an iterative
procedure.

\section{The Corrected Proof: Optimization Goal}

At last, let us begin. We will cast our search for a diagonal matrix
$\mtxA' \defeq \mtxQtDQ$ as an optimization problem. Define:

\begin{nedqn}
  f(\mtxA')
& = &
  \sum_{i \ne j} \left(A_{i, j}\right)^2
\end{nedqn}

That is, $f(\mtxA')$ is the sum of squares of off-diagonal elements. Of
course, $f(\mtxA') = 0$ iff $\mtxA'$ is diagonal. So $f$ is a measure of
``badness'' that I want to minimize. To show that $\mtxA$ is always
diagonalizable is to show that $f$ achieves a minimum of exactly zero on
the space of rotated representations $\mtxA'$.

Instead of writing $f$ as a function of $\mtxA'$, we could write $f$ as
a function of $\mtxQ$. Or even less redundantly, as a function of
$\frac{n(n-1)}{2}$ angles $\theta_{i, j}, \forall i \ne j$.

This last view helps: the space of $\theta_{i, j}$ is really just points
on the surface of a unit hypersphere, which is a compact set. (TODO: JRN
informs me this intuition is not quite correct.) We can apply the
extreme value theorem since $f$ is continuous. That is: $f$ achieves a
minimum somewhere.

The only question remains: is $\min_{\mtxA'} f(\mtxA') = 0$?

\section{How Rotations Change $\mtxA$}

We also have that $f$ is differentiable. Thus we know that if there is a
minimum for $f$ at $\mtxA'$, we must have that all derivatives here are
zero. Let's consider the simplest rotation between two basis vectors
$\bvec{i}, \bvec{j}$:

\begin{nedqn}
  \mtxQ
& \defeq &
  \bvec{1}\bvec{1}\tran
  %
  + \ldots
  %
  + \left(
    \cos\theta \bvec{i} + \sin\theta \vec{j}
  \right)
  \bvec{i}\tran
  \\
&&
  \phantom{\bvec{1}}
  + \ldots
  %
  + \left(
    -\sin\theta \bvec{i} + \cos\theta \vec{j}
  \right)
  \bvec{j}\tran
  %
  + \ldots
  %
  + \bvec{n}\bvec{n}\tran
\end{nedqn}

I would like to know what happens to $f(\mtxQtDQ)$ as we change $\theta$
near zero. Let's first study $\mtxA'_{k_1, k_2}(\theta)$ where $k_1, k_2
\not\in \setof{i, j}$:

\begin{nedqn}
  \mtxA'_{k_1, k_2}(\theta)
& = &
  \bvec{k_1}\tran (\mtxQtDQ) \bvec{k_2}
  \\
& = &
  \bvec{k_1}\tran \mtxA \bvec{k_2}
  \\
& = &
  \mtxA_{k_1, k_2}
  \nednumber\label{change:k1:k2}%
\end{nedqn}

As we see, the only changed elements should be in rows (and columns) $i$
and $j$. Let's next consider what happens to $\mtxA'_{k, i}(\theta)$
where $k \not\in \setof{i, j}$.

\begin{nedqn}
  \mtxA'_{k, i}(\theta)
& = &
  \bvec{k}\tran (\mtxQtDQ) \bvec{i}
  \\
& = &
  \bvec{k}\tran \mtxA \left(
    \cos\theta \bvec{i} + \sin\theta \bvec{j}
  \right)
  \\
& = &
  \cos\theta \mtxA_{k, i} + \sin\theta \mtxA_{k, j}
  \nednumber
  \label{change:k:i}
\end{nedqn}

Well, that makes sense! Naturally, because $\mtxA'(\theta)$ is
symmetric, we know that $\mtxA'_{i, k}(\theta) = \mtxA'_{k, i}(\theta)$.
Let's next ask about $\mtxA'_{k, j}(\theta)$. Here we go:

\begin{nedqn}
  \mtxA'_{k, j}(\theta)
& = &
  \bvec{k}\tran (\mtxQtDQ) \bvec{j}
  \\
& = &
  \bvec{k}\tran \mtxA \left(
    -\sin\theta \bvec{i} + \cos\theta \bvec{j}
  \right)
  \\
& = &
  -\sin\theta \mtxA_{k, i} + \cos\theta \mtxA_{k, j}
  \nednumber\label{change:k:j}%
\end{nedqn}

Naturally, this is just the ``opposite'' of what is done for $\mtxA'_{k,
i}$. Again, we know $\mtxA'_{k, j} = \mtxA'_{j, k}$. Last, we have for
$\mtxA'_{i, j}(\theta)$:

\begin{nedqn}
  \mtxA'_{i, j}(\theta)
& = &
  \bvec{i}\tran (\mtxQtDQ) \bvec{j}
  \\
& = &
  \left(
    \cos\theta\bvec{i} + \sin\theta\bvec{j}
  \right)\tran
  %
  \mtxA
  %
  \left(
    -\sin\theta\bvec{i} + \cos\theta\bvec{j}
  \right)
  \\
& = &
  - \sin\theta \cos\theta \mtxA_{i, i}
  + \cos^2\theta \mtxA_{i, j}
  - \sin^2\theta \mtxA_{j, i}
  + \sin\theta \cos\theta \mtxA_{j, j}
  \\
& = &
  \left(
    \sin\theta \cos\theta
  \right)
  \left(
    \mtxA_{j, j} - \mtxA_{i, i}
  \right)
  +
  \left(
    \cos^2\theta - \sin^2\theta
  \right)
  \mtxA_{i, j}
  \nednumber\label{change:i:j}%
\end{nedqn}

\section{First Order Conditions}

Let us now begin examining $\fptheta f(\mtxQtDQ)$. Why? Because, at a
minimum, we know that these partials must be zero. We hope to show that
a minimum can only occur if only all off-diagonal entries are zero.

First, let's look at:

\begin{nedqn}
  \fptheta \left[
    \left( \mtxA'_{k_1, k_2}(\theta) \right)^2
  \right]
& = &
  \fptheta \left[ \left( \mtxA_{k_1, k_2} \right)^2 \right]
=
  0
  \nedcomment{substitution from \eqref{change:k1:k2}}%
\end{nedqn}

As expected: those elements don't change regardless of $\theta$, so they
don't cause any change in $f$. Next, let's look at:

\begin{nedqn}
  \fptheta \left[
    \left( \mtxA'_{k, i}(\theta) \right)^2
  \right]
& = &
  \fptheta \left[
    \left(
      \cos\theta \mtxA_{k, i} + \sin\theta \mtxA_{k, j}
    \right)^2
  \right]
  \nedcomment{substitution from \eqref{change:k:i}}%
  \\
& = &
  2
  \left(
    \cos\theta \mtxA_{k, i} + \sin\theta \mtxA_{k, j}
  \right)
  \\&&
  \phantom{2\big(}
  \tfptheta \bigl[
    \cos\theta \mtxA_{k, i} + \sin\theta \mtxA_{k, j}
  \bigr]
  \nedcommenthard{chain rule}%
  \\
& = &
  2 \left(
    \cos\theta \mtxA_{k, i} + \sin\theta \mtxA_{k, j}
  \right)
  \\&&\phantom{2\big(}
  \left(
    -\sin\theta \mtxA_{k, i} + \cos\theta \mtxA_{k, j}
  \right)
  \nednumber%
\end{nedqn}

We want to evaluate this derivative at $\theta = 0$, since we want to
see what happens when we make small changes to $\mtxA$. Thus, we obtain:

\begin{nedqn}
  \fptheta \left[
    \left( \mtxA'_{k, i}(\theta) \right)^2
  \right]
  % Move closer. Represents function application.
  \!\!
  \big(0\big)
& = &
  2
  \left( \mtxA_{k, i} \right)
  \left( \mtxA_{k, j} \right)
  \nednumber\label{change:at:zero:k:i:}
\end{nedqn}

Let us do the corresponding calculation for $\mtxA'_{k, j}$:

\begin{nedqn}
  \fptheta \left[
    \left( \mtxA'_{k, j}(\theta) \right)^2
  \right]
& = &
  \fptheta \left(
    -\sin\theta \mtxA_{k, i} + \cos\theta \mtxA_{k, j}
  \right)^2
  \nedcommenthard{substitution from \eqref{change:k:j}}%
  \\
& = &
  2
  %
  \left(
    -\sin\theta \mtxA_{k, i} + \cos\theta \mtxA_{k, j}
  \right)
  %
  \\&&
  \phantom{2(}
  \tfptheta \left[
    -\sin\theta \mtxA_{k, i} + \cos\theta \mtxA_{k, j}
  \right]
  \\
& = &
  2
  \left(
    -\sin\theta \mtxA_{k, i} + \cos\theta \mtxA_{k, j}
  \right)
  \\&&
  \phantom{2(}
  \left(
    -\cos\theta \mtxA_{k, i} - \sin\theta \mtxA_{k, j}
  \right)
  \nednumber%
\end{nedqn}

Again, substituting $\theta = 0$, we obtain:

\begin{nedqn}
  \fptheta \left[
      \left( \mtxA'_{k, j}(\theta) \right)^2
  \right]
  % Move closer: function application.
  \!\!
  \big(0\big)
& = &
  2
  \left( \mtxA_{k, j} \right)
  \left( -\mtxA_{k, i} \right)
  \nednumber\label{change:at:zero:k:j:}
\end{nedqn}

Do you see what has happened here? The change to $f$ due to change in
$\mtxA'_{k, i}$ (equation \ref{change:at:zero:k:i:}) be canceled out by
the change to $f$ due to change in $\mtxA'_{k, j}$ (equation
\ref{change:at:zero:k:j:}). (And likewise for $\mtxA'_{i, k}$ and
$\mtxA'_{j, k}$.)

Thus, when rotating between $\bvec{i}, \bvec{j}$, the only meaningful
difference (if any) to $f$ is going to come from change at positions
$\mtxA'_{i, j}, \mtxA'_{j, i}$.

\section{
  Calculation Of
  $
    \fptheta \left[
      \left( \mtxA'_{i, j}(\theta) \right)^2
    \right]
  $
}

Let's do this. First we simply apply the chain rule.

\begin{nedqn}
  \fptheta \left[
    \left( \mtxA'_{i, j}(\theta) \right)^2
  \right]
& = &
  2
  \mtxA'_{i, j}(\theta)
  \fptheta \left[ \mtxA'_{i, j}(\theta) \right]
  \nednumber\label{partial:i:j:error}%
\end{nedqn}

Next, let's recall our formula for $\mtxA'_{i, j}$.

\begin{nedqn}
  \mtxA'_{i, j}(\theta)
& = &
  {
    (\sin\theta\cos\theta)
    (\mtxA_{j, j} - \mtxA_{i, i})
    +
    (\cos^2\theta - \sin^2\theta)
    \mtxA_{i, j}
  }
  \nedcommenthard{i.e., formula \eqref{change:i:j}}%
\end{nedqn}

The chain rule wants us to calculate the derivative for this:

\begin{nedqn}
  \fptheta \left[
    \mtxA'_{i, j}(\theta)
  \right]
& = &
  \fptheta \Bigl[
    (\sin\theta \cos\theta)
    (\mtxA_{j, j} - \mtxA_{i, i})
    +
    (\cos^2\theta - \sin^2\theta)
    \mtxA_{i, j}
  \Bigr]
  \\
& = &
  (\cos^2\theta - \sin^2\theta)
  (\mtxA_{j, j} - \mtxA_{i, i})
  \\&&
  \phantom{(}
  +
  (-2\cos\theta\sin\theta - 2\sin\theta\cos\theta)
  \mtxA_{i, j}
  \\
& = &
  (\cos^2\theta - \sin^2\theta)
  (\mtxA_{j, j} - \mtxA_{i, i})
  -
  4
  \cos\theta \sin\theta
  \mtxA_{i, j}
  \nednumber\label{partial:i:j}%
\end{nedqn}

Next, we must evaluate $\mtxA'_{i, j}(0)$ and $\fptheta \mtxA'_{i,
j}(0)$.

\begin{nedqn}
  \mtxA'_{i, j}(0)
& = &
  (\sin0 \cos0)
  (\mtxA_{j, j} - \mtxA_{i, i})
  +
  (\cos^2 0 - \sin^2 0) \mtxA_{i, j}
  \\
& = &
  \mtxA_{i, j}
  \nednumber%
  \label{aprime:i:j:at:zero}%
  %
\shortintertext{and}
  %
  \fptheta \left[
    \mtxA'_{i, j}(0)
  \right]
& = &
  (\cos^2 0 - \sin^2 0)
  (\mtxA_{j, j} - \mtxA_{i, i})
  -
  4
  \left( \cos 0 \sin 0 \right) \mtxA_{i, j}
  \\
& = &
  (\mtxA_{j, j} - \mtxA_{i, i})
  \nednumber%
  \label{partial:i:j:at:zero}%
\end{nedqn}

We may now evaluate the partial for $(\mtxA'_{i, j})^2$ at $\theta = 0$
by substitution:

\begin{nedqn}
  \fptheta \left[
    \left( \mtxA'_{i, j}(0) \right)^2
  \right]
& = &
  2
  \mtxA'_{i, j}(0)
  \fptheta \left[ \mtxA'_{i, j}(0) \right]
  \\
& = &
  2
  \mtxA_{i, j}
  (\mtxA_{j, j} - \mtxA_{i, i})
  \nednumspace%
  \nednumber%
\end{nedqn}

Of course, we may use this to evaluate $\fptheta f(0)$, as has been
our goal:

\begin{nedqn}
  \fptheta \left[ f(\theta) \right]
  \!
  (0)
& = &
  \fptheta \left[
    \left( \mtxA'_{i, j}(0) \right)^2
  \right]
  +
  \fptheta \left[
    \left( \mtxA'_{j, i}(0) \right)^2
  \right]
  \\
& = &
  2 \fptheta \left[
    \left( \mtxA'_{i, j}(0) \right)^2
  \right]
  \\
& = &
  4
  \mtxA_{i, j}
  (\mtxA_{j, j} - \mtxA_{i, i})
  \nednumber%
\end{nedqn}

\section{Summary Thus Far}

Let's sum up. We used these facts:

\begin{itemize}
  \item there is no change to entries outside the $i$th and $j$th
  rows/columns,

  \item the effect on $f$ of changes at positions $(k, i)$ cancel with
  the effect of changes at $(k, j)$ (for $k \not\in \setof{i, j}$),

  \item likewise for positions $(i, k)$ and $(j, k)$,

  \item changes at $(i, i)$ and $(j, j)$ never count toward the error
  because they are on-diagonal,

  \item that leaves only positions to consider $(i, j)$ and $(j, i)$.
\end{itemize}

We may finally say that if $\mtxA$ minimizes $f$, then:

\begin{IEEEeqnarray*}{rrCl}
  &
  \fptheta \left[ f(0) \right]
  =
  0
\\
  \Rightarrow{} &
  4 \mtxA_{i, j} (\mtxA_{j, j} - \mtxA_{i, i})
  & = &
  0
\\
  \Rightarrow{} &
  \mtxA_{i, j} (\mtxA_{j, j} - \mtxA_{i, i})
  & = &
  0
  \nednumber%
\end{IEEEeqnarray*}

Which of course implies:

\begin{nedqn}
  \mtxA_{i, j} = 0
& \quad\text{or}\quad &
  (\mtxA_{j, j} - \mtxA_{i, i}) = 0
\end{nedqn}

The first condition is exactly what we want: it would impose a
requirement that the matrix $\mtxA$ have zero elements off the diagonal
in order to minimize $f$. That is: $f$ would achieve a minimum only for
a diagonal matrix. Since we know $f$ indeed achieves a minimum, it must
be that $\mtxA$ is diagonalizable.

But the second condition is an ``escape hatch.'' It says that the entry
off the diagonal needn't be zero so long as $\mtxA_{j, j} = \mtxA_{i,
i}$. Uh-oh?

% TODO: Up to here in my review.

\section{Second Order Conditions}

To check out what is happening we should investigate the second order
conditions. In particular: we know that $f$ has a maximum, for the same
reason it has a minimum: the extreme value theorem!

So let's go ahead and examine the second partial.

\begin{nedqn}
  % Second derivative
  \fpthetax \left[
    \left( \mtxA'_{i, j}(\theta) \right)^2
  \right]
& = &
  % Derivative of derivative
  \fptheta \biggl[
    \fptheta \left[
      \left( \mtxA'_{i, j}(\theta) \right)^2
    \right]
  \biggl]
  \\
& = &
  % Chain rule.
  \fptheta \biggl[
    2
    \mtxA'_{i, j}(\theta)
    \fptheta \left[
      \mtxA'_{i, j}(\theta)
    \right]
  \biggl]
  \nedcomment{chain rule}%
  \\
& = &
  % Product rule
  2
  \biggl[
    % Squared first derivative
    \left(
      \fptheta \left[
        \mtxA'_{i, j}(\theta)
        \right]
    \right)^2
    +
    % Present value times second derivative
    \mtxA'_{i, j}(\theta)
    \fpthetax \left[
      \mtxA'_{i, j}(\theta)
    \right]
  \biggl]
  \nedcommenthard{product rule}%
  \\
& = &
  2 \Big[
    \mtxA'_{i, j}(\theta)
    \fpthetax \left[
      \mtxA'_{i, j}(\theta)
    \right]
  \Big]
  \nednumber\label{second:derivative:squared}%
\end{nedqn}

The last step is justified because we wouldn't bother with the second
order test unless $\fptheta \left[ \mtxA'_{i, j}(\theta) \right] = 0$.
We've already calculated $\mtxA'_{i, j}(\theta)$ previously, so let's
focus on the second derivative: $\fpthetax \left[ \mtxA'_{i, j}(\theta)
\right]$.

\begin{nedqn}
  % Second derivative.
  \fpthetax \left[
    \mtxA'_{i, j}(\theta)
  \right]
& = &
  % Derivative of derivative.
  \fptheta \Big[
    \fptheta \left[
      \mtxA'_{i, j}(\theta)
    \right]
  \Big]
  \\
& = &
  \fptheta \Big[
    {
      (\cos^2\theta - \sin^2\theta)
      (\mtxA_{j, j} - \mtxA_{i, i})
    }
    -
    {
      4
      \cos\theta
      \sin\theta
      \mtxA_{i, j}
    }
  \Big]
  \nedcommenthard{formula \ref{partial:i:j}}%
  \\
& = &
  \fptheta \Big[
    -4
    \cos\theta
    \sin\theta
    \mtxA_{i, j}
  \Big]
  \nedcomment{by presumption $\mtxA_{i, i} = \mtxA_{j, j}$}%
  \nednumspace%
  \nednumber%
\end{nedqn}

Notice that I've used our presumption that $\mtxA_{i, i} = \mtxA_{j,
j}$, since otherwise the first-order condition would have already
ensured that an off-diagonal partial is zero. We continue:

\begin{nedqn}
  \fpthetax \left[
    \mtxA'_{i, j}(\theta)
  \right]
&=
  \fptheta \Big[
    -4
    \cos\theta
    \sin\theta
    \mtxA_{i, j}
  \Big]
  \\
&=
  -4
  \left(
    \cos^2\theta - \sin^2\theta
  \right)
  \mtxA_{i, j}
  \nednumber%
  \nednumber\label{second:derivative}%
\end{nedqn}

And we may now plug formula \ref{second:derivative} into formula
\ref{second:derivative:squared}.

\begin{nedqn}
  \fpthetax \left[
    \left(
      \mtxA'_{i, j}(\theta)
    \right)^2
  \right]
& = &
  2
  \Big[
    \mtxA'_{i, j}(\theta)
    \fpthetax \left[
      \mtxA'_{i, j}(\theta)
    \right]
  \Big]
  \\
& = &
  2
  \mtxA'_{i, j}(\theta)
  \Big(
    -4
    \left(
      \cos^2\theta - \sin^2\theta
    \right)
    \mtxA_{i, j}
  \Big)
  \\
& = &
  -8 \mtxA'_{i, j}(\theta)
  \left(
    \cos^2\theta - \sin^2\theta
  \right)
  \mtxA_{i, j}
  \nednumber%
\end{nedqn}

There is nothing left to do but evaluate at $\theta = 0$:

\begin{nedqn}
  \fpthetax \left[
    \left(
      \mtxA'_{i, j}(0)
    \right)^2
  \right]
& = &
  -8
  \left(
    \mtxA'_{i, j}(0)
  \right)
  \left(
    \cos^2 0 - \sin^2 0
  \right)
  \mtxA_{i, j}
  \\
& = &
  -8
  \mtxA_{i, j}
  \mtxA_{i, j}
  \\
& = &
  -8
  \left(
    \mtxA_{i, j}
  \right)^2
  \nednumber
\end{nedqn}

And there you have it. This second derivative must always be negative,
since it is $-8$ (always negative) times $\left( \mtxA_{i, j} \right)^2$
(always positive). Note that we assumed $\mtxA_{i, j} \ne 0$, otherwise
there was no reason to go down this road. That saves us from any further
derivative testing.


\section{An Example}

Are we fucked? Presumably not. Let's just look at one matrix to make
ourselves feel a little better:

\begin{equation*}
  \begin{bmatrix}
    1 & 2 \\
    2 & 1
  \end{bmatrix}
\end{equation*}

What happens when we try to rotate it?

\begin{equation*}
  \begin{split}
      &\begin{bmatrix}
        \cos\theta & -\sin\theta \\
        \sin\theta & \cos\theta
      \end{bmatrix}\tran
      \begin{bmatrix}
        1 & 2 \\
        2 & 1
      \end{bmatrix}
      \begin{bmatrix}
        \cos\theta & -\sin\theta \\
        \sin\theta & \cos\theta
      \end{bmatrix}
    \\
    ={}&
      \begin{bmatrix}
        \cos\theta  & \sin\theta \\
        -\sin\theta & \cos\theta
      \end{bmatrix}
      \begin{bmatrix}
        1\cos\theta + 2\sin\theta & -1\sin\theta + 2\cos\theta \\
        2\cos\theta + 1\sin\theta & -2\sin\theta + 1\cos\theta
      \end{bmatrix}
    \\
    ={}&
      \begin{bmatrix}
        1 \cos^2\theta + (2 + 2) \cos\theta \sin\theta + 1 \sin^2\theta
        &
        2 \cos^2\theta + (-1 + 1) \cos\theta \sin\theta - 2\sin^2\theta
        \\
        2 \cos^2\theta + (-1 + 1) \cos\theta \sin\theta - 2\sin^2\theta
        &
        1\cos^2\theta + (-2 - 2) \cos\theta\sin\theta + 1\sin^2\theta
      \end{bmatrix}
    \\
    ={}&
      \begin{bmatrix}
        1 + 4\cos\theta\sin\theta
        &
        2 \left( \cos^2\theta - \sin^2\theta \right)
        \\
        2 \left( \cos^2\theta - \sin^2\theta \right)
        &
        1 - 4\cos\theta\sin\theta
      \end{bmatrix}
    \end{split}
\end{equation*}

Okay let's calculate $\fptheta f$ for this matrix!

\begin{equation*}
  \begin{split}
      \fptheta \left[f(\theta)\right]
    ={}&
      \fptheta \left[
        \left( \mtxA'_{2, 1}(\theta) \right)^2
        + \left( \mtxA'_{1, 2}(\theta) \right)^2
      \right]
    \\
    ={}&
      4
      \mtxA'_{2, 1}(\theta)
      \fptheta \left[ \mtxA'_{2, 1}(\theta) \right] \\
    \\
      \mtxA'_{2, 1}(\theta)
    ={}&
      2 \left( \cos^2\theta - \sin^2\theta \right) \\
      \mtxA'_{2, 1}(0)
    ={}&
        2 \left( \cos^2 0 - \sin^2 0 \right) \\
    ={}&
      2 \\
    \\
      \fptheta \left[ \mtxA'_{2, 1}(\theta) \right]
    ={}&
      \fptheta \left[
        2 \left( \cos^2\theta - \sin^2\theta \right)
      \right] \\
    ={}&
      2 \big( -2\cos\theta \sin\theta - 2\sin\theta\cos\theta \big) \\
    ={}&
      -8 \cos\theta \sin\theta \\
      \fptheta \left[ \mtxA'_{2, 1}(0) \right]
    ={}&
      -8 \cos 0 \sin 0 \\
    ={}&
      0 \\
    \\
      \fptheta \left[f(0)\right]
    ={}&
      4
      \mtxA'_{2, 1}(0)
      \fptheta \left[ \mtxA'_{2, 1}(0) \right] \\
    ={}&
      4 \cdot 2 \cdot 0 \\
    ={}& 0
  \end{split}
\end{equation*}

All our calculations check out with what we've done before. And we know
that, because this is a 2-by-2 real symmetric matrix, it must be
diagonalizable. And still the first partial is zero\dots

Well, the solution is clear: what is the story with the \textit{second}
partial?

\begin{equation*}
  \begin{split}
      \fpthetax \left[f(\theta)\right]
    ={}&
      \fptheta \Big[ \fptheta \left[f(\theta)\right] \Big] \\
    ={}&
      \fptheta \Big[
        4
        \mtxA'_{2, 1}(\theta)
        \fptheta \left[ \mtxA'_{2, 1}(\theta) \right]
      \Big] \\
    ={}&
      4 \Big[
        \fptheta \left[ \mtxA'_{2, 1}(\theta) \right]
        \fptheta \left[ \mtxA'_{2, 1}(\theta) \right]
        +
        \mtxA'_{2, 1}(\theta)
        \fpthetax \left[ \mtxA'_{2, 1}(\theta) \right]
      \Big] \\
    \\
      \fpthetax \left[ \mtxA'_{2, 1}(\theta) \right]
    &=
      \fptheta \Big[ -8 \cos\theta \sin\theta \Big] \\
    &= -8 \left( \cos^2\theta - \sin^2\theta \right) \\
      \fpthetax \left[ \mtxA'_{2, 1}(0) \right]
    &= -8 \left( \cos^2 0 - \sin^2 0 \right) \\
    &= -8 \\
    \\
      \fpthetax \left[f(0)\right]
    ={}&
      4 \Big[
        \fptheta \left[ \mtxA'_{2, 1}(0) \right]
        \fptheta \left[ \mtxA'_{2, 1}(0) \right]
        +
        \mtxA'_{2, 1}(0)
        \fpthetax \left[ \mtxA'_{2, 1}(0) \right]
      \Big] \\
    ={}&
      4 \Big[
        \left(0 \cdot 0\right)
        +
        \left(2 \cdot -8\right)
      \Big] \\
    ={}& -64
  \end{split}
\end{equation*}

Wonderful. In conclusion, we see that indeed the second partial is
negative, which means that we are at a local maximum.

\section{Conclusion}

Let's recap what we have done.

\begin{enumerate}
  \item We proved (elsewhere, previously) that rotation of basis keeps
  symmetric matrices symmetric.

  \item We then defined an ``error'' function $f$ to minimize. The error
  is minimal (zero) precisely when all off-diagonal entries are zeroed.

  \item Since the space of rotations of $\mtxA$ is compact, and because
  $f$ is continuous, we know it achieves a minimum in this space.

  \item We were unsure whether the minimum of $f$ on the space of
  rotations is actually zero.

  \item We use the first derivative test to discover that for any
  optimal $\mtxA$, either $\mtxA_{i, j} = \mtxA_{j, i} = 0$ (as desired)
  OR $\mtxA_{i, i} = \mtxA_{j, j}$ (leaving open the possibility that
  $\mtxA_{i, j} \ne 0$).

  \item We then examined what the second derivative would be at a
  first-order critical point, in the undesired case that $\mtxA_{i, j}
  \ne 0$, and thus necessarily $\mtxA_{i, i} = \mtxA_{j, j}$.

  \item The second derivative test showed us this must be a local
  \textit{maximum}. Thus we can preclude this scenario at a global
  minimum.

  \item We have foreclosed any escape. A global minimum must have
  $\mtxA_{i, j} = \mtxA_{j, i} = 0$, else it would be possible to
  improve.
\end{enumerate}

Does this proof have any advantage over the much more succinct proof JRN
gave me?

This proof accords with an intuition that you can always iteratively
keep improving your matrix by doing little rotations. It shows that you
can never get stuck. Indeed, you can use gradient descent if you want.
So this proof is much more clearly tied to a notion of
\textit{procedure}, which is certainly related to intuition.

Second, the proof says that, if you believe things work in two
dimensions, you can just keep picking pairs of basis vectors to fix, and
do your best. Remember, you might ``disrupt'' previous work, but to the
extent you disrupt previous work, you're helping other work. That's what
we know from the cancelation of the $\mtxA_{k, i}, \mtxA_{k, j}$ terms.

Probably that would have been a simpler proof right there. You have this
iterative process now, and it must converge to an answer, so the answer
must exist\dots

One goal was to limit the magical use of symmetry. Here it is still used
somewhat: you use symmetry when you presume that rotation of basis
preserves symmetry. Likewise, you use symmetry when you assume that
fixing things in one subspace won't have an overall negative impact.

TODO: Come back and think more about whether this proof really does feel
less mysterious?

\end{document}
