\documentclass[11pt, oneside]{amsart}

\usepackage{geometry}
\geometry{letterpaper}

\usepackage{ned-common}
\usepackage{ned-abstract-algebra}
\usepackage{ned-calculus}
\usepackage{ned-linear-algebra}

\begin{document}

\title{Quaternions}
\maketitle

\begin{enumerate}
  \item We want a system that allows us to simply write rotations.

  \item Consider rotations in a 2-dimensional plane. We know that these
  can be written in matrix format as:

  \begin{nedqn}
    \begin{bmatrix}
      \cos \theta & -\sin \theta \\
      \sin \theta & \cos \theta
    \end{bmatrix}
  \end{nedqn}

  \item The result of rotating a vector $\vv$ by $\theta$ degrees is
  written as $\mtxQ \vv$.

  \item If $\mtxQ_1, \mtxQ_2$ are two rotations, then the successive
  operation of both rotations is given by $\mtxQ_2 \mtxQ_1$.

  \item Note that this matrix approach easily generalizes from two to
  three dimensions and in fact to arbitrary numbers of dimensions.

  \item This matrix way is not the only way to represent, apply, and
  compose rotations to vectors. Consider the two-dimensional case.
  Represent $\vv$ as $\theta$, where $\vv = \cos\theta \ve_1 +
  \sin\theta \ve_2$. Next, represent $\mtxQ$ as $\phi$. Then $\mtxQ \vv$
  may be represented as $\phi + \theta$.

  \item Likewise, we may represent $\mtxQ_2 \mtxQ_1$ as $\phi_1 +
  \phi_2$. Note of course that this won't work if we are rotating in
  more than two dimensions. Also note that if $\vv$ is already written
  as a linear combination of $\ve_1, \ve_2$, and we want $\mtxQ\vv$ in
  the same format, then the matrix approach may be preferred.

  \item Certainly, we should note that the matrix formalism is less
  dense than the angle representation in the two-dimensional case. It
  may also be more convenient if we want to know the result in terms of
  angle.

  \item Let's consider a formalism for three-dimensional rotation. A
  rotation occurs about some axis: call this $\vu$. How much rotation?
  Call this $\theta$ radians. Note that $\vu$ should be normalized, so
  it only has two degrees of freedom; a rotation can be defined by three
  free parameters. Sometimes we combine $\vu$ and $\theta$ as $\theta
  \vu$ which is called the \define{Euler vector}. This concisely
  combines the three parameters, but it is not very useful for combining
  successive rotations (or calculating the result of a single rotation
  on a vector $\vv$). There is no obvious way to do this.

  \item Let's consider the two-dimensional case a little bit more. We
  noted that we can view $\vu$ as a rotation $\mtxQ$, where $\mtxQ \ve_1
  = \vu$. Then the application of a rotation transformation to $\vu$ can
  be seen simply as another successive rotation composition.

  \item Equivalently, we can represent our $\mtxQ$ transformation as a
  vector $\vu$, where $\vu = \mtxQ \ve_1$. Representation as $\vu$ loses
  no information; there is a unique angle $\theta$ by which $\ve_1$ is
  transformed to $\vu$. If we recovered that, we could easily rotate
  $\ve_2$ by $\theta$ to compute the second column of $\mtxQ$.

  \item A representation as $\vu$ will be convenient if it can also be
  used to accumulate rotations. Let's write $\vu\vv$ to mean the result
  of rotating $\vv$ by $\vu$.

  We will require $\ve_1 \ve_1 = \ve_1$ and $\ve_1 \ve_2 = \ve_2$. We
  will also want $\ve_2 \ve_1 = \ve_2$ and $\ve_2 \ve_2 = -\ve_1$.

  Because rotation in two dimensions commutes, we have $\vu\vv =
  \vv\vu$.

  \item Last, we know that rotation distributes: $\parens{\vu_1 +
  \vu_2}\vv = \vu_1\vv + \vu_2\vv$. I've talked about that elsewhere,
  but won't bother here.

  \item Commutativity and distributivity imply that a definition of
  $\ve_1\ve_1 = \ve_1, \ve_1\ve_2 =\ve_2, \ve_2\ve_2=-\ve_1$ suffices to
  allow easy computation. Because $\ve_1$ acts as an identity, we simply
  call it $1$. Because $\ve_2$ acts as a root of $-1$, we call it $i$,
  after the imaginary unit.

  \item Let's now consider rotations in three dimensions.
\end{enumerate}

% Source: https://eater.net/quaternions
% Source: https://en.wikipedia.org/wiki/Quaternions_and_spatial_rotation

\end{document}
