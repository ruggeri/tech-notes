\section{Cyclic Groups and Subgroups}

\subsection{Definition}

\begin{definition}
  An element $g \in G$ is a \define{generator} if every element of the
  group $a$ is equal to some $g^i$. We say that the group $G$ is
  \define{cyclic}.
\end{definition}

\begin{remark}
  Finite cyclic groups of the same order are isomorphic. In particular
  we can always consider $\Zmodn$ (with addition as the group
  operation). Note that zero is the identity (its inverse is itself).

  Infinite cyclic groups are isomorphic to $\Z$.
\end{remark}

\subsection{Order Of Elements}

\begin{proposition}
  If $g$ generates $G$, and $k$ divides $|G|$, then the order of $g^k$
  is of course $|G|/k$. More generally for any $k$: $|g^k|$ is
  $\frac{|g|}{\gcd(|g|, k)}$.
\end{proposition}

\begin{proof}
  Consider $g^k$ raised to the $\frac{|g|}{\gcd(|g|, k)}$ power. This is
  $g$ to the power of $\frac{|g|k}{\gcd(|g|, k)}$. This is the least
  common multiple of $|g|, k$. Thus it is $g^{|g|}$ to the something.
  Which means it is one to the something. Which means it is $1$.

  Because $\frac{|g|k}{\gcd(|g|, k)}$ is the \emph{least} common
  multiple of $|g|, k$, this is the first time that $g^k$ will wrap to a
  power of $|g|$. Thus $|g^k| = \frac{|g|}{\gcd(|g|, k)}$ as desired.
\end{proof}

\begin{remark}
  If $k$ is coprime with $|g|$, then $|g^k| = |G|$. Thus $g^k$ generates
  the entire group. There are thus $\varphi(|G|)$ generators of a cyclic
  group $G$.
\end{remark}

\begin{remark}
  Note that $|g^k|$ always divides the order of the cyclic group.
\end{remark}

\subsection{Subgroups of Cyclic Groups}

\begin{theorem}
  Every subgroup of a cyclic subgroup is cyclic.
\end{theorem}

\begin{proof}
  Choose any two $g^a, g^b \in H$. I want to show that \emph{both}
  elements are generated by a common $g^k$. If I do this, I can then
  consider $g^k, g^c$, to find a common generator. Repeating, I will
  find a minimal $g^k$ that generates the entire group.

  Let's begin. Now, if $a$ divides $b$ (or vice versa), we are done.
  Otherwise, consider $\gcd(a, b)$. By Bezout's lemma, we can always
  find $x, y$ such that $xa - yb = \gcd(a, b)$. Thus, by closure of a
  subgroup, we know that $g^{\gcd(a, b)}$ is in the subgroup.

  But this implies that there exists $k_a, k_b$ such that $g^{\gcd(a,
  b)}$ can be exponentiated to yield $g^a, g^b$. Thus $g^a, g^b$ are
  both generated by this element.
\end{proof}

\begin{remark}
  We thus see that for any set of $g^{a_i}$, the subgroup generated by
  the collection is exactly equal to the cyclic subgroup generated by
  $g^k$, where $k = \gcd\parens{\setof{a_i}}$.
\end{remark}

\begin{remark}
  We can say more than this! By Bezout's lemma again, there exists $x,
  y$ such that $xk - y|G| = \gcd\parens{k, |G|}$. Note that Bezout shows
  that you can never choose $x, y$ to get \emph{smaller} than
  $\gcd\parens{k, |G|}$.

  This shows that every subgroup is generated by some $g^{k'}$ where
  $k'$ \emph{divides} $|G|$.
\end{remark}

\begin{theorem}
  By the preceding remark, we've shown that subgroups of a cyclic group
  $|G|$ can be put into \emph{bijection} with divisors of $|G|$.

  There is thus a \emph{unique} subgroup of size $k$ for each divisor of
  $|G|$.
\end{theorem}

\subsection{Subgroups Generated By Sets}

\begin{definition}
  Consider an acyclic group $G$. Consider any subset $A$. You can
  \define{generate a subgroup from $A$} by taking all finite products of
  members of $A$ (and their inverses).
\end{definition}

\begin{remark}
  The book discusses how there can be many more distinct ``words''
  formed from $A$ if $G$ is \emph{non-abelian}.
\end{remark}

\begin{remark}
  The book describes drawing a ``lattice'' of subgroups of $G$. This
  lattice is a DAG where the vertices are subgroups of $G$, and the
  edges are from $A$ to $B$ if (a) $A \subgroup B$, but (b) there is no
  $C$ such that $A \subgroup C \subgroup B$.

  Of course isomorphic groups have the same lattice. However, apparently
  some non-isomorphic groups \emph{also} have the same lattice!
\end{remark}

\subsection{Notes On Factoring Cyclic Groups}

\begin{remark}
  Here I go into my own exploration apart from the book.
\end{remark}

\begin{definition}
  The \define{join} of two subgroups $A, B$ consists of all products
  $ab$. If $A, B$ are disjoint, then the join has size $|A||B|$. (I'm
  assuming abelian I guess?).

  A group $G$ is \define{factored} if I can separate it into disjoint
  subgroups whose join is $G$.
\end{definition}

\begin{theorem}
  A cyclic group $G$ of size $\prod_i p_i^{k_i}$ can be factored into
  disjoint subgroups $G_i$ of size $p_i^{k_i}$.
\end{theorem}

\begin{proof}
  We've seen that for a cyclic group $G$, there is exactly \emph{one}
  cyclic subgroup of size $k$ for each $k$ dividing $|G|$.

  So let $G_i$ be the subgroup of size $p_i^{k_i}$. The intersection of
  any two $G_i, G_j$ must be trivial. Any subgroup of these must have
  size dividing both $|G_i|, |G_j|$, which are necessarily coprime. Thus
  the $G_i$ are disjoint.

  Each $G_i$ is generated by some element $g_i$. I claim that $g =
  \prod_i g_i$ is a generator of $G$. Why? Because the $|g|$ is the
  least common multiple of $p_i^{k_i}$. Thus $|g| = |G|$, and so $g$
  generates the whole group.
\end{proof}

\begin{remark}
  Note that I don't need to assume $G$ is cyclic. Take all elements with
  order dividing $p_i^{k_i}$ and group them as $G_i$. Each $G_i$ is
  clearly a subgroup. And each $G_i$ is clearly disjoint. And the join
  must still be all of $G$.

  You may note: chose a minimal \emph{set} of generators $G'_i$ for each
  $G_i$. Then take all products $\prod_i g_i$ for $g_i \in G'_i$. This
  generates $G$. I won't prove it, but of course if $G'_i$ has just one
  element $g_i$, we're back to the cyclic case.
\end{remark}

\begin{theorem}
  A cyclic group $G$ of size $p^k$ cannot be further factored into
  smaller subgroups that join together back to $G$.
\end{theorem}

\begin{proof}
  Consider any two proper subgroups $A, B$. I claim that one contains
  the other.

  WLOG, let's assume that $|A| \leq |B|$. Then I claim that every $ab$
  has order dividing $|B|$. Why? Because $|A|, |B|$ are both powers of
  $p_i$. So the smaller power must divide the larger one.

  Since every $ab$ has order dividing $|B|$, the join has size at most
  $|B|$. Which means the join is exactly $B$. Which means that $A$ must
  have already been contained in $B$.

  But that means our task is hopeless. We cannot factor into disjoint
  subgroups; there is simply a series of subgroups each contained in the
  other Matryoshka style. And the join of these subgroups never gives
  anything new.
\end{proof}

\begin{theorem}
  An \emph{acyclic} group of size $p^k$ is decomposable.
\end{theorem}

\begin{proof}
  Consider a maximal cyclical subgroup $G_1$ generated by $g_1$. This
  has size $p^{k_1}$.

  Next, let's partition $G$ into equivalence classes. We'll say that two
  elements $g, g'$ are equivalent if $g = g_1^i g'$. That is, we are
  saying that two elements are equivalent if they differ only by a
  factor of some element of $G_1$. There are $p^{k_2} = p^{k - k_1}$
  equivalence classes, each containing $p^{k_1}$ elements.

  I claim that by careful selection from the equivalence classes, we can
  construct a disjoint subgroup $G_2$. I won't go into precisely how.
  Presumably we'll learn when we cover quotient groups.

  Anyway, I claim that you have now decomposed $G$. You have two
  disjoint subgroups, so their join should be $G$.
\end{proof}

\begin{remark}
  Since $G$ is acyclic, perhaps $G_2$ is acyclic (since $G_1$ is cyclic
  by assumption). In this case $G_2$ can be further factored.

  Or perhaps $G_2$ is \emph{also} cyclic. In that case it must
  \emph{not} be decomposable. Note that in this case the join of $G_1,
  G_2$ is not as simple as taking the two generators $g_1, g_2$ and
  forming $g = g_1 g_2$ that generates all of $G$.
\end{remark}

\begin{theorem}[Fundamental Theorem of Finite Abelian Groups]

  Every finite abelian group can be uniquely factored into a product of
  prime power cyclic groups.
\end{theorem}

\begin{proof}
  We've proven this by the above work.
\end{proof}

\begin{corollary}
  Every finite group of prime size is cyclic.
\end{corollary}

\begin{remark}
  How many abelian groups are there of size $p^k$? We know that this
  must decompose into a product of cyclic groups of size $p^i$. How many
  distinct ways are there to ``partition'' $k$ into non-negative
  integers $i$? This is called the \define{partition function}, which is
  denoted $p(k)$.

  Apparently there is no (known?) closed form for $p(k)$. Though it can
  be calculated through a recurrence relation.
\end{remark}

\begin{remark}
  The number of abelian groups of size $\prod_i p^{k_i}$ is thus equal
  to $\prod_i p(k_i)$.
\end{remark}

\subsection{Finite Fields}

\begin{remark}
  These notes are based off of:
  \url{https://web.stanford.edu/class/ee392d/Chap7.pdf}. Those notes are
  part of EE392d, a coding theory course at Stanford. This chapter is
  from the reading notes for that class.
\end{remark}

\begin{definition}
  A \define{monoid} is almost a group. The difference is that not every
  element need be invertible. The integers (either positive or
  non-negative) are a monoid with respect to multiplication.

  Like groups/abelian groups, monoids may be either commutative or
  non-commutative.
\end{definition}

\begin{definition}
  A \define{ring} is a set with $+, \cdot$ operations. The set is an
  abelian group with respect to $+$ (the identity is written 0).

  The group is a \define{monoid} with respect to multiplication. That
  is: multiplication is associative, and there is an identity 1.
  Elements, (even non-zero elements) may not be invertible.

  The distributive law is required to hold: $a(b + c) = ab + ac$.
\end{definition}

\begin{definition}
  A \define{commutative ring} is one in which the multiplication
  operation commutes. When a ring has a multiplication which is not
  commutative, we will emphasize this by calling it a
  \define{non-commutative ring}. When we just write ``ring,'' we'll
  assume it is a commutative ring.
\end{definition}

\begin{definition}
  A \define{field} is a commutative ring where every non-zero element
  has an inverse.
\end{definition}

\begin{definition}
  We can talk about a ``non-commutative field.'' When the multiplication
  operation is non-commutative, we call this a \define{division ring} or
  \define{skew field}. But we won't study these now.

  Wedderburn's little theorem (which I have not studied) proves that all
  finite skew fields are in fact proper commutative fields. The only
  ``true'' skew fields are infinite.
\end{definition}

\begin{theorem}
  $\Zmodn$ is a field exactly when $n$ is prime.
\end{theorem}

\begin{proof}
  The addition and multiplication operations are both commutative. The
  distributive law holds. These are facts on the integer operations, in
  terms of which $\Zmodn$ is defined.

  Both a 0 and 1 elements exist. Additive inverses always exist. Last,
  note that for all $x \in \Zmodn$, $x\Zmodn$ is a permutation when $n$
  is prime. Thus all $x$ are invertible when $n$ is prime. But this is
  not so when $n$ is composite.
\end{proof}

\begin{theorem}
  The \define{integers} of a \emph{finite} field $\Field$ are the sums
  generated by repeated addition of 1: $0, 1, 1+1, 1+1+1, \ldots$. The
  integers of a \emph{finite} field are always themselves a subfield.
\end{theorem}

\begin{proof}
  Presuming that $\Field$ is finite, the set of integers of $\Field$ is
  also finite, and must wrap back to 0 eventually. (This is how we use
  our important presumption that $\Field$ is finite.) So the integers
  are a cyclic group generated by $1$ with respect to addition.

  Next question: are the integers of $\Field$ a \define{subfield}? We
  consider the product $ij$ of integers $i, j$. But we can use the
  distributive law:

  \begin{nedqn}
    ij
  \eqcol
    \parens{1_1 + 1_2 + \ldots + 1_i} \parens{1_1 + 1_2 + \ldots + 1_j}
  \\
  \eqcol
    1_1\parens{1_1 + 1_2 + \ldots + 1_j}
    + \ldots +
    1_2\parens{1_1 + 1_2 + \ldots + 1_j}
  \end{nedqn}

  \noindent
  This shows that $ij$ is just the result of adding $1$ to itself $ij$
  times. Which shows that the integers are isomorphic to $\Zmodn$. If
  $n$ were not prime, then multiplication by some $i$ in $\Zmodn$ is not
  injective. In this case, $\Zmodn$ is not a field, and neither can
  $\Field$ be.

  Since of course we've assumed $\Field$ \emph{is} a field, we know that
  $n$ must be prime and $\Zmodn$ must be a subfield.
\end{proof}

\begin{corollary}
  There are always a prime number of units. We call this the
  \define{characteristic} of the field $\Field$.
\end{corollary}

\begin{corollary}
  If the characteristic of $\Field$ is $p$, then any field element $x$
  added to itself $p$ times will always be zero.
\end{corollary}

\begin{proof}
  \begin{nedqn}
    x_1 + x_2 + \ldots + x_p
  \eqcol
    \parens{1_1 + 1_2 + \ldots + 1_p} x
  \\
  \eqcol
    0x
  \\
  \eqcol
    0
  \end{nedqn}
\end{proof}

\begin{theorem}
  Every finite field of $p$ elements is isomorphic to $\Zmodp$. We
  denote this $\Field_p$ or even more commonly as $\GF{p}$. We call this
  the \define{Galois Field} of $p$ elements.
\end{theorem}

\begin{proof}
  We know that for a finite field $\Field$ with $p$ elements, the
  integers are a subfield. But Lagrange implies that the number of
  integers must divide $p$. Since the number of integers must always be
  at least two (the integers always contains 0 and 1), the number of
  integers must not trivially divide $p$. Thus the number of integers is
  in fact $p$, and thus $\Field \cong \GF{p}$.
\end{proof}

\begin{definition}
  A polynomial over $\Field$ is an expression $f_0 + f_1x + \ldots + f_m
  x^m$. The ``coefficients'' $f_i$ are taken from $\Field$. We can add
  polynomials coordinate-wise.

  We define the standard multiplication of polynomials by
  \define{convolution}. For two polynomials $f, g$, their product is a
  new polynomial $h$ where the coefficients of $h$ are given by:

  \begin{nedqn}
    h_i
  \eqcol
    \sum_{j = 0}^i
    f_j
    g_{i - j}
  \end{nedqn}
\end{definition}

\begin{proposition}
  We denote the space of all polynomials over a field $\Field$ as
  $\Field[x]$. This space is a (commutative) ring.
\end{proposition}

\begin{proof}
  The space is a group with respect to addition. There are additive
  inverses (the polynomial with coefficients $-f_0, \ldots, -f_m$). We
  see that there is an additive identity (the polynomial with all zero
  coefficients). A polynomial has a finite number of terms. A
  \define{degree $n$} polynomial has $n+1$ terms.

  The space is a commutative monoid with respect to multiplication.
  Multiplication is clearly commutative. Proving it is associative takes
  just a little work that I won't bother with. The polynomial 1 is the
  multiplicative identity.

  It is simple to verify that the distributive law holds, so this proves
  that we have a ring.
\end{proof}

\begin{remark}
  We normally think of polynomials as functions, where $x$ is a
  placeholder for an input to the function. Over $\R$, two polynomials
  are identical if and only if they define the same function.

  This is not necessarily true in other contexts. The reading gives an
  example that both $x, x^2$ define the same function over $\Zmod{2}$,
  even though the functions they define are both the same.

  This emphasizes that maybe we shouldn't think of polynomials quite
  like functions. They are simply expressions in a syntactic sense. The
  notation $\Field[x]$ is really just meaning expressions written in a
  certain form. The $[x]$ part means ``adjoin'' an element to define a
  richer set of expressions. We could write $\Field[x, y]$ to define an
  even fuller space of polynomials.
\end{remark}

\begin{proposition}
  $\F[x]$ is a \define{cancellation ring}: whenever $xy = xz$, we may
  infer $y = z$. However, $\F[x]$ is not a field.

  A cancellation ring is also called an \define{integral domain}. That
  name sounds dumb to me, though.
\end{proposition}

\begin{proof}
  Consider $xy = xz$. Then $x \parens{y - z} = 0$ (distributive law).
  But this clearly only happens if $x = 0$ or $y - z = 0$. So it is safe
  to cancel when we know $x \ne 0$.

  Even though $F[x]$ is a cancellation ring, it doesn't have inverses.
  Multiplying by a polynomial can never result in a new polynomial of
  lower degree. So \emph{no} polynomial (besides the integers) can have
  an inverse.
\end{proof}

\begin{proposition}
  A finite cancellation ring $R$ must be a field.
\end{proposition}

\begin{proof}
  For every non-zero element $a$ and pair of distinct elements $b, c$,
  we know $ab \ne bc$. Thus each $ab$ is distinct. There are $\order{R}$
  distinct products, and exactly $\order{R}$ elements in the ring. So
  there exists some $b$ such that $ab = 1$.

  Note this proof doesn't work for \emph{infinite} cancellation rings
  like the integers or polynomials.
\end{proof}

\begin{remark}
  It's clear that $\Zmodp$ is a finite field. But $\Zmodp \times \Zmodp$
  is more tricky. The identity is clearly $1 \times 1$. But note that $1
  \times 0$ does not have an inverse! So this is not a finite field!

  So does there exist a finite field of size $p^k$? Focus on the
  multiplicative subgroup a moment. If this factors, then you can have
  non-zero elements like $0 \times x$ which have no inverse. So I
  presume that the multiplicative subgroup must itself be cyclic.

  An example of a field that works is if you take polynomials modulo
  $q(x)$, where $q$ is a $k+1$-degree irreducible polynomial. Then: (1)
  every non-zero polynomial has an inverse, (2) the space consists of
  all $k$-degree polynomials.
\end{remark}
