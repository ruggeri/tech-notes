\documentclass[11pt, oneside]{amsart}

\usepackage{geometry}
\geometry{letterpaper}

\usepackage{ned-common}
\usepackage{ned-abstract-algebra}

\begin{document}

\title{Abstract Algebra}
\maketitle

\section{Basic Group Definitions And Properties}

\begin{remark}
  These notes follow Dummit and Foote's Abstract Algebra textbook.
\end{remark}

\subsection{Definitions and Propositions}

\begin{definition}
  A \define{group} is a set with a binary operation. The binary
  operation must be \define{associative}. There must be an
  \define{identity element}. (The identity must be both a left and right
  identity. That is, the identity must commute.) And there must be
  \define{inverses}.

  An \define{abelian group} is \define{commutative}. That is: $ab = ba$.
  If not abelian, then every non-identity element might not commute.
\end{definition}

\begin{proposition} Each of the following is true:
  \begin{enumerate}
    \item The identity $1$ element is unique.
    \item Each inverse $\inv{a}$ is unique.
    \item $\parensinv{ab} = \inv{b} \inv{a}$.
  \end{enumerate}
\end{proposition}

\begin{proof}
  Say that $a$ was also an identity element. Then what would be $a1$? Is
  it $1$ or is it $a$?

  Say that both $x$ and $y$ were inverses of $a$. Then what is $xay$? Is
  it $x$ or is it $y$?

  It is clear that $\inv{b}\inv{a}$ will invert $ab$.
\end{proof}

\begin{proposition}
  Both left and right cancelation hold. That is $ab = ac$ implies $b =
  c$.
\end{proposition}

\begin{proof}
  Say not. Then multiply on the right by $\inv{b}$ and the left by
  $\inv{a}$. We get $1 = c \inv{b}$. But we've already shown that
  because inverses are unique we must have $b = c$.
\end{proof}

\subsection{Permutations, Homomorphisms, and Isomorphisms}

\begin{definition}
  The \define{symmetric group} on $A$ $S_A$ is the group of all
  bijections $\sigma: A \to A$. The operation for $S_A$ is function
  composition. Elements of the symmetric group are also called
  \define{permutations}.
\end{definition}

\begin{proposition}
  We can always decompose an element of $S_A$ into disjoint
  \define{cycles}.
\end{proposition}

\begin{definition}
  A \define{homomorphism} is a map $\varphi: A \to B$ such that
  $\varphi(xy) = \varphi(x) \varphi(y)$.

  An \define{isomorphism} is a homomorphism that is bijective.
\end{definition}

\begin{proposition}
  The inverse of an isomorphism $\inv{\varphi}$ is itself an
  isomorphism.
\end{proposition}

\begin{proof}
  I prove only that it is a homomorphism. It is clearly bijective.

  \begin{nedqn}
    \varphi(ab)
  \eqcol
    \varphi(a) \varphi(b)
  \\
    \inv{\varphi}\parens{\varphi(ab)}
  \eqcol
    \inv{\varphi}\parens{\varphi(a) \varphi(b)}
  \\
    \alpha \beta
  \eqcol
    \inv{\varphi}\parens{\alpha \beta}
  \end{nedqn}

  \noindent
  Here I've introduced $\alpha, \beta$ defined as $\inv{\varphi}(a),
  \inv{\varphi}(b)$. The new notation demonstrates that $\inv\varphi$ is
  a homomorphism.
\end{proof}

\begin{remark}
  All symmetric groups over the same number of elements can be put into
  isomorphism in the obvious way. There are $n!$ such isomorphisms
  (corresponding to $n!$ ways to match off corresponding elements).
\end{remark}

\subsection{Group Actions}

\begin{definition}
  A \define{group action} $\cdot$ of a group $G$ on a set $A$ satisfies:

  \begin{enumerate}
    \item $g_1 \cdot (g_2 \cdot a) = (g_1 g_2) \cdot a$,
    \item $1 \cdot a = a$.
  \end{enumerate}
\end{definition}

\begin{remark}
  Each element $g$ implies a permutation $\sigma_g$ on $A$. Why must
  $\sigma_g$ be bijective? Because it must have an inverse
  $\inv\sigma = \sigma_{\inv{g}}$!

  Thus any group action implicitly defines a \emph{homomorphism} from
  $G$ into $S_A$.
\end{remark}

\section{Subgroups}

\begin{definition}
  $H$ is a \define{subgroup} of $G$ (written $H \subgroup G$) if $H$ is
  closed under multiplication and inverses (and, by implication,
  contains $1$).
\end{definition}

\begin{proposition}[The Subgroup Criterion]
  $H$ is a subgroup of $G$ if for all $a, b \in H$, we have $a\inv{b}
  \in H$ too.
\end{proposition}

\begin{proof}
  Clearly this implies that $1 \in H$. But then this implies that $1
  \inv{b} \in H$. And therefore $a\parensinv{\inv b} = ab$ must be in
  $H$, too.
\end{proof}

\subsection{Centralizers, Normalizers}

\begin{remark}
  I'm not sure I care too much about these concepts.
\end{remark}

\begin{definition}
  The \define{centralizer} of a subset $A$ (written: $C_G(A)$) is the
  set of elements $g$ such that $ga = ag$. These are the elements that
  \emph{commute} with elements of $A$.

  Another (annoying?) way to write this: $ga\inv{g} = a$.
\end{definition}

\begin{proposition}
  $C_G(A)$ is a subgroup of $G$.
\end{proposition}

\begin{proof}
  Simple:

  \begin{nedqn}
    g_1 g_2 a
  \eqcol
    g_1 a g_2 \nedcomment{Because $g_2$ in $C_G(A)$}
  \\
  \eqcol
    a g_1 g_2 \nedcomment{Because $g_1$ in $C_G(A)$}
  \end{nedqn}

  It's similarly trivial to show inverses are contained.
\end{proof}

\begin{remark}
  Note that $C_G(\setof{1})$ is all of $G$. In general, if $A \subset
  B$, then $C_G(B) \subset C_G(A)$.
\end{remark}

\begin{definition}
  The \emph{center} of $G$ (denote $Z(G)$) consists of all elements $x$
  such that for all elements $g$ we have: $gx = xg$.

  The center is equal to $C_G(G)$.

  Alternatively: it is the intersection of all centralizers. It is only
  those elements that would commute with every other element.

  Alternatively: the center is the \emph{largest} subset of $G$ such
  that $C_G(Z(G)) = G$. It is \emph{all} the elements that commute with
  every other element.
\end{definition}

\begin{definition}
  The \define{normalizer} of $A$ (denoted $N_G(A)$) consists of all $g$
  such that $gA = Ag$.
\end{definition}

\begin{remark}
  This is a generalization of the centralizer. The centralizer is very
  specific: it says that $ga = ag$. But it appears that the normalizer
  allows some shuffling around of the elements of $A$.

  That is: if $g \cdot a$ is seen as a left group action, it corresponds
  to one permutation. If $a \cdot g$ is seen as a \emph{right} group
  action, it may correspond to a \emph{second} group action.

  Basically: the normalizer consists of those elements $g$ that, when
  acting on either the left or the right, permute the elements of $A$.
\end{remark}

\begin{remark}
  The normalizer is a subgroup of $G$. It's easy to see why. If $g_1$
  permutes the elements of $A$, and $g_2$ does likewise, then $g_1 g_2$
  does as well.
\end{remark}

\begin{remark}
  Note that $N_G(\setof{1}) = G$. But \emph{also} note that $N_G(G) =
  G$. As we increase $S$, $N_G(S)$ must handle more elements, but it may
  shuffle them amongst a larger set. So containment is not
  straightforward.
\end{remark}

\begin{remark}
  In summary: $C_G(A) \subgroup N_G(A) \subgroup G$.
\end{remark}

\subsection{Stabilizers, Kernels}

\begin{definition}
  Consider a group $G$ acting on a set $S$. The \define{stabilizer}
  $G_s$ is the set of $g$ such that $g \cdot s = s$.
\end{definition}

\begin{remark}
  Notice right away that $G_s$ is a subgroup of $G$.
\end{remark}

\begin{definition}
  The intersection of all stabilizers is called the \define{kernel} of
  $S$. These are the $g$ such that $g \cdot s = s$ for all $s \in S$.
\end{definition}

\begin{remark}
  Clearly the kernel is a subgroup of every stabilizer.
\end{remark}

\subsection{Relation of Centralizers/Normalizers and Stabilizers/Kernels}

\begin{remark}
  Consider $G$ acting on the powerset $\powerset{G}$ by $g \cdot S
  \mapsto gS\inv{g}$. Then the normalizer of $S$ is equivalent to the
  \emph{stabilizer} of $S$. It is every element of $G$ that merely
  permutes $S$.
\end{remark}

\begin{remark}
  Next, consider just the normalizer $N_G(S)$. Consider the action $g
  \cdot s \mapsto g s \inv{g}$.

  The kernel of this action are exactly those elements that commute with
  every element of $S$. It is the \emph{centralizer} $Z_G(S)$.
\end{remark}

\begin{remark}
  Lastly, if we set $S = G$, then the center $Z(G)$ is equal to the
  kernel of $g \cdot s \mapsto g s \inv{g}$.
\end{remark}

\section{Cyclic Groups and Subgroups}

\subsection{Definition}

\begin{definition}
  An element $g \in G$ is a \define{generator} if every element of the
  group $a$ is equal to some $g^i$. We say that the group $G$ is
  \define{cyclic}.
\end{definition}

\begin{remark}
  Finite cyclic groups of the same order are isomorphic. In particular
  we can always consider $\Zmodn$ (with addition as the group
  operation). Note that zero is the identity (its inverse is itself).

  Infinite cyclic groups are isomorphic to $\Z$.
\end{remark}

\subsection{Order Of Elements}

\begin{proposition}
  If $g$ generates $G$, and $k$ divides $|G|$, then the order of $g^k$
  is of course $|G|/k$. More generally for any $k$: $|g^k|$ is
  $\frac{|g|}{\gcd(|g|, k)}$.
\end{proposition}

\begin{proof}
  Consider $g^k$ raised to the $\frac{|g|}{\gcd(|g|, k)}$ power. This is
  $g$ to the power of $\frac{|g|k}{\gcd(|g|, k)}$. This is the least
  common multiple of $|g|, k$. Thus it is $g^{|g|}$ to the something.
  Which means it is one to the something. Which means it is $1$.

  Because $\frac{|g|k}{\gcd(|g|, k)}$ is the \emph{least} common
  multiple of $|g|, k$, this is the first time that $g^k$ will wrap to a
  power of $|g|$. Thus $|g^k| = \frac{|g|}{\gcd(|g|, k)}$ as desired.
\end{proof}

\begin{remark}
  If $k$ is coprime with $|g|$, then $|g^k| = |G|$. Thus $g^k$ generates
  the entire group. There are thus $\varphi(|G|)$ generators of a cyclic
  group $G$.
\end{remark}

\begin{remark}
  Note that $|g^k|$ always divides the order of the cyclic group.
\end{remark}

\subsection{Subgroups of Cyclic Groups}

\begin{theorem}
  Every subgroup of a cyclic subgroup is cyclic.
\end{theorem}

\begin{proof}
  Choose any two $g^a, g^b \in H$. I want to show that \emph{both}
  elements are generated by a common $g^k$. If I do this, I can then
  consider $g^k, g^c$, to find a common generator. Repeating, I will
  find a minimal $g^k$ that generates the entire group.

  Let's begin. Now, if $a$ divides $b$ (or vice versa), we are done.
  Otherwise, consider $\gcd(a, b)$. By Bezout's lemma, we can always
  find $x, y$ such that $xa - yb = \gcd(a, b)$. Thus, by closure of a
  subgroup, we know that $g^{\gcd(a, b)}$ is in the subgroup.

  But this implies that there exists $k_a, k_b$ such that $g^{\gcd(a,
  b)}$ can be exponentiated to yield $g^a, g^b$. Thus $g^a, g^b$ are
  both generated by this element.
\end{proof}

\begin{remark}
  We thus see that for any set of $g^{a_i}$, the subgroup generated by
  the collection is exactly equal to the cyclic subgroup generated by
  $g^k$, where $k = \gcd\parens{\setof{a_i}}$.
\end{remark}

\begin{remark}
  We can say more than this! By Bezout's lemma again, there exists $x,
  y$ such that $xk - y|G| = \gcd\parens{k, |G|}$. Note that Bezout shows
  that you can never choose $x, y$ to get \emph{smaller} than
  $\gcd\parens{k, |G|}$.

  This shows that every subgroup is generated by some $g^{k'}$ where
  $k'$ \emph{divides} $|G|$.
\end{remark}

\begin{theorem}
  By the preceding remark, we've shown that subgroups of a cyclic group
  $|G|$ can be put into \emph{bijection} with divisors of $|G|$.

  There is thus a \emph{unique} subgroup of size $k$ for each divisor of
  $|G|$.
\end{theorem}

\subsection{Subgroups Generated By Sets}

\begin{definition}
  Consider an acyclic group $G$. Consider any subset $A$. You can
  \define{generate a subgroup from $A$} by taking all finite products of
  members of $A$ (and their inverses).
\end{definition}

\begin{remark}
  The book discusses how there can be many more distinct ``words''
  formed from $A$ if $G$ is \emph{non-abelian}.
\end{remark}

\begin{remark}
  The book describes drawing a ``lattice'' of subgroups of $G$. This
  lattice is a DAG where the vertices are subgroups of $G$, and the
  edges are from $A$ to $B$ if (a) $A \subgroup B$, but (b) there is no
  $C$ such that $A \subgroup C \subgroup B$.

  Of course isomorphic groups have the same lattice. However, apparently
  some non-isomorphic groups \emph{also} have the same lattice!
\end{remark}

\subsection{Notes On Factoring Cyclic Groups}

\begin{remark}
  Here I go into my own exploration apart from the book.
\end{remark}

\begin{definition}
  The \define{join} of two subgroups $A, B$ consists of all products
  $ab$. If $A, B$ are disjoint, then the join has size $|A||B|$. (I'm
  assuming abelian I guess?).

  A group $G$ is \define{factored} if I can separate it into disjoint
  subgroups whose join is $G$.
\end{definition}

\begin{theorem}
  A cyclic group $G$ of size $\prod_i p_i^{k_i}$ can be factored into
  disjoint subgroups $G_i$ of size $p_i^{k_i}$.
\end{theorem}

\begin{proof}
  We've seen that for a cyclic group $G$, there is exactly \emph{one}
  cyclic subgroup of size $k$ for each $k$ dividing $|G|$.

  So let $G_i$ be the subgroup of size $p_i^{k_i}$. The intersection of
  any two $G_i, G_j$ must be trivial. Any subgroup of these must have
  size dividing both $|G_i|, |G_j|$, which are necessarily coprime. Thus
  the $G_i$ are disjoint.

  Each $G_i$ is generated by some element $g_i$. I claim that $g =
  \prod_i g_i$ is a generator of $G$. Why? Because the $|g|$ is the
  least common multiple of $p_i^{k_i}$. Thus $|g| = |G|$, and so $g$
  generates the whole group.
\end{proof}

\begin{remark}
  Note that I don't need to assume $G$ is cyclic. Take all elements with
  order dividing $p_i^{k_i}$ and group them as $G_i$. Each $G_i$ is
  clearly a subgroup. And each $G_i$ is clearly disjoint. And the join
  must still be all of $G$.

  You may note: chose a minimal \emph{set} of generators $G'_i$ for each
  $G_i$. Then take all products $\prod_i g_i$ for $g_i \in G'_i$. This
  generates $G$. I won't prove it, but of course if $G'_i$ has just one
  element $g_i$, we're back to the cyclic case.
\end{remark}

\begin{theorem}
  A cyclic group $G$ of size $p^k$ cannot be further factored into
  smaller subgroups that join together back to $G$.
\end{theorem}

\begin{proof}
  Consider any two proper subgroups $A, B$. I claim that one contains
  the other.

  WLOG, let's assume that $|A| \leq |B|$. Then I claim that every $ab$
  has order dividing $|B|$. Why? Because $|A|, |B|$ are both powers of
  $p_i$. So the smaller power must divide the larger one.

  Since every $ab$ has order dividing $|B|$, the join has size at most
  $|B|$. Which means the join is exactly $B$. Which means that $A$ must
  have already been contained in $B$.

  But that means our task is hopeless. We cannot factor into disjoint
  subgroups; there is simply a series of subgroups each contained in the
  other Matryoshka style. And the join of these subgroups never gives
  anything new.
\end{proof}

\begin{theorem}
  An \emph{acyclic} group of size $p^k$ is decomposable.
\end{theorem}

\begin{proof}
  Consider a maximal cyclical subgroup $G_1$ generated by $g_1$. This
  has size $p^{k_1}$.

  Next, let's partition $G$ into equivalence classes. We'll say that two
  elements $g, g'$ are equivalent if $g = g_1^i g'$. That is, we are
  saying that two elements are equivalent if they differ only by a
  factor of some element of $G_1$. There are $p^{k_2} = p^{k - k_1}$
  equivalence classes, each containing $p^{k_1}$ elements.

  I claim that by careful selection from the equivalence classes, we can
  construct a disjoint subgroup $G_2$. I won't go into precisely how.
  Presumably we'll learn when we cover quotient groups.

  Anyway, I claim that you have now decomposed $G$. You have two
  disjoint subgroups, so their join should be $G$.
\end{proof}

\begin{remark}
  Since $G$ is acyclic, perhaps $G_2$ is acyclic (since $G_1$ is cyclic
  by assumption). In this case $G_2$ can be further factored.

  Or perhaps $G_2$ is \emph{also} cyclic. In that case it must
  \emph{not} be decomposable. Note that in this case the join of $G_1,
  G_2$ is not as simple as taking the two generators $g_1, g_2$ and
  forming $g = g_1 g_2$ that generates all of $G$.
\end{remark}

\begin{theorem}[Fundamental Theorem of Finite Abelian Groups]

  Every finite abelian group can be uniquely factored into a product of
  prime power cyclic groups.
\end{theorem}

\begin{proof}
  We've proven this by the above work.
\end{proof}

\begin{remark}
  How many abelian groups are there of size $p^k$? We know that this
  must decompose into a product of cyclic groups of size $p^i$. How many
  distinct ways are there to ``partition'' $k$ into non-negative
  integers $i$? This is called the \define{partition function}, which is
  denoted $p(k)$.

  Apparently there is no (known?) closed form for $p(k)$. Though it can
  be calculated through a recurrence relation.
\end{remark}

\begin{remark}
  The number of abelian groups of size $\prod_i p^{k_i}$ is thus equal
  to $\prod_i p(k_i)$.
\end{remark}

% **TODO**: Up to section 3.1 p73

\end{document}
