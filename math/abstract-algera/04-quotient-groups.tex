\section{Quotient Groups}

\subsection{Fibers and Quotient Groups}

\begin{definition}
  Let $\varphi: G \to H$ be a homomorphism. For each $h\in H$, the
  \define{fiber} of $G$ mapping to $h$ is the set of $g$ such that
  $\varphi(g) = h$. We will often denote fiber of $G$ mapping to $h$ as
  $G_h$.

  We call $G_1$ the \define{kernel} of the $\varphi$. This is denoted
  $\ker \varphi$.
\end{definition}

\begin{proposition}
  It is clear that $\ker \varphi$ is a subgroup of $G$.

  It is also clear that for every fiber $G_h$ and $g \in G_h$, $g G_1
  \subset G_h$.

  In fact, for any $g \in G_h$, $g G_1 = G_h$. Thus any $g \in G_h$ is a
  ``representative'' of $G_h$.
\end{proposition}

\begin{proof}
  Let us say that $a, b \in G_h$. I claim that $a, b$ differ by a
  factor $x \in \ker\varphi$. Consider $x = a \inv{b}$. Then:

  \begin{nedqn}
    \varphi(x)
  \eqcol
    \varphi(a) \varphi\parens{\inv{b}}
  \\
  \eqcol
    \varphi(a) \inv{\varphi(b)}
    \nedcomment{a simple fact of homomorphisms}
  \\
  \eqcol
    h \inv{h}
  \\
  \eqcol
    1
  \end{nedqn}

  Thus every fiber $G_h$ is equal to $g_h G_1$ for some $g_h$ (you can
  choose any member of $G_h$ to be the representative).
\end{proof}

\begin{definition}
  Consider the set of fibers $G_h$. We can make this a group by using
  the operation $G_a G_b \mapsto G_{ab}$. This is called the
  \define{quotient group}.
\end{definition}

\begin{remark}
  By construction, the quotient group is isomorphic to the \emph{image}
  of $\varphi: G \to H$. Each fiber is in one-to-one relationship with
  each element $h$ in the image.

  The image of $\varphi$ is of course a subgroup of $H$.
\end{remark}

\subsection{(Normal) Subgroups and Quotient Groups}

\begin{theorem}
  For any (normal) subgroup $K$ of $G$, there exists a homomorphism
  $\varphi$ of $G$ such that $K = \ker\varphi$.

  Since $\varphi$ is somewhat irrelevant, we may denote the quotient
  group $G/K$, referencing only the kernel.
\end{theorem}

\begin{remark}
  Let $\varphi$ map elements of $G$ to corresponding \define{cosets} of
  $K$ $gK$.

  Define this product on cosets: $(aK)(bK) = (ab)K$. Since a
  coset has many non-unique representatives $a, a'$, is this operation
  well-defined? You can easily prove yes. (Wait! Only if $K$ is
  \emph{normal}!)

  We've thus defined a coset group, and $\varphi$ is a homomorphism (in
  fact, isomorphism!) from $G$ to the coset space. The kernel of
  $\varphi$ is exactly $K$.

  Thus we see that the concept of quotient groups is really about
  subgroups and cosets, rather than about homomorphisms.

  The homomorphism $\varphi$ we've defined is sometimes called the
  \define{natural projection} of $G$ onto $G/K$.
\end{remark}

\begin{remark}
  Not so fast! We've implicitly relied on the fact that $K$ is a
  \define{normal} subgroup of $G$. A normal subgroup has $gN = Ng$ for
  all $g \in G$. Of course, in an \emph{abelian} group every subgroup is
  normal.

  If the subgroup is normal, then the multiplication of cosets we're
  talking about is well-defined. Because:

  \begin{nedqn}
    (aK)(bK)
  \eqcol
    (aKb)K
  \\
  \eqcol
    (ab)KK
  \\
  \eqcol
    (ab)K
  \end{nedqn}
\end{remark}

\begin{definition}
  An element $gn\inv{g}$ is a \define{conjugate} of $g$. The set
  $gA\inv{g}$ is a \define{conjugate} of $A$.

  An element $g$ \define{normalizes} $A$ if $gA\inv{g} = A$. That is,
  $g$ is in the \emph{normalizer} of $A$ (denoted $N_G(A)$).

  A set $N$ is \define{normal} if every element of $G$ normalizes $N$.
  In this case, the normalizer of $N$ is all of $G$.
\end{definition}

\begin{remark}
  \TODO{We should explore non-abelian groups to find some subgroups that
  are not normal!}

  \TODO{We should explore examples of how non-normal subgroups would
  fail to give a well-defined definition of coset multiplication.}
\end{remark}

\subsection{Lagrange's Theorem}

\begin{theorem}[Lagrange's Theorem]

  For a subgroup $H$ of $G$, every coset $gH$ is disjoint
  from every other distinct coset $g'H$. Also, every coset $gH$ has
  exactly $|H|$ elements.

  Presuming $|G|$ is finite, there are thus $|G|/|H|$ distinct, disjoint
  cosets.

  Thus, $|H|$ divides $|G|$. Likewise, the quotient group has size
  $|G|/|H|$ which also divides $|G|$.
\end{theorem}

\begin{remark}
  If $G$ is infinite, then things could be more sophisticated. In that
  case, if $H$ is finite, then of course $G/H$ must be infinite.

  If $H$ is infinite, $G/H$ may or may not be infinite. For instance,
  consider $G = \Z \otimes \Z$. Alternatively, consider $G =
  \bigotimes_{i \in \mathbb{N}} \Z$.
\end{remark}

\begin{corollary}[Fermat's Little Theorem]

  $g^{|G|} = 1$ for all $g \in G$.
\end{corollary}

% **TODO**: Up to section 3.2 p91.
