\section{Subgroups}

\begin{definition}
  $H$ is a \define{subgroup} of $G$ (written $H \subgroup G$) if $H$ is
  closed under multiplication and inverses (and, by implication,
  contains $1$).
\end{definition}

\begin{proposition}[The Subgroup Criterion]
  $H$ is a subgroup of $G$ if for all $a, b \in H$, we have $a\inv{b}
  \in H$ too.
\end{proposition}

\begin{proof}
  Clearly this implies that $1 \in H$. But then this implies that $1
  \inv{b} \in H$. And therefore $a\parensinv{\inv b} = ab$ must be in
  $H$, too.
\end{proof}

\subsection{Centralizers, Normalizers}

\begin{remark}
  I'm not sure I care too much about these concepts.
\end{remark}

\begin{definition}
  The \define{centralizer} of a subset $A$ (written: $C_G(A)$) is the
  set of elements $g$ such that $ga = ag$. These are the elements that
  \emph{commute} with elements of $A$.

  Another (annoying?) way to write this: $ga\inv{g} = a$.
\end{definition}

\begin{proposition}
  $C_G(A)$ is a subgroup of $G$.
\end{proposition}

\begin{proof}
  Simple:

  \begin{nedqn}
    g_1 g_2 a
  \eqcol
    g_1 a g_2 \nedcomment{Because $g_2$ in $C_G(A)$}
  \\
  \eqcol
    a g_1 g_2 \nedcomment{Because $g_1$ in $C_G(A)$}
  \end{nedqn}

  It's similarly trivial to show inverses are contained.
\end{proof}

\begin{remark}
  Note that $C_G(\setof{1})$ is all of $G$. In general, if $A \subset
  B$, then $C_G(B) \subset C_G(A)$.
\end{remark}

\begin{definition}
  The \emph{center} of $G$ (denote $Z(G)$) consists of all elements $x$
  such that for all elements $g$ we have: $gx = xg$.

  The center is equal to $C_G(G)$.

  Alternatively: it is the intersection of all centralizers. It is only
  those elements that would commute with every other element.

  Alternatively: the center is the \emph{largest} subset of $G$ such
  that $C_G(Z(G)) = G$. It is \emph{all} the elements that commute with
  every other element.
\end{definition}

\begin{definition}
  The \define{normalizer} of $A$ (denoted $N_G(A)$) consists of all $g$
  such that $gA = Ag$.
\end{definition}

\begin{remark}
  This is a generalization of the centralizer. The centralizer is very
  specific: it says that $ga = ag$. But it appears that the normalizer
  allows some shuffling around of the elements of $A$.

  That is: if $g \cdot a$ is seen as a left group action, it corresponds
  to one permutation. If $a \cdot g$ is seen as a \emph{right} group
  action, it may correspond to a \emph{second} group action.

  Basically: the normalizer consists of those elements $g$ that, when
  acting on either the left or the right, permute the elements of $A$.
\end{remark}

\begin{remark}
  The normalizer is a subgroup of $G$. It's easy to see why. If $g_1$
  permutes the elements of $A$, and $g_2$ does likewise, then $g_1 g_2$
  does as well.
\end{remark}

\begin{remark}
  Note that $N_G(\setof{1}) = G$. But \emph{also} note that $N_G(G) =
  G$. As we increase $S$, $N_G(S)$ must handle more elements, but it may
  shuffle them amongst a larger set. So containment is not
  straightforward.
\end{remark}

\begin{remark}
  In summary: $C_G(A) \subgroup N_G(A) \subgroup G$.
\end{remark}

\subsection{Stabilizers, Kernels}

\begin{definition}
  Consider a group $G$ acting on a set $S$. The \define{stabilizer}
  $G_s$ is the set of $g$ such that $g \cdot s = s$.
\end{definition}

\begin{remark}
  Notice right away that $G_s$ is a subgroup of $G$.
\end{remark}

\begin{definition}
  The intersection of all stabilizers is called the \define{kernel} of
  $S$. These are the $g$ such that $g \cdot s = s$ for all $s \in S$.
\end{definition}

\begin{remark}
  Clearly the kernel is a subgroup of every stabilizer.
\end{remark}

\subsection{Relation of Centralizers/Normalizers and Stabilizers/Kernels}

\begin{remark}
  Consider $G$ acting on the powerset $\powerset{G}$ by $g \cdot S
  \mapsto gS\inv{g}$. Then the normalizer of $S$ is equivalent to the
  \emph{stabilizer} of $S$. It is every element of $G$ that merely
  permutes $S$.
\end{remark}

\begin{remark}
  Next, consider just the normalizer $N_G(S)$. Consider the action $g
  \cdot s \mapsto g s \inv{g}$.

  The kernel of this action are exactly those elements that commute with
  every element of $S$. It is the \emph{centralizer} $Z_G(S)$.
\end{remark}

\begin{remark}
  Lastly, if we set $S = G$, then the center $Z(G)$ is equal to the
  kernel of $g \cdot s \mapsto g s \inv{g}$.
\end{remark}
