\documentclass[11pt, oneside]{amsart}

\usepackage{geometry}
\geometry{letterpaper}

\usepackage{ned-common}
\usepackage{ned-calculus}

\begin{document}

\title{Make Electronics}
\maketitle

\section{Ch 1: The Basics}

\begin{enumerate}
  \item Multimeters are either \define{manual} or \define{auto-ranging}.
  Auto-ranging can take a second, so the author prefers manual.

  \item Will measure volts, amps, ohms. Can offer continuity testing (to
  see if circuit is connected). Higher end models can measure
  capacitance, test transistors (you plug the ends in the holes).

  \item The author suggests double-ended banana clips. This is handy for
  freeing up hands.

  \item \define{Potentiometers} vary resistance. A \define{fuse} blows
  if the current is too great.

  \item They suggest a first experiment of licking a 9 volt battery.
  This presumably doesn't hit you with too much current because the
  resistance is relatively high. Also: probably the battery has a
  reaction rate limit that constricts amperage?

  \item \define{Voltage} is the energy liberated per electron traveling
  from one side of a battery to another. \define{Current} is number of
  charges passing over a point per second.

  \item Materials have \define{resistivity}, which is the amount of
  \define{resistance} they provide current per unit of cross-sectional
  area. Resistance is derived from voltage and current. Why materials
  resist current/where resistivity comes from is a mystery to me at
  present.

  \item In the \define{hydraulic analogy}, voltage is like pressure.
  Gasses don't like to be under pressure, and they wish to expand to
  release stored energy. The greater the pressure, the greater the force
  exerted on the container.

  \item Recall that \define{work} (the energy expended) is the integral
  of force acting through a distance. Which is to say: the greater the
  pressure energy, the greater the force that will be acted upon in the
  same distance.  Which means, through the same distance, the force
  applied is greater, meaning greater acceleration of the particle being
  acted upon.

  \item Which explains why high pressures want to accelerate substances
  to high velocity. They are \define{resisted} presumably by
  \define{turbulence}, which is presumably from imperfections in the
  walls of the pipes the gas expands.

  \item A last note: the resistance doesn't change the number of
  electrons that will travel. Higher resistance doesn't mean that an
  electron must do more work to get through the material. It simply
  means that an electron can only accelerate to a certain speed. Which
  limits the rate of energy expenditure. That is: the \define{power}
  consumption of the circuit.

  \item 120 volts is standard in the US. 240 volt is for heavy-duty
  appliances. Household appliances may draw up to 2 amps maybe. Circuit
  breakers cut amperage at 20 amps.

  \item We did blow a fuse! It was fun!

  \item \define{Direct current} constantly pushes electrons from side to
  the other. \define{Alternating current} is generated by generators,
  and switches the voltage from one side to the other. 60hz is common.
  Generators induce current by spinning through a magnetic field.

  \item Right plug of outlet is hot, while left is neutral. The third
  hole is the electrical ground. Should AC have polarity? I think the
  idea is that there is a positive/negative draw on the hot plug,
  relative to both the neutral and ground plugs. But the ground plug is
  (I think) used to dump current rather than send current through the
  circuit to the neutral plug.

  \item Resistor marking is: Digit 1, Digit 2, Num Zeroes, Tolerance.

  \item LED wants positive power on the longer leg. There is a
  \emph{forward voltage} limit and a \emph{forward current} limit.
  Violating either of these will damage the LED.

  \item The greater resistors, when wired into series with the LED,
  result in less light from the LED. Less resistance means more light.

  \item Common breakpoints for resistors: 1.0, 1.5, 2.2, 3.3, 4.7, 6.8,
  10.0. These correspond to plus-or-minus 20\% tolerance.

  \item We pry open a potentiometer. We can see it's just a wiper
  running a long a resistive track. We have a three-prong version. As
  the wiper gets closer to the left prong, it gets farther from the
  right prong.

  \item We then use the potentiometer to control lighting the LED. But
  we also let it send too much current and burn out the LED. So we
  really want a resistor plus a potentiometer so that the current never
  runs above a certain limit.

  \item An LED with too little voltage will not even light up at all. It
  needs a threshold voltage. Measuring the LED voltage drop shows it
  consumes 2.1V when wired with the 470 ohm resistor.

  \item But if I increase the resistance of the potentiometer so that
  it's a total 1.47kΩ resistance, then the voltage consumed by the LED
  drops. If I plug it into the diode measurer, it measures 1.3V drop.

  \item I believe that the LED needs to consume 1.3V to light at all.
  That's the threshold. As you try to run more and more current through
  the LED, the LED will rise in resistance to try to block that current.
  But if the voltage is too high, the current will be too great to
  resist, and the LED will break down.

  \item Note: if a potentiometer is in series with a fixed resistor, the
  voltage drop across the resistor will increase as we turn down the
  potentiometer. That's because the battery voltage is constant.

  \item With the LED in series with potentiometer, as you turn up
  resistance on the potentiometer, the LED potential drop is
  less-and-less. That is expected. Once the voltage available to the LED
  drops below a fixed amount, the current will stop entirely, though.

  \item I guess that while a resistor consumes energy to push through
  electrons at a rate, an LED consumes energy to push electrons across
  \emph{at all}.

  \item This is \emph{Ohm's Law}: $V = IR$.

  \item Here is a thought. A resistor is like a crowded ballroom. If you
  snake your way through it, not pushing anyone, you can get through
  slowly with little energy. Or you can barrel through and push you way
  by and get through faster. But the faster you want to go, you must use
  more energy.

  \item An LED, on the other hand, is like a staircase. You must expend
  energy to travel up at all. But the speed of travel is irrelevant. An
  LED then, has a nonlinear resistance. It appears infinite at low
  voltage, but appears zero above the threshold voltage. Of course, it
  is not ideal: as current rises, the energy needed to push through will
  also rise.

  \item When you wire two resisters in parallel, then the effective
  resistance is halved. This is because to push a current of $I$ through
  the parallel pair, only $\frac{I}{2}$ current must go through each
  resistor. And thus only half the energy per charge must be used.

  \item They do a useful calculation. Say you have an LED which is
  specified to want a forward voltage of 2V, at which point it will pass
  a current of 20mA. How big a resistor must you use with a 9V battery?

  \item They say: the resistor should drop the voltage by 7V. It should
  pass a current of 20mA. If you do the math, you want a resistance of
  350 ohms.

  \item They mention that even a low-resistance wire (say: 0.2 ohm) will
  draw a lot of power if you run a large current (say 15A). This will
  consume 3V of the voltage. If you're running a 12V battery,
  approximately 1/4 of the energy is being lost to the wire!

  \item We may make an observation about \emph{power}. Power is the
  \emph{amount of energy released/consumed per second}. A high voltage
  source attached to a high resistance component will use all its energy
  to move a charge across the component. But the great resistance means
  that a small amount of current will travel, so only a small amount of
  energy will be consumed per second.

  \item As in: $P = V I = I^2 R = V^2 / R$.

  \item For reference: a car battery delivers \emph{hundreds of amps}!

  \item BTW, resistors are rated by \emph{wattage}. This is implicitly a
  measure of how much current they can allow to pass, so I don't know
  why this isn't simply measured in amps?

  \item As a (mistaken) convention, we imagine that electricity travels
  from positive to negative. Arrows point positive to negative. Positive
  pole is marked as red, negative as black.

  \section{Ch2: Switching}

  \item Solid core 22-gauge hookup wire is what we'll use. They mention
  that you could buy stranded wire, which is useful in some
  circumstances, but frustrating with the breadboard. Likewise, you can
  buy precut jumpers, but he doesn't like the color coding. And you can
  even get jumpers with plugs, but he finds those messy.

  \item Components: toggle switch (single-pole, double throw), tactile
  switch (AKA, a button), relay (nonlatching type). Trimmer
  potentiometers are simply potentiometers that plug into a breadboard.
  Transistors. Ceramic and electrolytic capacitors.

  \item A SPDT switch has three junctions. One is always connected, and
  of the other two only one is engaged. This is how a pair of light
  switches can be wired together so that only when both are thrown the
  same way does the light turn on.

  \item A DP switch has two \emph{poles}. That means when the switch is
  closed both poles close and two circuits are made. A DPST is what you
  see in movies.

  \item Pushbottons can be NO (normally open) or NC (normally closed).
  Maybe an NO button would be a doorbell.

  \item Schematic for resistor is a squiggle line. A battery is a short
  and longer parallel line; the longer one is positive. A diode is an
  arrow into a parallel line. Some diagonal lines show this is an LED.

  \item Voltage or a circuit is sometimes labeled $V_c$ or $V_{cc}$. It
  is common to label the negative terminal as \emph{ground}. There may
  be many ground symbols in a chart (rather than connect them to the
  negative terminal in the diagram).

  \item Here's how grounding works. The third wire leads directly to the
  earth. This is connected to, for instance, the chassis of the device.
  If a charge builds on an ungrounded chasis for whatever reason, and
  you touch touch it, the charge could pass through you to the earth
  (which I guess wants to accept the charge). But if the chassis is
  grounded, (1) the charge won't build up, and (2) you aren't a
  preferred route to discharge the charge because you have higher
  resistance than a direct connection to Earth.

  \item A potentiometer will have a little arrow next to the resistor
  squiggle. A pushbutton has a little plunger.

  \item When unconnected wires must cross in a planar diagram, we don't
  draw a dot. But if they must connect, there \emph{is} a dot.

  \item We did an experiment with a DPDT relay. We threw a switch by
  pressing a button. The voltage supplied to the relay runs a current
  through a wire, generating a magnetic field, and pulling a lever. The
  lever changes an electrical contact from A to B. This connects circuit
  A instead of B.

  \item Other relays are \emph{latching}. They will retain the state.
  Ours is non-latching. Latching relays have four voltage terminals, for
  two inductors.

  \item There is a \emph{coil voltage}. This is the intended voltage to
  apply the coil. There is a \emph{set voltage} which is the minimum at
  which the coil will set. There is a \emph{power consumption} specified
  for the coil to operate at the intended voltage. This can be
  equivalently specified in mA.

  \item Lastly, there is a \emph{switching capacity} which is the
  maximum allowed amps to run through the relay. One presumes this
  should be high, given there should not be much/any resistance to
  current flow.

  \item For Experiment 8 they have us build the relay circuit on the
  breadboard. They have you make jumpers - you strip a little off a
  hookup wire, cut and shift a bit of insulation so the length exposed
  is good, and then clip. Use needle-nose pliers to bend the ends at
  right angles.

  \item I have a dual bus breadboard. The bus runs along the side of the
  breadboard. It is used to distribute power. Each row is connected,
  except for a break in the middle.

  \item We get the relay to ``buzz'' by connecting the tactile switch
  from the typical output to the inductor. Thus the inductor is only
  activated when the button is depressed. But this cuts power to the
  inductor. The oscillation between the two states causes the relay to
  buzz.

  \item To slow things down, we attach a capacitor in parallel with the
  inductor. Now, when the switch is depressed, the voltage is used to
  first charge the capacitor. As it charges, the resistance of the
  capacitor rises, and thus current begins to run through the inductor.
  This closes the switch.

  \item Now no more charging of the capacitor happens. In fact, it
  begins to \emph{discharge}. The inductor resists the discharging of
  the capacitor. When the capacitor is drained, the gate closes again.

  \item So let's think about the capacitor. It is effectively a
  ``rechargeable battery.'' It is as if two big plates. You can push an
  electron onto one plate. If uncharged, the plate accepts the excess
  electron without too much trouble. To keep things net neutral, an
  electron from the opposing plate. However, no electrons cross the
  \emph{dielectric} separating the plates.

  \item  If you connect a capacitor to the terminals of a battery, the
  battery pushes electrons onto one plate, while the opposite plate
  sends electrons to the other terminal of the battery. But as electrons
  are pushed onto the capacitor plate, it requires more and more energy
  to push another electron onto the plate. Remember: battery voltage
  \emph{is} simply energy liberated by transfer of electrons. Thus: the
  capacitor is gaining voltage as it charges (accepts electrons).

  \item The charging will stop when the voltage of the capacitor reaches
  the voltage of the battery. If the battery is too strong for the
  capacitor, the buildup of electrons on one plate will become so great
  that they will start to jump across the dielectric. This will destroy
  the capacitor.

  \item Consider the \emph{rate} of charging. As the capacitor accepts
  more and more charge, it requires more and more energy to fit a next
  electron onto the plate. The energy released by sending an electron
  from the battery terminal to the plate is dropping. At the same time,
  there is some \emph{resistance} in the circuit. Less and less energy
  is available to barrel through a resistor. Thus, as the capacitor
  charges, the current flowing into the capacitor is dropping.

\newcommand{\vbat}{V_\text{bat}}
\newcommand{\vcap}{V_\text{cap}}

  \begin{nedqn}
    \vbat - \vcap(t)
  \eqcol
    I(t) R
  \\
    \fpartial{q} \vcap
  \eqcol
    \inv{C}
    \nedcomment{this is the capacitance}
  \\
    \fpartial{t} \vcap
  \eqcol
    \fpartial{t} q \inv{C}
  \\\eqcol
    I(t) \inv{C}
  \\\eqcol
    \frac{\vbat - \vcap(t)}{RC}
  \\
    \fpartial{t} \parens{
      \vbat - \vcap(t)
    }
  \eqcol
    0 - \frac{\vbat - \vcap(t)}{RC}
  \\
    \Delta V
  \eqcol
    \alpha_0
    \expf{-t/RC}
    \nedcomment{solution of diffeq}
  \\
    \alpha_0
  \eqcol
    \vbat
  \end{nedqn}

  \noindent
  And now we see where the exponential charging of capacitors comes
  from!

  \item The unit $C$ is the \define{capacitance}. This means is the
  amount of charge that a capacitor will store for a given voltage. It
  is measured in \define{Farads}, which are coulombs of charge per volt.

  \item Note: the time it takes to charge the capacitor to a given
  percentage is based not on the voltage, but on the time constant $RC$.
  The greater the $R$, the more electrons are delayed on their way into
  the capacitor. The bigger the $C$, the more electrons are needed to
  charge the capacitor by one volt.

  \item It is interesting that the capacitance doesn't rise as more
  electrons are moved into the capacitor. What \emph{does} change is the
  voltage of the capacitor. So the electron pressure exerted by the
  battery is reduced (relatively) as the capacitor voltage increases.

  \item In other words: capacitance is like the volume of a room. The
  volume of a room does not change as you fill it with air. The air
  pressure does change. The change in room air pressure means that you
  will need to blow harder (expend more energy) to move more air into
  the room, because the air that it already inside is retarding further
  air from entering.

  \item What makes a capacitor dangerous is not necessarily the
  capacitance. It is the \emph{voltage} that the capacitor can
  withstand. Your body can take a large amount of charge passing through
  it at a very slow rate. For instance, you could be connected to a 9V
  battery for the rest of your (long) life. But your body \emph{cannot}
  take being connected to 115V for even a short period of time. A
  capacitor rated for high-wattage is dangerous!!

  \item There are two kinds of capacitors: \emph{ceramic} and
  \emph{electrolytic}. Electrolytics are the tin-can ones. Electrolytic
  capacitors have polarity (long leg is positive!), ceramic capacitors
  don't care. \TODO{How do elecrolytic capacitors work?}

  % p74: read suggestions for checking circuits!
\end{enumerate}

\end{document}
