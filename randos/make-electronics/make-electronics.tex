\documentclass[11pt, oneside]{amsart}

\usepackage{geometry}
\geometry{letterpaper}

\usepackage{ned-common}
\usepackage{ned-abstract-algebra}

\begin{document}

\title{Make Electronics}
\maketitle

\section{Ch 1: The Basics}

\begin{enumerate}
  \item Multimeters are either \define{manual} or \define{auto-ranging}.
  Auto-ranging can take a second, so the author prefers manual.

  \item Will measure volts, amps, ohms. Can offer continuity testing (to
  see if circuit is connected). Higher end models can measure
  capacitance, test transistors (you plug the ends in the holes).

  \item The author suggests double-ended banana clips. This is handy for
  freeing up hands.

  \item \define{Potentiometers} vary resistance. A \define{fuse} blows
  if the current is too great.

  \item They suggest a first experiment of licking a 9 volt battery.
  This presumably doesn't hit you with too much current because the
  resistance is relatively high. Also: probably the battery has a
  reaction rate limit that constricts amperage?

  \item \define{Voltage} is the energy liberated per electron traveling
  from one side of a battery to another. \define{Current} is number of
  charges passing over a point per second.

  \item Materials have \define{resistivity}, which is the amount of
  \define{resistance} they provide current per unit of cross-sectional
  area. Resistance is derived from voltage and current. Why materials
  resist current/where resistivity comes from is a mystery to me at
  present.

  \item In the \define{hydraulic analogy}, voltage is like pressure.
  Gasses don't like to be under pressure, and they wish to expand to
  release stored energy. The greater the pressure, the greater the force
  exerted on the container.

  \item Recall that \define{work} (the energy expended) is the integral
  of force acting through a distance. Which is to say: the greater the
  pressure energy, the greater the force that will be acted upon in the
  same distance.  Which means, through the same distance, the force
  applied is greater, meaning greater acceleration of the particle being
  acted upon.

  \item Which explains why high pressures want to accelerate substances
  to high velocity. They are \define{resisted} presumably by
  \define{turbulence}, which is presumably from imperfections in the
  walls of the pipes the gas expands.

  \item A last note: the resistance doesn't change the number of
  electrons that will travel. Higher resistance doesn't mean that an
  electron must do more work to get through the material. It simply
  means that an electron can only accelerate to a certain speed. Which
  limits the rate of energy expenditure. That is: the \define{power}
  consumption of the circuit.

  \item 120 volts is standard in the US. 240 volt is for heavy-duty
  appliances. Household appliances may draw up to 2 amps maybe. Circuit
  breakers cut amperage at 20 amps.

  \item \TODO{blow a fuse!}

  \item \define{Direct current} constantly pushes electrons from side to
  the other. \define{Alternating current} is generated by generators,
  and switches the voltage from one side to the other. 60hz is common.
  Generators induce current by spinning through a magnetic field.

  \item Right plug of outlet is hot, while left is neutral. The third
  hole is the electrical ground. Should AC have polarity? I think the
  idea is that there is a positive/negative draw on the hot plug,
  relative to both the neutral and ground plugs. But the ground plug is
  (I think) used to dump current rather than send current through the
  circuit to the neutral plug.

  \item Resistor marking is: Digit 1, Digit 2, Num Zeroes, Tolerance.

  \item LED wants positive power on the longer leg. There is a
  \emph{forward voltage} limit and a \emph{forward current} limit.
  Violating either of these will damage the LED.

  \item The greater resistors, when wired into series with the LED,
  result in less light from the LED. Less resistance means more light.

  \item Common breakpoints for resistors: 1.0, 1.5, 2.2, 3.3, 4.7, 6.8,
  10.0. These correspond to plus-or-minus 20\% tolerance.

  \item We pry open a potentiometer. We can see it's just a wiper
  running a long a resistive track. We have a three-prong version. As
  the wiper gets closer to the left prong, it gets farther from the
  right prong.

  \item We then use the potentiometer to control lighting the LED. But
  we also let it send too much current and burn out the LED. So we
  really want a resistor plus a potentiometer so that the current never
  runs above a certain limit.

  \item An LED with too little voltage will not even light up at all. It
  needs a threshold voltage. Measuring the LED voltage drop shows it
  consumes 2.1V when wired with the 470 ohm resistor.

  \item But if I increase the resistance of the potentiometer so that
  it's a total 1.47kΩ resistance, then the voltage consumed by the LED
  drops. If I plug it into the diode measurer, it measures 1.3V drop.

  \item I believe that the LED needs to consume 1.3V to light at all.
  That's the threshold. As you try to run more and more current through
  the LED, the LED will rise in resistance to try to block that current.
  But if the voltage is too high, the current will be too great to
  resist, and the LED will break down.

  \item Note: if a potentiometer is in series with a fixed resistor, the
  voltage drop across the resistor will increase as we turn down the
  potentiometer. That's because the battery voltage is constant.

  \item With the LED in series with potentiometer, as you turn up
  resistance on the potentiometer, the LED potential drop is
  less-and-less. That is expected. Once the voltage available to the LED
  drops below a fixed amount, the current will stop entirely, though.

  % Up to p25!
\end{enumerate}

\end{document}
