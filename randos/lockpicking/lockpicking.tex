\documentclass[11pt, oneside]{amsart}

\usepackage{geometry}
\geometry{letterpaper}

\usepackage{ned-common}
\usepackage{ned-abstract-algebra}

\begin{document}

\title{Lockpicking}
\maketitle

\section{Definitions}

\begin{definition}
  The \define{cylinder} is the non-rotating part of the lock.
\end{definition}

\begin{definition}
  The \define{plug} is the rotating part of the lock. The plug rotates
  inside the cylinder.
\end{definition}

\begin{definition}
  The \define{shear line} is the interface between the plug and the
  cylinder.
\end{definition}

\begin{definition}
  The \define{driver pins} are pins that cross the sheer line. They
  naturally sit partway in the cylinder, and partway in the plug. If you
  try to turn the plug without the key, the driver pins will bind.
\end{definition}

\begin{definition}
  The \define{key pins} are pins that sit below the driver pins. They
  sit entirely within the plug. When the key is inserted, the top of the
  key pin is lifted to exactly the shear line. This implies that the
  driver pin bottom is lifted to exactly the shear line. The plug can
  rotate.
\end{definition}

\begin{remark}
  When no key is inserted, springs drive the driver and key pins
  downward toward the bottom of the \define{keyway}.
\end{remark}

\section{Picking}

\begin{definition}
  The \define{tension bar} is a strip of metal that tries to rotate/turn
  the plug. The tension bar is shaped so that you can access the pins
  with a pick.
\end{definition}

\begin{definition}
  A \define{lock pick} is a tool that allows you to push a pin up or
  down.
\end{definition}

\begin{proposition}
  The basic technique is to apply tension to the plug. This will cause
  the plug to bind. In particular: exactly one of the pins will be
  bound. The other pins will be free to move up-and-down.

  Push the bound pin upward slowly until it ``clicks.'' This happens
  when the driver pin is now set correctly. You should feel the plug
  rotate a little as the next pin is bound. Repeat.
\end{proposition}

\begin{proposition}
  In the beginning a bound pin is \define{underset}. As you lift the pin
  you try to set it. With an underset pin, you can push up on the key
  pin, but when pressure is released the key pin falls again.

  It can be hard to tell the difference between underset and
  \define{set}. When a pin is set you should get some rotation out of
  the lock.
\end{proposition}

\begin{proposition}
  A pin can be \define{overset}. This is when the key pin crosses the
  shear line. This is easy to detect, because the key pin will no longer
  drop when pressure is released.
\end{proposition}

\section{Security Pins}

\begin{proposition}
  When a pin is bound, you push it until the bottom of the driver pin
  reaches the shear line. You should feel a click and the plug will turn
  a little.
\end{proposition}

\begin{definition}
  A \define{spool pin} is one where the diameter of the pin is narrow
  like a spool for yarn. If the pin binds, you can try to push it up as
  usual. But when you get to the end of the spool, you cannot keep
  pushing straightforwardly. The end of the spool cannot slide through.
\end{definition}

\begin{proposition}
  The spool pin works because it gets caught at an angle. To defeat the
  spool pin, push on the side of the pin that is opposite the direction
  of rotation. This should be riding a little lower than the side that
  is rotating.

  Pressing on the opposite side should tend to reduce the angle of the
  caught spool pin. As it straightens out, it will want to turn the plug
  backward, so that the spool end can slide out. This is
  \define{counter-rotation}.
\end{proposition}

\begin{proposition}
  When you feel counter-rotation, ease up a little on the tension, and
  let the spool slide out.
\end{proposition}

\begin{definition}
  A \define{mushroom pin} is very much like a spool pin. However, it
  gracefully tapers on one end. The lower end is still like a spool.
  This is mushroom shaped.
\end{definition}

\begin{proposition}
  The intent of the mushroom design is to make it a little less obvious
  that you are dealing with a spool pin. With a spool pin, when the top
  of the spool clears the shear line, you'll get an ``exaggerated''
  rotation out of the plug. That's because the plug will turn ``too
  far'' as the shear line falls into the narrow part of the spool.

  That helps you know that you're dealing with a spool. The mushroom
  just makes this less clear.
\end{proposition}

% https://www.art-of-lockpicking.com/security-pins/
% http://lockpickernetwork.wikidot.com/security-pins

\section{TODO}

\begin{enumerate}
  \item Serrated Pins.
  % http://lockpickernetwork.wikidot.com/security-pins
  \item Sidebar
  % http://lockpickernetwork.wikidot.com/function-of-sidebars
  \item Paracentric keyway
  \item Rotating pins
\end{enumerate}

\end{document}
