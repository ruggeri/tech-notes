\documentclass[11pt, oneside]{amsart}

\usepackage{geometry}
\geometry{letterpaper}

\usepackage{ned-common}

\renewcommand{\define}[1]{\emph{#1}}
\newcommand{\add}{\text{add}}
\newcommand{\sus}{\text{sus}}
\newcommand{\M}{\text{M}}
\newcommand{\7}{$^7$}

\begin{document}

\title{Music Theory}
\maketitle

\section{Week \#1}

\begin{enumerate}
  \item You can say a sound is ``low'' or ``high'' but that doesn't mean
  they have a pitch - a singable musical quality to sound. Presumably
  low, unpitched sounds consist of a collection of pitches?

  \item They show how the standard notes are named A, B, C, \ldots They
  also show how they are written on the treble clef. They note that the
  treble clef is a stylized G.

  \item They note that octaves sound consonant.

  \item There are seven distinct notes in the diatonic scale
  (\define{heptatonic}). But there are 12 distinct notes in the
  chromatic scale. Successive notes in the chromatic scale are separated
  by a \define{semitone}.

  \item A second is the interval between successive notes in the
  diatonic scale. Likewise we have thirds, fourths, et cetera. They
  mention that the second F-G is a little different than B-C. But they
  don't go deep into that right now.

  \item They introduce C major scale. It is TTSTTTS. The order of
  tones/semitone intervals in a scale gives its \define{quality}. It's
  what makes the scale sound the way it does.

  \item The starting note in a scale is called the \define{tonic}. The C
  major scale is a \define{diatonic} scale. Diatonic means: can be
  played ascending on the white keys of the keyboard. There are seven
  diatonic scales.

  \item They introduce natural minor scale. This is also called the
  Aeolian mode. Ionian is the major scale. A melody played with the same
  notes can sound different if oriented around a different tonic. They
  then discuss all the different modes: Ionian, Dorian, Phrygian,
  Mixolydian, Aeolian, Locrian.

  \item They note that the \define{perfect fifth}, which exists in both
  minor and major, sounds nice. They also note that there is a
  difference in quality between the \define{minor third} and
  \define{major third}. A \define{triad} has three notes. The major
  triad has a perfect fifth and major third. A minor triad has a perfect
  fifth and minor third.

  \item They talk about about the triad built on B: BDF as a
  \define{diminished triad}. It has a \define{diminished fifth}
  (interval between B and F), and a minor third.

  \item They talk about how C, F, G are called \define{tonic,
  subdominant, and dominant} (in key of C). I, IV, V.

  \item They talk about how, in the melody, the melody note is typically
  in some harmonizing chord that ``backs'' it.
\end{enumerate}

\end{document}
