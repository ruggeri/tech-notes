\documentclass[11pt, oneside]{amsart}

\usepackage{geometry}
\geometry{letterpaper}

\usepackage{ned-common}
\usepackage{ned-calculus}

\begin{document}

\title{Ned Electronics}
\maketitle

\section{Batteries}

A \define{battery} is a source of \define{voltage}. One terminal of the
battery wants to send out an electron, while the other terminal of the
battery wants to receive an electron. This is energetically preferred;
the ``free energy'' difference can be harnessed to do work: light a
lightbulb, heat up a resistor, et cetera.

\TODO{Later, I may incorporate my battery notes. These describe how a
battery works at a chemical level.}

It is not necessary for an electron to move from the negative battery
terminal all the way to the positive battery terminal. If connected by a
conductive material (wire), all electrons in the wire can ``shift over''
by one, just like a bike chain can shift over all links by one.

\TODO{It should be explained why some materials conduct, and others do
not. Moreover: we should explain why electrons won't simply fly off
through the air.}

As described, some energy is released/harnessed as an electron moves
from one battery terminal to another. The amount of energy
(\define{Joules}) per coulomb of charge is called the \define{voltage}.
The unit of voltage is the \define{Volt}, which is equivalent to
\define{coulombs of charge per second}.

\section{Current and Resistors}

The simplest way to use a battery is to simply wire up the two
terminals. Electrons will start to flow through the wire.

How ``fast'' do the electrons flow? If we take a cross-section of the
wire, we can measure how many (coulombs of) electrons move across the
cross-section per second. This is called \define{current} or
\define{amperage}. The unit of current is the \define{ampere} or
\define{amp}, and it is equivalent to \define{coulombs of charge per
second}.

Note that current is the ``flow rate'' of electrons; it's not exactly
the ``speed'' at which electrons are moving. Remember that even if the
length of the wire is great, no electron needs to traverse the entire
distance of the wire from one battery terminal to the other. The
electrons may move as links in a chain do.

The next question is: for a given voltage, at what rate do electrons
pass through a given conductor? A wire offers a certain
\define{resistance}. The resistance is measured in \define{ohms}, which
is \define{joules per coulomb per ampere}. That is: the greater the
desired current, the more energy must be expended (per coulomb of
charge) to push it that fast through the wire.

This is an important point: when you wire a battery to a resistor, the
work that is performed in pushing each coulomb of charge through the
resistor \emph{is independent of the resistor's resistance}. This work
done per coulomb is \emph{always} the voltage of the battery. What
changes when you swap out one resistor for a higher resistance wire is
not the energy used to move charges through the resistor, but rather the
\emph{rate} at which those charges move.

One model for this is the \define{Drude model}. Imagine a resistor as a
crowded ballroom. An electron runs into the ballroom with a given
energy. It is trying to make its way out the door at the other end of
the ballroom. But it is moving quickly, and there are people in the way.
The electron bounces into people, giving up energy. The faster the
electron is moving, the harder it is to avoid people. More energy is
needed to push through quickly to the other side.

\TODO{This is inexact, but gives the rough idea. I should return and
make more formal how, if one is content to move more slowly, less energy
is needed.}

The resistance of a wire is determined by both the \define{resistivity}
of the material (how crowded the ballroom is) and (inversely) by the
cross-sectional area of the wire. The greater the cross-sectional area,
the more electrons can be trying to push forward simultaneously. The
resistance is also proportional to the length of the wire (how long a
ballroom to cross).

The relationship between voltage, current, and resistance is described
by \define{Ohm's Law}: $V=IR$.

When you wire two resisters in parallel, then the effective resistance
is halved. This is because to push a current of $I$ through the parallel
pair, only $\frac{I}{2}$ current must go through each resistor. And thus
only half the energy per charge must be used.

\subsection{Battery Capacity And Instantaneous Power Consumption}

No matter the choice of wire/resistance, a battery will always move the
same number of electrons from one terminal to the other. The battery
will keep moving electrons until the chemical inputs to the battery's
redox reaction are expended. Which is to say: the battery has a fixed
energy storage (measured in joules, or sometimes \define{Watt-hours}).

There is a question of \emph{how long} the battery will take to
discharge fully. Put another way: what is the energy consumption per
second? This is called \define{power}, and the unit is the
\define{Watt}. A watt is equivalent to \define{joules per second}.

We know that the voltage is the number of joules released per coulomb of
charge moved, and we know that the amperage is the coulombs of charge
moved per second. Therefore, we know that $P=VI$. By substitution, we
also know that $P=I^2 R$.

Consider what happens when we replace a resistor with resistance $R$
with a resistor with resistance $R' = \half R$. If we keep voltage
constant, the current becomes $I' = \frac{V}{R'} = 2 \frac{V}{R} = 2I$.
Thus the power consumption becomes $P = I'^2 R' = 4 I^2 \half R = 2 I^2
R = 2P$.

Thus, by halving the circuit resistance, we double the power
consumption. The amount of work done per coulomb stays the same, as does
the total work done until battery depletion, but the rate of energy
depletion (power consumption) doubles.

Another way of saying: $P = I^2 R = \frac{V^2}{R}$.

\section{Voltage Divider}

\newcommand{\Rtot}{R_\text{total}}
\newcommand{\Rload}{R_\text{load}}
\newcommand{\Rpot}{R_\text{pot}}

The book has us create a circuit where the input voltage is sent to one
end of a potentiometer, while the load circuit is connected to the
wiper, and ground is connected to the other end of the potentiometer.
This is called a \define{voltage divider}. Question: why bother
connecting the ground to the other end of the pot?

% Graphical example: https://en.wikipedia.org/wiki/Voltage_divider

If the potentiometer is set to $\alpha$, then the resistance in series
with the load circuit is $\alpha \Rpot$ and the resistance in parallel
with the load circuit is $\parens{1 - \alpha} \Rpot$.

When $\alpha = 0$, there is no series resistance. The full voltage is
applied to the load circuit. Note: the voltage drop across two parallel
paths must always be equal. This is because the energy expended by
following each path must be equal. For this reason, when you plug two
light bulbs in parallel, they don't get dimmer. The same current will
pass through both as before, aka the same instantaneous power is being
used by each bulb. So the power draw will double with the two parallel
bulbs.

On the other hand, when parallel paths are added, the effective total
resistance drops. Thus more current can be pushed by the same voltage.
The total current passed is:

\begin{nedqn}
  I_\text{total}
\eqcol
  \frac{V}{R_1} + \frac{V}{R_2}
\\
  \Rtot
\eqcol
  \frac{V}{I_\text{total}}
\\
\eqcol
  \frac{1}{\frac{1}{R_1} + \frac{1}{R_2}}
\end{nedqn}

\noindent
Here we see a downside to the voltage divider. It wastes some power.
Consider that $\alpha = 0$. Then $P = IV = V^2/\Rtot$. But:

\begin{nedqn}
  \frac{1}{\Rtot}
\eqcol
  \frac{1}{\Rload} + \frac{1}{\Rpot}
\intertext{thus}
  P
\eqcol
  \frac{V^2}{\Rload} + \frac{V^2}{\Rpot}
\end{nedqn}

If $\Rpot = \Rload$, then the power draw is two times it would be if the
potentiometer were not there. Half the energy is wasted. In general: the
smaller $\Rpot$ is, the more energy is wasted. This is an argument for
using large potentiometers when doing voltage division.

When $\alpha = 0$, we see that it could be desirable to not wire the
potentiometer at all to ground. In this case, there is no parallel path
to ground, so there would be no power wastage.

We must next consider what happens when $\alpha = 1$. In this
configuration, the potentiometer is entirely in series with the load.
But there is a short-circuit to ground in parallel with the load. Thus
the parallel part of the system has effective resistance zero. The total
resistance of the circuit is effectively $\Rpot$.

Thus, we expect that the voltage drop across $\Rpot$ will be effectively
$V$. There will be practically \emph{no} voltage drop across the either
of the parallel legs in the circuit. The voltage thus applied to the
load is effectively zero.

This shows us the importance of wiring the potentiometer not only to the
load, but also to ground. If not wired to ground, then the effective
resistance of the second half of the circuit is $\Rload$. The total
resistance is $\Rpot + \Rload$. The voltage drop across the pot is then
only $V \frac{\Rpot}{\Rpot + \Rload}$.

By wiring to ground, we may choose $\Rpot$ so that the wasted power is
acceptable. Then, no matter how we set the pot, the power wastage
remains the same. And we can adjust the voltage sent to the load from
literally zero to literally 100\% of the supply voltage $V$.

\section{Switches, Buttons, and Relays}

A mechanical \define{switch} is simply a disconnected wire that can be
\define{thrown} to complete an electronic circuit.

A \define{single throw} switch is simply open or closed. That is: when
the switch is open, there is no electrical connection. When it is
closed, the connection is made.

A \define{double throw} switch is not simply on-or-off. The input wire
is connected either to output A or output B. More complicated switches
can have even more throw positions.

Using a double-throw switch, we can select whether to light up either
light bulb A, or light bulb B. But we can also be fancy. Imagine a light
bulb controlled by \emph{two} switches connected in series. The switches
have two wires connecting corresponding throw positions. Only when both
switches are thrown correspondingly does the circuit complete.

(If you want a \emph{third} light switch, you will have to be more
fancy!)

A \define{double-pole} switch makes two connections simultaneously. A
double-pole, single-throw (DPST) switch is like the one you see in
monster movies. When you throw the switch, two connections go from
broken to completed.

\newcommand{\Ain}{A_\text{in}}
\newcommand{\Bin}{B_\text{in}}
\newcommand{\Aout}{A_\text{out}}
\newcommand{\Apout}{A'_\text{out}}
\newcommand{\Bout}{B_\text{out}}
\newcommand{\Bpout}{B'_\text{out}}

Our relay (AKA \define{bison}) is a form of double-pole, double-throw
(DPDT) switch. It has eight connections: two power the relay. But then
there are two inputs ($\Ain, \Bin$) and two pairs of outputs ($\Aout,
\Apout$ and $\Bout, \Bpout$).

Push buttons can be either normally open (NO) or normally closed (NC).
A doorbell is a NO button.

We've already mentioned our \define{relay}. It is an electromechanical
device. It contains a solenoid that will create a magnetic field whe
electricity runs through it. The magnetic field is used to move an
armature. The armature has electrical contacts that complete the
circuit. We will talk about how solenoids work later.

A relay can be either \define{latching} or \define{non-latching}. A
non-latching relay resets to its original state when voltage is removed.

\section{Capacitors}

A simple \define{capacitor} consists of two parallel metal plates. When
connected to a battery, one plate will accept electrons (building up an
electron excess), while the other will eject electrons (building an
electron deficit).

When the battery is removed, we can then connect the capacitor to a
circuit, and use it as a little battery. Electrons will want to flow
from the side with the excess to the side with the deficit. This
relieves the excess/deficit situation.

``How much'' does the capacitor want to restore balance? First note a
symmetry property: if releasing a coulomb of charge from a capacitor
releases $J$ joules, then pushing a coulomb of charge into the capacitor
\emph{consumes}/stores $J$ joules.

The battery voltage $V$ says that $J$ joules are released if a coulomb
of charge moves across the battery terminals (if a coulomb of charge is
stored in the capacitor). Thus, the capacitor should keep accepting
charges until the ``backpressure'' is also $V$ volts: $J$ joules would
be released if a coulomb of charge leaves the capacitor. In this case,
things would be in equilibrium.

Where does this backpressure come from? It comes from (a) the negative
plate getting crowded with electrons and (b) the positive plate becoming
short of electrons. Surely the backpressure must be going up as the
excess/deficit situation becomes more extreme. But how? Does the
backpressure grow logarithmically, linearly, quadratically with respect
to the number of charges pushed into the capacitor?

\subsection{Elastance/Capacitance}

Let's analyze this. Say that each plate has area $A$, and a total of $Q$
charges have been pushed onto the negative plate. Then the charge
density is $\sigma = \frac{Q}{A}$. Next: if the distance between plates
is small, we can approximate that capacitor plate has infinite area
(with uniform charge density $\sigma$). Feynman proved that, provided
that the plate is of infinite area, the electrical field is constant no
matter the distance from the plate.

\TODO{Can I give a formal proof that the field exerted by an infinite
sheet is uniform?}

Now, imagine how much energy would be released if a single electron were
to be ejected by the infinite-sized plate? As the electron was pushed
away from the plate, we would integrate the electrical force $F$ from
zero to infinity. This would be an infinite amount of energy (befitting
our infinite plate size).

However, note that, while we have not ejected any electrons from the
negative plate, we have, in a sense, ``partially'' ejected them. We've
pushed out electrons from the positive plate. Thus, the only energy to
be harnessed is that of moving an electron from the negative plate to
the positive plate. We should integrate from zero to $d$ the net
electrical force $2F$.

Therefore we see that: the energy required to move a charge into the
capacitor is (1) linearly proportional to the number of charges already
pushed in (increases charge density), (2) inversely proportional to the
capacitor plate area (reduces charge density), (3) linearly proportional
to the plate distance.

\define{Elastance} measures the second two factors. It is proportional
to $\frac{d}{A}$. Notice that it \emph{does not} incorporate the first
factor. Thus, elastance is the same no matter how much charge is pushed
into the capacitor. Note:

\begin{nedqn}
  \diff{V}
\eqcol
  S
  \diff{Q}
\end{nedqn}

That is: the joules required to push a next electron into a capacitor is
equal to $QS$: this grows linearly as the number of charges $Q$ is
pushed in.

So return to our battery hooked up to a capacitor. If the battery has a
voltage of $V$, we know that, in equilibrium, $V = QS$.

We may check out intuition. For a given voltage, we will push more
electrons into the capacitor if the plates are bigger (more space for
the electrons to fit). But we will push fewer electrons onto the plate
if the distance between plates is greater (the ``partial ejection''
effect is less). If the voltage is greater, the number of charges pushed
will also be greater.

It's more common to talk about the inverse of elastance:
\define{capacitance}. Capacitance is measured in \define{Farads}:
coulombs of charge pushed into a capacitor per volt. Since $C =
\inv{S}$, we know from a prior equation that $Q = VC$.

\subsection{Energy Storage In A Capacitor}

Note: $QS$ gives the energy required to store \emph{the next marginal}
coulomb of charge. You \emph{cannot} say that there are $Q$ charges and
$QS$ joules per coulomb of energy thus $Q^2 S$ joules of energy stored
in the capacitor. Instead, you ought to solve:

\begin{nedqn}
  \int_0^Q V(q) \diff{q}
\eqcol
  \int_0^Q Sq \diff{q}
\\
\eqcol
  \parens{\half S q^2} \intevalbar{0}{Q}
\\
\eqcol
  \half S Q^2
\end{nedqn}

\subsection{Charging/Discharging Rate}

We may now begin to ask: how \emph{quickly} does a capacitor reach its
maximum charge $Q = VC$?

It's important to remember that there is some resistance in the wires
that connect the capacitor to the battery. Imagine a resistor $R$ in
series with the capacitor.

As we push more and more charge into the capacitor, more-and-more energy
is required to push each charge. That leaves less-and-less energy
released by the battery available to push through the resistor $R$.
Thus, as the capacitor charges, the current flowing into the capacitor
is dropping.

\newcommand{\vbat}{V_\text{bat}}
\newcommand{\vcap}{V_\text{cap}}

\begin{nedqn}
  \vbat - \vcap(t)
\eqcol
  I(t) R
\\
  \fpartial{q} \vbat
\eqcol
  0
\\
  \fpartial{q} \vcap
\eqcol
  \inv{C}
  \nedcomment{defn of elastance/capacitance}
\\
  \fpartial{t} \vcap
\eqcol
  \fpartial{t} q
  \fpartial{q} \vcap
  \nedcomment{chain rule}
\\
\eqcol
  \fpartial{t} q \inv{C}
\\\eqcol
  I(t) \inv{C}
\\\eqcol
  \frac{\vbat - \vcap(t)}{RC}
\\
  \fpartial{t} \parens{
    \vbat - \vcap(t)
  }
\eqcol
  0 - \frac{\vbat - \vcap(t)}{RC}
\\
  \vbat - \vcap(t)
\eqcol
  C
  +
  \alpha
  \expf{-t/RC}
  \nedcomment{solution of diffeq}
\intertext{
  We know that $\vcap(0) = 0$, so $C + \alpha = \vbat$. Also,
  $\lim_{t\to\infty} \vcap(t) = \vbat$. Thus, $C = 0$. Thus:
}
\\
\\
  \vbat - \vcap(t)
\eqcol
  \vbat
  \expf{-t/RC}
  \nedcomment{solution of diffeq}
\end{nedqn}

\noindent
And now we see where the exponential charging of capacitors comes from!

Note: the time it takes to charge the capacitor to a given percentage of
the battery voltage is based not on the battery voltage itself, but on
the time constant $RC$. The greater the $R$, the more electrons are
delayed on their way into the capacitor. The bigger the $C$, the more
electrons are needed to charge the capacitor by one volt.

\subsection{Danger Of Capacitors}

The energy stored in a capacitor, if discharged through a human, can be
dangerous. Human skin is a pretty effective resistor; a relatively high
voltage is needed to drive significant current through a human.

There is the effect of internal burning due to ``Joule heating:'' the
energy dissipated by a resistor. Assuming fixed body resistance, both
current and time are factors. Though note that the same Joules delivered
over a longer time period may be safer because there is more opportunity
for the body to start to conduct the excess heat away from the
electrical path. (Also note: twice the current delivered over half the
time is still twice the joules delivered. This may make short-duration
high current deceptively dangerous.)

A more significant problem is heart fibrillation. Here, the heart goes
out of rhythm. Apparently even very small currents for small periods can
cause disruption. I don't think the main problem is heat damage to the
heart. It's that the heart goes out of rhythm.

Anyway: capacitors that have not been charged to a high voltage can't
drive very much current. But capacitors powered off mains voltage,
\emph{or} from a low-voltage battery that has been \define{stepped up}
to higher voltage (e.g., lightbulb flash) can be dangerous.

\subsection{Filtering}

A capacitor can be used as a \define{filter}. For instance, let's build
a \define{high-pass filter}. This will let fast alternating voltage go
through the capacitor, but block out low frequency voltage signal.

Simply wire a capacitor in series with a load circuit. Charge starts
coming into plate A and out of plate B. Then the voltage switches. Now
the capacitor \emph{discharges} briefly, and then begins charging in the
opposite direction: it accepts electrons into plate B (it flows out plate
A).

If the input voltage switches quickly relative to the time constant
$RC$, then the capacitor doesn't have time to charge significantly. Put
another way: the capacitor doesn't have time to build up its own stored
voltage. Put another way: the capacitor doesn't have time to start
offering its own resistance to the circuit.

Imagine a slowly alternating input voltage. The capacitor will charge
all the way up, and the current out of the capacitor will stop flowing.
Then, there will be a big spike when the input voltage changes. But
again, upon fully discharging and then charging fully the opposite way,
the capacitor will block any current.

The fast signal is only corrupted a little bit, but the slow signal is
greatly corrupted (``pops'' at each phase change, but quickly attenuated
to zero).

Note that we want to choose a time constant $RC$ that is appropriate for
the desired filtering. However, what if the resistance of the load is
inappropriate to achieve the desired time constant?

One solution is to wire a resistor $R$ in parallel with the load. This
ensures that, no matter the load circuit's resistance, the overall
resistance is never greater than $R$ (it may be less). This ensures that
the time constant is never more than $RC$, no matter the load
resistance. That means that frequencies lower than some threshold $f$
will \emph{always} be filtered out. Depending on how low the load
resistance is, possibly higher frequencies will also be attenuated.

To construct a \define{low-pass filter}, we instead wire the load in
parallel \emph{with the capacitor}. If the rate of voltage change is
slow relative to the time constant, then the capacitor has time to
charge all the way up and all the voltage is delivered to the load. If
the rate of change is fast relative to the time constant, then the
capacitor will steal the excess/deficit voltage by charging, and the
load circuit will not see the change in voltage.

% I think this also shows how you can do an audio cross-over? By
% simply wiring circuits in parallel with either the resistor or the
% capacitor?
%
% Sources:
% https://www.electronics-tutorials.ws/filter/filter_3.html
% https://www.electronics-tutorials.ws/filter/filter_2.html

\section{Tube Diodes}

A \define{diode} is a device that allows current to flow in one
direction, but not the other. Diodes originally consisted of a heated
cathode that throws electrons to a cylindrical anode shell. The entire
thing is encased in a vacuum tube. Thus diodes are sometimes called
\define{tubes}.

The phenomenon that is used is called \define{thermionic emission}. If
you heat a metal enough, it will bounce/``throw'' electrons away. It is
just like when you boil water: the water molecules will want to boil
away. It's not that the heated material prefers stuff to be away; this
is just a byproduct of jostling of the material. If there is no reason
for the molecules to go elsewhere, they may well return.

Here, the heated cathode throws away electrons into the vacuum of the
tube. The vacuum matters, because otherwise the gas molecules have
electrons and would not want thrown electrons to get near them. This is
like how boiling happens more easily at lower pressures.

Since there is a vacuum in the tube, the boiled electrons can bounce
away to the anode. If the cathode and anode have a negative potential
difference between them, the anode is happy to receive the electron.

Question: since the electrons are happier at the anode rather than the
cathode, why is it necessary at all to heat the cathode? Answer:
electrons don't like to simply jump off metal surfaces. They are happy
to move \define{through} metals, but unhappy to jump off of them (even
if they are replaced by another). The energy required to remove an
electron from the surface of a conductor is called the \define{work
function} of the conductor

If voltages are very high, then \define{field electron transmission} can
occur. In that case, there is spontaneous transmission from cold cathode
to anode. But for this to happen, the electric field must be great
enough that the work done by moving away \emph{at the metal's surface}
must exceed the work function. (Note, the electric field is measured in
\define{Volts-per-meter}).

Because only the cathode is heated, a diode can be used for
\define{rectification}. It can change alternating current into pulsing
DC current. This is because electrons cannot travel from the unheated
anode to the cathode. This is similar to a \define{valve} in plumbing.
Thus, diodes are also called valves.

\subsection{Current-to-Voltage Characteristics of a Tube Diode}

With no voltage difference between cathode and anode, still a small
current might run, because some high-energy electrons will be thrown all
the way to the anode. This is called \define{splash current}. I assume
that soon enough the cathode will collect charge and go negative and
start resisting the splash current (returning electrons to the cathode).

If there is a great voltage difference between cathode and anode, the
temperature of the cathode (and thus the rate of thermioinic emission)
becomes the limiting factor. This is the \define{temperature limited
regime}.

Quick question: if the diode is wired into a battery with sufficient
voltage to run the diode at maximum current, then where does the voltage
``go?'' Answer: the electrons are accelerated as they move through the
electric field. This velocity is a form of kinetic energy. When the
electrons smack into the anode, the kinetic energy is lost as waste
heat. The greater the voltage difference between cathode and anode, the
greater the kinetic energy imparted to the electrons when they arrive at
the anode.

To be clear: when wired directly to a battery, the cathode/anode
potential difference is equal to the potential across the terminals of
the battery. That is: the cathode and anode are charged to some $-Q, +Q$
sufficient to create a voltage difference across the plates equal to
$V$, the voltage of the battery.

Note that the diode is \define{non-Ohmic}. Increasing the voltage
difference across the diode plates does not necessarily imply a
proportionate increase in current. In the temperature-limited regime,
increasing the voltage will have no effect on current passed through the
diode. The increased voltage simply means that electrons will arrive
with more velocity at the anode. The power of the heat dissipated at the
anode should be $P=VI$, but $I$ need not vary in proportion with $V$.

To reiterate: where does the electric field come from? It is created in
the same way an electric field is created between capacitor plates. The
battery shoves excess electrons onto the cathode, and steals away
electrons from the anode. It builds up the electric field strength until
the voltage drop by moving an electron through the field becomes equal
to the battery voltage.

\subsection{Space Charge Limited Regime}

Let's consider the intermediate regime where voltage is relatively low.
This is called the \define{space charge limited regime}. As soon as any
voltage is applied, why shouldn't the diode immediately run at the
thermionic emission rate? After all, the electrons are entering an
electric field which is always directed toward the anode, right?

As electrons are thrown from the cathode, a cloud of them starts to
build up between cathode and anode. At points nearer the cathode, the
cloud tends to repel emitted electrons back toward the cathode. At
points nearer the anode, the cloud tends to push the emitted electrons
onward to the anode.

If the anode is at high voltage, then all emitted electrons are able to
``punch through'' the cloud and reach the anode. That's because the net
electric field is consistently directed toward the anode.

But if the anode voltage is low, then the electric field will be net
directed toward the cathode in the region closest to the cathode. This
limits the current toward the anode. Only electrons emitted with a
sufficiently high velocity can make it all the way to the anode. These
electrons will be \emph{decelerated} by the field, until they punch
through the cloud (at which point they are again accelerated).

Note: no matter the voltage we run the diode, no matter the density of
the electron cloud, the voltage drop across the diode is exactly equal
to the kinetic energy imparted per electron moving from cathode to
anode.

I've explained this in hand-wavy terms, but one can study further the
Child-Langmuir Law which describes the current in the space charge
limited regime. This is a ``three-halves'' law: the current grows
proportional to $V^{3/2}$. This is more-or-less linear. It is as if the
diode offers a constant resistance. Anyway, I am satisfied with this.

Again, it is very important to note: the voltage difference describes
the energy per coulomb difference between electrons at the cathode and
electrons at the anode. The rate at which electrons ``punch through''
the electron cloud may be non-linear (proportional to $V^{3/2}$), but
the kinetic energy that the electrons have on arrival to the anode is
always $V$ joules per coulomb, all of which is dissipated as heat.

\subsection{In Series With A Load}

The diode is \define{non-Ohmic}: it does not have a fixed resistance
$R$, and so its output current is nonlinear as a function of the voltage
$V$. If the diode is placed in series with a load, how do we determine
(1) the output current or (2) the voltage drop across the diode? We must
solve the equation:

\begin{nedqn}
  V
\eqcol
  \Delta V_\text{diode}
  +
  I_\text{diode}
  R
\end{nedqn}

This is actually just like what we do with any component, except that
the calculation is more intricate because there is not a simple
relationship between $\Delta V_\text{diode}$ and $I_\text{diode}$.

Note that even when $I_\text{diode}$ is nearly at its limit, $\Delta
V_\text{diode}$ may still be very high (because it is accelerating
electrons very fast). That is: $\Delta V_\text{diode}$ can be
arbitrarily high, even if the output current $I_\text{diode}$ has a
ceiling defined by the thermionic emission rate.

Of course, I once again emphasize that $\Delta V_\text{diode}$ will
be the joules per coulomb that are lost as heat.

If we reduce the resistance $R$, then $\Delta V_\text{diode}$ must
increase. But if we are in the temperature-limited regime, $I$ is not
significantly increased. How, then, does $\Delta V_\text{diode}$
increase? It's because the battery builds up more charge difference
across cathode/anode. Similarly, the electric field from cathode to
anode increases. And, last, the kinetic energy of electrons arriving at
the anode increases.

% Source: https://www.john-a-harper.com/tubes201/
% Source: http://www.r-type.org/articles/art-010b.htm
% Source: https://www.google.com/url?sa=t&rct=j&q=&esrc=s&source=web&cd=&ved=2ahUKEwjPkoqXzPHrAhWJrZ4KHavZDf8QFjABegQIChAE&url=http%3A%2F%2Fphysics.usask.ca%2F~angie%2Fp404%2Freports%2Fexp3%2Fp404-langmuirchild-lab3.doc&usg=AOvVaw00T-G_lNhpf_O7qHI8s_J-
% Source: https://electronics.stackexchange.com/questions/522152/what-causes-voltage-drop-in-a-vacuum-tube-diode

\section{Triodes and Tube Amplifiers}

A \define{triode} is similar to a diode, except there is a screen
between cathode and anode that modulates the current that runs from the
cathode to anode.

If the screen is set at the same voltage as the cathode, then it is as
if it were not present. The triode will work just like a diode. The
current is limited by either the thermionic emission rate or the
electron cloud. We've previously discussed the current-to-voltage
relationship in a thermionic diode.

When the grid is at a negative voltage, it will tend to inhibit the
transit of electrons across the diode. The effect is similar to the
effect of the electron cloud. If the voltage difference across cathode
and anode remains constant, then reducing the grid voltage relative to
the cathode will tend to reduce current.

When we studied the diode, we saw that the analysis was relatively
simple when the voltage across cathode and anode was held constant. The
situation is somewhat more complicated when the voltage difference is
not constant.

Imagine that the triode is in series with a load. If grid voltage is
decreased, then the outflow of electrons $I$ will tend to decrease. Say
that current drops to $I'$. Then the voltage drop across the load is
$I'R$: that is, the voltage drop across the resistor is reduced. This
means that the voltage drop across the triode $\Delta V_\text{triode}$
must increase, since $V = \Delta V_\text{triode} + I'R$ remains
constant.

We know that $\Delta V_\text{triode}$ must rise. Can we see how this
happens by looking at what is happening in the triode?

We know that a triode is basically two parallel plates, charged with
$-Q, +Q$ charge on the cathode/anode. If each plate induces an electric
field of strength $E$, then the net field between plates is uniformly
$2E$. The voltage drop across plates is $2Ed$.

Imagine interposing a middle plate (the control grid) with charge $-Q$.
Between negative plates, the net field is now $E$, and between the
middle and positive plates the field is now $3E$. But everything
integrates out still to a voltage drop of $2Ed$ across plates. In fact,
it doesn't matter the charge of the middle plate. Regardless, the effect
of the middle plate is anti-symmetric, and cancels out.

So why does interposing the intermediate plate of charge $-Q$
\emph{increase} the voltage? The point is: the charge of the
cathode/anode is \emph{not constant}!

We know that interposing the negatively charged control grid will reduce
current $I_\text{triode}$ across the triode, presuming that the voltage
across the plates is held constant. But if $I_\text{resistor}$ stays
constant, then a greater positive charge $+Q'$ is going to build up on
the anode. This will induce a correspondingly greater negative charge
$-Q'$.

This increase in charge difference across the plates will induce a
corresponding increase in voltage drop across the plates. This will tend
to increase $I_\text{triode}$ toward an equilibrium point. At the same
time, energy per charge moved across the plates will increase (it's
equal to the voltage drop across the plates). Thus there is less energy
remaining to push through the resistor. Thus $I_\text{resistor}$ will
fall toward an equilibrium point.

\end{document}

%   \item Here's how grounding works. The third wire leads directly to the
%   earth. This is connected to, for instance, the chassis of the device.
%   If a charge builds on an ungrounded chasis for whatever reason, and
%   you touch touch it, the charge could pass through you to the earth
%   (which I guess wants to accept the charge). But if the chassis is
%   grounded, (1) the charge won't build up, and (2) you aren't a
%   preferred route to discharge the charge because you have higher
%   resistance than a direct connection to Earth.

%   \item 120 volts is standard in the US. 240 volt is for heavy-duty
%   appliances. Household appliances may draw up to 2 amps maybe. Circuit
%   breakers cut amperage at 20 amps.

%   \item \define{Direct current} constantly pushes electrons from side to
%   the other. \define{Alternating current} is generated by generators,
%   and switches the voltage from one side to the other. 60hz is common.
%   Generators induce current by spinning through a magnetic field.

%   \item Right plug of outlet is hot, while left is neutral. The third
%   hole is the electrical ground. Should AC have polarity? I think the
%   idea is that there is a positive/negative draw on the hot plug,
%   relative to both the neutral and ground plugs. But the ground plug is
%   (I think) used to dump current rather than send current through the
%   circuit to the neutral plug.

%   \item LED wants positive power on the longer leg. There is a
%   \emph{forward voltage} limit and a \emph{forward current} limit.
%   Violating either of these will damage the LED.

%   \item The greater resistors, when wired into series with the LED,
%   result in less light from the LED. Less resistance means more light.

%   \item Common breakpoints for resistors: 1.0, 1.5, 2.2, 3.3, 4.7, 6.8,
%   10.0. These correspond to plus-or-minus 20\% tolerance.

%   \item We pry open a potentiometer. We can see it's just a wiper
%   running a long a resistive track. We have a three-prong version. As
%   the wiper gets closer to the left prong, it gets farther from the
%   right prong.

%   \item We then use the potentiometer to control lighting the LED. But
%   we also let it send too much current and burn out the LED. So we
%   really want a resistor plus a potentiometer so that the current never
%   runs above a certain limit.

%   \item An LED with too little voltage will not even light up at all. It
%   needs a threshold voltage. Measuring the LED voltage drop shows it
%   consumes 2.1V when wired with the 470 ohm resistor.

%   \item But if I increase the resistance of the potentiometer so that
%   it's a total 1.47kΩ resistance, then the voltage consumed by the LED
%   drops. If I plug it into the diode measurer, it measures 1.3V drop.

%   \item I believe that the LED needs to consume 1.3V to light at all.
%   That's the threshold. As you try to run more and more current through
%   the LED, the LED will rise in resistance to try to block that current.
%   But if the voltage is too high, the current will be too great to
%   resist, and the LED will break down.

%   \item The LED won't let any current pass until you hit a voltage
%   floor. Then, it will allow current to pass with very little
%   resistance. If you run too much current, you'll burn out the LED. The
%   amount of light that is output by the LED depends on the amount of
%   current being pushed through.

%   % Note: I think that there is no real limit on the voltage applied to
%   % the LED, provided you use an appropriate resistor. I believe the
%   % voltage limit is really just: (1) the voltage floor, plus (2) the
%   % amount of voltage to push the max-safe current through the LED's
%   % intrinsic resistance.
%   %
%   % Source:
%   % https://electronics.stackexchange.com/questions/151627/why-does-an-led-have-a-maximum-voltage/151653

%   \item \TODO{How do LEDs actually work?}

%   \item Note: if a potentiometer is in series with a fixed resistor, the
%   voltage drop across the resistor will increase as we turn down the
%   potentiometer. That's because the battery voltage is constant.

%   \item With the LED in series with potentiometer, as you turn up
%   resistance on the potentiometer, the LED potential drop is
%   less-and-less. That is expected. Once the voltage available to the LED
%   drops below a fixed amount, the current will stop entirely, though.

%   \item I guess that while a resistor consumes energy to push through
%   electrons at a rate, an LED consumes energy to push electrons across
%   \emph{at all}.

%   \item An LED, on the other hand, is like a staircase. You must expend
%   energy to travel up at all. But the speed of travel is irrelevant. An
%   LED then, has a nonlinear resistance. It appears infinite at low
%   voltage, but appears zero above the threshold voltage. Of course, it
%   is not ideal: as current rises, the energy needed to push through will
%   also rise.

%   \item They do a useful calculation. Say you have an LED which is
%   specified to want a forward voltage of 2V, at which point it will pass
%   a current of 20mA. How big a resistor must you use with a 9V battery?

%   \item They say: the resistor should drop the voltage by 7V. It should
%   pass a current of 20mA. If you do the math, you want a resistance of
%   350 ohms.

%   \item They mention that even a low-resistance wire (say: 0.2 ohm) will
%   draw a lot of power if you run a large current (say 15A). This will
%   consume 3V of the voltage. If you're running a 12V battery,
%   approximately 1/4 of the energy is being lost to the wire!

%   \item For reference: a car battery delivers \emph{hundreds of amps}!

% \subsection{Transistors}

% \begin{enumerate}
%   \item They next talk about transistors. The transistor has three legs:
%   \emph{collector}, \emph{base}, and \emph{emitter}.

%   \item Loosely speaking: the effective resistance of a transistor is
%   inversely proportional to the amount of current that passes through
%   the base.

%   \item They do an example where you hook a 9V battery to the
%   collector/emitter. You then complete a circuit by using your finger to
%   bridge the 9V positive terminal with the base. Your finger is like 40M
%   ohm resistance, so the current passing through is tiny. But the
%   transistor allows a much greater current to pass from collector to
%   emitter.

%   \item There are two types of tranistor: NPN (normal kind) and PNP. The
%   NPN switches a positive current, while the PNP switches a negative
%   current. That is: the NPN expects a positive input current at the
%   base/collector. With the PNP, a negative input current is expected at
%   base/collector.

%   \item This is a little different from the relay -- the relay expects a
%   directionality to the current flowing through the inductor that
%   opens/closes the relay, but I don't think it cares which way the
%   current flows through the switch itself.

%   \item The current that is passed by a transistor from collector to
%   emitter is a multiple of the current that is flowing from base to
%   emitter. The amplification factor can be measured by the multimeter
%   and it is about 200.

%   \item You shouldn't try to send too much current through either base
%   or collector. Also, you should not try to send current backward from
%   the emitter to the collector. That can damage the (NPN) transistor. I
%   assume the same is true of sending current backward through the base.

%   \item Transistors have longer term reliability (no moving parts), can
%   switch faster, and can be asily miniaturized. But transistors cannot
%   switch large currents (otherwise they'll be damaged). They also can't
%   switch \emph{alternating current} (because you can't send current
%   backward through them).

%   \item I will have to explain more about how they work later... The
%   book doesn't quite explain.
% \end{enumerate}

% https://en.wikipedia.org/wiki/Bipolar_junction_transistor
% https://en.wikipedia.org/wiki/Boost_converter
% https://en.wikipedia.org/wiki/Crystal_oscillator
% https://en.wikipedia.org/wiki/Electric_generator
% https://en.wikipedia.org/wiki/Electric_motor
% https://en.wikipedia.org/wiki/Field-effect_transistor
% https://en.wikipedia.org/wiki/Inductor
% https://en.wikipedia.org/wiki/Light-emitting_diode
% https://en.wikipedia.org/wiki/Photodiode (uses photoelectric effect)
% https://en.wikipedia.org/wiki/Relay
% https://en.wikipedia.org/wiki/Transformer
% https://en.wikipedia.org/wiki/Varistor
