\section{Characteristics of Force}

\begin{enumerate}

  \item Question: what \emph{is} force? Is it just definitional? What
  makes the concept useful, presumably, is that we have laws for
  calculating forces and thus predicting the acceleration that a mass
  will experience.

  \item Next he considers the drag force on an airplane. At moderate
  speeds this is found to be proportional to $F \sim v^2$. He points out
  this is a different kind of law than $F = ma$, because the drag force
  is really an approximation for the aggregation of a lot of
  \emph{fundamental} forces. One hint that this is not a fundamental law
  is that our approximation for the drag force as proportional to $v^2$
  breaks down at very low $v$. In that case, the drag is more directly
  proportional to $v$.

  \item He next discusses \emph{dry friction}. He mentions that the
  force arrises from bumps on the surfaces being deformed and snapped as
  the sliding occurs. This causes vibration and heat and loss of energy.
  In aggregate, we can empirically find a coefficient of friction $\mu$
  such that $F = \mu N$, where $F$ is the friction force and $N$ is the
  normal force to the surface.

  \item One way to verify this is to take a block and tilt the surface
  it lies on by $\theta$ radians until it slides. No matter the mass of
  the block, you'll always get the same $\theta$.

  \item He notes: friction of ``copper on copper'' is meaningless: pure
  copper placed on pure copper will adhere perfectly! The two pieces
  won't know they are separate! It's \emph{impurities} that cause
  friction to be less than the force required to tear apart copper.

  \item You can verify this with a tumbler on a glass table. Dragging
  the tumbler won't scratch the table, \emph{unless} you ``polish'' the
  table by cleaning it and the tumbler with water. That will greatly
  increase the adhesion, and you'll scratch the table as you drag the
  tumbler. You need more force, and the force is used to tear apart the
  glass...

  \item He now talks about \emph{molecular forces}. He first mentions
  that polar molecules (one where the centers of the positive and
  negative charges do not coincide) are a little complicated.

  \item But consider non-polar molecules. That includes symmetric
  \ce{O2}. At larger distances they attract each other (presumably
  because gravity), whereas at closer distances they start to repel each
  other (presumably because electrostatics).

  \item A maybe important note. At long distances you can pretend that
  nonpolar molecules have no charge. But as you get closer this breaks
  down. It's not like it was every truly valid to find the ``center of
  charge'' and try to calculate the net force just from this. The reason
  is that the force exerted is \emph{not directly proportional} to
  distance, but instead proportional to the square! The problem is just
  presumably greater in polar molecules.

  \item I think a special case would be if you had a spherical shell of
  mass and were considering the gravitational field inside the shell.

  \item Anyway, intermolecular repulsion explains intermolecular
  distance in a substance. It also explains how force is transferred by
  impacts of objects. And it explains friction. It explains why we don't
  fall through the floor.

  \item It also explains \emph{Hooke's law}. That's the force in
  springs. Basically, as you pull on a material and try to pull its
  atoms apart, a force is exerted against you (because attraction). For
  small distances of pulling apart, the force is roughly linear to the
  stretch.

  \item Again, the elastic force (if you want to call it that) is an
  approximation of a complicated aggregate of fundamental forces. The
  approximation will break down if you stretch the spring too much.
  Eventually you'll even just tear the spring apart and it won't be
  restored.

  \item He notes: the more turns of a spring you have per centimeter of
  length, the more stretch you're going to get out of the spring.

  \item But note: the intermolecular force is not \emph{fundamental}
  still.

  \item He now moves on to the next fundamental force (after gravity):
  the \emph{electrostatic force}. It is given by:

  \begin{nedqn}
    \vF
  \eqcol
    \frac{
      q_1 q_2
    }{
      4\pi \epsilon_0
      \normsq{\vr}
    }
    \vecu[r]
  \end{nedqn}

  $\vF$ is the force exerted by charge one on charge two. Here (1) $q_1,
  q_2$ are the charges (in coulombs), (2) $\vr$ is the displacement from
  charge one to charge two, (3) $\vecu[r]$ is simply
  $\frac{\vr}{\norm{\vr}}$ (the direction).

  Oh yeah, $4 \pi \epsilon_0$ is just the name of a constant. It's
  historical he says.

  \item He introduces the idea of a \emph{field of force}. The
  \emph{electric field} is simply:

  \begin{nedqn}
    \vE
  \eqcol
    \frac{
      q_1
    }{
      4\pi \epsilon_0
      \normsq{\vr}
    }
    \vecu[r]
  \end{nedqn}

  Note that $\vE$ is a function of location $\vr$ (that's what makes it
  a ``field'').  Now you can easily find the force that would be exerted
  on a second charge $q_2$ at position $\vr$ (offset from the first
  charge) by multiplying $q_2 \vE(\vr)$.

  You can likewise define a \emph{gravitational field}.

  \item Of course, the net field can be found simply via summation. This
  is a principle of \emph{superposition}. However, while superposition
  appears to be exactly true for electrical fields, there is some
  breakdown (because relativity) when doing superposition for
  gravitational fields. Of course, you need really great gravities for
  this.

  \item He now begins talking about \emph{magnetic} force. He talks
  about shooting electrons out of an ``electron gun'' to a fluorescent
  screen, but bending the the electrons with a magnet.

  \item He thus introduces a new field; the \emph{magnetic field}
  $\vB$. At present we won't talk about \emph{where} this field comes
  from. But he \emph{does} talk about \emph{how} the magnetic field
  \emph{affects} moving electrons.

  The force caused by the magnetic field upon the electron is normal to
  both the direction of travel and the direction of the magnetic field.
  A ``right hand rule'' is used to determine the direction of the force.

  Clearly this definition would degenerate if the magnetic field is
  directed in the same direction as the velocity of the electron. Thus,
  unsurprisingly, the magnitude of the force is proportional to (a) the
  magnitude of the magnetic field, (b) the magnitude of the velocity,
  and (c) $\sin \theta$ (where $\theta$ is the angle between the two).
  Note that this magnitude is sort of the ``opposite'' of the dot
  product.

  Anyway, we've just described the \emph{cross product}.

  To summarize, the \emph{Lorentz force} (the combined force of
  the electric and magnetic fields) on a moving electron is:

  \begin{nedqn}
    \vF
  \eqcol
    q \vE
    +
    q \vv \times \vB
  \end{nedqn}

  Anyway, this is all very interesting because it's one of the first
  times we've seen that a force depends not on the \emph{position} of
  things, but their \emph{relative velocity}.

  \item He repeats the demonstration translation/rotation doesn't change
  the fundamental laws. He further shows (very simply of course) that
  observing while traveling at a fixed velocity changes nothing. This
  completes Galilean invariance.

  \item Technically: he shows that if the forces are the same, then it
  the acceleration is invariant under Galilean transformation. But he
  hasn't quite shown the \emph{forces stay the same}. For instance, if
  you are moving, it'll look like a still electron is moving, too.
  Luckily, it'll also seem like the magnetic field is changin, so you
  should have no net impact, but he doesn't show this!

  \item But now he considers what happens when you are
  \emph{accelerating}. This makes everything look as if there was a
  force acting opposite your acceleration! Feynman calls this a ``pseudo
  force.'' Other people call these \emph{fictitious forces}. The
  \emph{centrifugal force} is an example of a pseudo force.

  \item Btw, an accelerating reference frame is called a
  \emph{non-inertial frame of reference}.

  \item (From Wikipedia) Acceleration can be anything, but the most
  commonly studied fictitious forces arrise from: (a) rectilinear
  acceleration (acceleration in a straight line), (b) centrifugal force
  (acceleration directed inward to the point of a circle), (c) Coriolis
  force (also from circular motion, but directed directed opposite the
  direction of rotation - perpindicular to the centrifugal force), (d)
  Euler force (something about changing rate of rotation).

  \item Feynman gives another (rectilinear) example: imagine
  accelerating a glass of water on the table. The water will slosh
  backward as if pulled against your acceleration.

  \item When you are in an accelerating frame of reference, it will look
  like there is a net force in the universe that you can't explain. It
  will look like there's some kind of gravity, for instance, pulling
  everything down against the direction of your acceleration.

  \item Einstein noticed: could our perception of gravity be caused
  simply by an accelerating reference frame? That is: could
  \emph{gravity} be a fictitious force? But that wouldn't work: if we
  think we're experiencing gravity from the earth because the Earth is
  accelerating us, why wouldn't people \emph{on the other side} just
  fall off?

  \item To preserve this idea, Einstein then wondered what might happen
  if the geometry of the universe was a little weird...

  \item Last he mentions the nuclear forces. I don't even know what was
  known of these forces. He doesn't use the words \emph{strong} or
  \emph{weak} forces. Anyway, these are forces that hold together the
  nucleus.

\end{enumerate}

\TODO{Can we always detect non-inertial frame of reference? I believe
the answer is yes?}
