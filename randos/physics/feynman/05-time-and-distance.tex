\section{Time and Distance}

\begin{enumerate}
  \item He mentions how to ``define'' time in terms of measuring it. For
  instance: you can say time is defined by how long it takes an
  hourglass to drain its sand.

  \item He mentions that we use pendulums to measure time in a
  grandfather clock. We can also use electrical oscillators. He mentions
  that below a certain limit, the concept of time may make no sense in
  the sense that it is impossible to measure it.

  \item He mentions radiocarbon dating.

  \item Next he discusses distance. He discusses direct measurement,
  using triangulation, using radar (which involves measuring time and
  assuming a velocity of the EM wave).

  \item He mentions how you might measure the \emph{cross sectional
  area} of a nucleus. You send a beam of particles through a material,
  and see how many get through. You have to do a little math because
  even a thin material will have many layers of atoms. The more layers,
  of course the more absorbtion. But because nucleii are spherical,
  there is a formula. (BTW I think this is what Fermi did.)

  \item He finally notes that measurements of distance and time will be
  \emph{relative} to the motion of an observer.

  \item And also: the uncertainty principle has already told us about a
  relationship between certainty in distance and certainty in momentum.
  It turns out this has implications for our ability to reliably measure
  time.
\end{enumerate}
