\subsection{Coriolis Force For Tangential Travel}

\begin{enumerate}
  \item We are asked to now consider the force experienced when walking
  in a circle along the rim of a spinning carousel. That is: tangential
  motion, rather than radial motion. We are walking at velocity $v'$
  relative to the rotating frame of reference, and the carousel itself
  is traveling at velocity $v$. So an outside observer knows that the
  centripetal force must be:

  \begin{nedqn}
    F
  \eqcol
    m (v + v')^2 / r
  \\
  \eqcol
    m v^2/r + 2m v v'/r + m v'^2 /r
  \end{nedqn}

  \item The first force is the centripetal force we'd experience even if
  standing still on the carousel. The last force is the centripetal
  force we'd experience from walking in a circle even if the carousel
  were not rotating.

  \item The middle term is interesting. Replace $v/r$ with $\omega$.
  Then this magnitude is $2m \omega v'$. This is the Coriolis force.

  \item \TODO{I definitely feel a little less sure about this force.} I
  guess I would say; if you start to run along the outside of the
  carousel, you will feel like you are struggling not to fall into a
  greater radial orbit about the center. It is as if you are being
  pulled outward. \TODO{I don't know, I feel like I understand this last
  part very poorly.}
\end{enumerate}
