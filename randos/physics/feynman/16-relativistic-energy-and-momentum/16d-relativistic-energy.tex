\subsection{Relativistic Energy}

\begin{enumerate}

  \item Let's now consider an \emph{inelastic collision}. In such,
  energy is ostensibly not conserved, though momentum will be.

  Imagine two identical objects moving at velocity $v$ hitting each
  other. Each object has mass $m_v = m_0/\sqrt{1 - v^2/c^2}$.

  \item What is the mass of the formed object? The formed object is
  motionless. Should it not therefore be $2m_0$? But if this happened,
  then mass would not be conserved. So not only is energy not being
  conserved: neither is mass!

  \item The formed object should have mass $2m_v$. Why? The components
  of the composite object cannot simply come to a stop. Their kinetic
  energy must go somewhere. It goes into the temperature of the formed
  object. Since the constituents are moving, they are not at rest. So
  they have excess mass above their rest mass.

  \item This is showing us that conservation of mass and conservation of
  energy are two sides of the same coin. If energy were lost in a
  collision, so would mass be lost.

  \item Let's consider an example in the other direction. Say some
  potential energy is stored in an object of mass $M$. You split the
  object into two parts, and they each rocket out at a velocity $v$ and
  a mass of $m_v$. Conservation of mass/energy requires that $M = 2m_v$.

  The objects pass through a medium, which retards it. The objects
  impart energy to the medium as they slow down. The constituent objects
  come to rest. They now have mass $m_0$.

  How much energy was transferred? Simple! It is exactly $2(m_v - m_0)$!

  \item One expects that the lost mass will be found elsewhere: in the
  medium to which the energy was transferred!

  \item He notes that you could in theory use measurements of mass
  before and afterward to determine the amount of energy stored in the
  chemical bonds of molecules.

\end{enumerate}
