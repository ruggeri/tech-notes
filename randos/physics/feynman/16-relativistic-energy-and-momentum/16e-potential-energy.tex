\subsection{Potential Energy}

\begin{enumerate}

  \item We now know that when an object A gains (loses) kinetic energy,
  it gains (loses) mass.

  If object A expends kinetic energy, and this kinetic energy is put
  into another object B, we know that object B will gain mass. Overall,
  the system conserves mass.

  \item Also, energy is conserved. The energy that object A has now
  exists inside object B as kinetic energy. The reason that object B has
  increased in mass is because of this ``stored'' kinetic energy.

  \item Can energy be stored other than as kinetic energy? If so, this
  stored energy must still increase the mass of object B if we are to
  conserve mass.

  But it is not clear to us how this should be. Relativity shows us
  clearly that \emph{stored kinetic energy} results in an increase in
  mass. But we never spoke of \emph{other} forms of stored energy!

  \item I guess my question is: how do other forms of energy storage fit
  in with the theory of special relativity? Presumably they increase
  mass? Or is mass not actually a conserved quantity?

  Or is it simply that there \emph{are no other forms} of energy
  storage? Was it always a convenient shorthand? \TODO{Return to this
  question!}

  \item Feynman notes that without knowing what is inside an object,
  it's hard to say how much of its energy is rest energy versus kinetic
  energy. What we do know from the outside is that the \emph{total}
  energy is $Mc^2$.

  He suggests that idea of ``inside'' might be spurious. He says a
  K-meson can disintigrate into either two pions \emph{or} three pions.

\end{enumerate}
