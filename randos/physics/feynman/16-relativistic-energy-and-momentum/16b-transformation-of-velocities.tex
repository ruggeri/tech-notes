\subsection{Lorentzian Transformation of Velocities}

\begin{enumerate}

  \item We next explore transformations of velocities. Let's first
  recall the equations to translate deltas in $x'$ and $t'$:

  \begin{nedqn}
    \Delta x'
  \eqcol
    \frac{
      \parens{x_1 - ut_1} - \parens{x_0 - ut_0}
    }{
      \sqrt{1 - u^2/c^2}
    }
  \\
  \eqcol
    \frac{\Delta x - u \Delta t}{
      \sqrt{1 - u^2/c^2}
    }
  \\
    \Delta t'
  \eqcol
    \frac{
      \parens{t_1 - ux/c^2} - \parens{t_0 - ux/c^2}
    }{
      \sqrt{1 - u^2/c^2}
    }
  \\
  \eqcol
    \frac{
      \Delta t - u\Delta x/c^2
    }{
      \sqrt{1 - u^2/c^2}
    }
  \end{nedqn}

  \item Let's consider an object A traveling at $v_y$ with no $x$
  velocity. Let's now consider observing it from a spaceship travelling
  with a velocity $u$ along the $x$ axis:

  \begin{nedqn}
    v_y'
  \eqcol
    \frac{\Delta y'}{\Delta t'}
  \\
  \eqcol
    \Delta y
    \frac{
      \sqrt{1 - u^2/c^2}
    }{
      \Delta t - u\Delta x/c^2
    }
  \\
  \eqcol
    v_y \sqrt{1 - u^2/c^2}
  \end{nedqn}

  This scenario is somewhat somewhat simple. Dilation of $x$ space is
  irrelevant, so the only factor involved is the time dilation effect.

  The new velocity $v_x'$ is also easy to analyze. It's easy beacuse
  $\Delta x = 0$.

  \begin{nedqn}
    v_x'
  \eqcol
    \frac{\Delta x'}{\Delta t'}
  \\
  \eqcol
    \frac{
      \Delta x - u \Delta t
    }{
      \Delta t - u \Delta x /c^2
    }
  \\
  \eqcol
    \frac{-u \Delta t}{\Delta t}
  \\
  \eqcol
    -u
  \end{nedqn}

  \item We can consider the change in velocity if the object A was
  travelling parallel to the $x$ axis:

  \begin{nedqn}
    v_x'
  \eqcol
    \frac{\Delta x'}{\Delta t'}
  \\
  \eqcol
    \frac{
      \Delta x - u \Delta t
    }{
      \Delta t - u \Delta x / c^2
    }
  \\
  \eqcol
    \frac{
      v_x - u
    }{
      1 - u v_x / c^2
    }
  \end{nedqn}

  Again that seems reasonable. If we match $u$ to equalize $v_x$ we find
  no more velocity $v_x'$. Likewise, if $v_x$ started out as zero, then
  as expected we find $v_x' = -u$. Notice that $v_x'$ can never be
  greater than $c$ (provided that $u, v_x$ are both less than $c$).

  Our finding for $v_y'$ is also reasonable:

  \begin{nedqn}
    v_y'
  \eqcol
    \frac{
      \Delta y'
    }{
      \Delta t'
    }
  \\
  \eqcol
    \frac{0}{\Delta t'}
  \\
  \eqcol
    0
  \end{nedqn}

  Which is as expected.

  \item Let's next examine what happens when neither $v_x, v_y$ are
  zero. Can we use our equations as before to treat these separately? It
  turns out not exactly. The math for $v_x'$ will work out as before.
  But let's next consider $v_y'$:

  \begin{nedqn}
    v_y'
  \eqcol
    \frac{
      \Delta y'
    }{
      \Delta t'
    }
  \\
  \eqcol
    \Delta y
    \frac{
      \sqrt{1 - u^2/c^2}
    }{
      \Delta t - u\Delta x/c^2
    }
  \\
  \eqcol
    v_y
    \frac{
      \sqrt{1 - u^2/c^2}
    }{
      1 - uv_x/c^2
    }
  \end{nedqn}

  We can see that when $v_x = 0$, this is just the same equation we had
  found before.

\end{enumerate}
