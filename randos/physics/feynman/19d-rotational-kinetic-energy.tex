\subsection{Rotational Kinetic Energy}

\begin{enumerate}
  \item We can ask: what is the rotational kinetic energy of a point
  mass? That is, what is the rotational kinetic energy of an point mass
  moving at speed $\omega$? Well, we know that:

  \begin{nedqn}
    \text{KE}
  \eqcol
    \half m v^2
  \\
  \eqcol
    \half m \parensq{\omega r}
  \\
  \eqcol
    \half I \omega^2
  \end{nedqn}

  \item Note that this is invariant with respect to origin, because $I$
  will scale appropriately to offset changes in $\omega^2$.

  \item We can generalize this further to break down the kinetic energy
  of a rigid body into two parts. Presume that a body is rotating with
  angular velocity $\vomega$ about its CoM $\vx_\CoM$. WLOG, assume that
  $\vx_\CoM = 0$. We allow that the CoM has velocity $\vv_\CoM$. Then:

  \begin{nedqn}
    \text{KE}
  \eqcol
    \sum_i \half m_i \normsq{\vv_i}
  \\
  \eqcol
    \sum_i \half m_i \normsq{\vomega \cross \vx_i + \vv_\CoM}
  \\
  \eqcol
    \sum_i
      \half m_i
      \parens{\vomega \cross \vx_i + \vv_\CoM} \cdot
      \parens{\vomega \cross \vx_i + \vv_\CoM}
  \\
  \eqcol
    \sum_i
    \half m_i
    \parens{
      \normsq{\vomega}\normsq{\vx_i}
      + 2 \parens{\vomega \cross \vx_i} \cdot \vv_\CoM
      + \normsq{\vv_\CoM}
    }
  \\
  \eqcol
    \parens{
      \sum_i \half m_i \normsq{\vx_i} \normsq{\vomega}
    }
    + \parens{
      \sum_i m_i \parens{\vomega \cross \vx_i} \cdot \vv_\CoM
    }
    + \parens {
      \sum_i \half m_i \normsq{\vv_\CoM}
    }
  \intertext{
    We show in a moment why, but the second term can be eliminated:
  }
  \eqcol
    \half I_\CoM \normsq{\vomega} + \half M \normsq{\vv_\CoM}
  \\
  \eqcol
    \text{KE}_\text{rotational} + \text{KE}_\text{translational}
  \end{nedqn}

  \noindent
  And now I show the elimination of the second term:

  \begin{nedqn}
    \sum_i m_i \parens{\vomega \cross \vx_i} \cdot \vv_\CoM
  \eqcol
    \parens{
      \vomega \cross \parens{\sum_i m_i \vx_i}
    } \cdot \vv_\CoM
  \\
  \eqcol
    \parens{\vomega \cross \vec{0}} \cdot \vv_\CoM
  \\
  \eqcol
    \vec{0}
  \end{nedqn}

  \item As ever, this breakdown will only be clean by choosing to
  consider rotation about the center of mass.

  \item Consider again the skater who moves their leg inward as they
  spin. We know their angular momentum has been conserved, but that
  their moment of inertia has gone down, and thus their angular velocity
  $\omega$ has gone up by the same proportion.

  The implication, though, is that there is more kinetic energy in the
  system. How did this happen? Well, note that, to maintain rotation, a
  centripetal force must be applied to the outside foot. This force is
  required to maintain circular motion.

  When we bring the leg in, we apply this force through a (radial)
  distance. Thus the change in kinetic energy is (1) the magnitude of
  the centripetal force, (2) times the change in the radius.
\end{enumerate}
