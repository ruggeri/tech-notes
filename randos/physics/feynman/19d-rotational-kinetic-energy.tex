\subsection{Rotational Kinetic Energy}

\begin{enumerate}
  \item We can ask: what is the rotational kinetic energy of an object?
  That is, what is the rotational kinetic energy of an object moving at
  speed $\omega$? Well, we know that:

  \begin{nedqn}
    \text{KE}
  \eqcol
    \half m v^2
  \\
  \eqcol
    \half m \parensq{\omega r}
  \\
  \eqcol
    \half I \omega^2
  \end{nedqn}

  \item Consider again the skater who moves their leg inward as they
  spin. We know their angular momentum has been conserved, but that
  their moment of inertia has gone down, and thus their angular velocity
  $\omega$ has gone up by the same proportion.

  The implication, though, is that there is more kinetic energy in the
  system. How did this happen? Well, note that, to maintain rotation, a
  centripetal force must be applied to the outside foot. This force is
  required to maintain circular motion.

  When we bring the leg in, we apply this force through a (radial)
  distance. Thus the change in kinetic energy is (1) the magnitude of
  the centripetal force, (2) times the change in the radius.
\end{enumerate}
