\subsection{Rotational Inertia}

\begin{enumerate}
  \item Consider a body that is rigid. A torque $\tau$ is applied at
  some point. We know that there may be internal force pairs, but we
  also know these are irrelevant. The instantaneous change in angular
  momentum $L = \sum_i L_i$ is $\tau$.

  \item What is the angular acceleration of the object? Consider:

  \begin{nedqn}
    \tau
  \eqcol
    \fpartial{t} L
  \\
  \eqcol
    \sum_i \fpartial{t} L_i
  \\
  \eqcol
    \sum_i \fpartial{t} \parens{m_i r_i^2 \omega_i}
  \\
  \eqcol
    \sum_i m_i r_i^2 \fpartial{t} \omega_i
  \\
  \eqcol
    \sum_i m_i r_i^2 \alpha_i
  \intertext{but note that all $\alpha_i$ are the same, so:}
  \eqcol
    \parens{\sum_i m_i r_i^2} \alpha
  \\
  \eqcol
    I \alpha
  \end{nedqn}

  \noindent
  where $I = \sum_i m_i r_i^2$ is the total rotational inertia.

  \item Great. We make a final note. We know that the translational
  momentum $\vp$ is conserved. We know that $\vp = M \vv$. And $M$ is
  assumed fixed. So $v$ cannot change.

  \item On the other hand, consider the angular momentum $L$. This is
  also conserved. We know that $L = I \omega$. But $I = \sum_i m_i
  r_i^2$. We cannot change $m_i$, but we \emph{can} change $r_i$! Thus
  we can change $I$!

  This is what a skater does in a spin. They start with their leg far
  out. Then, they bring their leg in toward their center of mass. This
  decreases the rotational inertia $I$ about the axis. Their angular
  momentum $L$ will be conserved, and this occurs by their angular
  velocity $\omega$ increasing to compensate!
\end{enumerate}
