\subsection{Coriolis Force For Tangential Travel}

\begin{enumerate}
  \item Next, we may consider the force experienced by walking along a
  circular path around a spinning carousel. Say the carousel spins with
  angular velocity $\omega_0$, and you want to walk around the central
  point with (apparent) angular velocity $\omega_1$.

  \item Note first that even if we are standing still on the carousel,
  we will experience the fictional \define{centrifugal} force, which is
  equal and opposite to the centripetal force that keeps redirecting our
  true tangential velocity us and keeps us rotating about the axis. This
  is $m r \omega_0^2$.

  \item Also, if the carousel were \emph{not} rotating, we expect to
  need to exert a centripetal force of $m r \omega_1^2$ to maintain a
  circular motion about a central point. (Incidentally, this is why we
  lean in toward the rotational axis when we are running in a circle).

  \item But, from the perspective of the spinning person, you must apply
  an \emph{additional} force to move along a circle painted on the
  carousel. We can calculate it:

  \begin{nedqn}
    F_\text{Centripetal}
  \eqcol
    m r \omega^2
  \\
  \eqcol
    m r \parensq{\omega_0 + \omega_1}
  \\
  \eqcol
    m r \omega_0^2 + 2m r \omega_0\omega_1 + mr \omega_1^2
  \\
  \eqcol
    F_{0, \text{Centripetal}}
    + 2m r \omega_0\omega_1
    + F_{1, \text{Centripetal}}
  \end{nedqn}

  For a person walking about the carousel, they've always felt a
  fictitious centrifugal force which they've ``overcome'' by applying a
  true (and needed, for circular motion) force $F_{0,
  \text{Centripetal}}$. And, in order to walk in a circle, they must
  apply the usual centripetal force $F_{1, \text{Centripetal}}$.

  But they may not expect that they need to apply a further force inward
  of $2m r \omega_0 \omega_1$. Rewriting with $v_\text{Tangential} =
  \omega_1 r$, we have that a force of magnitude $2m \omega_0
  v_\text{Tangential}$ is required.

  \item To the rotating observer on the carousel, this needed extra
  force is felt as necessary to oppose and negate a fictitious Coriolis
  force. If the observer wishes to talk \emph{with} the rotation of the
  carousel, the Coriolis force pulls them outward, whereas if the
  observer wishes to walk \emph{against} the rotation of the carousel,
  the Coriolis force pulls them inward.

  \item We see once again that the Coriolis force is perpendicular to
  $\vomega$ and $\vv$. In fact, one expects that it is equal to $-2
  m\vomega \cross \vv$.
\end{enumerate}
