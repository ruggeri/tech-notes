\subsection{Circular Motion And Forces}

\begin{enumerate}
  \item I next take a quick break to derive the centripetal force
  required to keep an object traveling at angular speed $\omega$ and
  radius $r$ in a circle.

  \begin{nedqn}
    x(t)
  \eqcol
    r\expf{i \omega t}
  \\
    v(t)
  \eqcol
    \omega i x(t)
  \\
    a(t)
  \eqcol
    - \omega^2 x(t)
  \end{nedqn}

  \noindent
  thus to maintain circular motion, we must apply a force $F$ directed
  inward with magnitude as follows:

  \begin{nedqn}
    F
  \eqcol
    m \omega^2 r
  \\
  \eqcol
    m \parensq{v/r} r
  \\
  \eqcol
    m v^2/r
  \end{nedqn}

  \item We may slightly generalize. We should allow that the center be
  moved from the origin. We simply modify that $x(t) = c + \expf{i
  \omega t}$, where $c$ is the center of rotation. The rest stays the
  same.

  \item We may wish to further generalize from two-dimensional space
  (where we used a basis of $1, i$) to a three-dimensional space (with
  basis $\vi, \vj, \vk$). I am at present too lazy to do this. I don't
  believe a simple exponential form can be used, because the exponential
  function is not typically extended to having arbitrary vector-space
  inputs. This is because the exponential function is typically defined
  in terms of a multiplication operation, and there is no typical
  multiplication on vectors.

  \item Can you not use cross product as a multiplication? While cross
  product is neither commutative nor associative, it is
  \emph{anticommutative} and obeys the Jacobi identity (which I won't
  describe right now). This means that you can treat $\R^3$ as a
  \define{Lie algebra}, and there is some work to define the exponential
  function for Lie algebras.

  \item Note that a vector space with a bilinear multiplication is
  called a \define{algebra}. There are associative and non-associative
  algebras. Lie algebras are a kind of non-associative algebra. A
  classic commutative algebra would be polynomials over a field, with
  the traditional polynomial multiplication. Note that an algebra
  typically doesn't contain vector inverses. An algebra can either be
  over a field (typical), or over a \define{ring} (scalars don't always
  have multiplicative inverses). This is called an \define{algebra over
  a ring} for purpose of distinction.

\end{enumerate}

\subsection{Coriolis Force For Radial Travel}

\begin{enumerate}

  \item Feynman notes: consider just your torso when moving a leg in.
  The torso angular velocity has increased, even though its inertia has
  not changed (the torso didn't move; only the leg). What happened to
  the torso when the leg was moved in? There must be some torque
  applied!

  \item Note that this torque cannot arise from the centripetal force,
  because the centripetal force is radial, and radial forces do not give
  torque.

  \item As the leg moves inward, it wants to increase its angular
  velocity so that angular momentum is conserved. But as the leg starts
  to speed up, it encounters the rotational inertia of the core body.
  The leg wants to run ahead of the body, but of course it is attached
  to the body. Thus the central body gets a push from the leg. The leg,
  for its part is pushed backward by the body.

  \item Imagine things this way: a central part and a more distant
  connected part have equal inertia $I$. They rotate at the same rate
  $\omega$. Then they are disconnected. They continue to rotate freely
  at $\omega$. The outer part goes then through these changes:

  \begin{nedqn}
    r \mapstocol r/2
  \\
    I \mapstocol I/4
  \\
    \omega \mapstocol 4\omega
  \\
    v \mapstocol 2v
  \end{nedqn}

  \noindent
  Now reconnect the two parts. It's clear that the core part is ``going
  around too slowly'' while the outer part is ``going around too
  quickly.'' We need to conserve $2I_0 \omega = I' \omega' = \frac{5}{4}
  I_0 \omega'$. That is: $\omega' = \frac{8}{5} \omega$. So there must
  be a torque against the outer part (slow it down) and a torque with
  the inner part (speed it up).

  \item The force that gives rise to this is called the \define{Coriolis
  force}. Like the centrifugal force, it is ``fictitious.'' It only
  appears when we try to move things in a rotating frame of reference.

  \item Feynman gives an example of a carousel. Say you are pushing a
  box of mass $m$ radially on a carousel. Assume that the carousel's
  rotational speed remains constant at $\omega$ (e.g., assume the
  carousel's inertia is much greater than the box's inertia of $m r^2$).

  As you push the box away from the center, the box is gaining angular
  momentum. There must be a torque applied; therefore there must be a
  tangential force. A `fictitious' one, but a force nonetheless.
  Consider:

  \begin{nedqn}
    L
  \eqcol
    I \omega
  \\
    L
  \eqcol
    m r^2 \omega
  \\
    \fpartial{t} L
  \eqcol
    2 m r \omega v_r
  \intertext{and since}
    \tau
  \eqcol
    \fpartial{t} L
  \intertext{we know}
    \tau
  \eqcol
    F r
  =
    2mr \omega v_r
  \\
    F
  \eqcol
    2m \omega v_r
  \end{nedqn}

  \noindent
  We call this the \define{Coriolis force}. Thus as you walk radially on
  a carousel, you will feel yourself leaning against the rotation, as
  your feet are accelerated out from underneath you.
\end{enumerate}
