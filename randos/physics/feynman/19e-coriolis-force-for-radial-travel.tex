\subsection{Circular Motion And Forces}

\begin{enumerate}
  \item I next take a quick break to derive the centripetal force
  required to keep an object traveling at angular speed $\omega$ and
  radius $r$ in a circle.

  \begin{nedqn}
    x(t)
  \eqcol
    r\expf{i \omega t}
  \\
    v(t)
  \eqcol
    \omega i x(t)
  \\
    a(t)
  \eqcol
    - \omega^2 x(t)
  \end{nedqn}

  \noindent
  thus to maintain circular motion, we must apply a force $F$ directed
  inward with magnitude as follows:

  \begin{nedqn}
    F
  \eqcol
    m \omega^2 r
  \\
  \eqcol
    m \parensq{v/r} r
  \\
  \eqcol
    m v^2/r
  \end{nedqn}

  \item We may slightly generalize. We should allow that the center be
  moved from the origin. We simply modify that $x(t) = c + \expf{i
  \omega t}$, where $c$ is the center of rotation. The rest stays the
  same.

  \item We may wish to further generalize from two-dimensional space
  (where we used a basis of $1, i$) to a three-dimensional space (with
  basis $\vi, \vj, \vk$). I am at present too lazy to do this. I don't
  believe a simple exponential form can be used, because the exponential
  function is not typically extended to having arbitrary vector-space
  inputs. This is because the exponential function is typically defined
  in terms of a multiplication operation, and there is no typical
  multiplication on vectors.

  \item Can you not use cross product as a multiplication? While cross
  product is neither commutative nor associative, it is
  \emph{anticommutative} and obeys the Jacobi identity (which I won't
  describe right now). This means that you can treat $\R^3$ as a
  \define{Lie algebra}, and there is some work to define the exponential
  function for Lie algebras.

  \item Note that a vector space with a bilinear multiplication is
  called a \define{algebra}. There are associative and non-associative
  algebras. Lie algebras are a kind of non-associative algebra. A
  classic commutative algebra would be polynomials over a field, with
  the traditional polynomial multiplication. Note that an algebra
  typically doesn't contain vector inverses. An algebra can either be
  over a field (typical), or over a \define{ring} (scalars don't always
  have multiplicative inverses). This is called an \define{algebra over
  a ring} for purpose of distinction.

  \item Note that the required centripetal force is quadratic in
  $\omega$ and linear in $r$, while momentum is quadratic in $r$ and
  linear in $\omega$. If momentum is conserved while $r$ changes, this
  implies that $F$ cannot stay constant. For instance, if $r$ is halved,
  then $\omega$ must be quadrupled. But then we see that the centripetal
  force must be increased by eight times.

  Thus, we cannot say that, if the centripetal force required to hold a
  mass at distance $r$ in circular motion is $F$, then a change $\Delta
  r$ to the radius results in $F \Delta r$ work being performed. This
  would undercalculate the work needed to bring reduce the angular
  inertia of the body. And if one plugged this incorrect work into the
  equation for rotational kinetic energy, they would underestimate the
  corresponding change in rotational speed.
\end{enumerate}

\subsection{Coriolis Force For Radial Travel}

\begin{enumerate}
  \item Feynman notes: consider just your torso when moving a leg in.
  The torso angular velocity has increased, even though its inertia has
  not changed (the torso didn't move; only the leg). What happened to
  the torso when the leg was moved in? There must be some torque
  applied!

  \item Note that this torque cannot arise from the centripetal force,
  because the centripetal force is radial, and radial forces do not give
  torque.

  \item As the leg moves inward, it needs to increase its angular
  velocity so that angular momentum is conserved. But as the leg starts
  to speed up, it encounters the rotational inertia of the core body.
  The leg wants to run ahead of the body, but of course it is attached
  to the body. Thus the central body gets a push from the leg. The leg,
  for its part is pushed backward by the body.

  \item Imagine things this way: a central part and a more distant
  connected part have equal inertia $I$. They rotate at the same rate
  $\omega$. Then they are disconnected. They continue to rotate freely
  at $\omega$. The outer part then goes through these changes:

  \begin{nedqn}
    r' \eqcol \frac{r}{2}
  \\
    I' \eqcol \frac{I}{4}
  \\
    \omega' \eqcol 4 \omega
  \end{nedqn}

  \noindent
  Now reconnect the two parts. It's clear that the core part is ``going
  around too slowly'' while the outer part is ``going around too
  quickly.'' They will need to ``agree'' on a final angular velocity
  $\omega''$.

  This is easily calculated. Originally the angular momentum was $L = I
  \omega + I \omega$. After drawing in the distant part, we have $L' =
  I\omega + I' \omega'$, but of course $L' = L$ by conservation of
  angular momentum. And last, we know that $L'' = L' = L$, because
  angular momentum continues to be conserved. So:

  \begin{nedqn}
    L''
  \eqcol
    L
  \\
    I \omega'' + I' \omega''
  \eqcol
    I \omega + I \omega
  \\
    \parens{I + \frac{I}{4}} \omega''
  \eqcol
    2 I\omega
  \\
    \omega''
  \eqcol
    \frac{8}{5} \omega
  \end{nedqn}

  \noindent
  This shows that to ``reconnect'' the outer part to the inner part, a
  torque must be applied to the inner part to accelerate it (angularly),
  and a corresponding torque must be applied to the outer part to
  decelerate it (angularly).

  \item Feynman asks us to consider a carousel. You are pushing a box of
  mass $m$ radially outward. There is a line drawn on the carousel, you
  want to push the box outward along this radial line.

  To remain on this radial line, you are effectively asking that the
  angular velocity of the box be held constant at $\omega$, the rate of
  the carousel's rotation, as you push the box outward. But since the
  box's angular inertia is increasing as you push it outward, we know
  this corresponds to an increase in angular momentum of the box. Thus,
  we know that a torque needs to be applied as you move the box outward.

  \item If one looks from the outside at the carousel, they can see that
  the box, even when not moving, is always accelerating toward the
  center. They can see that the box will need to undergo tangential
  acceleration as you move it outward, if it is to maintain its angular
  velocity.

  For someone who is on the carousel, and thus has a rotating frame of
  reference, they will experience that pushing the box outward (which
  does not look to them like acceleration), will somehow induce a force
  that drags it toward the right (for counter-clockwise rotation) or
  toward the left (for clockwise rotation).

  This unexpected sideways drag force is called the \define{Coriolis
  force}. To push the box radially, one needs to exert a force
  \emph{opposite} the Coriolis force if they want to follow the line
  drawn on the carousel.

  \item To counteract the Coriolis force, the pusher must push
  \emph{with} the rotation: they must push the box to their left (for
  counter-clockwise rotation) or to the right (for clockwise rotation).
  Alternatively, if they only push toward the center, they will see the
  object drift away to the right (for counter-clockwise rotation) or the
  left (for clockwise rotation).

  \item The Coriolis force is again a ``fictitious'' force, because it
  only appears in a rotating frame of reference. A rotating frame is
  non-inertial.

  \item Let's calculate the magnitude of the Coriolis force. One can
  find it from:

  \begin{nedqn}
    L
  \eqcol
    m r^2 \omega
  \\
    \fpartial{t} L
  \eqcol
    m r^2 \parens{\fpartial{t} \omega}
    + 2mr\omega \parens{\fpartial{t} r}
  \\
    -F_\text{Coriolis} r
  \eqcol
    2mr \omega \parens{\fpartial{t} r}
    \nedcomment{assuming $\omega$ is held constant}
  \\
    -F_\text{Coriolis}
  \eqcol
    2m\omega \fpartial{t} r
  \end{nedqn}

  \item Here I held $\omega$ constant, and found the magnitude of the
  force required to maintain this. Alternatively, presuming that $L$ is
  held constant, we know $\fpartial{t} \omega$ must have opposite sign
  from $\fpartial{t} r$. This dragging force (when moving away from the
  axis of rotation) or accelerating force (when moving toward the axis
  of rotation) is the Coriolis force.

  \item A final note and summary: when a rocketship launches from the
  earth (or a box is pushed to the outside of a carousel), no torque is
  applied if the motion is always radial. The rocketship or box does not
  gain any angular momentum as it moves away from the axis. It does not
  experience a torque. Instead, simply its angular velocity $\omega$
  falls. This does not imply a loss of angular momentum, since $r$ is
  increasing.

  Only if we insist that $\omega$ be held constant must there be a
  torque. Then there \emph{must} be a tangential acceleration to keep
  pace, thus a tangential force, thus a torque. This force/torque is not
  really ``counteracting'' any force/torque, since in fact it is
  increasing momentum/angular momentum.

  Only if we take the non-inertial, rotating frame is there a
  (fictitious!) force to counteract. This is the Coriolis force. It is
  only an \emph{apparent} force, not a ``real'' one. When you resist
  this force, you are applying a real force to counteract a fictitious
  one. Thus you are applying a net torque and thus increasing angular
  momentum.
\end{enumerate}
