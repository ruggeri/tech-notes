\subsection{Center of Gravity}

\begin{enumerate}
  \item We've shown that the acceleration of the center of mass $\va$ is
  always exactly equal to $\frac{\vF}{M}$. We've also shown that the
  instantaneous change in angular momentum $\fpartial{t} L$ is always
  exactly $\tau$.

  \item We know that rigid motion can be broken down into translation
  and rotation. Given a set of forces $\vF_i$, we know that we can use
  our definitions of net external force and net external torque to
  describe the instantaneous change in angular momentum.

  \item We know that a rigid body can undergo two kinds of motion:
  translation and rotation. We want to be able to consider these two
  kinds of motion separately.

  \item As discussed, if we know the net external force $\vF = \sum_i
  \vF_i$, we know the acceleration of $\vx$. Depending on the set of
  individual forces $\vF_i$, there may also be some tendency for points
  on the object to angularly accelerate.

  \item This angular acceleration is of course $\tau = \sum_i \tau_i$.
  But we can counteract this. Simply apply

  \item We want to break down the dynamic acceleration of a rigid body
  into translational and rotational accelerations. To cleanly separate
  these two kinds of acceleration, we want (1) the angular acceleration
  to effect no change in translational momentum $\vp$ and (2) the
  translational acceleration to effect no change in angular momentum
  $L$.

  \item We need to choose a center-point so specify torque and angular
  momentum. We shall see that the center of mass is the center-point
  which allows us to cleanly separate total external torque from changes
  in translational momentum and total external force from changes in
  angular momentum.

  \item If we choose any other center-point, we will have that a pure
  angular acceleration accelerates the center of mass. That means there
  will be a change in translational momentum. On the other hand, if the
  center of mass is chosen as being the center point, then any
  rotational acceleration will not accelerate the center point. Thus,
  there is no change in translational momentum.

  \item We hope that, likewise, pure translational acceleration will not
  cause a change in angular momentum. I will show tha



  One thought: angular momentum is
  actually scalar in two-dimensional space (rotation about a fixed
  point), but a three-vector in three-dimensional space (pitch, roll,
  and yaw).

  \item I first discuss two-dimensional space. Take the plane that (1)
  passes through the center of mass and (2)

  \item We shall see that this happens exactly when we take the
  reference axis to be through the center of mass. I first note: clearly
  we must take some point that is not fixed, but will itself accelerate
  by $\va$ if the body itself is translationally accelerated by $\va$.
  Otherwise, there is a torque about a fixed point ``off to the side''
  of the path of the accelerating body.

  \item We can say more than this. We had better consider rotation about
  the center of mass. Otherwise, if any other point is taken, then pure
  rotational acceleration about this point will be able to change the
  velocity of the center of mass. In that case, pure rotational
  acceleration will cause a change in translational momentum, which we
  do not want.

  \item As hinted before, when we choose the center of mass as our
  center-point, we ensure that pure translational acceleration by $\va$
  does not cause a change in angular momentum.

  \item That is: if we choose the center of mass as our axis of
  rotation, we are assured that rotational acceleration $\alpha$ will
  not change translational momentum $\vp$.

  \item If the body is rigid, then the distance $\vr_i$ from the center
  of mass cannot change. In that case $\alpha = \frac{\tau}{I}$.

  \item Consider an object that is experiencing a uniform translational
  acceleration. That is, $\va_i$ is constant everywhere. We denote this
  $\va$. The force at each subpart $i$ is $\vF_i = m_i \va$.

  \item I claim that there is a special point about which the torque is
  exactly zero. Even about this special point, we cannot say that
  $\tau_i = F_i r_i$, because $\vF_i$ will not typically perpendicular to
  $\vr_i$.

  \item To find $\tau_i$, we can use our lever-arm rule. Since all
  $\vF_i$ are parallel, we can project $\vr_i$ onto any plane
  perpendicular to the $\vF_i$. Once we do this, we are taking a sum:

  \begin{nedqn}
    \tau
  \eqcol
    \sum_i \vF_i \vr'_i
  \\
  \eqcol
    \sum_i m_i a \vr'_i
  \\
  \eqcol
    a \sum_i m_i \vr'_i
  \end{nedqn}

  But now we know that this is the projection of the center of mass into
  the plane. In particular, we can just choose the plane to be
  \emph{through} the center of mass, in which case the torque will be
  zero here.

  \item It is common that $a$ might be acceleration due to gravity, in
  which case we denote it $g$. This is why we sometimes use the term
  \define{center of gravity} is sometimes used somewhat interchangeably
  with center of mass. But CoG is not always exactly at the CoM if the
  gravity does not pull uniformly on the object. Still, the
  approximation that the two are equal is good at non-planetary scales.

  \item He very quickly finishes with the statement that, if the axis is
  taken through the center of mass, the torque is equal to the
  instantaneous change in angular momentum \emph{even if the center of
  mass is undergoing accelerations}.

  He says this quickly, but I think his point is: if every point on the
  object is undergoing a uniform acceleration, there will still be no
  torque on the object. That is: there will be no change in the angular
  momentum (about the CoM), and there will be no change in our
  calculation of the torque.

  \item I feel like my ``proof'' here is a little less than ironclad.
  But I do feel like I understand it intuitively and well enough. I wish
  to declare victory and move on.
\end{enumerate}
