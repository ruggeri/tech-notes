\subsection{Center of Gravity}

\begin{enumerate}
  \item He asks us to consider an object sitting in a uniform
  gravitational field. What is the torque about an origin?

  \begin{nedqn}
    \tau
  \eqcol
    \sum_i m_i g \vx_i
  \\
  \eqcol
    g \sum_i m_i \vx_i
  \end{nedqn}

  \noindent
  Feynman notes that the torque will be zero if the origin is chosen to
  be the center of mass, since $\sum_i  m_i \vx_i$ is the location of
  the center of mass. If the origin is the CoM, then this is defined to
  be zero.

  \item This is why \define{center of gravity} is sometimes used
  somewhat interchangeably with center of mass. But CoG is not always
  exactly at the CoM if the gravity does not pull uniformly on the
  object. But the approximation that the two are equal is good at
  non-planetary scales.

  \item He very quickly finishes with the statement that, if the axis is
  taken through the center of mass, the torque is equal to the
  instantaneous change in angular momentum \emph{even if the center of
  mass is undergoing accelerations}.

  He says this quickly, but I think his point is: if every point on the
  object is undergoing a uniform acceleration, there will still be no
  torque on the object. That is: there will be no change in the angular
  momentum (about the CoM), and there will be no change in our
  calculation of the torque.

  \item I feel like my ``proof'' here is a little less than ironclad.
  But I do feel like I understand it intuitively and well enough. I wish
  to declare victory and move on.
\end{enumerate}
