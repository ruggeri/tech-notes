\subsection{Center of Gravity}

\begin{enumerate}
  \item Feynman now notes that the center of mass is also often called
  the \define{center of gravity}. Imagine an object is in a uniform
  gravitational field. We will ``support'' the object at a point. This
  point is not allowed to be translated. That means that the object can
  \emph{only} rotate about this point.

  \item How can an object be constrained so that this point doesn't
  move/accelerate? We could assume that, whatever acceleration this
  point would experience, we will apply a uniform acceleration/force at
  every point in the body. This exactly counteracts the acceleration of
  our supported point.

  \item Thus, the only movement can be rotation about the point. Will
  the object rotate? It depends on the total torque about this point.
  Only if it is zero will the object balance.

  \item So what is the total torque? It is

  \begin{nedqn}
    \tau
  \eqcol
    \sum_i \tau_i
  =
    \sum_i m_i g r_i
  =
    g \sum_i m_i r_i
  \end{nedqn}

  \item Note that if and only if we choose the center of mass as the
  balance, then the net torque will sum out to zero.

  \item I did neglect to address one part. We should not just use the
  $\tau_i$ that come from the gravitational field. We should \emph{also}
  add in the force necessary to fix the balance point. Note that exactly
  when we select the center of gravity as the balance point, a uniform
  force throughout the object has zero torque.
\end{enumerate}

\subsection{Center Of Mass As Conventional Axis Of Rotation}

\begin{enumerate}
  \item We want to decompose the motion of a rigid body into translation
  and rotation.

  \item The acceleration of every point on the object will be broken
  into two parts: (1) a constant translational acceleration and (2)
  a constant angular acceleration.

  \item The axis of the rotation is a special point which will
  \emph{only} undergo translational acceleration, since the (2) angular
  acceleration will have no effect.

  \item There is a point whose acceleration is uniquely easy to
  calculate: the center of mass's. Thus we will use the center of mass
  as our axis of rotation. Thus the translational acceleration will be
  very simple to calculate.

  \item We must now calculate the torque about the center of mass. If
  the net force is $\vF$, we can imagine a counteracting force of
  magnitude $\frac{\vF}{M}m_i$ at each point $i$. This exactly
  counteracts the net force $\vF$, so the center of mass will stay
  fixed.

  \item Also, this force has no net-torque. Thus the net torque remains
  $\tau = \sum_i \tau_i$. The instantaneous change in angular momentum
  about the center of mass will remain the same after applying the
  fixing force.

  \item Thus we have shown that we can decompose the total motion
  easily: (1) a translational part found by assuming that all the mass
  and all the forces lies/act at the center of mass, and (2) the
  rotational component found by simply assuming that there is no net
  force on the object (even if that is not true).

  \item This is a more general version of the center-of-gravity story I
  told above. It is also a more ironclad proof.

  \item Note that this did \emph{not} assume a rigid body. I \emph{do}
  allow for non-rigid internal motion of the object. This will not
  affect the argument that instantaneous change in angular momentum will
  still be equal to the total torque about the center of mass. Of
  course, the angular acceleration may no longer be proportional to the
  torque, if the moment of inertia is allowed to change.
\end{enumerate}
