\subsection{Center of Gravity}

\begin{enumerate}
  \item Feynman now notes that the center of mass is also often called
  the \define{center of gravity}. Imagine an object is in a uniform
  gravitational field. We will ``support'' the object at a point. This
  point is not allowed to be translated. That means that the object can
  \emph{only} rotate about this point.

  \item Let's set aside momentarily \emph{how} we could support the
  object at a point. We will simply assume that this point is
  constrained so it does not move, and that only rotation about this
  point is allowed.

  \item In that case, will the object rotate? Assuming the object is a
  rigid body, it will rotate if and only if there is a total torque
  about the support point. It will not rotate if and only if total
  torque is zero.

  \item Let's calculate the torque. Assume we've chosen an axis and
  calculated all the various radial distances $r_i$.

  \begin{nedqn}
    \tau
  \eqcol
    \sum_i \tau_i
  =
    \sum_i m_i g r_i
  =
    g \sum_i m_i r_i
  \end{nedqn}

  \item Note that that the torque $\tau_i$ is proportional to the mass
  of the subpart $i$ and the radial distance $r_i$. The center of mass
  is the exact point defined to balance these terms. Thus the total
  torque is exactly zero if and only if we choose the center of gravity
  as the balance point. Thus, provided the balance point is fixed, the
  center of mass is also the center of gravity.

  \item Now we need to figure out how to fix the balance point.
  Hopefully we will do it in a way that does not affect the total
  torque. If we can do this, our calculation of the CoM as a point about
  which there is zero torque will still be valid.

  \item If the CoM wants to accelerate by $\va$, we could imagine apply
  a force $\vF_i = -\va m_i$ at every subpart $i$. This results in a net
  force of $\vF$, changing the acceleration of the CoM by $-\va =
  \frac{\vF}{M}$.

  \item Note that while this new imaginary force fixes the CoM, it has a
  net-zero torque about the CoM. So our prior calculation of total
  torque remains unaffected.

  \item Note that I could also have simply demanded a force of $\vF =
  -\va M$ be applied directly at the CoM. Presumably \emph{internal}
  forces would develop to cause a uniform deceleration of the rigid
  object by $-\va$. Of course, torque would continue to be conserved.
\end{enumerate}

\subsection{Support By Another Axis}

\begin{enumerate}
  \item Let's consider what happens if you try to support at another
  point.

  \item Imagine a ball that is falling through gravity. By symmetry, we
  do not expect it to rotate, but simply fall straight down.

  \item Consider supporting at any point other than the CoM. In
  particular, consider supporting at a point that does not lie on the
  line through the CoM and parallel to gravity.

  \item Then note that this point experiences the same acceleration
  $\va$ as every other point on the object.

  \item Of course, there is a torque about this off-center point.

  \item Now, consider the two kinds of ``fixing force'' I've considered.
  First consider the ``uniform'' fixing acceleration $\vF_i = -\va m_i$.
  This would fix the off-center point. And in fact every point on the
  ball. And, expectedly, it would have a torque that perfectly offsets
  the torque about the off-center axis due to gravity.

  \item Which shows that we should not consider this uniform fixing
  acceleration as ``support.'' It would say that we could perfectly
  balance the ball at an off-center point.

  \item What we really want to consider is a force applied directly at
  the off-center point. The amount of force required cannot ignore
  internal forces, so it is hard to say what this would be.

  \item If the fixing force is not exactly $\vF = -\va M$, then we know
  the center of mass will move, because there is a non-zero net force on
  the object.

  \item If the fixing force were exactly $\vF = -\va M$, then we know
  that the object will be constrained to fix the CoM (because net force
  is zero). But we know that we have applied a net torque to the object,
  which otherwise has a zero net torque about the CoM due to uniform
  gravity.

  \item Thus the object must rotate about the CoM. Which means that the
  object is not balanced. Also, note that the rotation about the CoM
  would require that the alleged balance point is in fact moved. So
  additionally the object is not even being properly supported at the
  off center point anyway.

  \item Maybe I will finish by saying: you can balance an object by
  applying forces that (1) negate the net force and (2) negate the net
  torque about the CoM. With regard to a uniform gravitational field,
  the net torque about the CoM is zero, and thus you must simply place
  your fixing forces such that they imply a net zero torque. This can be
  anywhere symmetric about the CoM.
\end{enumerate}

\subsection{Center Of Mass As Conventional Axis Of Rotation}

\begin{enumerate}
  \item We want to decompose the motion of a rigid body into translation
  and rotation.

  \item The acceleration of every point on the object will be broken
  into two parts: (1) a constant translational acceleration and (2)
  a constant angular acceleration.

  \item The axis of the rotation is a special point which will
  \emph{only} undergo translational acceleration, since the (2) angular
  acceleration will have no effect.

  \item There is a point whose acceleration is uniquely easy to
  calculate: the center of mass's. Thus we will use the center of mass
  as our axis of rotation. Thus the translational acceleration will be
  very simple to calculate.

  \item We must now calculate the torque about the center of mass. If
  the net force is $\vF$, we can imagine a counteracting force of
  magnitude $\frac{\vF}{M}m_i$ at each point $i$. This exactly
  counteracts the net force $\vF$, so the center of mass will stay
  fixed.

  \item Also, this force has no net-torque. Thus the net torque remains
  $\tau = \sum_i \tau_i$. The instantaneous change in angular momentum
  about the center of mass will remain the same after applying the
  fixing force.

  \item Thus we have shown that we can decompose the total motion
  easily: (1) a translational part found by assuming that all the mass
  and all the forces lies/act at the center of mass, and (2) the
  rotational component found by simply assuming that there is no net
  force on the object (even if that is not true).

  \item This is a more general version of the center-of-gravity story I
  told above. It is also a more ironclad proof.

  \item Note that this did \emph{not} assume a rigid body. I \emph{do}
  allow for non-rigid internal motion of the object. This will not
  affect the argument that instantaneous change in angular momentum will
  still be equal to the total torque about the center of mass. Of
  course, the angular acceleration may no longer be proportional to the
  torque, if the moment of inertia is allowed to change.
\end{enumerate}

\subsection{Can You Use Another Axis?}

\begin{enumerate}
  \item I think that you can use \emph{any} accelerating reference axis
  to decompose motion into translation and rotation. Simply, other axes
  are annoying for calculation purposes. First, you will have to
  calculate the acceleration of your reference point, which will be
  annoying if you do not choose CoM.

  \item The CoM is convenient because you can ignore internal forces. If
  you choose an axis that is sensitive to internal forces (any non-CoM
  fixed point on the body, for instance), you will have a lot of
  trouble.

  \item You should now be able to calculate torque about your chosen
  axis. However, this torque must be ``fixed-up'' by applying
  fictitious forces that fix the axis.

  \item Again, you will have a lot of pain if you have chosen an axis
  that is sensitive to internal forces. You cannot simply apply a force
  $\vF = -\va m_i$ everywhere and assume that this uniformly decelerates
  the entire body by $-\va$.

  \item Likewise, even if you apply $\vF = -\va M$ directly at your
  chosen axis, of course internal forces might develop that imply that
  your axis is not in fact decelerated by exactly $-\va$.

  \item Once you have found your fixing forces, you'll have to calculate
  the torque ``fix-up.'' We saw that this is zero if you chose the CoM.
  But of course it won't be if you choose any other axis.

  \item Thus we see that a distant second-best axis is one that is a
  fixed offset from the CoM. This will allow you to easily calculate its
  acceleration in the usual way, as well as the fixing force. You only
  have to do the torque fix-up, which should be easy to calculate.
\end{enumerate}
