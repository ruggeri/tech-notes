\section{The Theory of Gravitation}

\begin{enumerate}

  \item Historically, Brahe made measurements of the planets. Kepler
  then analyzed them and made some conclusions: the Earth moves in an
  ellipse around the sun, the radius from the sun to the Earth sweeps
  out the same area in the same time, et cetera.

  \item At the same time, Galileo had proposed \emph{inertia}: that an
  object will continue in motion if nothing is done to it. What could be
  done? Newton proposed \emph{force}, which accelerates an object. The
  direction of the force matters.

  \item Newton further noticed that if the Earth kept changing
  direction, there must be a force acting upon it. That suggested a
  gravitational force. The force would be directed toward the sun, and
  it would be inversely proportional to the \emph{square} of the
  distance between the two objects. Kepler's observations implied these
  conclusions.

  \item He mentions how gravitation explains tides. Why are there two
  high tides per day? Well, one occurs when the moon is right above.
  That's because the moon wants to pull water here, causing it to bulge
  here.

  \item The other bulge is at the opposite side of the Earth. Why is
  there a bulge here? The reason is that on this side there is the least
  pull on the water.

  \item He mentions a story. When observing the moons of Jupiter, we saw
  that they sometimes seemed to be moving faster than gravitation had
  predicted, and sometimes slower. Why? We realized that it takes
  \emph{light} some time to travel from the moons to the Earth, and that
  when Jupiter and its moons are closest to Earth, they seem to be
  moving faster.

  \item Another story: people noticed that Uranus was acting weird, even
  when we accounted for the action of Saturn and Jupiter. Thus it was
  posited that there might be an invisible planet we had not seen.
  Indeed there was! Pluto!

  \item Cavendish used an apparatus which was basically a rod connected
  to the ceiling. The ends of the aparatus were lead balls. He then
  attracted them with some other lead balls. This caused a twisting of
  the rope holding the rod. This implied a certain force was being
  exerted, which allowed him to determine $G$.

  \item Knowing $G$, one can then determine the weight of the earth. You
  start with a $1N$ weight and then plug into:

  \begin{nedqn}
    F
  \eqcol
    G \frac{m_1 m_2}{r^2}
  \end{nedqn}

  Note that you must know the radius of the Earth. But this was known
  already from a number of other experiments.

  \item Where does gravity come from? No one knows. He notes something
  that is demonstrably \emph{wrong}. Some people thought that maybe
  there are tiny particles coming from every direction that push on an
  object (with a net force of zero). The sun could ``block'' some of
  these particles coming at the Earth from the side, creating a net
  force toward the sun.

  \item A way to see this is false is that as the Earth rotates around
  the sun, it would be ``running into the wind,'' so to speak. That is:
  there would be a force that would be resisting its tangential
  velocity. That would tend to slow it down, and it would cause the
  Earth to fall into the sun.

  \item As ever, he notes how the electrical and gravitational forces
  are so different in magnitude, but also very interestingly that they
  are both inversely proportional to the square of distance. Does that
  suggest they are related?

  \item He observes that the mass of an object rotating around the sun
  is not relevant to its speed. All things at a radial distance $r$ from
  the sun will have the same speed, because the acceleration toward the
  sun is always the same.

  \item According to Newton, the gravitational force's effect is
  instantaneous. Thus, we could wiggle a mass at location A to
  instantaneously send an encoded signal to location B. Einstein makes
  arguments that this cannot be done. Therefore, there must be some
  propogation delay in the effect of gravity.

  \item Apparently Einstein's relativity theory suggests that anything
  with energy has mass. Thus light, which is electromagnetic energy, has
  mass, and thus it can be bent by gravity.

  \item He notes that there isn't an explanation of gravity in terms of
  quantum mechanics.

\end{enumerate}

TODO: Can I figure out why the orbits are in ellipses?
