\section{Center of Mass; Moment of Inertia}

\begin{enumerate}
  \item We can ask: what is the rotational kinetic energy of an object?
  That is, what is the rotational kinetic energy of an object moving at
  speed $\omega$? Well, we know that:

  \begin{nedqn}
    KE
  \eqcol
    \half m v^2
  \\
  \eqcol
    \half m \parensq{\omega r}
  \\
  \eqcol
    \half I \omega^2
  \end{nedqn}

  \item Consider again the skater who moves their leg inward as they
  spin. We know their angular momentum has been conserved, but that
  their moment of inertia has gone down, and thus their angular velocity
  $\omega$ has gone up by the same proportion.

  The implication, though, is that there is more kinetic energy in the
  system. How did this happen? Well, note that, to maintain rotation, a
  centripetal force must be applied to the outside foot. This force is
  required to maintain circular motion.

  When we bring the leg in, we apply this force through a (radial)
  distance. Thus the change in kinetic energy is (1) the magnitude of
  the centripetal force, (2) times the change in the radius.

  \item Feynman notes: consider just your torso when moving a leg in.
  The torso angular velocity has increased, even though its moment of
  inertia has not changed. What happened to the torso when the leg was
  moved in? There must be some torque applied!

  This torque is the \define{Coriolis force}, and like the centrifugal
  force, it is ``fictitious.'' It only exists if our frame of reference
  is already rotating.

  \item The Coriolis force can also be observed if you are on a
  carousel. If you start from an inner point and try to walk to an outer
  point, you will feel the need to lean in the direction of rotation.
  The reason is that your angular momentum is increasing as you move
  outward. Thus, some torque must be applied to increase your angular
  momentum.

  We can calculate the magnitude of the Coriolis force. If you are
  moving radially at a velocity $v_r$, then:

  \begin{nedqn}
    \fpartial{t} L
  \eqcol
    \fpartial{t} I \omega
  \\
  \eqcol
    \fpartial{t} m r^2 \omega
  \\
  \eqcol
    m \omega 2r v_r
  \intertext{let's focus on in on the torque/force:}
    \tau
  \eqcol
    m \omega 2 r v_r
  \\
    F_\text{coriolis} r
  \eqcol
    m \omega 2 r v_r
  \\
    F_\text{coriolis}
  \eqcol
    2m \omega v_r
  \end{nedqn}
\end{enumerate}
