\section{Center of Mass; Moment of Inertia}

\subsection{Some Properties of CoM}

\begin{enumerate}
  \item The center of mass is like the average position of the
  \emph{grams} in the object. This is what it means to be waited by
  mass. The CoM might not be in the material of the object, but if the
  object is bounded, of course the CoM must lie within those bounds.

  \item The center of mass will always lie within any planes of symmetry
  that exist for the object.

  \item Consider subobjects A and B, with masses $M_A, M_B$. Now combine
  the objects. The center of mass is $\frac{M_A}{M} \vx_A +
  \frac{M_B}{M} \vx_B$, where $M = M_A + M_B$. This is because:

  \begin{nedqn}
    \frac{1}{M} \sum_i m_i \vx_i
  \eqcol
    \frac{1}{M}
    \parens{
      \sum_{i \in A} m_i \vx_i
      +
      \sum_{i \in B} m_i \vx_i
    }
  \\
  \eqcol
    \frac{1}{M}
    \parens{
      M_A \vx_A + M_B \vx_B
    }
  \end{nedqn}

  \item Feynman then goes on a long philosophical ramble about how
  Newton's laws ``reproduce'' themselves on greater and greater levels.
  That is: if Newton's laws apply to individual particles, they also
  apply to aggregates. Feynman notes that these are the kind of laws
  that we are most likely to find first, because they occur at larger
  and thus observable scales. He notes that the true laws at atomic
  level are not exactly Newton's, but when aggregates are studied,
  Newton's are a good approximation (and get better as we aggregate
  further).

\end{enumerate}

\subsection{Center of Gravity}

\begin{enumerate}
  \item He asks us to consider an object sitting in a uniform
  gravitational field. What is the torque about an origin?

  \begin{nedqn}
    \tau
  \eqcol
    \sum_i m_i g \vx_i
  \\
  \eqcol
    g \sum_i m_i \vx_i
  \end{nedqn}

  \noindent
  Feynman notes that the torque will be zero if the origin is chosen to
  be the center of mass, since $\sum_i  m_i \vx_i$ is the location of
  the center of mass. If the origin is the CoM, then this is defined to
  be zero.

  \item This is why \define{center of gravity} is sometimes used
  somewhat interchangeably with center of mass. But CoG is not always
  exactly at the CoM if the gravity does not pull uniformly on the
  object. But the approximation that the two are equal is good at
  non-planetary scales.

  \item He very quickly finishes with the statement that, if the axis is
  taken through the center of mass, the torque is equal to the
  instantaneous change in angular momentum \emph{even if the center of
  mass is undergoing accelerations}.

  He says this quickly, but I think his point is: if every point on the
  object is undergoing a uniform acceleration, there will still be no
  torque on the object. That is: there will be no change in the angular
  momentum (about the CoM), and there will be no change in our
  calculation of the torque.

  \item I feel like my ``proof'' here is a little less than ironclad.
  But I do feel like I understand it intuitively and well enough. I wish
  to declare victory and move on.
\end{enumerate}

\subsection{Tips and Tricks For CoM and Inertia Calculations}

\begin{enumerate}
  \item He introduces a theorem of Pappus. Say that a volume is
  generated by taking a closed planar area and moving in through space.
  If the motion is entirely perpendicular to the planar area, then every
  point in the planar area moves through the same distance. The volume
  is thus area time distance.

  \item But Pappus further considers rotating the area through space, or
  in fact any such motion. His theorem says that the volume generated is
  equal to the area times \emph{the distance the plane's CoM moves}.

  \item Feynman then uses this to compute the center of mass of some
  planar shapes.

  \item Feynman next calculates the inertia for rotating a rod of length
  $L$ about one end. It is:

  \begin{nedqn}
      I
    \eqcol
      \int_0^L x^2 \frac{M}{L} \dx
    \\
    \eqcol
      \frac{L^3}{3} \frac{M}{L}
    \\
    \eqcol
      \frac{L^2M}{3}
  \end{nedqn}

  \item He does a similar calculation and computes that swinging about
  the CoM is $\frac{L^2M}{12}$

  \item He next develops the \define{parallel-axis theorem}. He asks us
  to consider swinging a rod about an axis. Reduce the rod to simply a
  point mass at its CoM. Then the inertia about the axis is $M
  x_\text{CM}^2$. But when we rotate the rod, we not only rotate the
  CoM, but also rotate \emph{about} the CoM. Thus we should also correct
  by $I_C = \sum_i m_i x_i^2$, the inertia about the center of mass
  (moment-of-inertia).

  \item He proves this like so: let $x_i = x_i' + x_\text{CM}$
  ($x_\text{CM}$ is the location of the center of mass). Then:

  \begin{nedqn}
    x_i^2
  \eqcol
    x_i'^2 + 2 x_i' x_\text{CM} + x_\text{CM}^2
  \end{nedqn}

  \noindent
  Then consider $\sum_i m_i x_i^2$. Then $\sum_i m_i x_i'^2$ is simply
  the moment of inertia and $\sum_i m_i x_\text{CM}^2 = M
  x_\text{CM}^2$. The middle term sums $\sum_i m_i x_i'$, but this is
  zero because it is a CoM calculation where origin is already defined
  to be CoM.

  \item This proves the theorem for one dimension, but it holds in
  three. However, it is important that the axes under consideration be
  \emph{parallel}!
\end{enumerate}

\subsection{Rotational Kinetic Energy}

\begin{enumerate}
  \item We can ask: what is the rotational kinetic energy of an object?
  That is, what is the rotational kinetic energy of an object moving at
  speed $\omega$? Well, we know that:

  \begin{nedqn}
    \text{KE}
  \eqcol
    \half m v^2
  \\
  \eqcol
    \half m \parensq{\omega r}
  \\
  \eqcol
    \half I \omega^2
  \end{nedqn}

  \item Consider again the skater who moves their leg inward as they
  spin. We know their angular momentum has been conserved, but that
  their moment of inertia has gone down, and thus their angular velocity
  $\omega$ has gone up by the same proportion.

  The implication, though, is that there is more kinetic energy in the
  system. How did this happen? Well, note that, to maintain rotation, a
  centripetal force must be applied to the outside foot. This force is
  required to maintain circular motion.

  When we bring the leg in, we apply this force through a (radial)
  distance. Thus the change in kinetic energy is (1) the magnitude of
  the centripetal force, (2) times the change in the radius.
\end{enumerate}

\subsection{Coriolis Force}

\begin{enumerate}

  \item Feynman notes: consider just your torso when moving a leg in.
  The torso angular velocity has increased, even though its moment of
  inertia has not changed. What happened to the torso when the leg was
  moved in? There must be some torque applied!

  This torque is the \define{Coriolis force}, and like the centrifugal
  force, it is ``fictitious.'' It only exists if our frame of reference
  is already rotating.

  \item The Coriolis force can also be observed if you are on a
  carousel. If you start from an inner point and try to walk to an outer
  point, you will feel the need to lean in the direction of rotation.
  The reason is that your angular momentum is increasing as you move
  outward. Thus, some torque must be applied to increase your angular
  momentum.

  We can calculate the magnitude of the Coriolis force. If you are
  moving radially at a velocity $v_r$, then:

  \begin{nedqn}
    \fpartial{t} L
  \eqcol
    \fpartial{t} I \omega
  \\
  \eqcol
    \fpartial{t} m r^2 \omega
  \\
  \eqcol
    m \omega 2r v_r
  \intertext{let's focus on in on the torque/force:}
    \tau
  \eqcol
    m \omega 2 r v_r
  \\
    F_\text{coriolis} r
  \eqcol
    m \omega 2 r v_r
  \\
    F_\text{coriolis}
  \eqcol
    2m \omega v_r
  \end{nedqn}
\end{enumerate}
