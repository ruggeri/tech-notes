\subsection{The Lorentzian Transformation}

\begin{enumerate}

  \item Was this the death of Maxwell's laws? They were difficult to
  reject, because they had worked so well to explain other experimental
  resuls. But on the other hand the Michelson-Morley result seemed to
  totally contradict the concept of a luminiferous aether.

  \item H.A. Lorentz suggested a solution. He didn't suggest changing
  Maxwell's laws exactly. He suggested that a change of frame of
  reference did not mean a Galilean transformation of space and time. He
  suggested a different form of transformation:

  \begin{nedqn}
    x'
  \eqcol
    \frac{x - ut}{\sqrt{1 - u^2/c^2}}
  \\
    t'
  \eqcol
    \frac{t - ux/c^2}{\sqrt{1 - u^2/c^2}}
  \end{nedqn}

  For simplicity we assume that velocity is only along the $x$ axis.
  This transformation is called a \define{Lorentzian transformation}.

  So here is what Lorentz was saying: what if your perception of $x$
  position and the rate of time \emph{changed} depending on your
  velocity? If that were the case, it would ``undo'' your ability to
  detect the ostensible change in the speed of light.

  \item Notice in particular that lengths of an object at velocity are
  stretched (rather, shrunk) by a factor of $\sqrt{1 - u^2/c^2}$ (since
  this is less than one). That would explain the Michelson-Morley null
  result. With this factor of correction, the distances equalize and
  thus we don't expect the light to go out of phase.

  \item Lorentz proved that Maxwell's laws are \emph{invariant} under a
  Lorentzian transformation. But Newton's laws are invariant under a
  Galilean transformation. If a change in inertial reference frame
  results in a Galilean transformation being observed, then
  electrodynamics lets us observe the change. If a change in inertial
  reference frame results in a Lorentzian transformation, then mechanics
  will allow us to detect the change.

  So even though the Lorentzian transformation lets us reconcile the
  principle of relativity and the principle of invariant light speed
  within the context of electrodynamics, it \emph{breaks} the principle
  of relativity in the context of mechanics.

  This is what Einstein resolved. He realized that you could very
  slightly tweak Newton's laws so that they \emph{were} invariant under
  a Lorentzian transformation. All you need to do is assume that
  \emph{mass is observed to change at different velocities}. We shall
  return to this point.

  \item To summarize: Maxwell disturbed people because (1) he posited
  this luminiferous aether that somehow didn't interact with matter, (2)
  he broke the principle of relativity. Except Michelson-Morley found
  that relativity \emph{didn't} get broken somehow. That's a relief, but
  does that really mean Maxwell was wrong? How, if Maxwell is working so
  well?

  Maybe Maxwell is okay says Lorentz. But Lorentz's solution implies
  that relativity should be broken for mechanics. And we don't observe
  that either! So we need Einstein!

\end{enumerate}
