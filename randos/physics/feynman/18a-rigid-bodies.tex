\subsection{Rigid Bodies}

\begin{enumerate}
  \item We're going to explore what happens when \define{external}
  forces are applied to an object (or \define{system}). Within the
  system, \define{internal} forces between subparts of the system may
  develop. We hope that we will be able to \emph{ignore} the effect of
  the internal forces.

  \item Two extreme systems are (1) a system where there are \emph{no}
  interaction effects (totally ``non-rigid''), and (2) a system which is
  a \define{rigid body}: every point of the object remains at a fixed
  and unvarying distance to every other point of the object.

  \item What kind of transformations can a rigid body undergo? These
  transformations must be \define{isometries} of the object.

  \item We know that translating all points in the object does not
  deform it. We also know that rotation about an axis does not deform
  the object.

  \item Mirroring is also an isometry, but it is not possible to
  continuously perform on a rigid body. So we are restricted to
  roto-translations.
\end{enumerate}
