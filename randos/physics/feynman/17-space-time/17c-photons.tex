\subsection{Photons}

\begin{enumerate}

  \item Last, Feynman discusses a little about photons. Photons travel
  at $c$ in every rest frame. To prove this, remember our formula for
  relativistic change in velocity:

  \begin{nedqn}
    c'
  \eqcol
    \frac{
      c - u
    }{
      1 - uc / c^2
    }
  \\
  \eqcol
    \frac{
      c - u
    }{
      c \parens{c - u}
    }
  \\
  \eqcol
    c
  \end{nedqn}

  \item Photons have zero rest mass. Feynman says photons have energy $E
  = \hslash f$. I'm not sure what his units are here.

  \item He also notes that the momentum of a photon is $p = \hslash f /
  c$.

  \item He notes that if you are assuming $c = 1$, then energy and
  momentum are equal. Again: I'm not sure what the units are here...

  \item By possessing momentum, this implies that photons have
  (relativistic) mass. Moreover, by saying energy equals momentum, it
  implies the rest mass of a photon must be zero.

\end{enumerate}
