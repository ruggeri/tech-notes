\section{Atoms in Motion}

\begin{enumerate}

  \item A drop of water is really a mass of moving, jiggling \ce{H2O}
  molecules.

  \item The molecules do have some level of attraction to each other.
  They don't just fly apart. If the water is put on a slope, the
  molecules that that start moving at the front will pull the ones
  behind forward with them.

  \item But if we heat the water (jiggling is really just heat), then
  they will fly apart. This is boiling the water. If there is high
  enough heat, the force of the jiggling will overcome the attractive
  force of the water, and it will become steam.

  \item Next, let's consider a gas contained in a vessel. The gas
  molecules will bounce against the walls, which is pressure.

  \item If you heat the gas, they will have greater velocity, and it
  will take more force to stop them.

  \item It's noted that the force required to contain the molecules is
  really the pressure times the area. The pressure relates to the number
  of collisions \emph{in a certain area} (and the velocity of the
  molecules).

  \item If we double the number of molecules, we will double the
  pressure. That's because we're doubling the number of collisions.

  \item Likewise, if we increase the temperature, we will increase the
  pressure, because the velocity of the colliding molecules will be
  greater.

  \item But note: doubling the temperature will also increase the
  \emph{rate} of collisions. Why doesn't doubling temperature quadruple
  the pressure?

  \item The reason is that the temperature is a measure of average
  kinetic energy (you sum the total kinetic energy of all molecules and
  then divide by the number of molecules). As we know, if you double the
  kinetic energy, you only increase the velocity by a factor of
  $\sqrt{2}$. So we are hitting $\sqrt{2}$ times ``harder'', and we are
  hitting $\sqrt{2}$ times more often. Thus we get an increase in
  pressure of $2$ times.

  \item Feynman doesn't discuss what happens if we halve the volume
  (keeping the temperature constant). I claim the pressure will double.
  Why? Because the number of collisions per second will double. Consider
  how many molecules are within a distance of $\epsilon$ from the sides
  of the vessel. How many molecules are within a distance of
  $2\epsilon$? For a small enough $\epsilon$, the ratio is 2. Which
  means that approximately twice as many molecules are within "collision
  range."

  \item He talks about slow compression of a gas. If we slowly compress
  the gas, he claims it will become hotter. He claims that if we are
  compressing the sides, then when a molecule bounces off a side, it
  will return with a greater velocity. Likewise slow decompression will
  cool a gas.

  \item He returns to discuss of phase change. If we cool water, it
  stops jiggling so much, and settles down into a latice structure
  determined by the intermolecular attraction between water molecules.
  This is now a solid because water molecules really don't want to break
  from the latice. If you pick up one end of an icicle, you lift the
  other.

  \item One interesting note is that ice takes up a greater volume than
  liquid water. The reason is particular to the latice of frozen water.
  There are lots of ``holes.'' This is unusual: most other substances
  don't need more space when arranging in a lattice. They take up
  \emph{less} space because there is no need for room for jiggling.

  \item There is an interesting note. Even in ice molecules of water do
  jiggle. The ice has a temperature: it has heat. At absolute zero, the
  water molecules will \emph{still} do some jiggling! Helium, for
  instance, will jiggle enough at its absolute zero temperature that it
  will break out of any lattice. Thus helium cannot be frozen (at normal
  pressures).

  \item Next he asks us to imagine a puddle of water in air. The water
  molecules are jiggling around, and every once in a while one will be
  thrown into the air, and escape from the puddle. Thus the water will
  evaporate.

  \item If we sealed the water in a jar, then it would not entirely
  evaporate. The reason is that evaporated water will sometimes
  \emph{return} to the puddle. As the water evaporates, the amount of
  water in the air will increase, and likewise the rate of return will
  increase. When the rate of return balances the rate of leaving,
  evaporation will stop.

  \item We can get water to keep evaporating if we continuously replace
  the moist air with dry air. We often use a fan to circulate dry air in
  and moist air out. Of course, if the air is very moist, a fan won't be
  very helpful.

  \item He next asks which molecules are ejected? The ones with the
  greatest velocity: the hotest ones! So as the water evaporates, the
  puddle cools.

  \item If, in a sealed vessel, the puddle cools as molecules leave, but
  no heat is gained when molecules return, then the puddle would get
  colder and colder without bound.

  \item But in fact the water \emph{does} gain some energy when
  molecules return. When they return, they accelerate toward the puddle.
  Why? Because of the intermolecular attraction! In balance: when a fast
  molecule leaves, the puddle cools (and the molecule slows down to
  break free). But when it returns, it speeds up back to the original
  rate, and the puddle warms. Things balance out and the puddle remains
  at constant temperature.

  \item To ensure that the puddle keeps cooling, you can make sure that
  the rate of evaporation exceeds the rate of return. This is what
  happens when you blow on soup.

  \item He mentions that gasses can dissolve in the water just like
  water can evaporate. The gas molecules can push into the water and
  kind of get lost in there.

  \item Next, we discuss how a salt may dissolve in water. Salts like
  \ce{NaCl} consist of ions \ce{Na+} and \ce{Cl-} in a lattice. Note
  that the ``molecule'' \ce{NaCl} isn't quite like a molecule of
  \ce{H2O}. In \ce{H2O} the hydrogen and the oxygen live very close to
  each other, while in \ce{NaCl} the ions live quite far apart.

  \item The two hydrogens and one oxygen are \emph{covalently} bonded,
  while the sodium and chloride are \emph{ionically} bonded. In ionic
  bonds, electrons are shared very unequally (basically, they move from
  one element to the other). Whereas with covalent bonds, there is
  relatively equal sharing. I think covalent/ionic is kind of on a
  spectrum. It seems that with covalent bonds, it's easier to see or say
  which two \ce{H} are bonded to which one \ce{O}. Whereas in a salt, it
  is not possible to easily identify which \ce{Na+} is bound to which
  \ce{Cl-}, because they're in a relatively equal grid.

  \item He talks about how salts dissolve in water. \ce{H2O} has a
  positive end (the hydrogens) and a negative end (the oxygen). They can
  kind of surround the \ce{Na+} and \ce{Cl-} ions appropriately, and
  break them out of the latice structure.

  \item As ever, there is a rate of the crystal dissolving, but also a
  rate of it \emph{precipitating}. Only when the rates match is the
  solution in equilibrium.

  \item If we heat the solution, will we get it to dissolve faster (or
  more)? It's hard to say: the rate of dissolving will increase as water
  more frequently comes in to interact with the crystal, but the rate of
  precipitation will also increase. Which effect is greater depends on a
  number of factors.

  \item Now we talk about chemical reactions! He talks about burning
  carbon in air. The key is that the attraction between carbon and
  oxygen is greater than oxygen-oxygen and carbon-carbon.

  \item When the carbon in the lattice is pulled away to join an oxygen,
  it leaves with a significant velocity. The new \ce{CO} molecule leaves
  with a great velocity. Energy is released as kinetic energy, which is
  heat. We also sometimes get \emph{electromagnetic radiation} AKA
  light AKA flame.

  \item Note that, in general, when an \ce{O2} molecule is ripped apart
  to pair with a \ce{C} atom, we normally form \emph{two} \ce{CO}
  molecules. But carbon dioxide (\ce{CO2}) is produced when a \ce{CO}
  molecule interacts/intercepts one of those \ce{O} atoms.

  \item He mentions something about how burning carbon quickly in a low
  oxygen environment will result in lower amounts of \ce{CO2} and more
  energy being released? He doesn't really explain that, but it isn't
  super important.

  \item He now digresses to discuss a little how chemists can determine
  the atomic formulae of chemicals. He notes that there are both
  chemical methods and physical methods. He notes that chemical names
  can be complicated, but they are trying to do something complicated:
  describe the atoms and their arrangement in a chemical.

  \item The chapter ends on fluffy observations, including that all life
  is from complex interactions of atoms.

\end{enumerate}
