\subsection{Proper Time}

\begin{enumerate}

  \item Consider a \define{world line} through space time. The simplest
  kind of world line is one where we stay at a fixed location,
  experience no acceleration, and just live for a given time. That is:
  our world line corresponds to a single fixed inertial frame of
  reference.

  \item Consider the Minkowski norm between two points $\vmu_0 = (0, 0,
  0, 0)$ and $\vmu_1 = (0, 0, 0, \ttilde_1)$. Then $\norm{\vmu_1 -
  \vmu_0} = \ttilde_1$. Therefore the Minkowski norm corresponds to the
  time difference $\Delta \ttilde = \ttilde_1$.

  \item We already showed that the Minkowski norm is invariant under
  Lorentz transform. So consider a Lorentz transformation whereby

  \begin{nedqn}
    \vmu_0
  \eqcol
    \parens{0, 0, 0, 0}
  \\
  & \mapsto &
    \vmu'_0 = \parens{0, 0, 0, 0}
  \\
    \vmu_1
  \eqcol
    \parens{x_1, y_1, z_1, \ttilde_1}
  \\
  & \mapsto &
    \vmu'_1 = \parens{0, 0, 0, \ttilde'_1}
  \end{nedqn}

  \item The Minkowski norm $\norm{\vmu_1 - \vmu_0} = \norm{\vmu'_1 -
  \vmu'_0}$. Clearly the norm in the new coordinate system is the time
  $\ttilde'_1$ a traveler experiences on their journey to $(x, y, z)$ at
  uniform speed $\norm{\parens{x, y, z}} / \ttilde_1$.

  Note: the traveler does not experience that they have ``gone''
  anywhere. Note that when the traveler ``arrives'' they would disagree
  with the duration of the ``journey'' when comparing notes with someone
  who has stayed at space position $(0, 0, 0)$ (in the original
  coordinate system).

  \item What is going on here? Perhaps at $\vmu_0$ Earth Twin's friend
  (Space Twin) left Earth. Later, Earth Twin receives a radio
  transmission from Space Twin. The transmission says: ``I'm turning
  around.'' Earth Twin calculates from (a) the direction of the signal,
  (b) the time it was received, and (c) the invariant speed of light in
  a vacuum that the signal was broadcast from $\vmu_1 = \parens{x_1,
  y_1, z_1, \ttilde_1}$.

  \item Space Twin has a somewhat different experience. She started at
  Earth with $\vmu'_0 = \parens{0, 0, 0, 0}$. That is, she ``zero-ed''
  her position and time reading to the same moment in space-time that
  Earth Twin has.

  Space Tiwn left Earth with a velocity $\vv = \parens{x_1 / \ttilde_1,
  y_1 / \ttilde_1, z_1 / \ttilde_1}$. Note: this is Space Twin's
  velocity \emph{as observed by Earth Twin}. As far as Space Twin is
  concerned, Earth Twin was the one moving away. Space Twin considers
  her velocity to be zero: $\vv' = (0, 0, 0)$.

  \item Space Twin will ``travel'' for a while before sending her signal
  to Earth Twin. She decides to ``travel'' for time equal to
  $\ttilde'_1$. Thus she travels to a point in space-time $\vmu_1' =
  \parens{0, 0, 0, \ttilde'_1}$.

  \item When Space Twin sends her signal, she claims that $\ttilde'_1$
  time has passed, whereas Earth Twin believes that it must have taken
  her time $\ttilde_1$ to get to the point $(x_1, y_1, z_1)$ to send her
  signal. The contradiction in their calculations is explained entirely
  by relativistic effects.

  \item Note that despite the ``contradiction,'' Earth Twin can
  calculate that

  \begin{nedqn}
    \norm{\parens{x_1, y_1, z_1, \ttilde_1}}
  \ltcol
    \ttilde_1
  \end{nedqn}

  In fact it equals $\ttilde'_1$. Thus we call the Minkowski norm of
  this vector the \define{proper time}. It is the time that a traveler
  along the straight-line path from $\vmu'_0$ to $\vmu'_1$ would observe
  when travelling between those points in space-time.

  Note that the proper time of a straight-line world path is invariant
  in every frame of reference.

  \item Space Twin may decide to return to Earth (perhaps to settle the
  matter of the differing clocks). She just reverses her journey. In her
  series of frames, this has taken time equal to $2\ttilde'_1$, whereas
  to the Earth observer it has taken time $2\ttilde_1$.

  \item Where has the symmetry been broken? The reason is that Earth
  Twin has taken a \emph{straight line path through space-time}. Space
  Twin has taken a \emph{broken line path through space-time}.

  Earth Twin \emph{knows} that they have taken a straight line path
  because they felt no acceleration. Space Twin \emph{knows} they have
  taken a broken line path because they \emph{did} feel an acceleration.
  Their circumstances are \emph{not} symmetric.

  \item Analogue: consider if Alice walks straight up Valencia Street
  from 24th Street to 16th Street. Bob leaves the same time, but he
  walks a block east to Mission Street first, up 8 blocks, and then one
  block west to Valencia again.

  Alice is already waiting for Bob. Bob says: how could you have beaten
  me? Alice says: I \emph{knew} I couldn't have gotten here after you,
  because I took a straight line path. No one could have beaten me.

  \item The Minkowski norm is ``weird'' in the sense that it assigns
  \emph{lesser norm} to broken line paths through space-time than to
  straight line paths connecting their endpoints (in space-time).

  \item Consider the ``arc length'' of a path through space-time. Here
  the arc length is of course measured with respect to the Minkowski
  norm. The arc length is of course equal in all inertial reference
  frames: that follows from our proof that the Minkowski norm is
  invariant under a change of inertial reference frame.

  \item Further, we may approximate a path through space-time with a
  series of line segments. An observer who follows each segment at
  uniform speed will observe that the proper time (equal to the arc
  length) of that segment has passed. Thus, in the limit, the arc length
  of the path is exactly equal to the proper time experienced by the
  traveler along the path.

\end{enumerate}
