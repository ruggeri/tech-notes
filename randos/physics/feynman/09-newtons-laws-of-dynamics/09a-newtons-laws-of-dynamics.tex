\section{Newton's Laws of Dynamics}

\begin{enumerate}

  \item Newton's first law is Galileo's observation: that if no force is
  applied to an object, it will keep moving without change.

  \item Newton's second law tells how velocity can change. Specifically,
  it says that the change in momentum is proportional to the force.

  \item He defines momentum as velocity times \emph{inertia}. Here
  \emph{mass} is really just a way of measuring inertia.

  \item He mentions that velocity is three-dimensional, and that speed
  is the norm of velocity.

  \item Next he considers a spring. He doesn't say how a spring works,
  but he says that the force it exerts on a hanging weight is
  proportional to the length of the spring. Say that the force pulling
  upward is $F = -kx$.

  \item For an object with mass $m$, the force pulling downward is $gm$.
  Thus we are in equilibrium when $kx = gm$, implying $x =
  \frac{gm}{k}$.

  \item Now, let's use this equilibrium position as our reference point;
  let's make it position zero. The \emph{excess} force if we pull the
  object down by $x$ is now $-kx$. Feynman says: assume that $k = m$. In
  that case, we have that the acceleration is $a = -x$.

  \item He now wants to describe the position of the object \emph{as a
  function of time}. What we have here is a \emph{differential
  equation}. That's because $x$ is really a function of $t$ itself!

  \item \TODO{I should probably go back and study differential equations.}

  \item Of course we know that the acceleration is zero when $x = 0$,
  but that's not very relevant here. In reality we expect oscillation
  between $-x$ and $x$.

  \item He shows how we can numerically approximate the solution. Low
  and behold, the solution is $-x \cos t$.

  \item That makes total sense. When an object follows circular motion,
  its velocity is always perpindicular to its position. In fact, the
  velocity vector is at a $90\deg$ angle to the position vector. And
  further, the \emph{acceleration} must be at a $90\deg$ angle to the
  velocity. Which of course means that the acceleration is just a
  reflection through the origin! That gives us our $a = -x$!

  \item He then does a similar numerical approximation to show that the
  orbit of a planet around a stationary sun is elliptical. \TODO{can I
  derive this exactly through mathematics?}

  \item I'll note his differential equations. Consider $x, y$ are
  offsets from the sun. Then note that we have that $v = \frac{F}{m} =
  \frac{GM}{r^2}$. To put this in terms of $x, y$, he notes that
  $\frac{v_x}{v} = \frac{x}{r}$ (and likewise for $v_y$). This follows
  because of similar triangles. Thus we have $v_x = \frac{GMx}{r^3}$
  (and likewise for $v_y$). With the constraint that $r = \sqrt{x^2 +
  y^2}$, we can now solve the differential equation.

  \item Last, he shows that in principle, by doing a numerical
  approximation we can calculate all planetary motion with as many
  planets as we wish.

\end{enumerate}
