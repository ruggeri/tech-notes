\subsection{Pedagogical Note}

I found this chapter very challenging. A first problem is that the
entire discussion is motivated by a conclusion of Maxwell's laws, and we
have studied no EM at this point.

The only thing we really need to know from Maxwell is that light speed
is constant in the aether. But we have never studied any waves. So we
don't know why propagation through a medium should be constant.

Further, the Michelson-Morley experiment is based on detecting whether
some light waves are in or out of phase. But we don't know anything
about phase.

He not only talks about space/time dilation. He also talks about changes
in relativistic mass. I suppose this is important, because the reason to
bring up relativity in the context of our mechanics discussion must
relate to mass/inertia/force. But the horse is before the cart - he
doesn't derive or really justify the relativistic mass formula.
Presumably that happens in the next chapter.

He does a final confusing thing: he ``derives'' the relativistic mass
formula from the concept of total energy $mc^2$. But this assumes that
total energy ``is a thing'' and incorporates both rest energy and
kinetic energy. Where did this total energy concept come from? An
exploration of gas behavior with the relativistic mass formula! We
literally have no intuition for this concept of total energy except that
it is implied by the relativistic mass formula. So how can it build our
intuition for the relativistic mass formula??

I think the student is justified to be frustrated by this chapter.
