\subsection{Center of Mass}

\begin{enumerate}
  \item If we put some forces $\vF_i$ on a body and allow it to start
  moving, it is maybe hard to know how it will move. This is especially
  the case if the system is not rigid: if the system contains many parts
  that may move relative to each other.

  \item Let's consider a system where the parts are allowed to freely
  move relative to each other. Then $\vF_i = m_i \fpartialsq{t} \vx_i$.
  Assuming non-relativistic speeds, we can move the $m_i$ term inside:
  $\vF_i = \fpartialsq{t} m_i \vx_i$. Let's try summing all forces:

  \begin{nedqn}
    \sum_i \vF_i
  \eqcol
    \sum_i \fpartialsq{t} m_i \vx_i
  \\
  \eqcol
    M
    \fpartialsq{t} \sum_i \frac{m_i}{M} \vx_i
  \intertext{We denote $\vx = \sum_i \frac{m_i}{M} \vx_i$:}
  \\
  \eqcol
    M
    \fpartialsq{t} \vx
  \end{nedqn}

  \noindent
  This is another motivation for considering the total external force
  $\vF = \sum_i \vF_i$ an important quantity.

  \item We call $\vx$ the \define{center of mass}. We've talked in terms
  of acceleration. But let's make an (equivalent) investigation of
  momenta:

  \begin{nedqn}
    \sum_i \vp_i
  \eqcol
    \sum_i m_i \fpartial{t} \vx_i
  \\
  \eqcol
    M \fpartial{t} \sum_i \frac{m_i}{M} \vx_i
  \\
  \eqcol
    M \fpartial{t} \vx
  \end{nedqn}

  \noindent
  This justifies us in calling $\vp = \sum_i \vp_i$ the \define{total
  translational momentum}. Unsurprisingly:

  \begin{nedqn}
    \vF
  \eqcol
    \sum_i \vF_i
  \\
  \eqcol
    \sum_i \fpartial{t} \vp_i
  \\
  \eqcol
    \fpartial{t} \sum_i \vp_i
  \\
  \eqcol
    \fpartial{t} \vp
  \end{nedqn}

  \item Here is a key takeaway: the instantaneous change in the center
  of mass is the same no matter on what subpart we apply a force
  $\vF_i$. Compared to applying a force on a miniscule subpart, applying
  $\vF_i$ to a massive subpart will tend to accelerate it more slowly,
  but the instantaneous change in momentum will still be the same.

  \item We've specifically explored a scenario where $\fpartial{t} \vp_i
  = \vF_i$. But what if subparts can exert forces upon each other? For
  instance: what if subparts are exerting forces to pull each other
  together (or push each other apart)? Basically: what if the net force
  is on subpart $i$ is not simply the external force $\vF_i$, but also
  includes some internal forces?

  Indeed, these internal forces can change $\fpartial{t} \vp_i$. But
  they will not change the aggregate $\fpartial{t} \vp$. Per Newton,
  these internal forces always come in equal and opposite pairs. Thus
  internal forces, when both halves of the force pair are accounted for,
  cannot change the net instantaneous change in total momentum.

  Thus internal forces are irrelevant when accounting for change in the
  system's total momentum, AKA change in the position of the center of
  mass.

  \item Thus, we see that, even for a rigid body - for \emph{any} body -
  we still know that $\vF = M \fpartial{t} \vx$.

  \item I even mean \emph{spaceships}. A spaceship cannot move its its
  center of mass. If the spaceship accelerates some exhaust (mass)
  backward, the spaceship gains velocity forward. If you account for the
  exhaust, the center of mass doesn't change. But if you only focus on
  the spaceship (if the exhuast ``exits'' the system under
  consideration), then yes, the center of  mass does change.
\end{enumerate}
