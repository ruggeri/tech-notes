\subsection{Total Rotational Inertia}

\begin{enumerate}
  \item Now, imagine a body that is constrained to rotate about an axis.
  This means the velocity of every particle in the body must be
  perpendicular to its radial distance from the axis. Additionally, the
  angular velocity of every particle must be an equal $\omega$.

  \item By conservation of angular momentum, we know the total
  instantaneous change in angular momentum is:

  \begin{nedqn}
    \tau
  \eqcol
    \fpartial{t} L
  \\
  \eqcol
    \sum_i \fpartial{t} L_i
    \nedcomment{external only}
  \end{nedqn}

  \item Consider each point $i$ on the object. We've assumed that $r_i$
  does not change: that the object must rotate about this axis. (We of
  course also assume that $m_i$ does not change). And we've even assumed
  that $\omega_i$ is always equal to a common $\omega$.

  \item Then we have:

  \begin{nedqn}
    \tau
  \eqcol
    \fpartial{t} L
  \\
  \eqcol
    \sum_i \fpartial{t} L_i
    \nedcomment{external and internal}
  \\
  \eqcol
    \sum_i m_i r_i^2 \fpartial{t} \omega
  \\
  \eqcol
    \parens{\sum_i m_i r_i^2}
    \fpartial{t} \omega
  \\
  \eqcol
    \parens{\sum_i I_i}
    \fpartial{t} \omega
  \end{nedqn}

  \item Thus, by setting total $I = \sum_i I_i$, we have:

  \begin{nedqn}
    \tau
  \eqcol
    I \alpha
  \end{nedqn}

  \noindent
  for aggregate bodies.

  \item Of course, the preceding has assumed that total rotational
  inertia $I$ is constant. But even if net external torque $\tau$ is
  zero, $\alpha$ might be non-zero if $\fpartial{t} I$ is non-zero.

  \item This is what happens when a figure skater moves their leg from
  outside to inside, reducing their rotational inertia. Their angular
  momentum remains constant (net zero torque). But because the inertia
  is decreasing, their angular velocity $\omega$ increases to
  compensate. They experience a positive angular acceleration $\alpha$.

  \item More generally, we may be applying a non-zero torque to a
  system, but not see the expected angular acceleration, provided the
  system is undergoing a simultaneous change in its rotational inertia.

  \item Last, we constrained our body to rotate about a fixed axis. But
  how do we choose an axis such that the body is constrained to rotate
  about it?
\end{enumerate}
