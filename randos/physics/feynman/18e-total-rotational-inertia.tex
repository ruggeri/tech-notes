\subsection{Total Rotational Inertia}

\begin{enumerate}
  \item For a point mass, we've related $L_i$ with $\omega_i$ by
  defining $L_i = I_i \omega_i$. But we have this synthetic quantity $L
  = \sum_i L_i$. Does this relate to some ``aggregate'' angular velocity
  $\omega$ of anything? Is there any corresponding aggregate rotational
  inertia $I$?

  \item Consider an aggregate object that is rotating with an angular
  velocity $\omega$. Then each subpart $i$ has angular momentum $L_i =
  I_i \omega$. The total angular momentum is then:

  \begin{nedqn}
    L
  =
    \sum_i L_i
  =
    \parens{\sum_i I_i} \omega
  \end{nedqn}

  \noindent
  Thus we are motivated to define $I = \sum_i I_i = \sum_i m_i r_i^2$ as
  the \define{total rotational inertia} of the object.

  \item We know that the net external torque is equal to the
  instantaneous change in total angular momentum. Now, a net external
  torque does not always correspond to pure rotation of the object. For
  instance, the object may not be rigid, so different parts may
  experience different angular accelerations $\alpha_i$.

  \item However, we may consider the important case where (1) $m_i, r_i$
  are fixed and (2) all $\omega_i$ are required to stay equal. This
  corresponds to the requirement that the object is rigid and
  constrained to rotate about the axis. In this case, we do have that

  \begin{nedqn}
    \tau
  \eqcol
    \fpartial{t} L
  \\
  \eqcol
    \fpartial{t} \sum_i L_i
  \\
  \eqcol
    \sum_i m_i r_i^2 \fpartial{t} \omega
  \\
  \eqcol
    I \alpha
  \end{nedqn}

  \item Of course, the preceding has assumed that total rotational
  inertia $I$ is constant. But even if net external torque $\tau$ is
  zero, $\alpha$ might be non-zero if $\fpartial{t} I$ is non-zero.

  \item This is what happens when a figure skater moves their leg from
  outside to inside, reducing their rotational inertia. Their angular
  momentum remains constant (net zero torque). But because the inertia
  is decreasing, their angular velocity $\omega$ increases to
  compensate. They experience a positive angular acceleration $\alpha$.

  \item More generally, we may be applying a non-zero torque to a
  system, but not see the expected angular acceleration, provided the
  system is undergoing a simultaneous change in its rotational inertia.
\end{enumerate}
