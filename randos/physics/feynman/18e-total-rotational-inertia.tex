\subsection{Total Rotational Inertia}

\begin{enumerate}
  \item We know that for subparts $i$, $L_i = m_i r_i^2 \omega_i$. And
  we know that (inclusive of internal torques) $\tau_i = \fpartial{t}
  L_i = I_i \alpha$.

  \item \TODO{We should note that this is true only if the distance
  $r_i$} is fixed! Which is why we will want to consider only rotations
  about the center of mass...

  \item Per our discussion, we know that $\tau = \fpartial{t} L$, where
  $L$ has been defined simply as the sum of the angular momenta $L_i$ of
  the various subparts $i$. But is there any similar sense in which $L$
  relates to some $\alpha$?

  \item Well, let us consider that the object cannot be deformed. Then
  its we must have that $\alpha_i$ is constant for all subparts $i$. We
  will denote this $\alpha$.

  \item We thus know that for every subpart $i$, when inclusive of
  internal torques, $\tau_i = m_i r_i^2 \alpha$. But how do we calculate
  $\alpha$ from the external torques? Similar to our discussion of net
  translational force, we hope to ignore entirely the subparts. Thus we
  hope that we can calculate $\alpha$ from the aggregate external torque
  $\tau$.

  \item Indeed we can! We know that, $\sum_i \tau_i$ is constant whether
  we take $\tau_i$ to be just the external torque, or inclusive of
  external torques. Thus:

  \begin{nedqn}
    \tau
  \eqcol
    \sum_i m_i r_i^2 \alpha
  \\
  \eqcol
    \parens{\sum_i m_i r_i^2} \alpha
  \end{nedqn}

  \item This justifies us in calling the sum $\sum_i m_i r_i^2$ the
  \define{total rotational inertia} of the object.

  \item We make a final note: whereas we cannot change the translational
  inertia of an object (which is $M = \sum_i m$), we \emph{can} change
  the rotational inertia. This is what a figure skater does as they
  enter a spin. They move their leg from outside to inside, reducing
  their angular inertia. Their angular momentum remains constant. But
  because the inertia is decreasing, their angular velocity $\omega$
  increases to compensate.
\end{enumerate}
