\section{Probability}

\begin{enumerate}

  \item He begins by summarizing what is frequentist probability.

  \item He defines the binomial distribution.

  \item Next he explores random walks. After $N$ steps, we expect to be
  a distance of $0$ away from the origin. But if we consider $d^2$, we
  get an expectation of $N$. That is: the variance in the outcome is
  linear with the number of steps. He derives this, but I won't bother
  here.

  \item We can take the root-mean-square, $\sqrt{\expectation{d^2}}$.
  This is supposedly a more useful number in terms oof the original
  distance, though I'm not sure that's obvious... I guess this number is
  supposed to be more similar to $\expectation{\abs{d}}$.

  \item Next he discusses \emph{Brownian motion}. Here, the step
  \emph{size} also varies. We shall say that the \emph{average} step
  size is $1$, in which case he shows that $\expectation{d^2}$ is still
  $N$.

  \item He must (and he does) introduce the concept of \emph{probability
  density}. He doesn't show (of course), that for large $N$, it doesn't
  matter what the distribution is on step size. All that matters is the
  expected value.

  \item Of course, we now arrive at the Gaussian distribution.

  \item He hints how this will be useful in our study of gasses. He
  shows that Maxwell will use these kinds of tools to find a probability
  distribution over the velocity of particles in a gas.

  \item Of course this is important in determining the pressure.

  \item He last talks a bit about uncertainty again. He mentions that
  really by uncertainty we're talking about the standard deviations. So
  we have:

  \begin{nedqn}
    \sigma_x \sigma_v
  \geqcol
    \frac{\hslash}{2m}
  \end{nedqn}

  \item The uncertainty principle is part of saying that even the most
  fundamental description of nature requires us to speak in terms of
  probabilities.

  \item We'll eventually see that the \emph{radius of the atom} is in
  some sense related to the standard deviation of a probability
  distribution saying how far away an electron will be from the
  nucleus...
\end{enumerate}
