\subsection{Angular Momentum, Acceleration, and Inertia}

\begin{enumerate}
  \item We've already defined torque in the preceding section. Let's
  next consider how torque changes the angular speed about an axis. In
  doing so, we'll first focus on a point mass.

  \item Consider that a force $\vF$ is applied at a distance $r$ from an
  axis. Let us assume that the force is exactly tangential to the
  radius. Then we know that the instantaneous change in velocity is:

  \begin{nedqn}
    \vF
  \eqcol
    m \fpartial{t} \vv
  \end{nedqn}

  \item We may translate this to terms of \define{angular velocity} and
  \define{angular acceleration}. A tangential velocity $v$ translates to
  an angular velocity $\omega = \frac{v}{r}$. Presuming that a force
  $\vF$ is tangential, then $\vv$ is also tangential. Thus:

  \begin{nedqn}
    F
  \eqcol
    m \fpartial{t} v
  \\
    F
  \eqcol
    m \fpartial{t} \omega r
  \\
    F
  \eqcol
    mr \fpartial{t} \omega
  \intertext{or, denoting $\alpha = \fpartial{t} \omega$:}
    F
  \eqcol
    m r \alpha
  \end{nedqn}

  \item Now we will translate to torque. Since $\tau = Fr$, we know:

  \begin{nedqn}
    \tau
  \eqcol
    m r^2 \alpha
  \end{nedqn}

  \noindent
  We call the quantity $I = m r^2$ the \define{moment of inertia} for
  the body about the specified axis. Just as torque is relative to a
  choice of axis, so is moment of inertia. Thus we write:

  \begin{nedqn}
    \tau
  \eqcol
    I \alpha
  \end{nedqn}

  \item Of course, if a constant torque is applied for time $t$:

  \begin{nedqn}
    \tau t
  \eqcol
    I \Delta \omega
  \end{nedqn}

  \item We last introduce one more formalism: we call $L = I\omega$ the
  \define{angular momentum} of the particle. Thus torque represents the
  instantaneous change in the angular momentum. That is:

  \begin{nedqn}
    \tau
  \eqcol
    \fpartial{t} L
  \end{nedqn}

  \item Note that this whole time I have just been talking in new
  mathematical terms about an old subject. Nothing has really changed.
  This is all symbols and formalism.
\end{enumerate}
