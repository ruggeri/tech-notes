\subsection{Coriolis Force}

\begin{enumerate}

  \item Feynman notes: consider just your torso when moving a leg in.
  The torso angular velocity has increased, even though its inertia has
  not changed (the torso didn't move; only the leg). What happened to
  the torso when the leg was moved in? There must be some torque
  applied!

  \item Note that this torque cannot arise from the centripetal force,
  because the centripetal force is radial, and radial forces do not give
  torque.

  \item As the leg moves inward, it wants to increase its angular
  velocity so that angular momentum is conserved. But as the leg starts
  to speed up, it encounters the rotational inertia of the core body.
  The leg wants to run ahead of the body, but of course it is attached
  to the body. Thus the central body gets a push from the leg. The leg,
  for its part is pushed backward by the body.

  \item Imagine things this way: a central part and a more distant
  connected part have equal inertia $I$. They rotate at the same rate
  $\omega$. Then they are briefly disconnected. The outer part goes
  through these changes:

  \begin{nedqn}
    r \mapstocol r/2
  \\
    I \mapstocol I/4
  \\
    \omega \mapstocol 4\omega
  \\
    v \mapstocol 2v
  \end{nedqn}

  \noindent
  Now reconnect the two parts. It's clear that the core part is ``going
  around too slowly'' while the outer part is ``going around too
  quickly.'' We need to conserve $L = \frac{5}{4} \omega$. So there must
  be a torque against the outer part (slow it down) and a torque with
  the inner part (speed it up).

  \item The force that gives rise to this is called the \define{Coriolis
  force}. Like the centrifugal force, it is ``fictitious.'' It only
  appears when we try to move things in a rotating frame of reference.

  \item The Coriolis force can also be observed if you are on a
  carousel. If you start from an inner point and try to walk to an outer
  point, you will feel the need to lean in the direction of rotation.
  The reason is that your angular momentum is increasing as you move
  outward. Thus, some torque must be applied to increase your angular
  momentum.

  We can calculate the magnitude of the Coriolis force. If you are
  moving radially at a velocity $v_r$, then:

  \begin{nedqn}
    \fpartial{t} L
  \eqcol
    \fpartial{t} I \omega
  \\
  \eqcol
    \fpartial{t} m r^2 \omega
  \\
  \eqcol
    m \omega 2r v_r
  \intertext{let's focus on in on the torque/force:}
    \tau
  \eqcol
    m \omega 2 r v_r
  \\
    F_\text{coriolis} r
  \eqcol
    m \omega 2 r v_r
  \\
    F_\text{coriolis}
  \eqcol
    2m \omega v_r
  \end{nedqn}
\end{enumerate}
