\subsection{Total Translational Force}

\begin{enumerate}
  \item Consider a set of external forces $\vF_i$ acting on subparts $i$
  of a body at points $\vx_i$.

  \item If no internal forces develop, then $\vx_i$ will start to
  change. The change is described by $\va_i = \frac{\vF_i}{m_i}$, per
  the definitions of force and mass.

  \item However, if internal forces \emph{do} develop, then we don't
  actually know how any particular $\vx_i$ will change! Is there
  anything we \emph{do} know about the object's motion, regardless of
  internal forces?

  \item If we want to ignore internal forces, then in a sense we want to
  ignore that the object is composed of separate parts. Could we pretend
  that all the forces $\vF_i$, instead of operating at different
  subparts, are all operating on a single aggregate? In that case, the
  total force on the aggregate would be $\vF = \sum_i \vF_i$. We call
  this the \define{net external translational force}.

  \item Likewise, we would want need to aggregate the inertia of the
  various parts. Summing the inertia of the various subparts would give
  us an aggregate \define{total translational inertia} of $M = \sum_i
  m_i$.

  \item Is there any sense in which the object behaves as if it is a
  point-mass, and that all the force and all the mass are concentrated
  there? Let us imagine. The location of this point-mass we will denote
  $\vx$. We don't yet know what $\vx$ should be. We only know that we
  want that $\va = \frac{\vF}{M}$.

  \item Is there any way to define $\vx$ in terms of just the subparts'
  positions $\vx_i$ and masses $m_i$?

  \begin{nedqn}
    \va
  \eqcol
    \frac{\vF}{M}
  \\
  \eqcol
    \frac{1}{M} \sum_i \vF_i
  \\
  \eqcol
    \frac{1}{M} \sum_i \va_i m_i
  \\
    \fpartialsq{t} \vx
  \eqcol
    \sum_i \frac{m_i}{M} \fpartialsq{t} \vx_i
  \\
    \fpartialsq{t} \vx
  \eqcol
    \fpartialsq{t}
    \sum_i \frac{m_i}{M} \vx_i
  \end{nedqn}

  \item This tells us that if we want to imagine that the object is
  collapsed to a point $\vx = \sum_i \frac{m_i}{M} \vx_i$, then the net
  external force $\vF$ and the net translational inertia $M$ will tell
  us the instantaneous change in $\vx$. We call $\vx$ the \define{center
  of mass}.

  \item Our proof above relied on the assumption that $\va_i =
  \frac{\vF_i}{m_i}$. I've previously said that $\vF_i$ is supposed to
  be the \emph{external} force. But if we stick to that, then $\va_i$ is
  only going to equal $\frac{\vF_i}{m_i}$ if there are no internal
  forces.

  \item So let's say that $\vF_i$ is maybe \emph{not} simply the
  external force, but actually inclusive of all the internal forces.
  This will allow us to be assured that $\va_i = \frac{\vF_i}{m_i}$.
  Then it is true that the center of mass $\vx$ accelerates per $M \va =
  \sum_i \vF_i$.

  \item The question then is: does $\sum_i \vF_i$ change whether we take
  $\vF_i$ to be just the external force, or inclusive of all internal
  forces on subpart $i$?

  \item And here comes the last point: internal forces always come in
  equal and opposite pairs. So even if we take $\vF_i$ to possibly be
  inclusive of internal forces, the sum $\sum_i \vF_i$ will still come
  out the same as if we only considered external forces. Thus, when
  defining $\vF = \sum_i \vF_i$, we can ignore any internal forces that
  develop.

  \item If we like, we can talk in terms of change of momentum
  $\fpartial{t} \vp_i$ instead of mass-weighted acceleration $m \va_i$.
  It's all equivalent. It is simple to show that if we define $\vp = M
  \vv$, then we will have $\vF = \fpartial{t} \vp$, and that $\vp =
  \sum_i \vp_i$. It can be convenient to talk in terms of momentum,
  because it is easier to aggregate the subpart momentums $\vp_i$ than
  to aggregate the subpart accelerations $\va_i$.

  \item Feynman gives an example of spaceships. If there is no external
  force of the spaceship, then no matter the internal forces, there can
  be no change in spaceship momentum. But doesn't that eliminate the
  possibility of propulsion? Not exactly. A spaceship moves by ejecting
  some mass (exhaust), and then considering that mass as ``no longer
  part of the spaceship.'' No exhaust, no propulsion.
\end{enumerate}
