\newcommand{\cross}{\times}

\subsection{Tips and Tricks For CoM and Inertia Calculations}

\begin{enumerate}
  \item He introduces a theorem of Pappus. Say that a volume is
  generated by taking a closed planar area and moving in through space.
  If the motion is entirely perpendicular to the planar area, then every
  point in the planar area moves through the same distance. The volume
  is thus area time distance.

  \item But Pappus further considers rotating the area through space, or
  in fact any such motion. His theorem says that the volume generated is
  equal to the area times \emph{the distance the plane's CoM moves}.

  \item Feynman then uses this to compute the center of mass of some
  planar shapes.

  \item Feynman next calculates the inertia for rotating a rod of length
  $L$ about one end. It is:

  \begin{nedqn}
      I
    \eqcol
      \int_0^L x^2 \frac{M}{L} \dx
    \\
    \eqcol
      \frac{L^3}{3} \frac{M}{L}
    \\
    \eqcol
      \frac{L^2M}{3}
  \end{nedqn}

  \item He does a similar calculation and computes that swinging about
  the CoM is $\frac{L^2M}{12}$
\end{enumerate}

\subsection{Parallel Axis Theorem}

\newcommand{\CoM}{\text{CoM}}

\begin{enumerate}
  \item Consider rotating a body not about its center of mass, but some
  other, parallel axis. We may ask: what is the rotational inertia of
  the body, when considered about this axis.

  \item We know that $I = \sum_i \normsq{\vx_i} m_i$, where $\vx_i$ is
  the position relative of particle $i$ to the axis.

  \item We would like to break down the motion into \emph{two} kinds of
  rotation: rotation of the center of mass about the reference axis, and
  rotation of the body about the center of mass.

  \item We'll write $\vy_i = \vx_i - \vx_\CoM$, and see what happens
  when we apply this substitution for our rotational inertia equation.

  \begin{nedqn}
    I
  \eqcol
    \sum_i \normsq{\vx_i} m_i
  \\
  \eqcol
    \sum_i \normsq{\vy_i + \vx_\CoM} m_i
  \\
  \eqcol
    \sum_i
    \normsq{\vy_i} m_i
    + 2 \vy_i \cdot \vx_\CoM m_i
    + \normsq{\vx_\CoM} m_i
  \intertext{
    the middle term is zero, because the CoM is the mass-weighted center
    of the body, thus
  }
  \\
    I
  \eqcol
    \sum_i \normsq{\vy_i} m_i
    + \normsq{\vx_\CoM} m_i
  \\
  \eqcol
    I_\CoM + \normsq{\vx_\CoM} M
  \end{nedqn}

  \item I will prove a similar theorem about torque. Here, I will use a
  vectorized version of torque: $\vtau_i = \vx_i \cross \vF_i$. Then:

  \begin{nedqn}
    \vtau
  \eqcol
    \sum_i \vtau_i
  \\
  \eqcol
    \sum_i \vx_i \cross \vF_i
  \\
  \eqcol
    \sum_i \parens{\vy_i + \vx_\CoM} \cross \vF_i
  \\
  \eqcol
    \parens{\sum_i \vy_i \cross \vF_i}
    + \parens{\sum_i \vx_\CoM \cross \vF_i}
  \\
  \eqcol
    \vtau_\CoM + \vx_\CoM \cross \vF
  \end{nedqn}

  \item Here is a similar proof breaking down angular momentum:

  \begin{nedqn}
    \vL
  \eqcol
    \sum_i \vx_i \cross \vp_i
  \\
  \eqcol
    \sum_i \vx_i \cross (m_i \vv_i)
  \\
  \eqcol
    \sum_i m_i \parens{\vy_i + \vx_\CoM} \cross \vv_i
  \\
  \eqcol
    \sum_i m_i \parens{\vy_i \cross \vv_i} + m_i \parens{\vx_\CoM \cross \vv_i}
  \\
  \eqcol
    \vL_\CoM + \vx_\CoM \cross M\vv_\CoM
  \\
  \eqcol
    \vL_\CoM + \vx_\CoM \cross M (\vomega \cross \vx_\CoM)
  \\
  \eqcol
    \vL_\CoM + \normsq{\vx_\CoM} M \vomega
  \end{nedqn}

  \item Here I have used the distributive rule of cross products only. I
  will prove that shortly. The last equation uses a trick that $\vx
  \cross \vy \cross \vx$ is equal to $\normsq{\vx} \vy$.

  \item This shows that if we have a non-zero torque about the center of
  mass, but a zero total force on a body, then the torque about any
  parallel axis remains the same as if we chose the axis through the
  center of mass.

  \item Correspondingly, if there is zero torque about the center of
  mass, but a total force exists on the body, then the torque is the
  same as if all the total force were being applied at the center of
  mass.

  \item We can further see that, if an object is spinning about its
  center of mass without its center of mass moving, then the angular
  momentum about any axis is the same as if we just considered an axis
  through the center of mass.

  \item Likewise, if the object is not spinning, but is being
  translated, then we see that the angular momentum is the same as if
  all the mass were located at the center of mass.

  \item It is very convenient to note that if we break down $\vtau$ into
  $\vtau_\CoM$ and $\vx_\CoM \cross \vF$, then these respectively are
  equal to $\fpartial{t} \vL_\CoM$ and $\fpartial{t} \vx_\CoM \cross M
  \vv_\CoM$.

  \item Thus we see that we can break down torque and angular momentum
  into ``internal'' and ``external'' parts.

  \item Question: why break down specifically with regard to the CoM?
  Can we not use any other location? I believe yes (as a matter of
  algebra), but we will lose the correspondence that $\vtau_\CoM =
  \fpartial{t} \vL_\CoM$ and $\vx_\CoM \cross \vF = \fpartial{t}
  \vx_\CoM \cross M \vv_\CoM$.

  \item These decompositions gives us another reason to believe that $I
  = I_\CoM + M \normsq{\vx_\CoM}$. This is true because to rotate the
  body about an external axis, we must use the external torque to (1)
  get the body rotating about its center of mass and (2) get the center
  of mass rotating about the external axis.

  \item \TODO{Ideally, I would be able to explain why it is intuitive
  that a body spinning about its center of mass has angular momentum
  $\vL$ about its CoM equal to its angular momentum about any other
  origin point.} Algebraically we showed this simply and trivially, but
  I don't think its intuitive that this should be the case.
\end{enumerate}

\subsection{The Cross Product}

\begin{enumerate}
  \item We will use the cross-product as a sort of opposite of the dot
  product. Whereas the dot-product $\vx \cdot \vy$ is $\norm{\vx}
  \norm{\vy}$ when $\vx, \vy$ are parallel, we want $\norm{\vx \cross
  \vy} = \norm{\vx} \norm{\vy}$ when $\vx$ and $\vy$ are perpendicular.

  \item Let us start by defining the cross-product of unit vectors. We
  define:

  \begin{nedqn}
    \vi \cross \vj
  \eqcol
    \vk
  \\
    \vj \cross \vk
  \eqcol
    \vi
  \\
    \vk \cross \vi
  \eqcol
    \vj
  \end{nedqn}

  \item We see that the a cross-product of two basis vectors maps to the
  third (orthogonal) basis vector.

  \item As a mnemonic, consider your right hand. Assume that the vectors
  $\vi$, $\vj$, and $\vk$ are oriented such that you can rotate your
  hand so that index, middle, and thumb fingers correspond to each.

  \item Then you can always find the output of $\vx \cross \vy$ by
  orienting your index finger with $\vx$, your middle finger with $\vy$,
  and your thumb is oriented in the direction of the output.

  \item Note that the right-hand rule should imply that the
  cross-product is \define{anti-symmetric}: $\vx \cross \vy =
  -\parens{\vy \cross \vx}$. This still produces a mutually orthogonal
  vector, but with the opposite orientation. We will find this
  convenient.

  \item Thus we have:

  \begin{nedqn}
    \vj \cross \vi
  \eqcol
    -\vk
  \\
    \vk \cross \vj
  \eqcol
    -\vi
  \\
    \vi \cross \vk
  \eqcol
    -\vj
  \end{nedqn}

  \item Next, I will define:

  \begin{nedqn}
    \vi \cross \vi
  \eqcol
    \vec{0}
  \\
    \vj \cross \vj
  \eqcol
    \vec{0}
  \\
    \vk \cross \vk
  \eqcol
    \vec{0}
  \end{nedqn}

  \noindent
  We need this if we want the norm of the cross-product to reflect not
  only the product of the norms of the input vectors, but also how
  orthogonal these vectors were to begin with.

  \item I can actually generalize in this way:

  \begin{nedqn}
    \alpha \parens{
      \cos\theta \vi + \sin\theta \vj
    }
    \cross
    \beta \parens{
      \cos\phi \vj - \sin\phi \vi
    }
  \eqcol
    \alpha\beta \sinf{\frac{\pi}{2} + \phi - \theta} \vk
  \\
    \alpha \parens{
      \cos\theta \vj + \sin\theta \vk
    }
    \cross
    \beta \parens{
      \cos\phi \vk - \sin\phi \vj
    }
  \eqcol
    \alpha\beta \sinf{\frac{\pi}{2} + \phi - \theta} \vi
  \\
    \alpha \parens{
      \cos\theta \vk + \sin\theta \vi
    }
    \cross
    \beta \parens{
      \cos\phi \vi - \sin\phi \vk
    }
  \eqcol
    \alpha\beta \sinf{\frac{\pi}{2} + \phi - \theta} \vj
  \end{nedqn}

  \item This corresponds exactly with the calculation we want to perform
  when the input vectors lie in the $(i,j)$, $(j,k)$, or $(i,k)$ planes.

  \item Or even more generally: for perpendicular $\vx, \vy$, we want:

  \begin{nedqn}
    \alpha \parens{
      \cos\theta \vx + \sin\theta \vy
    }
    \cross
    \beta \parens{
      \cos\phi \vy - \sin\phi \vx
    }
  \eqcol
    \alpha\beta
    \sinf{\frac{\pi}{2} + \phi - \theta}
    \parens{\vx \cross \vy}
  \end{nedqn}

  \item Let's consider another kind of relationship. Consider two
  non-parallel $\vx, \vy$. We expect that the norm of $\vx \cross \vy$
  is proportional to $\norm{\vx} \norm{\vy} \sin\theta$, where $\theta$
  is the angle between the two.

  Now, consider $\parens{\vx + \alpha \vx} \cross \vy$. Effectively we
  have scaled $\vx$ by $1 + \alpha$. Thus the cross product is
  $\parens{1 + \alpha} \vx \cross \vy$. But note this is exactly equal
  to $\parens{\vx \cross \vy} + \parens{\alpha\vx \cross \vy}$. That is:
  we can distribute the cross product over this sum.

  We next consider $\parens{\vx + \alpha \vy} \cross \vy$. We know that
  this will change the projection of $\vx + \alpha \vy$ onto $\vy$: by
  exactly $\vy$. However, the perpendicular component of $\vx + \alpha
  \vy$ remains unchanged. Thus $\parens{\vx + \alpha \vy} \cross \vy$
  remains $\vx \cross \vy$. Note that $\alpha \vy \cross \vy = 0$, so we
  can say that $\parens{\vx + \alpha \vy} \cross \vy = \vx \cross \vy +
  \alpha \vy \cross \vy$. Again, we see that the cross product
  distributes over addition.

  \item But we must extend our notion outside the planes defined by the
  basis vectors. Only then will the story be complete. Consider
  perpendicular $\vx, \vy$. We want to consider rotating $\vx$ in the
  plane defined by $\vx, \vx \cross \vy$. Let $\vz = \frac{\vx \cross
  \vy}{\norm{\vy}}$. Then:

  \begin{nedqn}
      \parens{
        \cos\theta \vx + \sin\theta \vz
      } \cross \vy
    \eqcol
      \cos\theta \parens{\vx \cross \vy} - \sin\theta \vx
  \end{nedqn}

  \noindent
  as this rotates the output away from $\vx$ exactly as the first input
  is rotated toward $\vz$. Also, note that $-\vx = \vz \cross \vy$.
  Thus, we have our typical distribution law for perpendicular unit
  vectors $\vx, \vy, \vz = \vx \cross \vy$:

  \begin{nedqn}
      \alpha \parens{
        \cos\theta \vx + \sin\theta \vz
      } \cross \beta \vy
    \eqcol
      \alpha \beta \parens{
        \cos\theta \parens{\vx \cross \vy}
        + \sin\theta \parens{\vz \cross \vy}
      }
  \end{nedqn}


  \item Last, we should inquire about how to extend our definition to
  non-basis vectors. This will be accomplished by adopting a
  distribution law:

  \begin{nedqn}
    &&
    \parens{
      \alpha_i \vi + \alpha_j \vj + \alpha_k \vk
    }
    \cross
    \parens{
      \beta_i \vi + \beta_j \vj + \beta_k \vk
    }
  \\
  &&
  \qquad
  =
    \parens{
      \alpha_j \beta_k - \alpha_k \beta_j
    } \vi
    +
    \parens{
      \alpha_k \beta_i - \alpha_i \beta_k
    } \vj
    +
    \parens{
      \alpha_i \beta_j - \alpha_j \beta_i
    } \vk
  \end{nedqn}
\end{enumerate}
