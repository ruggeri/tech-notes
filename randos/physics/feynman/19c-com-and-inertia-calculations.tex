\subsection{Tips and Tricks For CoM and Inertia Calculations}

\begin{enumerate}
  \item He introduces a theorem of Pappus. Say that a volume is
  generated by taking a closed planar area and moving in through space.
  If the motion is entirely perpendicular to the planar area, then every
  point in the planar area moves through the same distance. The volume
  is thus area time distance.

  \item But Pappus further considers rotating the area through space, or
  in fact any such motion. His theorem says that the volume generated is
  equal to the area times \emph{the distance the plane's CoM moves}.

  \item Feynman then uses this to compute the center of mass of some
  planar shapes.

  \item Feynman next calculates the inertia for rotating a rod of length
  $L$ about one end. It is:

  \begin{nedqn}
      I
    \eqcol
      \int_0^L x^2 \frac{M}{L} \dx
    \\
    \eqcol
      \frac{L^3}{3} \frac{M}{L}
    \\
    \eqcol
      \frac{L^2M}{3}
  \end{nedqn}

  \item He does a similar calculation and computes that swinging about
  the CoM is $\frac{L^2M}{12}$

  \item He next develops the \define{parallel-axis theorem}. He asks us
  to consider swinging a rod about an axis. Reduce the rod to simply a
  point mass at its CoM. Then the inertia about the axis is $M
  x_\text{CM}^2$. But when we rotate the rod, we not only rotate the
  CoM, but also rotate \emph{about} the CoM. Thus we should also correct
  by $I_C = \sum_i m_i x_i^2$, the inertia about the center of mass
  (moment-of-inertia).

  \item He proves this like so: let $x_i = x_i' + x_\text{CM}$
  ($x_\text{CM}$ is the location of the center of mass). Then:

  \begin{nedqn}
    x_i^2
  \eqcol
    x_i'^2 + 2 x_i' x_\text{CM} + x_\text{CM}^2
  \end{nedqn}

  \noindent
  Then consider $\sum_i m_i x_i^2$. Then $\sum_i m_i x_i'^2$ is simply
  the moment of inertia and $\sum_i m_i x_\text{CM}^2 = M
  x_\text{CM}^2$. The middle term sums $\sum_i m_i x_i'$, but this is
  zero because it is a CoM calculation where origin is already defined
  to be CoM.

  \item This proves the theorem for one dimension, but it holds in
  three. However, it is important that the axes under consideration be
  \emph{parallel}!
\end{enumerate}
