\section{Work and Potential Energy (A)}

\begin{enumerate}

  \item He starts by examining gravitational potential energy. He
  defines this as $mgh$. He defines kinetic energy as $\half m v^2$.

  \item If you take the derivative of kinetic energy with respect to
  time, you get $mv \fpartial[v]{t} = mav = Fv$. Note this works
  vectorwise too:

  \begin{nedqn}
    \half m \normsq{\vv}
  \eqcol
    \half m (\vv \cdot \vv)
  \\
    \fpartial{t} \left(
      \half m (\vv \cdot \vv)
    \right)
  \eqcol
    \half m \vv \cdot \va
    +
    \half m \va \cdot \vv
  \\
  \eqcol
    m \va \cdot \vv
  \\
  \eqcol
    \vF \cdot \vv
  \end{nedqn}


  \item Now imagine we are an object that is moving through a field of
  force. As we move through time our position $\vs$ is changing. I
  define that the \define{work} done to move the object through the
  field is equal to $\int \vF \cdot \diff{\vs}$.

  How is the total work changing with time? This is equal to $\vF \cdot
  \fpartial[\vs]{t} = \vF \cdot \vv$.

  We thus see that, at any time $t$, the rate of change in work
  performed exactly equals the rate of change in kinetic energy.

  \item We can define a \define{change in potential energy} that is
  caused by moving an object from point A to point B as the integral
  $-\int_A^B \vF \cdot \diff{\vs}$. Of course, we see that a change in
  potential energy exactly offsets the change in the kinetic energy.

  \item What is this \define{potential energy}? From a fixed starting
  point A, we say that the object's potential energy is the supremum of
  $\int \vF \cdot \diff{\vs}$ when integrated over any path (finite or
  infinite).

  Of course, the potential energy could be infinite. For instance, if we
  are considering a gravitational field, the field gets very strong as
  you approach a point mass, so the potential energy blows up.

  On the other hand, if you consider (repulsive) electrostatics, even if
  you integrate out to infinity, because the force is dropping
  proportional the square of the distance, you only get a finite
  potential energy...

  \item By the nature of the definition of potential, I think it is
  clear that there can be no change of total energy. Consider the
  potential at point A. We know that it can be \emph{no greater} than
  the potential at point B plus the change in potential when moving from
  point B to point A. We can make the symmetric argument, so that the
  change in potential really is the integral of $\int \vF \cdot
  \diff{\vs}$ from A to B. But moving from A to B changes kinetic energy
  by precisely that amount.

  \item I believe this ``proves'' conservation of energy. But it
  basically comes down to the definition of potential energy.

  \item In another document I've explored why \emph{energy} is a better
  measure of ability to do ``work'' than \emph{momentum}. One hint is
  that energy can be stored even in things that are not moving (momentum
  cannot). A second is that kinetic energy and potential energy are
  interchangeable.

  It is confused to think: ``if I am holding up a weight via a pulley,
  and I am pulley on the rope, am I not expending energy? Aren't I
  getting tired because I am expending energy on holding the heavy
  object via the rope?''

  But you can see something is wrong with this thinking if you simply
  tie your end of the rope to a bolt on the ground and walk away. Is the
  earth expending ``energy'' to hold the suspended object?

  The perceived energy that is expended by the human comes from the
  chemical energy that is consumed in order to apply an electrical
  current (or whatever) to contract your muscles. A side product is
  lactic acid which you feel as fatigue.

  Note of course that the bolt can hold the object forever. So it seems
  wrong to think that \emph{time} would be a factor in energy
  dissipation, because the bolt can suspend the object for
  \emph{infinite time} while expending \emph{zero energy}.

  On the other end, imagine the amount of work that can be performed by
  dropping a weight from $h_1$ to $h_2$. If you tie a rope to
  the weight, you can lift a second object as the weight drops. You can
  lift it quickly or slowly (it for instance depends on the initial
  velocity of the lifting weight - that doesn't need to be zero). But
  regardless, but no matter what, $\int \vF \cdot \diff{\vs}$ will
  always be constant.

  \item Now, another question is: can you drop a weight of \SI{10}{N} a
  distance of \SI{1}{m} to lift a weight a weight of \SI{1}{N} a
  distance of \SI{10}{m}?  More important, could we life the \SI{10}{N}
  weight with just a \SI{1}{N} weight??

  That would make energy truly \emph{fungible}. It would mean that we
  could transform any energy storeage A to perform any work B.

  We can start to see the answer in pulleys, but I won't go further into
  this.

  \item Back to the book. He mentions that we \define{Joule} is
  preferred to the synonymous \define{Newton meter}.

  \item Last, he works some examples. First, imagine a charged planar
  plate extending outward to infinity. What electric field does it
  produce? He notes that by symmetry, it is everywhere perpindicular to
  the plane. He also notes that it is everywhere constant in magnitude.
  That is maybe surprising: why doesn't it change when we are ``closer''
  to the plane?

  I think you can see this by examining the absurdity of ``closer to the
  plane.'' If the plane extends infinitely, how could you ever ``see''
  that you were closer to it? By moving closer, everything looks the
  same as before, maybe only the units of distance have changed.

  He considers two parallel infinite plates with opposite charge. Note
  that the electric field is (a) zero ouside the plates, and (b) doubled
  inside the plates.

  \item He also considers the gravitational field induced by a thin
  spherical shell on a point outside the shell. He shows we can
  simplify and just assume all the mass is at the center of gravity.

  By showing this for a shell of infinitesimal width, he can extend the
  argument to a sphere, justifying our treatment of planets as point
  masses.

  He also notes that the field is exactly zero \emph{within} a spherical
  shell. Thus a Dyson sphere has no gravitational impact on anyone
  inside.

  I might note: you can't do the same thing to cancel out the effect of
  charges. We discussed that in a previous chapter.
\end{enumerate}
