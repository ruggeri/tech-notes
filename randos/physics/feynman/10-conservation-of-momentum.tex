\section{Conservation of Momentum}

\begin{enumerate}

  \item He first notes that even as soon as you get to three bodies, you
  cannot find a closed form solution for position in terms of time. You
  have to do something like his numerical approach.

  \item Thus, it can be helpful to find certain principles that let you
  simplify.

  \item Newton's third law says that all forces have an equal and
  opposite force. It follows trivially that momenum must be conserved.

  \item However, we will see that Newton's third law is \emph{not
  correct} in QM. And yet we will also see that momentum is still
  conserved.

  \item So he now tries to derive conservation of momentum from another
  angle. He talks about \emph{Galilean invariance}. This says that the
  laws of physics will all work the same no matter the velocity of the
  observer.

  \item He now does a series of thought experiments. He studies first
  perfectly \emph{inelastic} collisions. Inelastic collisions are the
  kind where energy is ``lost'' due to deformation. Let's assume in
  these collisions that impacted objects ``stick together.''

  \item Imagine two identical objects moving at the same (but opposite)
  velocities impact. Because we're studying ineleastic collisions, we
  assume that the two will stick together. Will they be moving overall
  left, or overall right? By symmetry, neither. So the velocities ought
  to cancel out.

  \item He next asks what would happen if we impact an unmoving impacted
  object A with an identical moving object B. He says: let's ``run along
  side'' the experiment so it looks like the two objects both have equal
  (but opposite) velocities. Then we expect them to stand still, from
  our reference frame. But that means that they are moving in another
  reference frame.

  \item From this kind of argument, we can see that momentum is always
  conserved.

  \item He next asks what should happen if we impact two objects of
  differing mass. When we talked about equal masses, we used a principle
  of symmetry in our very first step (everything else came from Galilean
  invariance). But now with differing masses, the symmetry is broken.

  \item I bet there \emph{is} a good reason coming from symmetry for why
  momentum should still be conserved with differing masses. But his
  argument is tortured. He wants to establish a ``base case'' (A is
  unmoving, B is moving) for each pair of masses, and then use Galilean
  invariance to generalize to any velocities.

  \item He moves on to consider \emph{elastic} collisions where the two
  objects bounce off each other. He notes that only elastic collisions
  can preserve energy. First kinetic energy will be turned into
  potential energy (on compression), and then converted back out into
  kinetic energy again (on decompression).

  \item Because of conservation of energy, we must have that the speeds
  afterward are equal to the speeds in the beginning. But conservation
  of energy would not itself tell us that the velocities should be
  \emph{opposite} each other. That is described from conservation of
  \emph{momentum}.

  \item He explains that rockets eject mass backward at a great
  velocity, and thus by conservation of momentum must move forward. Of
  course, there is no need for ``atmosphere'' for the rocket engine to
  ``push against.'' This can all happen in the vacuum of space.

  \item Relativity: he mentions that mass changes as velocity changes.
  But still momentum gets conserved.

  \item He asks: can you ``hide'' momentum? I think he's asking: can you
  store momentum in a field, kind of like how you can store potential
  energy? He notes that you can't store momentum in random fluctuations,
  because if an object has a net momentum, it must be drifting.

  \item But he notes that because EM propagation is not instantaneous,
  momentum can be ``hidden'' in EM waves. Basically: if you wiggle an
  electric charge, the change in force does not arrive immediately to a
  second electrical charge far away. When the wave arrives, there will
  be a change in velocity of the other charge: a change in momentum.

  \item He says: in the interim the missing momentum is in the EM wave.
  This relates to light waves having mass, and thus they can be bent by
  gravity.

  \item Wikipedia note: there is a theorem (Noether's theorem) which
  says that certain invariance laws (like Galilean invariance) will
  always imply certain conservation laws (like conservation of
  momentum).

\end{enumerate}
