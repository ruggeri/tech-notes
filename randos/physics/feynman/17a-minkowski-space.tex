\subsection{Minkowski Space}

\begin{enumerate}

  \item We call a position in a \define{space-time} $(x, y, z, ct)$ an
  \define{event}. I also call it a \define{four-vector}. It turns out
  that $ct$ is a more convenient set of units to use.

  \item I shall tend to write $\ttilde = ct$ and $\vtilde = v/c$. Of
  course we always have $\vtilde < 1$.

  \item We may now more conveniently write:

  \begin{nedqn}
    x'
  \eqcol
    \frac{x - \vtilde \ttilde}{\sqrt{1 - \vtilde^2}}
  \\
    \ttilde'
  \eqcol
    \frac{
      \ttilde - \vtilde x
    }{
      \sqrt{1 - \vtilde^2}
    }
  \end{nedqn}

  This \emph{almost} looks like a rotation. However, it is not
  quite.

  \item Lorentzian transforms stretch space. Therefore, Lorentzian
  transforms are not \define{isometries} of three-dimensional space.
  Moreover, Lorentzian transformations are not isometries even if we
  consider space-time to be a four-dimensional Euclidean space.

  That is: Lorentz transformations do not preserve the Euclidean norm of
  points in four-dimensional space. They do not preserve the Euclidean
  inner product between points in four-dimensional space.

  \item Isometries of a Euclidean space are generated by flips,
  translations, and rotations. These basis elements generate the
  \define{Euclidean group} of isometries for a Euclidean space.

  \item Is there a way to look at space-time such that Lorentzian
  transformations don't ``warp'' or ``stretch'' our view? If there is,
  then this conception of space-time might be more ``real'' than any
  specific view of space-time from a specific vantage point.

  \item It turns out that there is an (almost) inner product with
  respect to which both Euclidean and Lorentzian transformations are
  isometries:

  \begin{nedqn}
    \innerprod{\vmu_1}{\vmu_2}
  \eqcol
    \parensq{\Delta \ttilde}
    - \parensq{\Delta x}
    - \parensq{\Delta y}
    - \parensq{\Delta z}
  \end{nedqn}

  This is called the \define{Minkowski inner product}. But that is
  technically an abuse: an inner product is supposed to be
  \define{positive definite}. That is, we ought to have
  $\innerprod{\mu}{\mu} > 0$. The Minkowski inner product is really an
  \define{indefinite bilinear form}.

  People sometimes define $\norm{\mu} = \sqrt{\innerprod{\mu}{\mu}}$ as
  the \define{Minkowski norm}, but this is obviously sometimes
  imaginary, and is thus not really a norm. Likewise, the Minkowski
  inner product is sometimes called the \define{Minkowski norm squared},
  though this is similarly an abuse.

  \item We may verify that Lorentzian transformations preserve distance
  between space-time four-vectors. Note:

  \begin{nedqn}
    \ttilde' - x'^2
  \eqcol
    \parensq{
      \frac{
        \ttilde - \vtilde x
      }{
        \sqrt{1 - \vtilde^2}
      }
    }
    -
    \parensq{
      \frac{
        x - \vtilde \ttilde
      }{
        \sqrt{1 - \vtilde^2}
      }
    }
  \\
  \eqcol
    \parensinv{
      1 - \vtilde^2
    }
    \parens{
      \parensq{\ttilde - x \vtilde}
      -
      \parensq{x - \vtilde \ttilde}
    }
  \\
  \eqcol
    \parensinv{
      1 - \vtilde^2
    }
    \parens{
      \ttilde^2 - 2x \ttilde \vtilde + x^2 \vtilde^2
      -
      x^2 + 2x \ttilde \vtilde - \ttilde^2 \vtilde^2
    }
  \\
  \eqcol
    \parensinv{
      1 - \vtilde^2
    }
    \parens{
      \parens{1 - \vtilde^2} \ttilde^2
      -
      \parens{1 - \vtilde^2} x^2
    }
  \\
  \eqcol
    \ttilde^2 - x^2
  \end{nedqn}

  This shows that there will be no change in Minkowski distance between
  two points in space-time by applying a Lorentzian transformation. Of
  course, a Galilean transformation would not respect this metric.

  \item The metric for space-time defines the \emph{geometry} of
  space-time. When we change our velocity, it \emph{looks} like we have
  warped the geometry of space. For instance, we know that spatial
  distances get stretched. But if we think of space-time \emph{as a
  whole}, and we think outside the box of our spatially-focused
  geometry, we might see that there is no \emph{overall warping} of
  space-time.

  A space with this kind of geometry is called a \define{Minkowski
  space}.

  \item We've seen that Minkowski norm can sometimes be imaginary.

  In general, the norm of a four-vector at time $t$ will be real if $ct
  > \norm{\parens{x, y, z}}$. That is: only if the four-vector lies
  within the \define{light-cone} centered at the origin.

  The \define{affective past} consists of four-vectors within the light
  cone where $t < 0$. The \define{affective future} consists of
  four-vectors within the light cone where $t > 0$.

  \item Points in space-time that are outside our present light-cone
  will \emph{eventually} come to be in our light-cone. The light cone
  always grows with time. But points outside our present light-cone are
  not part of our ``now.''

  \item BTW, the \define{Poincaré group} are the isometries of Minkowski
  space. I guess it has ten basis elements? I'm actually sure which ones
  are the minimum necessary.

  Certainly there are (a) three rotations ($xy, xz, yz$), (b) four
  coordinate shifts, ($x, y, z, t$), (c) four reflections ($x, y, z,
  t$), (d) three \define{boosts} ($xt, yt, zy$). That would make for 12
  basis vectors, but certainly some of those are not necessary...

\end{enumerate}
