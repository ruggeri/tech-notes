\subsection{Conservation of Angular Momentum}

\begin{enumerate}
  \item Consider again a system in which parts allowed to rotate
  freely. That is, there are no forces between parts. Then, as we know,
  $\vF_i = m \fpartial{t} \vv$.

  We want to know how the system will start rotating. Of course, fix any
  choice of axis. Throw away all the parts of $\vF_i$ that aren't acting
  tangentially; we don't care about radial movement.

  Then we can define $\tau_i = r_i F_i$. We know that $\tau_i =
  \fpartial{t} L_i$. And if we let $L = \sum_i L_i$, then we know that
  $\tau = \sum_i \tau_i = \fpartial{t} L$. These are all just definitions.

  \item One thing that is worth noting: the change in angular momentum
  depends on $\tau$, and this totally does depend on where forces are
  applied. Though we do see that it doesn't matter where a \emph{torque}
  is applied. That is, if we apply a force in a different place, but
  adjust its scale appropriately, the total external torque will remain
  the same. Thus the total instantaneous change in angular momentum will
  remain unchanged.

  \item We've assumed that $\fpartial{t} L_i = \tau_i$. But, as when we
  considered the center of mass, we must now contemplate the effect of
  \emph{internal forces}. As we know, internal forces always come in
  equal and opposite pairs. Assume a pair of equal and opposite forces
  of magnitude $F$ are each applied at the same location distance $r$
  from the axis. Then this translates to a pair of equal and opposite
  torques: $\tau_1 = Fr, \tau_2 = -Fr$.

  \item But we must consider ``non-contact'' force pairs, where the
  force may apply at different points. Consider two charged particles.
  Each exerts a force on the other, no matter their distance from each
  other. Bad news: two particles may be at different distances $r_1,
  r_2$ from the reference axis. Clearly $Fr_1 \ne Fr_2$.

  \item One thing will save us. Newton didn't precisely say, but he
  assumed that the equal/opposite force pair acts all along one line.
  For instance: an interaction force is always pulling two points
  directly together or pushing two points directly apart.

  \item So let us consider the \emph{direction} of the force, too. Let
  us say that $\vF$ is tangential at the point of application $\vx_1$.
  Thus indeed $\tau_1 = F r_1$. We know an opposite force $-\vF$ applies
  at some other point $\vx_2$. We know $\vx_2$ may be some distance from
  $\vx_1$. But we also now know that $\vx_2$ lies on the line defined by
  $\vx_1$ and $\vF$.

  In this case, because $\vF_1$ is already assumed tangential when
  applied to $\vx_1$, we know that $r_2 > r_1$. This is because $\vx_1$
  is the point on the line closest to the reference axis of rotation. So
  indeed $r_2 F > r_1 F$. But note that $-\vF$ will \emph{not} be
  applied tangentially at $\vx_2$. In fact, we will prove that in as
  much as $r_2$ diverges from $r_1$, so $-\vF$ will diverge from
  being tangential at $\vx_2$.

  \item I have included a diagram. It shows a force $\vF$ applied at
  distance $r$ from the axis of rotation. The force $\vF$ is not applied
  tangentially; it is applied at an angle of $\theta$ degrees off
  tangential. Thus the tangential force is $\cosf{\theta} \norm{\vF}$
  and the torque is $\tau = r \cosf{\theta} \norm{\vF}$.

  Feynman says that we can instead look at the ``lever arm.'' We get
  this from extending $\vF$ until it is in fact tangential. The diagram
  shows that the length of the lever arm is $r \cosf{\theta}$. So we can
  say that torque is:

  \begin{nedqn}
    \tau
  \eqcol
    r \cosf{\theta} F
  \\
  \eqcol
    \textrm{lever arm length} \cdot F
  \end{nedqn}

  \item This formulation is very helpful in proving conservation of
  angular momentum. Two equal/opposite paired forces, even if they
  aren't applied at the same location, will both share a lever arm
  (because the force pair acts along the same line). Thus we know that
  $\tau_1 = -\tau_2$.
\end{enumerate}
