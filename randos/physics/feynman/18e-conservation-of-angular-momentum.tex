\subsection{Conservation of Angular Momentum}

\begin{enumerate}
  \item So far

  \item This all seems like a lot of formalism. Let's see if it is
  useful. Consider again a system in which parts allowed to rotate
  freely. That is, there are no forces between parts. Then, as we know,
  $\vF_i = m \fpartial{t} \vv$.

  We want to know how the system will start rotating. Of course, fix any
  choice of axis. Throw away all the parts of $\vF_i$ that aren't acting
  tangentially; we don't care about radial movement.

  Then, as we know, $\tau_i = \fpartial{t} L_i$. And if we let $L =
  \sum_i L_i$, of course: $\tau = \fpartial{t} L$.

  \item One thing that is worth noting: the change in angular momentum
  depends on $\tau$, and this totally does depend on where forces are
  applied. Though we do see that it doesn't matter where a \emph{torque}
  is applied.

  \item Okay, what about systems in which there are also internal
  forces? As we showed with regard to translational momentum, we will
  show that angular momentum is conserved by a closed system.

  Now, we know that forces come in equal and opposite pairs. But we need
  to additionally reason about \emph{where} forces occur, since torque
  cares about that.

  We need an additional fact: the equal and opposite forces occur on the
  same line.
\end{enumerate}
