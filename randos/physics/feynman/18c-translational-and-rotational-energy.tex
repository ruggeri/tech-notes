\subsection{Translational and Rotational Energy}

\begin{enumerate}
  \item Let's assume that there is no net torque on the object, so that
  it only undergoes translation. Then $\vF \cdot \vx$ is the work done
  by the forces upon the object. Because the object is undergoing
  translation, note that for every part $i$, $\vv_i$ is always directed
  exactly in the direction of the net force $\vF$. Also, we know that
  $\vv_i$ is equal for all subparts of the object.

  Since $\vF, \vx, \vv$ are all directed in the same direction, we can
  treat them as scalar. If we assume that all work done must be
  converted to kinetic energy, then:

  \begin{nedqn}
    Fx
  \eqcol
    \sum_i \half m_i v^2
  \\
  \eqcol
    \half M v^2
  \end{nedqn}

  \item Likewise, we can assume that there is no net force acting on the
  object. In that case, it may still rotate about its CoM. Consider a
  rotation by $\theta$ degrees. As we discussed, the work done is $\tau
  \theta$. We will once again assume that all energy is converted to
  kinetic energy.

  We will use one more fact. The $v_i$ will not all be equal. However,
  we know that the object is undergoing rotation, and thus $v_i = r_i
  \omega$, where $\omega$ is the angular velocity. Using this, we can
  determine:

  \begin{nedqn}
    \tau \theta
  \eqcol
    \sum_i \half m_i v_i^2
  \\
  \eqcol
    \half \sum_i m_i \parensq{r_i \omega}
  \\
  \eqcol
    \half \parens{\sum_i m_i r_i^2} \omega^2
  \end{nedqn}

  \item We thus choose to introduce $I = \sum_i m_i r_i^2$. This is
  called the \define{rotational inertia}, or, commonly, \define{moment
  of inertia}. When something is called a ``moment,'' it means
  ``weighted by distance from a point or axis.''
\end{enumerate}
