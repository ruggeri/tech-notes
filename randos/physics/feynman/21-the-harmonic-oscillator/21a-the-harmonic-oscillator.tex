\subsection{Mass On a Spring}

\begin{enumerate}
  \item Let's consider a mass suspended by a spring.

  \item Let's assume it's a \define{linear} spring. The force exerted by
  the spring is linear with respect to its extension $x$. This implies
  that there is a point where no force is exerted by the spring.

  \item Gravity exerts a force downward on our mass. We know that there
  is a corresponding extension of the spring that exerts an equal but
  opposite force upward. We will call this position of $x$ zero. This is
  an equilibrium position: if the mass had no velocity, the mass gravity
  and spring extension forces would balance and the mass would not move.

  \item We'll say that an upward displacement of the mass is denoted by
  a positive $x$, while a downward displacement of the mass is denoted
  by a negative $-x$. The total force on the mass is $-kx$, where $k$ is
  the \define{spring constant}. I suppose its units are newtons per
  meter.

  \item Note: we aren't trying to describe right now why the
  relationship between spring extension and spring force is linear. But
  it seems like a natural assumption.

  \item He notes that the acceleration $\fpartialsq{t} x$ is
  $\frac{-kx}{M}$, where $M$ is the mass of the object. For simplicity,
  he will assume that $k/M=1$. So now we want to find those trajectory
  functions $\xf{t}$ that satisfy the equation:

  \begin{nedqn}
    \xf{t} \eqcol -\fpartialsq{t} \xf{t}
  \end{nedqn}

  \item What function $\xf{t}$ would satisfy this? Certainly $\xf{t} = 0$
  would do.

  \item Is $\xf{t} = \sinf{t}$ also a solution? In this case, we have
  $\fpartialsq{t} \xf{t} = -\sinf{t}$, which satisfies the differential
  equation, as desired.

  \item When $t = 0$, $\sinf{t} = 0$, so there is no initial force to
  start the mass moving. Why is it moving? Because it must have an
  initial velocity! This is why $\fpartial{t} \sinf{t} = \cosf{t} = 1$
  when $t = 0$.

  \item Likewise, consider $\xf{t} = \cosf{t}$. This satisfies the
  equation above. This corresponds to no initial velocity, but an
  initial \emph{extension} of the spring. Thus there is initial force
  that starts it moving.

  \item In fact, we can see that any superposition $a_0 \sinf{t} + b_0
  \cosf{t}$ is a solution. Of course, this space consists of exactly
  phase and scale shifts of a sinusoidal with fixed period. These other
  solutions correspond to initially extending the spring a certain
  amount but also giving it some initial velocity.

  \item This maybe makes sense. All phase-shifts should be solutions,
  because a change in arbitrary starting point $t = 0$ corresponds to a
  phase-shift. And all scaling by a constant should be a solution, since
  that is just defined by your original definition of the unit of
  distance.

  \item Still, it is interesting to note that the \define{period} of the
  oscillation is entirely determined by $k/M = 1$. The initial velocity
  and initial extension of the spring are irrelevant.

  \item What if we doubled the scale of time? Then we should have the
  same solutions, but with half the period. \TODO{Can we describe why,
  mathematically, we would expect to get different solutions.}
\end{enumerate}

\subsection{Differential Equations}

\begin{enumerate}
  \item Feynman says many systems will follow similar rules as our mass
  and spring.

  \item We are really studying \define{linear differential equations}.
  These are equations of the form:

  \begin{nedqn}
    b(t)
  \eqcol
    a_0 \xf{t}
    + a_1 \fpartial{t} \xf{t}
    + \ldots + a_n \frac{\partial^n}{\partial t^n} \xf{t}
  \end{nedqn}

  \item We are typically given the $a_i$ and $b$ and asked to find $x$.

  \item The $a_i$ are typically constant. In more complex problems
  (which I do not consider here), the $a_i$ could be themselves
  functions of $t$.

  \item We will begin by studying \define{homogeneous} linear
  differential equations. That means $b(t) = 0$. The key property of
  homogeneous differential equations is that linear combinations of
  solutions are also solutions. Our spring example is a homogeneous
  linear differential equation (with constant coefficients):

  \begin{nedqn}
    0 = \xf{t} + \fpartialsq{t} \xf{t}
  \end{nedqn}
\end{enumerate}

\subsection{Complex Version}

\begin{enumerate}
  \item We said that there are two independent solutions to the (real)
  differential equation problem. All others appear to be linear
  combinations of the two. \TODO{We should prove that a degree two
  differential equation has at most two independent solutions!}

  \item Let's consider the complex valued version of the problem. Then
  we can consider $\xf{t} = e^{it} = \cosf{t} + i \sinf{t}$. This is a
  solution to the equation.

  \item Equally well would $\xf{t} = e^{-it} = \cosf{-t} + i \sinf{-t} =
  \cosf{t} - i \sinf{t}$ be a solution. This corresponds to rotation in
  the opposite direction. We didn't worry about direction of rotation in
  the real-valued version, because a change in direction could look the
  same as a phase shift.

  \item On the other hand, because we have complex coefficients, we can
  explain a change in phase simply as a scaling by a complex value.

  \item We can recover a real-valued solution like $\xf{t} = \cosf{t}$ by
  summing two exponentials:

  \begin{nedqn}
    \half e^{it} + \half e^{-it}
  \eqcol
    \half \parens{
      \cosf{t} + i \sinf{t}
    }
    +
    \half \parens{
      \cosf{t} - i \sinf{t}
    }
  =
    \cosf{t}
  \end{nedqn}

  \item This is as usual. By changing from a real-valued problem to a
  complex-valued problem we (1) make phase shifts just multiples, (2)
  allow directionality to make a difference.

  \item \TODO{JRN and I discussed a proof, based on the Fourier
  transform and the Fundamental Theorem of Algebra, that a degree $n$
  homogenous linear differential equation always has a solution vector
  space of dimension $n$}.
\end{enumerate}

\subsection{Mass On a Spring II}

\begin{enumerate}
  \item Now, let us consider when $k/M \ne 1$. In this case, we have

  \begin{nedqn}
    \fpartialsq{t} \xf{t}
  \eqcol
    \frac{k \xf{t}}{M}
  \\
    \xf{t}
  \eqcol
    -\frac{M}{k} \fpartialsq{t} \xf{t}
  \end{nedqn}

  \noindent
  Note that the factor of $\frac{M}{k}$ cancels out the factor of
  $\frac{k}{M}$ in the acceleration formula.

  \item In this case, we want $\cosf{\sqrt{\frac{k}{M}} t}$. Then we
  will have

  \begin{nedqn}
    \fpartialsq{t} \cosf{\sqrt{\frac{k}{M}} t}
  \eqcol
    - \frac{M}{k} \cosf{\sqrt{\frac{k}{M}} t}
  \end{nedqn}

  \noindent
  which satisfies the differential equation.

  \item This makes sense. If the mass is increased, it will decrease the
  acceleration of the mass. But doubling the mass will not quite double
  the time period. Likewise, halving the mass will not quite double the
  period either.

  \item Interestingly, changing $k, M$ do not change the amplitude of
  the oscillation. This makes the most sense when we consider a mass on
  a spring that has been stretched out and is at rest, waiting to be
  released. That is, when $\xf{t} = A \cosf{\omega t}$. Changing the mass
  or spring constant won't change the fact that the system reaches the
  opposite extreme when $x(t_0) = -A$. It just changes how long it takes
  the system to reach the other extreme.

  \item To summarize. Solutions are of the form:

  \begin{nedqn}
    \xf{t}
  \eqcol
    a \sinf{\omega \parens{t - t_0}}
  \end{nedqn}

  \item You can call $\omega = \sqrt{\frac{k}{M}}$ an \define{angular
  velocity} if you like. It's how fast the system cycles from one
  extreme and back again. We call $a$ the \define{amplitude}. $t_0$ is
  the \define{phase shift}.

\end{enumerate}

\subsection{Harmonic Motion and Circular Motion}

\begin{enumerate}
  \item Because we know about complex numbers, we already found that a
  circular motion is a more general solution to the differential
  equation. But Feynman is going to show this through some non-algebraic
  means.

  \item He recounts some equations of circular motion:

  \begin{nedqn}
    a \eqcol \omega^2 R
    \nedcomment{See prior work about centripetal force}
  \\
    x \eqcol R \cosf{\theta}
  \\
    y \eqcol R \sinf{\theta}
  \end{nedqn}

  \item He then consider the acceleration in the $x$ dimension. This
  involves a projection of $a$ into the $x$-dimension, which depends on
  the angle between $a$ and $e_x$ $\theta$. And because acceleration is
  oriented inward, we need a negative sign:

  \begin{nedqn}
    a_x
  \eqcol
    -\omega^2 R \cdot \cosf{\theta}
  =
    -\omega^2 x
  \end{nedqn}

  \item Now, he notes that this shows that the acceleration in the $x$
  dimension is exactly proportional to the extension in the $x$
  dimension. That means we have set up the same differential equation as
  with the spring scenario. Thus, circular motion, when projected into
  one dimension, looks like oscillation of a linear spring.

  \item He proposes an experiment that should demonstrate the
  equivalence. I don't entirely understand the experiment, but it is not
  particularly relevant. It is something like: suspend a bicycle wheel
  from a spring. Attach a crank pin to the bicycle wheel. Extend the
  wheel/spring and release. Project a light from behind. The shadow of
  the pin projected on the screen should oscillate.

  \item Next, simultaneous spin the wheel. If the displacement of the
  spring is equal to the diameter of the wheel, then there is an angular
  velocity at which the crank pin's shadow should look stationary.

  \item Look, this isn't too important to us. We know algebraically that
  spring motion and circular motion are related. He just hasn't
  introduced the math for it yet.

  \item He notes that he will soon be introducing the needed math. He
  also notes that sometimes it will be convenient to treat spring motion
  as if it were circular motion. We imagine a second dimension of the
  motion, perform calculations and analysis. At the end, we'll throw
  away the imaginary component by simply projecting into the one
  dimension of linear travel we care about. It's just a matter of
  mathematical convenience.
\end{enumerate}

\subsection{Initial Conditions}

\begin{enumerate}
  \item We know that the initial position and velocity of the mass
  determines the solution to the differential equation.

  \item The solution can be found relatively easily. First we find $A$
  from the position equation:

  \begin{nedqn}
    \xf{t}
  \eqcol
    A \cosf{\omega_0 t} + B \sinf{\omega_0 t}
  \\
    \xf{0}
  \eqcol
    A \cosf{0} + B \sinf{0}
  \\
    A
  \eqcol
    \xf{0}
  \end{nedqn}

  \item Next, we find $B$ from the velocity equation:

  \begin{nedqn}
    \vf{t}
  \eqcol
    -A \omega_0 \sinf{t} + B \omega_0 \cosf{\omega_0 t}
  \\
    v(0)
  \eqcol
    -A \omega_0 \sinf{0} + B \omega_0 \cosf{0}
  \\
    B
  \eqcol
    \frac{v(0)}{\omega_0}
  \end{nedqn}

  \item Thus we simply see that:

  \begin{nedqn}
    \xf{t}
  \eqcol
    \xf{0} \cosf{\omega_0 t} + \frac{\vf{0}}{\omega_0} \sinf{\omega_0 t}
  \end{nedqn}

  \item Could we have guessed this answer? Yes, certainly. If the
  initial configuration $\xf{0}, \vf{0}$ is given, then we need to
  decompose this into a superposition of $\cosf{\omega_0 t},
  \sinf{\omega_0 t}$. But only $\cosf{\omega_0 t}$ can explain the
  initial extension $\xf{0}$. Likewise, only $\sinf{\omega_0 t}$ can
  explain the initial velocity. But, if $B=1$, then the initial velocity
  $\vf{0}$ would be $\omega_0$. So we need to set $B =
  \frac{\vf{0}}{\omega_0}$.

  \item This superposition form is helpful, but it doesn't directly tell
  me the amplitude nor the phase shift of the solution. That is, I want
  the form $\xf{t} = a \cosf{\omega_0 (t - t_0)}$. I can of course
  recover this in the usual way. First, we need to find the $\theta_0$
  such that:

  \begin{nedqn}
    \parens{A, B}
  \eqcol
    \norm{\parens{A, B}}
    \parens{\cosf{\theta_0}, \sinf{\theta_0}}
  \end{nedqn}

  \noindent
  This (inevitably) involves an application of the inverse tangent
  function. But now we can say:

  \begin{nedqn}
    \xf{t}
  \eqcol
    \norm{\parens{A, B}}
    \parens{
      \cosf{\theta_0} \cosf{\omega_0 t}
      + \sinf{\theta_0} \sinf{\omega_0 t}
    }
  \\
  \eqcol
    \norm{\parens{A, B}}
    \cosf{\omega_0 t - \theta_0}
  \intertext{now we set  $t_0 = \frac{\theta_0}{\omega_0}$}
  \\
  \eqcol
    \norm{\parens{A, B}}
    \cosf{\omega_0 (t - t_0)}
  \end{nedqn}

  \noindent
  This shows us that we want:

  \begin{nedqn}
    a
  \eqcol
    \sqrt{A^2 + B^2}
  \\
    t_0
  \eqcol
    \atantwof{A, B} / \omega_0
  \end{nedqn}

  \noindent
  Notice my use of the two argument $\atantwo$ function. This is
  basically just $\atanf{\frac{B}{A}}$. However, $\frac{B}{A}$ doesn't
  tell us the quadrant that $\parens{A, B}$ lies in. Also, it is
  undefined if $A = 0$. Thus the two argument arctan function is quite
  useful. You can see its definition on Wikipedia, if you like.

  \item Next, let's find $a, t_0$ directly from $\parens{\xf{0},
  \vf{0}}$. First, we know that we can put the initial conditions in
  polar form for some $\alpha, \beta$:

  \begin{nedqn}
    \parens{\xf{0}, \vf{0}}
  \eqcol
    \alpha
    \parens{
      \cosf{\beta}, \sinf{\beta}
    }
  \end{nedqn}

  \noindent
  Next, let us rewrite $\xf{t}, \vf{t}$ in terms not of time $t$, but of
  angular displacement $\theta$:

  \begin{nedqn}
    \xf{\theta}
  \eqcol
    a \cosf{\theta - \theta_0}
  \\
    \vf{\theta}
  \eqcol
    -a \sinf{\theta - \theta_0}
  \intertext{This implies, for $\theta = 0$}
    \xf{\theta = 0}
  \eqcol
    a \cosf{-\theta_0}
  =
    a \cosf{\theta_0}
  =
    \xf{t = 0}
  \\
    \vf{\theta = 0}
  \eqcol
    -a \sinf{-\theta_0}
  =
    a \sinf{\theta_0}
  =
    \vf{t = 0}
  \end{nedqn}

  \noindent
  What does this imply? By uniqueness of polar decomposition, it implies
  that the polar decomposition $\parens{\alpha, \beta}$ must be equal to
  the polar decomposition $\parens{a, \theta_0}$.

  Thus we must have

  \begin{nedqn}
    a
  \eqcol
    \norm{\parens{\xf{0}, \vf{0}}}
  \\
    \theta_0
  \eqcol
    \atantwof{\xf{0}, \vf{0}}
  \\
  \intertext{converting to time from angular displacement}
    t_0
  \eqcol
    \atantwof{\xf{0}, \vf{0}} / \omega_0
  \end{nedqn}

  \item Have we learned anything? Maybe that as we scale $\cosf{t}$, so
  we scale $\xf{0}, \vf{0}$. And that as we rotate $\cosf{t - t_0}$ by
  changing $t_0$, so we rotate from $\parens{\xf{0}, \vf{0}} =
  \parens{1, 0}$ toward $\parens{\xf{0}, \vf{0}} = \parens{0, 1}$.

\end{enumerate}

\subsection{Kinetic and Potential Energy}

\begin{enumerate}
  \item Let's calculate the kinetic and potential energy in a spring.

  \item The potential energy in a spring with extension $x_0$ is:

  \begin{nedqn}
    \text{PE}
  \eqcol
    \int_{x_0}^0 -F(x) \dx
  =
    \int_{x_0}^0 -kx \dx
  =
    \frac{k x_0^2}{2}
  \end{nedqn}

  \item When $x = a$ the spring is at maximum extension and $v = 0$.
  Thus, this shows that the total energy in the system is always
  $\frac{ka^2}{2}$.

  \item Moreover, we can say that the potential energy, at any given
  time, is

  \begin{nedqn}
    \text{PE} \parens{t}
  \eqcol
    \half k x(t)^2
  \\
  \eqcol
    \half k a^2 \cos^2 \parens{\omega \parens{t - t_0}}
  \end{nedqn}

  \item Next, we can find the kinetic energy just as easily:

  \begin{nedqn}
    \text{KE} \parens{t}
  \eqcol
    \half M \vf{t}^2
  \\
  \eqcol
    \half
    M
    \parensq{
      -a \omega_0 \sinf{\omega_0 \parens{t - t_0}}
    }
  \\
  \eqcol
    \half
    M
    \parens{
      a^2 \frac{k}{M} \sinf{\omega_0 \parens{t - t_0}}
    }
  \\
  \eqcol
    \half
    k a^2 \sin^2 \parens{\omega_0 \parens{t - t_0}}
  \end{nedqn}

  \noindent
  As expected, we see that $\text{PE}(t) + \text{KE}(t) = \half k a^2$.

  \item Here we see that if you double the spring constant, you double
  the energy of the system. We also see that the energy is proportional
  to the square of the amplitude of the oscillation. We also see the
  energy in the system is independent of the mass.

  \item Last, he notes that the average potential energy is half the
  maximum potential energy. How does he know this? Well:

  \begin{nedqn}
    \text{PE}_\text{AVG}
  \eqcol
    \int_0^{2\pi} \half k a^2 \cos^2 \parens{\theta} \dtheta
  \intertext{here I paramaterize by $\theta$ rather than $t$}
  \\
  \eqcol
    \half k a^2 \int_0^{2\pi} \cos^2 \parens{\theta} \dtheta
  \end{nedqn}

  \noindent
  It would be sufficient to know: $\int_0^{2\pi} \cos^2 \parens{\theta}
  \dtheta = \half$. But I have proven this previously by a symmetry
  argument about the norm of all unit vectors.

  \item Thus, we know that

  \begin{nedqn}
    \text{PE}_\text{AVG}
  \eqcol
    \half \text{PE}_\text{MAX}
  \\
    \text{KE}_\text{AVG}
  \eqcol
    \half \text{KE}_\text{MAX}
  \end{nedqn}

\end{enumerate}
