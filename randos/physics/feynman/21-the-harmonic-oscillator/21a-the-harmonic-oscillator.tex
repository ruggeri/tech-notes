\begin{enumerate}
  \item Feynman says that we will study the harmonic oscillation. You
  can create a mechanical version with a mass attached to a spring. But
  Feynman says there are many analogues. We are really studying
  \define{linear differential equations}. These are equations of the
  form:

  \begin{nedqn}
    b(t)
  \eqcol
    a_0 x + a_1 \frac{\partial x}{\partial t} + \ldots + a_n \frac{\partial^n x}{\partial t^n}
  \end{nedqn}

  \item In this example, the $a_i$ are constant in $t$. A more general
  linear differential equation would allow $a_i$ to be an arbitrary
  function of $t$. We won't study that right now though.

  \item In a linear differential equation problem, we are given the
  function $b$ and the coefficients $a_i$, and asked to find $x(t)$.

  \item Let's think of some of the simplest examples we can. First,
  let's say that $b(t) = 0$ for all $t$. This is called a
  \define{homogeneous} linear differential equation. Next, let's assume
  that $a_0 = 1$, while $a_i = 0$ for all $i > 0$. Then certainly $x(t)
  = 0$.

  \item But before we try to solve more kinds of equations like this, we
  should follow Feynman, who is now going to talk about a specific
  mechanical case (a spring with a mass attached).

  \item He's going to consider a linear spring. The force exerted by the
  spring is linear. Which also implies that there is a point where no
  force is exerted by the spring.

  Of course, we have attached a mass. Gravity exerts a force downward.
  But we know that there is a corresponding extension of the spring that
  exerts an equal but opposite force upward. We will call this position
  zero. This is an equilibrium position: if the mass had no velocity,
  the mass gravity and spring extension forces would balance and the
  mass would not move.

  \item We'll say that an upward displacement of the mass is denoted by
  a positive $x$, while a downward displacement of the mass is denoted
  by a negative $-x$. The total force on the mass is $-kx$, where $k$ is
  the \define{spring constant}.

  \item He notes that the acceleration $\fpartialsq{t} x$ is
  $\frac{-kx}{M}$, where $M$ is the mass of the object. For simplicity,
  he will assume that $k/M=1$. So now we have a differential equation to
  solve:

  \begin{nedqn}
    x(t) - \fpartialsq{t} x(t) = 0
  \end{nedqn}

  \noindent
  What function $x(t)$ would satisfy this? Certainly $x(t) = 0$ would
  do. Next, is $x(t) = \sinf{t}$ is a solution? In this case, we have
  $x(t) = \fpartialsq{t} x(t) = 0$, as desired. But, there could be some
  initial velocity $\fpartialsq{t}$. This would suggest that the spring
  would stretch (or compress) until a minimum (or maximum) $x$ value is
  obtained, at which point the process reverses. The mass is restored to
  its initial position, and the cycle continues.

  \item Of course, any $x(t) = A \sinf{t}$ would also be a solution to
  the equation. This would just correspond to more or less velocity
  imparted to the object. In general, a solution $f(t)$ to a
  (homogenous) differential equation always implies that $Af(t)$ is also
  a solution.

  \item Could not the object be stretched out to start from a
  displacement $x$, imparted no initial velocity, but then allowed to
  oscillate? Yes, naturally. We know that it will suffice to consider
  $x(0) = 1$. In that case, $x(t) = \cosf{t}$ is the most natural
  choice. And of course $x(t) = A \cosf{t}$ is a solution that allows
  for any $x(0) = A$.

  \item And this is an interesting point: any \emph{superposition} of
  solutions is also a solution. Thus, we know that any $A\cosf{t} +
  B\cosf{t}$ must be a solution to the differential equation.

  \item Equivalently, we can say that the solutions are defined by $x(t)
  = \text{Re}\parens{c_0 e^{it}}$. When $c_0 = 1$, this corresponds to
  x(t) = $\cosf{t}$. When $c_0 = i$, this corresponds to $-\sinf{t}$.

  \item You might ask: shouldn't $x(t) = \text{Re}\parens{c_0 e^{-it}}$
  also be a solution? In this case, we do not need it. That's because
  we're restricting $x(t)$ to be real. And $\text{Re}\parens{c_0
  e^{-it}}$ is equal to $\text{Re}\parens{ic_0 e^{it}}$. That is: the
  direction of rotation doesn't matter if we only care about the real
  component.
\end{enumerate}
