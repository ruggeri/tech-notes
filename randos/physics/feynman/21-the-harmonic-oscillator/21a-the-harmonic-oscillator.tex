\begin{enumerate}
  \item Feynman says that we will study the harmonic oscillation. You
  can create a mechanical version with a mass attached to a spring. But
  Feynman says there are many analogues. We are really studying
  \define{linear differential equations}. These are equations of the
  form:

  \begin{nedqn}
    b(t)
  \eqcol
    a_0 x(t)
    + a_1 \fpartial{t} x(t)
    + \ldots + a_n \frac{\partial^n}{\partial t^n} x(t)
  \end{nedqn}

  \item In this example, the $a_i$ are constant in $t$. A more general
  linear differential equation would allow $a_i$ to be an arbitrary
  function of $t$. We won't study that right now though. We have
  \emph{constant coefficients}.

  \item In a linear differential equation problem, we are given the
  function $b$ and the coefficients $a_i$, and asked to find $x(t)$.

  \item Let's think of some of the simplest examples we can. First,
  let's say that $b(t) = 0$ for all $t$. This is called a
  \define{homogeneous} linear differential equation. Next, let's assume
  that $a_0 = 1$, while $a_i = 0$ for all $i > 0$. Then certainly $x(t)
  = 0$. In fact, for any choice of $a_i$, we always have $x(t) = 0$ is a
  solution of a homogenous linear differential equation.

  \item But before we try to solve more kinds of equations like this, we
  should follow Feynman, who is now going to talk about a specific
  mechanical case (a spring with a mass attached).

  \item He's going to consider a linear spring. The force exerted by the
  spring is linear with respect to its extension $x$. Which also implies
  that there is a point where no force is exerted by the spring.

  Of course, we have attached a mass. Gravity exerts a force downward.
  But we know that there is a corresponding extension of the spring that
  exerts an equal but opposite force upward. We will call this position
  zero. This is an equilibrium position: if the mass had no velocity,
  the mass gravity and spring extension forces would balance and the
  mass would not move.

  \item We'll say that an upward displacement of the mass is denoted by
  a positive $x$, while a downward displacement of the mass is denoted
  by a negative $-x$. The total force on the mass is $-kx$, where $k$ is
  the \define{spring constant}.

  \item He notes that the acceleration $\fpartialsq{t} x$ is
  $\frac{-kx}{M}$, where $M$ is the mass of the object. For simplicity,
  he will assume that $k/M=1$. So now we have a differential equation to
  solve:

  \begin{nedqn}
    x(t) + \fpartialsq{t} x(t) = 0
  \end{nedqn}

  \noindent
  What function $x(t)$ would satisfy this? Certainly $x(t) = 0$ would
  do. Next, is $x(t) = \sinf{t}$ is a solution? In this case, we have
  $\fpartialsq{t} x(t) = -\sinf{t}$, which satisfies the differential
  equation, fas desired.

  \item But since $\sinf{0} = 0$, which is the equilibrium position, why
  would the mass move at all? Well, there could be some initial velocity
  $\fpartial{t} x(0) \ne 0$! This would suggest that the spring would
  stretch (or compress) until a minimum (or maximum) $x$ value is
  obtained, at which point the process reverses. The mass is restored to
  its initial position, and the cycle continues.

  \item Of course, any $x(t) = A \sinf{t}$ would also be a solution to
  the equation. This would just correspond to more or less velocity
  imparted to the object. In general, a solution $f(t)$ to a
  (homogenous) differential equation always implies that $Af(t)$ is also
  a solution.

  \item Could not the object be stretched out to start from a
  displacement $x$, imparted no initial velocity, but then allowed to
  oscillate? Yes, naturally. We know that it will suffice to consider
  $x(0) = 1$. In that case, $x(t) = \cosf{t}$ is the most natural
  choice. And of course $x(t) = A \cosf{t}$ is a solution that allows
  for any $x(0) = A$.

  \item And this is an interesting point: any \emph{superposition} of
  solutions is also a solution. Thus, we know that any $A\cosf{t} +
  B\cosf{t}$ must be a solution to the differential equation.

  \item It is interesting to note that the \define{period} of the
  oscillation in no way depends on (1) the initial velocity or (2) the
  initial position. It has been fixed because $k/M = 1$.

  \item We've been talking in terms of $\sin, \cos$. These are
  real-valued functions which are the solutions to the differential
  equation. What if we consider complex-valued functions? Then we can
  consider the analogue of $\cosf{t}$, which $x(t) = e^{it} = \cosf{t} +
  i \sinf{t}$. A complex analogue of $\sinf{t}$ is $-i e^{it} = \sinf{t}
  - i\cosf{t}$. This is a (complex) scalar multiple of our first
  solution, $e^{it}$.

  \item We had a two-dimensional real vector space as the solution space
  for the real-valued problem. Is there a second dimension to the
  complex-valued problem? Yes. It is: $e^{-it}$. This has the
  ``opposite'' frequency, 180deg out of phase. This rotates in the
  opposite direction.

  \item Are not $\cosf{t}, \sinf{t}$ still complex-valued solutions?
  Yes. These real-valued solutions are simply the sum of $\cosf{t} =
  \half e^{it} + \half e^{-it}$. Note that $\cosf{t}, \sinf{t}$ span the
  complex-valued solution space, but you have to use complex
  coefficients.

  \item We see that, per usual, the real-valued problem has a
  two-dimensional solution space. And we see that the complex-valued
  problem extends the real-valued solution space, but (1) the
  real-valued solution space is not a closed subspace over the complex
  numbers, and (2) the complex-valued solution space remains
  two-dimensional.

  \item We want to ask: are there any other solutions to this
  differential equation? How do we know that a solution must be a linear
  combination of two sinusoidals of frequency $\frac{1}{2\pi},
  -\frac{1}{2\pi}$?

  \item \TODO{JRN and I discussed a proof, based on the Fourier
  transform and the Fundamental Theorem of Algebra, that a degree $n$
  homogenous linear differential equation always has a solution vector
  space of dimension $n$}.

  \item Now, let us consider when $k/M \ne 1$.
\end{enumerate}
