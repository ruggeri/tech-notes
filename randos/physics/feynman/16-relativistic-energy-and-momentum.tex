\section{Relativistic Energy and Momentum}

\begin{enumerate}

  \item He notes a potential paradox. Imagine Alice and Bob. Bob flies
  away, and returns. He is younger than Alice now.

  But couldn't Bob imagine that he is still, and it was \emph{Alice} who
  flew away? Why did she come back older? Doesn't this violate symmetry?

  The reason: we need to do accelerations to make this experiment work.
  Bob experiences accelerations, but Alice does not. That breaks the
  symmetry and explains the difference.

  \item We next explore transformations of velocities. Let's first
  recall the equations to translate deltas in $x'$ and $t'$:

  \begin{nedqn}
    \Delta x'
  \eqcol
    \frac{
      \parens{x_1 - ut_1} - \parens{x_0 - ut_0}
    }{
      \sqrt{1 - u^2/c^2}
    }
  \\
  \eqcol
    \frac{\Delta x - u \Delta t}{
      \sqrt{1 - u^2/c^2}
    }
  \\
    \Delta t'
  \eqcol
    \frac{
      \parens{t_1 - ux/c^2} - \parens{t_0 - ux/c^2}
    }{
      \sqrt{1 - u^2/c^2}
    }
  \\
  \eqcol
    \frac{
      \Delta t - u\Delta x/c^2
    }{
      \sqrt{1 - u^2/c^2}
    }
  \end{nedqn}

  \item Let's consider an object A traveling at $v_y$ with no $x$
  velocity. Let's now consider observing it from a spaceship travelling
  with a velocity $u$ along the $x$ axis:

  \begin{nedqn}
    v_y'
  \eqcol
    \frac{\Delta y'}{\Delta t'}
  \\
  \eqcol
    \Delta y
    \frac{
      \sqrt{1 - u^2/c^2}
    }{
      \Delta t - u\Delta x/c^2
    }
  \\
  \eqcol
    v_y \sqrt{1 - u^2/c^2}
  \end{nedqn}

  This scenario is somewhat somewhat simple. Dilation of $x$ space is
  irrelevant, so the only factor involved is the time dilation effect.

  The new velocity $v_x'$ is also easy to analyze. It's easy beacuse
  $\Delta x = 0$.

  \begin{nedqn}
    v_x'
  \eqcol
    \frac{\Delta x'}{\Delta t'}
  \\
  \eqcol
    \frac{
      \Delta x - u \Delta t
    }{
      \Delta t - u \Delta x /c^2
    }
  \\
  \eqcol
    \frac{-u \Delta t}{\Delta t}
  \\
  \eqcol
    -u
  \end{nedqn}

  \item We can consider the change in velocity if the object A was
  travelling parallel to the $x$ axis:

  \begin{nedqn}
    v_x'
  \eqcol
    \frac{\Delta x'}{\Delta t'}
  \\
  \eqcol
    \frac{
      \Delta x - u \Delta t
    }{
      \Delta t - u \Delta x / c^2
    }
  \\
  \eqcol
    \frac{
      v_x - u
    }{
      1 - u v_x / c^2
    }
  \end{nedqn}

  Again that seems reasonable. If we match $u$ to equalize $v_x$ we find
  no more velocity $v_x'$. Likewise, if $v_x$ started out as zero, then
  as expected we find $v_x' = -u$. Notice that $v_x'$ can never be
  greater than $c$ (provided that $u, v_x$ are both less than $c$).

  Our finding for $v_y'$ is also reasonable:

  \begin{nedqn}
    v_y'
  \eqcol
    \frac{
      \Delta y'
    }{
      \Delta t'
    }
  \\
  \eqcol
    \frac{0}{\Delta t'}
  \\
  \eqcol
    0
  \end{nedqn}

  Which is as expected.

  \item Let's next examine what happens when neither $v_x, v_y$ are
  zero. Can we use our equations as before to treat these separately? It
  turns out not exactly. The math for $v_x'$ will work out as before.
  But let's next consider $v_y'$:

  \begin{nedqn}
    v_y'
  \eqcol
    \frac{
      \Delta y'
    }{
      \Delta t'
    }
  \\
  \eqcol
    \Delta y
    \frac{
      \sqrt{1 - u^2/c^2}
    }{
      \Delta t - u\Delta x/c^2
    }
  \\
  \eqcol
    v_y
    \frac{
      \sqrt{1 - u^2/c^2}
    }{
      1 - uv_x/c^2
    }
  \end{nedqn}

  We can see that when $v_x = 0$, this is just the same equation we had
  found before.

  \item Next Feynman wants to derive the formula for relativistic mass.
  He considers two identical particles that approach each other and then
  deflect at an angle. We may orrient axes such that: (a) particle A
  approaches with velocity $\vv$, (b) after approach, particle A
  maintains horizontal velocity $\vv_x$, (c) after approach,
  particle A reverses vertical velocity $\vv_y$ to $-\vv_y$.


  Particle B does just like particle A, except its velocity is always
  exactly opposite to A's.

  The masses are always the same, and the velocities are always
  opposite, so net momentum is always zero.

  \item He now imagines changing the frame by moving at $\vv_x$
  horizontally. We know from the above work that the velocity of
  particle A transforms to:

  \begin{nedqn}
    v'_{y, A}
  \eqcol
    v_y \frac{
      \sqrt{1 - v_x^2/c^2}
    }{
      1 - v_x^2/c^2
    }
  \\
    v'_{x, A}
  \eqcol
    \frac{
      v_x - v_x
    }{
      1 - v_x^2/c^2
    }
  \\
  \eqcol
    0
  \end{nedqn}

  With regard to particle B, we have:

  \begin{nedqn}
    v'_{y, B}
  \eqcol
    v_y \frac{
      \sqrt{1 - v_x^2/c^2}
    }{
      1 + v_x^2/c^2
    }
  \\
    v'_{x, B}
  \eqcol
    \frac{
      -v_x - v_x
    }{
      1 - (v_x)(-v_x)/c^2
    }
  \\
  \eqcol
    \frac{-2v_x}{1 + v_x^2/c^2}
  \end{nedqn}

  Therefore:

  \begin{nedqn}
    \frac{v'_{y, B}}{v'_{y, A}}
  \eqcol
    \frac{
      1 - v_x^2/c^2
    }{
      1 + v_x^2/c^2
    }
  \end{nedqn}

  \item This is correct, but rather than write this in terms of $v_x$,
  Feynman wants to write this in terms of $v'_{x, B}$. The easiest way
  to do so is through a trick.

  Note that if we change the \emph{new} frame of reference by the
  velocity $v'_{x, B}$, then the vertical velocity $v'_{y, B}$ becomes
  exactly $-v'_{y, A}$. Basically, particle A would become just like
  particle B, and vice versa.

  But then this means:

  \begin{nedqn}
    -v'_{y, A}
  \eqcol
    v'_{y, B}
    \frac{
      \sqrt{1 - \parensq{v'_{x, B}}/c^2}
    }{
      1 - \parensq{v'_{x, B}}/c^2
    }
  \\
    v'_{y, B}
  \eqcol
    -
    v'_{y, A}
    \sqrt{1 - \parensq{v'_{x, B}}/c^2}
  \\
    \frac{
      v'_{y, B}
    }{
      v'_{y, A}
    }
  \eqcol
    -\sqrt{1 - \parensq{v'_{x, B}}/c^2}
  \end{nedqn}

  \item Now, if we assume that momentum is conserved, this means that:

  \begin{nedqn}
    \frac{
      m'_A
    }{
      m'_B
    }
  \eqcol
    \sqrt{1 - \parensq{v'_{x, B}}/c^2}
  \end{nedqn}

  Feynman wants us to now consider letting $v'_{y, A} \to 0$. If we do
  this, then $m'_A \to m_0$. At the same time, $m'_B$, which is the mass
  of the particle at velocity $v'_B$ will tend toward $v'_{x, B}$, since
  the $v'_{y, B}$ component will tend toward zero. In which case:

  \begin{nedqn}
    m(v'_{x, B})
  \eqcol
    \frac{
      m_0
    }{
      \sqrt{1 - v'_{x, B}}
    }
  \end{nedqn}

  And now we see that this is the relativistic mass formula we've been
  looking for!

\end{enumerate}
