\section{Conservation of Energy}

\begin{enumerate}

  \item He starts talking about \emph{the conservation of energy}. He
  explains that we don't know what \emph{fundamentally constitutes}
  energy, but that it comes in many forms. You can convert between the
  forms, but everything always comes out equal.

  \item He talks about \emph{gravitational potential energy}. He first
  says: presume that there is no perpetual motion. Next, consider an
  ``ideal'' ``reversible'' machine for lifting balls (presumption is via a
  lever). He shows that if any machine (reversible or otherwise) could
  beat a reversible machine, we could achieve perpetual motion. As a
  corrolary, all reversible machines are equally effective.

  \item He next makes a symmetry argument to show that any reversible
  machine must conserve $wh$: weight times height. More generally,
  because it doesn't matter whether the force is due to gravity or
  electricity, we preserve force times distance.

  \item He shows some examples of a \emph{screw jack} and a balanced rod
  to work some other examples of conservation of energy. The screw jack
  shows that a much smaller force, acted through a longer distance, can
  do the work to lift a much heavier object.

  \item I would say that conservation comes from reversibility AKA
  symmetry.

  \item He next begins to talk about \emph{kinetic energy}. He doesn't
  yet use that $F = ma$, so he can't give a precise equation. But he
  does know (by symmetry) that the change in gravitational potential
  energy is equal to the amount of energy converted into kinetic energy.

  \item He mentions that there is elastic energy. He mentions that there
  is \emph{heat energy}. He mentions that loss of other forms of energy
  to heat energy is what stops perpetual motion machines. He notes that
  elastic energy really is a form of chemical AKA electrical energy.

  \item He mentions that there is nuclear energy which somehow comes
  from the forces holding the nucleus together. And last there is
  \emph{mass energy} which is energy that exists by virtue of mass
  existence. We know this exists because when a particle and its
  antiparticle annhiliate each other, energy is released. Thus there was
  energy in their masses...

  \item The other conservation laws are for \emph{linear momentum} and
  \emph{angular momentum}.

  \item He says that in a sense we don't understand conservation well,
  because we don't know precisely \emph{what} is conserved. It's not
  like there are little ``blocks'' of energy that are being conserved.

  \item He does hint that the conservation of energy is implied by the
  laws of quantum mechanics \emph{plus} a principle that the result of
  an experiment will never depend on the ``absolute time'' at which the
  experiment is run. Likewise, he hints that conservation of linear
  momentum is implied by QM plus a principle that it doesn't matter
  \emph{where} an experiment is run. Last, conservation of angular
  momentum is implied by QM and a principle that orientation of
  apparatus does not matter.

  \item He mentions some other conservation laws that are easier to
  understand. There is \emph{conservation of charge}: the net charge is
  always the same. There is also conservation of \emph{baryons} and
  \emph{leptons} (whatever those are!).

  \item He mentions that although energy is conserved, the amount of
  \emph{useful} energy (the amount that can be harvested to do work by
  humans) tends to be decreasing. This is thermodynamics and entropy.

\end{enumerate}

\TODO{Could we prove the conservation of energy from Newton's Third
Law?}

\TODO{Can we demonstrate a simple machine that will convert any stored
potential energy into a force $F$ for a given distance $d$, for any
choice of $F, d$?}
