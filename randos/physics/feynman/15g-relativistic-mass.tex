\subsection{Relativistic Mass}

\begin{enumerate}

  \item We now get to Einstein. Consider a force acting on an object.
  We've traditionally assumed that this implies a constant acceleration
  on the object (which has constant mass).

  However, relativity prohibits that we should be able to go faster than
  light. If we emit a photon on the front of our spaceship, both the
  spaceship pilot and a ``stationary'' observer see the photon move away
  from the spaceship at light speed. But if the spaceship were
  \emph{faster} than light speed, the photon would go backward. And that
  would not be compatible.

  \item So we see that force cannot accelerate us at constant speed
  forever. But the tweak is quite simple. Instead of talking about
  constant acceleration of a constant mass, let's talk about a constant
  \emph{change in momentum} to a \emph{changing mass}.

  Here we note: mass is really a concept of \emph{inertia}. Inertia is
  how quickly an object accelerates as a force is applied. Basically: as
  we increase the speed, an object's inertia is also increasing.

  \item Feynman gives an example of how synchrotrons need much stronger
  electromagnetic fields to do deflection of fast moving particles,
  because the inertia of the particles is so much greater than would be
  expected except for relativistic effects.

  \item We do not presently derive the equation (that's the next chapter
  I think), but it is:

  \begin{nedqn}
    m
  \eqcol
    \frac{m_0}{\sqrt{1 - v^2/c^2}}
  \end{nedqn}

  Since mass will become infinite if $v = c$, this shows that no object
  would be able to accelerate past $c$, since its mass (aka inertia)
  would become so great.

  What is $m_0$? It is the \define{rest mass} of the object. The rest
  mass is relative to a frame of reference. $m_0$ will be different in
  different frames, I guess it is smallest if you are standing next to
  the mass and it is not moving...

\end{enumerate}
