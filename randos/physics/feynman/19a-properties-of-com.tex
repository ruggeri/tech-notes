\subsection{Some Properties of CoM}

\begin{enumerate}
  \item The center of mass is like the average position of the
  \emph{grams} in the object. This is what it means to be waited by
  mass. The CoM might not be in the material of the object, but if the
  object is bounded, of course the CoM must lie within those bounds.

  \item The center of mass will always lie within any planes of symmetry
  that exist for the object.

  \item Consider subobjects A and B, with masses $M_A, M_B$. Now combine
  the objects. The center of mass is $\frac{M_A}{M} \vx_A +
  \frac{M_B}{M} \vx_B$, where $M = M_A + M_B$. This is because:

  \begin{nedqn}
    \frac{1}{M} \sum_i m_i \vx_i
  \eqcol
    \frac{1}{M}
    \parens{
      \sum_{i \in A} m_i \vx_i
      +
      \sum_{i \in B} m_i \vx_i
    }
  \\
  \eqcol
    \frac{1}{M}
    \parens{
      M_A \vx_A + M_B \vx_B
    }
  \end{nedqn}

  \item Feynman then goes on a long philosophical ramble about how
  Newton's laws ``reproduce'' themselves on greater and greater levels.
  That is: if Newton's laws apply to individual particles, they also
  apply to aggregates. Feynman notes that these are the kind of laws
  that we are most likely to find first, because they occur at larger
  and thus observable scales. He notes that the true laws at atomic
  level are not exactly Newton's, but when aggregates are studied,
  Newton's are a good approximation (and get better as we aggregate
  further).

\end{enumerate}
