\subsection{Principles}

\begin{enumerate}

  \item The \emph{principle of relativity} says that the result of an
  experiment should be the same in every inertial reference frame (one
  with constant linear velocity). This implies that it is impossible to
  detect whether we are are in motion (with no acceleration).

  \item Note that this is not true of an accelerating frame. In an
  accelerating frame we will experience a ``fictitious force'' which can
  be detected. The fictitious force will not have any explanation from
  inside the frame.

  I imagine myself floating in a box in space. I have been knocked out
  but come to. When I awake, if I am drifting inside the box, that is
  fine. I just had some kind of initial velocity I guess from past
  actions. But if I am \emph{accelerating} inside the box, then there
  is either (1) some force acting on me inside the box (in which case:
  where is it?), or (2) there is a force acting \emph{on the box}.

  \TODO{This could be a good time to study Foucault's Pendulum.}

  \item The principle of relativity is related to the concept of a
  \emph{Galilean transformation}. A Galilean transformation assumes that
  if you are traveling at a velocity $v$, then this changes your
  relative spatial coordinates over time, but simply in the expected
  way. It does not expect any space or time dilation.

  Newton's laws are invariant under a Galilean transformation. Which
  means we expect the principle of relativity to stay true in any
  inertial frame of reference.

  \item Studying electromagnetism, Maxwell proposed some equations. An
  implication of his equations was that light always travels at the same
  speed (in a vacuum): $c$. In particular: even if the source of light
  was moving, the light propagates at the same speed $c$. This is
  invariant. This principle is called the \emph{principle of invariant
  light speed}.

  It turns out that this is also the case with regard to sound. If the
  source of a sound is moving, the sound still propagates through the
  medium at a constant rate. I believe that makes sense, because the
  sound is a disturbance in the medium at a point, which needs to
  propagate through the medium. The sound is not itself a ``thing'' that
  is moving. It is a disturbance that is propagating.

  I should note: if the \emph{medium is moving}, then that will affect
  the rate of travel relative to a ``stationary'' observer within the
  moving medium. That's why sound can ``blow'' in the wind. But if you
  treat the medium as fixed and unmoving, the propagation rate remains
  the same.

  The (fictional) ``aether'' through which electromagnetic waves
  supposedly propagate was called the \emph{luminiferous aether}. One
  weird thing: the luminiferous aether seems to have no interaction with
  massive bodies moving through space. It doesn't retard them in any
  detectable way. So that was already suspicious to people.

  \item If the wave theory of light/luminiferous aether/principle of
  invariant light speed is true, then that threatens the the principle
  of relativity. If I am in a box travelling at $u$ relative to the
  aether, then I should observe light traveling at speed $c - u$. But in
  that case, I can detect my velocity inside the box!

  Basically, Maxwell's theory was saying that not all reference frames
  are equally privileged. Maxwell's theory says that there is a baseline
  reference frame in which the luminiferous aether appears stationary
  and in which light really does travel at it's ``correct'' rate of $c$.

  \item So you have these two principles: the principle of relativity
  (which says you can't detect inertial reference frames), and the
  principle of invariant light speed (from Maxwell). And these seem to
  be irreconcilable, because a Galialean transformation ought to be
  detectable.

  At first, they thought that Maxwell must be wrong. But every way they
  could try to fix Maxwell suggested results that were not observed
  experimentally. Presumably these were \emph{variable light speed}
  theories.

\end{enumerate}
