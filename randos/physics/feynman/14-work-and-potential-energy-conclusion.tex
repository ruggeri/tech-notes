\section{Work and Potential Energy (conclusion)}

\begin{enumerate}

  \item He mentions the example of holding a weight, and how that
  \emph{feels} like work. But that's because of the nature of how
  muscles work.

  \item He gives an example use of conservation of energy. Imagine a
  frictionless track. It undulates, but the start is \SI{5}{m} higher
  than the end. What's the velocity of a car on the track at the end?
  It's trivially found by converting the change in gravitational
  potential energy into kinetic energy.

  \item He talks about \define{conservative} forces. These are one where
  the same work is done to move an object from A to B no matter the path
  taken.

  \item He right away defines \define{potential energy} from this. He
  basically says: pick any point A as your reference point. Call
  potential energy the work done to get to B. Now, your choice of A will
  determine potential energy \emph{up to a constant}. But that doesn't
  matter, especially when you talk about conservation of energy.

  You just need a relative point! You don't need somewhere with the
  ``lowest'' energy.

  \item He next asks what escape velocity should be. He says: at the
  start, an object has a velocity (kinetic energy) plus a gravitational
  potential energy (we can put this in terms of the radius of the
  earth). At the end, if the object is just escaped, the velocity should
  be zero, and the distance out will approach infinity so the
  gravitational potential energy should be zero.

  We start with a definition of gravitational potential energy:

  \begin{nedqn}
    U(r)
  \eqcol
    -\frac{G M m}{r}
  \end{nedqn}

  Why? Because to move someone from $a$ to $b$, we find:

  \begin{nedqn}
    \int_a^{b} \frac{G M m}{r^2} \dr
  \eqcol
    \left(
      - \frac{G M m}{r}
    \right)
    \intevalbar{a}{b}
  \\
  \eqcol
    -\frac{G M m}{b}
    -
    \left(
      -\frac{G M m}{a}
    \right)
  \end{nedqn}

  Anyway, let's assume the Earth's radius is $r_0$. Recall that we said
  that initial kinetic energy must equal initial potential energy.
  Therefore:

  \begin{nedqn}
    \half mv^2
  \eqcol
    \frac{G M m}{r_0}
  \\
    v^2
  \eqcol
    2 \frac{G M}{r_0}
  \\
  \eqcol
    2gr_0
    \nedcomment{$g = \frac{G M}{r_0^2}$}
  \end{nedqn}

  I want to note: you can pick a different reference point for your
  gravitational potential energy. In that case, everything is translated
  by a constant and in fact nothing changes overall!

  \item He mentions that in a system with many planets, we should be
  able to work things out by breaking down into pairwise potential
  energies, and then just summing up for needed results.

  He mentions that this breaks down at molecular levels, which I find
  \emph{very unexpected}. I don't know if he means in terms of some kind
  of quantum situation. He does say you can do this superposition
  approach toward potential energies as an approximation...

  He hints that maybe there's something where the line between kinetic
  and potential energies may blur at the molecular level?

  \item He says that electrical and gravitational forces are both
  conservative. And he says that all known forces are conservative.

  He mentions that nothing from Newton suggests that forces \emph{must}
  be conservative.

  \item He mentions that friction looks like energy loss until we notice
  the heat is causes.

  \item \TODO{The reasoning here is a work in progress!}

  It's maybe worth mentioning part of what makes energy weird.
  You'd think that expending $X$ units of energy would always yield a
  constant change of $Y$ in the velocity. But that isn't what happens.

  Alternate example: burning fuel to make your rocketship go? How does
  that get explained?

  Alternate example: How do cars respect conservation of momentum? Earth
  gets pushed backward.

  Note, you don't need to eject material. You use energy that is densely
  stored to rotate a wheel and thus push back on the earth.

  If no gas is being burned and the pistons are just rotating freely,
  you don't need to push them. But to accelerate them you need to inject
  more gas, and it's harder and harder to push on something that's
  already moving really fast.

  Is that true? I mean, I think that RPM in the engine is mediated by
  the gearbox... But regardless when the gears are meshed, that mediates
  how much force is needed to further accelerate?

  For rockets... Why does burning another gallon of fuel not result in
  the same acceleration? Imagine you store energy in terms of electrical
  plates. The plates are paired with equal charges. To accelerate, you
  release a plate, and you push yourself away from that plate.

  I guess that to accelerate your large spacecraft an appreciable
  amount, you need to eject a bunch of mass. That is conservation of
  momentum. I guess each kilogram of mass you eject, you gain more
  velocity per ejected kilogram (because the mass of the rocket is
  decreasing).

  I guess it also depends on the velocity of the ejected plate. I guess
  if the ship is accelerating more easily, then the time that it takes
  for a constant energy of $Fd$ to be expended will be less, reducing
  efficiency.

  Nailed it.

  \item \TODO{I feel like I should work an example of a pulley, and
  especially with a focus on tension}.
\end{enumerate}
