\section{Work and Potential Energy (conclusion)}

\begin{enumerate}

  \item He mentions the example of holding a weight, and how that
  \emph{feels} like work. But that's because of the nature of how
  muscles work.

  \item He gives an example use of conservation of energy. Imagine a
  frictionless track. It undulates, but the start is \SI{5}{m} higher
  than the end. What's the velocity of a car on the track at the end?
  It's trivially found by converting the change in gravitational
  potential energy into kinetic energy.

  \item He talks about \define{conservative} forces. These are one where
  the same work is done to move an object from A to B no matter the path
  taken.

  \item He right away defines \define{potential energy} from this. He
  basically says: pick any point A as your reference point. Call
  potential energy the work done to get to B. Now, your choice of A will
  determine potential energy \emph{up to a constant}. But that doesn't
  matter, especially when you talk about conservation of energy.

  You just need a relative point! You don't need somewhere with the
  ``lowest'' energy.

  \item He next asks what escape velocity should be. He says: at the
  start, an object has a velocity (kinetic energy) plus a gravitational
  potential energy (we can put this in terms of the radius of the
  earth). At the end, if the object is just escaped, the velocity should
  be zero, and the distance out will approach infinity so the
  gravitational potential energy should be zero.

  We start with a definition of gravitational potential energy:

  \begin{nedqn}
    U(r)
  \eqcol
    -\frac{G M m}{r}
  \end{nedqn}

  Why? Because to move someone from $a$ to $b$, we find:

  \begin{nedqn}
    \int_a^{b} \frac{G M m}{r^2} \dr
  \eqcol
    \left(
      - \frac{G M m}{r}
    \right)
    \intevalbar{a}{b}
  \\
  \eqcol
    -\frac{G M m}{b}
    -
    \left(
      -\frac{G M m}{a}
    \right)
  \end{nedqn}

  Anyway, let's assume the Earth's radius is $r_0$. Recall that we said
  that initial kinetic energy must equal initial potential energy.
  Therefore:

  \begin{nedqn}
    \half mv^2
  \eqcol
    \frac{G M m}{r_0}
  \\
    v^2
  \eqcol
    2 \frac{G M}{r_0}
  \\
  \eqcol
    2gr_0
    \nedcomment{$g = \frac{G M}{r_0^2}$}
  \end{nedqn}

  I want to note: you can pick a different reference point for your
  gravitational potential energy. In that case, everything is translated
  by a constant and in fact nothing changes overall!

  \item He mentions that in a system with many planets, we should be
  able to work things out by breaking down into pairwise potential
  energies, and then just summing up for needed results.

  He mentions that this breaks down at molecular levels, which I find
  \emph{very unexpected}. I don't know if he means in terms of some kind
  of quantum situation. He does say you can do this superposition
  approach toward potential energies as an approximation...

  He hints that maybe there's something where the line between kinetic
  and potential energies may blur at the molecular level?

  \item He says that electrical and gravitational forces are both
  conservative. And he says that all known forces are conservative.

  He mentions that nothing from Newton suggests that forces \emph{must}
  be conservative.

  \item He mentions that friction looks like energy loss until we notice
  the heat is causes.

  \item I want to copy a proof I did elsewhere about the relationship
  between kinetic energy and work:

  \begin{nedqn}
    \int_0^{d_0} F(d) \dd
  \eqcol
    \int_0^{d_0} m a(d) \dd
    \nedcomment{definition of force}
  \\\eqcol
    m \int_0^{t_0} a(t) d'(t) \dt
    \nedcomment{change variables to time}
  \\\eqcol
    m \int_0^{t_0} v'(t) v(t) \dt
    \nedcomment{write in terms of velocity}
  \\\eqcol
    m \int_0^v v \dv
    \nedcomment{change of variables to velocity}
  \\\eqcol
    \frac{1}{2} m v^2
  \end{nedqn}

  I guess this proof is unnecessary because Feynman already gave a proof
  earlier that work equals the change in kinetic energy.

  \item I want to resolve a confusion about the relation between stored
  energy and kinetic energy. Basically: if I expend half my rocket fuel,
  I get to a velocity $v_0$. But if I expend the next half, I only get
  to velocity $\sqrt{2}v_0$. Why?

  We know this holds for gravitational energy. Since the acceleration
  due to gravity is $g$, then the distance traveled over a time $t$ is
  $\half t^2$. Which means that if we double the height we drop from, we
  only increase the time by a factor of $\sqrt{2}$.

  Let's consider a rocket ship that will be boosted in two stages. It is
  powered by electrostatics. In stage one, it drops a negatively charged
  plate on the launch pad. This causes a force to be exerted on the ship
  pushing it upward.

  After that force dies out, the ship is moving at constant velocity. It
  drops a second negatively charged plate, giving it a new push. I will
  imagine that this second plate, rather than flying backward in space,
  lands on a nearby planet. It thus mirrors the first plate's action,
  except the rocket ship is already moving at a given velocity.

  We can now see why we are not going to get double the change in
  velocity from the second stage. The reason is that the rocket ship is
  already moving quickly out of the range of the second stage plate.
  Thus, there is less \emph{time} during which the rocket ship is within
  an effective range to receive further pushing.

  Okay I didn't do all the math but I think the idea is clear.

  \TODO{I still feel like I should do a math proof to show the change in
  velocity given an input of energy \emph{when there is already an
  initial velocity!}}

  \item \TODO{I feel like I should work an example of a pulley, and
  especially with a focus on tension}.
\end{enumerate}
