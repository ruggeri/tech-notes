\subsection{Center of Mass}

\begin{enumerate}
  \item Consider a system that is experiencing many forces. That is,
  each part of the system $i$ is acted upon with force $\vF_i$. Thus,
  each part is accelerated according to:

  \begin{nedqn}
    \vF_i
  \eqcol
    m_i
    \fpartialsq{t}{\vx_i}
  \end{nedqn}

  \item Different parts of the system are moving in different directions
  at different speeds. However, we would like to ask how the system,
  ``in aggregate,'' is moving in response to the many forces. Our hope
  is that we can, simply, aggregate the various forces $\vF_i$ into an
  aggregate force:

  \begin{nedqn}
    \vF
  \eqcol
    \sum_i \vF_i
  \end{nedqn}

  \noindent
  Perhaps this is possible. But, pray tell, in what respect does the
  system respond to such a force? Well, note:

  \begin{nedqn}
    \vF
  \eqcol
    \sum_i \vF_i
  \\
  \eqcol
    \sum_i m_i \fpartialsq{t}{\vx_i}
  \\
  \eqcol
    M \sum_i \frac{m_i}{M} \fpartialsq{t}{\vx_i}
  \\
  \eqcol
    M \fpartialsq{t} \sum_i \frac{m_i}{M} \vx_i
  \end{nedqn}

  \noindent
  Feynman notes that we are assuming non-relativistic speeds: that $m_i$
  does not change wrt $t$.

  \item This motivates us to define the \define{center of mass}:

  \begin{nedqn}
    \vx
  \eqcol
    \sum_i m_i \vx_i / M
  \end{nedqn}

  \noindent
  Having defined this, it is a matter of substitution that:

  \begin{nedqn}
    \vF
  \eqcol
    M
    \fpartialsq{t} \vx
  \end{nedqn}

  \item Thus, if we choose to view the center of mass as the ``true
  location'' of the system, we can see how the system is moved when
  acted upon by a set of forces. We do not need to know on what parts
  those forces are acting; we can simply just roll all the $\vF_i$
  blindly up into one $\vF$.

  \item We must not assume that when the total exterior force $\vF =
  \sum_i \vF_i = 0$, therefore the system is motionless. This is not at
  all a safe assumption! Any number of things can happen to the system:
  it can contract (everything moves toward center of mass), it can
  expand (everything moves away), or many other transformations
  (provided they do not move the center of mass). These are all
  transformations that ``move'' the system without moving its center of
  mass.
\end{enumerate}
