\subsection{Four Momentum}

\begin{enumerate}

  \item Under rotation in 3D space, the momentum vector \emph{will}
  change (direction; not magnitude). But the momentum is in a sense
  ``invariant'' in that it will undergo an \emph{opposite} rotation to
  that which was applied to the space.

  We say that the three-momentum is \define{contravariant} under
  rotations of Euclidean three-space.

  \item Three-momentum doesn't tell you mass or velocity; it tells you a
  mix of those things. For instance: an object with zero momentum
  certainly has no velocity, but we don't have enough information to
  know the mass of the object.

  Therefore, it is obvious that three-momentum doesn't have enough
  information for you to know how it will transform if you change the
  velocity of the reference frame. This is true even if you are ignoring
  any special relativistic effects.

  \item The relativistic version of momentum is called
  \define{four-momentum}. It is: $(p_x, p_y, p_z, m)$.

  They sometimes say that the ``mass'' component of this vector is
  really an ``energy'' component. This follows from $E = mc^2$. So
  really if we correct by a factor of $c^2$ we can consider the mass as
  energy. In which case the vector represents momentum and energy.

  \item No matter the frame, the Minkowski norm of the four-momentum is
  always invariant:

  \begin{nedqn}
    \normsq{
      \parens{p_x, p_y, p_z, m c^2}
    }
  \eqcol
    m^2 c^4 - \parensq{m \norm{v}}
  \\
  \eqcol
    \parens{c^2 - \normsq{v}}
    m^2 c^2
  \\
  \eqcol
    \parens{c^2 - \normsq{v}}
    \parensq{
      \frac{m_0 c}{\sqrt{1 - \normsq{v} / c^2}}
    }
  \\
  \eqcol
    \parens{c^2 - \normsq{v}}
    \parensq{
      \frac{
        m_0 c^2
      }{
        c\sqrt{1 - \normsq{v} / c^2}
      }
    }
  \\
  \eqcol
    \parens{c^2 - \normsq{v}}
    \parensq{
      \frac{
        m_0 c^2
      }{
        \sqrt{c^2 - \normsq{v}}
      }
    }
  \\
  \eqcol
    \parensq{m_0 c^2}
  \\
  \eqcol
    E_0^2
  \end{nedqn}

  Thus we see that the Minkowski norm of the four-momentum is always
  equal to the rest energy, no matter the frame we have used to express
  the four-momentum. And of course, this is always constant.

  Unsurprisingly, neither total energy nor momentum are invariant under
  changes of inertial reference frame. But we've just shown: $E^2 - p^2
  = E_0^2$.

  \item Note that energy and momentum are still \emph{conserved} by
  interactions within a fixed reference frame. In a fixed reference
  frame mass is neither destroyed, nor is energy ever lost.

  \item Last, I ought to show that the four-momentum vector is
  contravariant to Lorentz transformations applied to space-time. Having
  done that, we can conclude that four-momentum is more ``real'' than
  three-momentum or energy alone.

\end{enumerate}
