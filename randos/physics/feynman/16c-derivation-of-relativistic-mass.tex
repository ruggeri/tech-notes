\subsection{Derivation of Relativistic Mass}


\begin{enumerate}

  \item Next Feynman wants to derive the formula for relativistic mass.
  He considers two identical particles that approach each other and then
  deflect at an angle. We may orrient axes such that: (a) particle A
  approaches with velocity $\vv_A$, (b) after approach, particle A
  maintains horizontal velocity $v_{x, A}$, (c) after approach, particle
  A reverses vertical velocity $v_{y, A}$ to $-v_{y, A}$.

  Particle B does just like particle A, except its velocity $\vv_B = -
  \vv_A$ at all times.

  \item If we assume that relativistic mass is a function of rest mass
  and the magnitude of velocity, then the masses of A and B are always
  equal, and the net momentum is always zero.

  \item He now imagines changing the frame by moving at $v_x$
  horizontally. We know from the above work that the velocity of
  particle A transforms to:

  \begin{nedqn}
    v'_{y, A}
  \eqcol
    v_y \frac{
      \sqrt{1 - v_x^2/c^2}
    }{
      1 - v_x^2/c^2
    }
  \\
    v'_{x, A}
  \eqcol
    \frac{
      v_x - v_x
    }{
      1 - v_x^2/c^2
    }
  \\
  \eqcol
    0
  \end{nedqn}

  With regard to particle B, we have:

  \begin{nedqn}
    v'_{y, B}
  \eqcol
    v_y \frac{
      \sqrt{1 - v_x^2/c^2}
    }{
      1 + v_x^2/c^2
    }
  \\
    v'_{x, B}
  \eqcol
    \frac{
      -v_x - v_x
    }{
      1 - (v_x)(-v_x)/c^2
    }
  \\
  \eqcol
    \frac{-2v_x}{1 + v_x^2/c^2}
  \end{nedqn}

  Therefore:

  \begin{nedqn}
    \frac{v'_{y, B}}{v'_{y, A}}
  \eqcol
    \frac{
      1 - v_x^2/c^2
    }{
      1 + v_x^2/c^2
    }
  \end{nedqn}

  \item This math is correct, but rather than write this formula in
  terms of $v_x$, Feynman wants to write this in terms of $v'_{x, B}$.
  The easiest way to do so is through a trick.

  Note that if we change the \emph{new} frame of reference by the
  velocity $v'_{x, B}$, then the vertical velocity $v'_{y, B}$ becomes
  exactly $-v'_{y, A}$. Basically, particle A would become just like
  particle B, and vice versa. This follows from the symmetery of our
  initial setup.

  We may conclude:

  \begin{nedqn}
    -v'_{y, A}
  \eqcol
    v'_{y, B}
    \frac{
      \sqrt{1 - \parensq{v'_{x, B}}/c^2}
    }{
      1 - \parensq{v'_{x, B}}/c^2
    }
  \\
    v'_{y, B}
  \eqcol
    -
    v'_{y, A}
    \sqrt{1 - \parensq{v'_{x, B}}/c^2}
  \\
    \frac{
      v'_{y, B}
    }{
      v'_{y, A}
    }
  \eqcol
    -\sqrt{1 - \parensq{v'_{x, B}}/c^2}
  \end{nedqn}

  \item Now, if we assume that momentum is conserved, this means that:

  \begin{nedqn}
    \frac{
      m'_A
    }{
      m'_B
    }
  \eqcol
    \sqrt{1 - \parensq{v'_{x, B}}/c^2}
  \end{nedqn}

  Feynman wants us to now consider letting $v'_{y, A} \to 0$. If we do
  this, then $m'_A \to m_0$. At the same time, $m'_B$, which is the mass
  of the particle at velocity $v'_B$ will tend toward $m(v'_{x, B})$,
  since the $v'_{y, B}$ component will tend toward zero. In which case:

  \begin{nedqn}
    m(v'_{x, B})
  \eqcol
    \frac{
      m_0
    }{
      \sqrt{1 - v'_{x, B}}
    }
  \end{nedqn}

  And now we see that this is the relativistic mass formula we've been
  looking for!

\end{enumerate}
