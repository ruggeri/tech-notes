\subsection{Rigid Bodies: When Motion Doesn't Happen}

\begin{enumerate}
  \item Let us consider a \define{rigid body}: one where the parts have
  a fixed and unvarying distance form each other.

  We know that translating all points in the object does not deform it.
  We also know that rotation about its center of mass does not deform
  the object.

  Any other other distance preserving transformation can be broken down
  into rotation about CoM followed by translation of the center of mass.

  \item We reiterate that a rigid body will not translate when the net
  force $\vF = 0$. On the one hand, we know this because (1) translation
  moves the center of mass, and (2) when $\vF = 0$, the center of mass
  will not move.

  We may also calculate more directly. Consider if the object translated
  by $vx$. Then:

  \begin{nedqn}
    \sum_i \vF_i \cdot \vx
  \eqcol
    \parens{\sum_i \vF_i} \cdot \vx
  \\
  \eqcol
    \vF \cdot \vx
  \\
  \eqcol
    0 \cdot \vx
  \\
  \eqcol
    0
  \end{nedqn}

  \noindent
  So it is not energetically preferred to translate the object.

  \item We may next ask: when does the object not undergo rotation?
  Well, consider a rotation by $\theta$ radians. Consider that each
  force $\vF_i$ is applied at a distance $r_i$ from the center of mass.

  Then, by turning, how much work is done?

  \begin{nedqn}
    \sum_i \vF_i \vx_i
  \eqcol
    \sum_i \vF_i r_i \theta
  \end{nedqn}

  We denote $\tau_i = \vF_i r_i$; this is called \define{torque}. When
  the torques sum to $\tau = 0$, this is precisely when rotation is not
  energetically preferred.

  \item Thus we see that an object at rest will stay at rest when (1)
  net force is zero, and (2) net torque is zero. It is possible for
  both, either, or neither to be zero.
\end{enumerate}
