\section{The Special Theory of Relativity}

\begin{enumerate}

  \item The \emph{principle of relativity} says that the result of an
  experiment should be the same in every inertial reference frame (one
  with constant linear velocity). This implies that it is impossible to
  detect whether we are are in motion (with no acceleration).

  \item Note that this is not true of an accelerating frame. In an
  accelerating frame we will experience a ``fictitious force'' which can
  be detected. The fictitious force will not have any explanation from
  inside the frame.

  I imagine myself floating in a box in space. I have been knocked out
  but come to. When I awake, if I am drifting inside the box, that is
  fine. I just had some kind of initial velocity I guess from past
  actions. But if I am \emph{accelerating} inside the box, then there
  is either (1) some force acting on me inside the box (in which case:
  where is it?), or (2) there is a force acting \emph{on the box}.

  \item The principle of relativity is related to the concept of a
  \emph{Galilean transformation}. A Galilean transformation assumes that
  if you are traveling at a velocity $v$, then this changes your
  relative spatial coordinates over time, but simply in the expected
  way. It does not expect any space or time dilation.

  \item Studying electromagnetism, Maxwell proposed some equations. An
  implication of his equations was that light always travels at the same
  speed (in a vacuum): $c$. In particular: even if the source of light
  was moving, the light propagates at the same speed $c$. This is
  invariant. This principle is called the \emph{principle of invariant
  light speed}.

  It turns out that this is also the case with regard to sound. If the
  source of a sound is moving, the sound still propagates through the
  medium at a constant rate. I believe that makes sense, because the
  sound is a disturbance in the medium at a point, which needs to
  propagate through the medium. The sound is not itself a ``thing'' that
  is moving. It is a disturbance that is propagating.

  I should note: if the \emph{medium is moving}, then that will affect
  the rate of travel relative to a ``stationary'' observer within the
  moving medium. That's why sound can ``blow'' in the wind. But if you
  treat the medium as fixed and unmoving, the propagation rate remains
  the same.

  The (fictional) ``aether'' through which electromagnetic waves
  supposedly propagate was called the \emph{luminiferous aether}. One
  weird thing: the luminiferous aether seems to have no interaction with
  massive bodies moving through space. It doesn't retard them in any
  detectable way.

  \item If the wave theory of light/luminiferous aether/principle of
  invariant light speed is true, then that threatens the the principle
  of relativity. If I am in a box travelling at $u$ relative to the
  aether, then I should observe light traveling at speed $c - u$. But in
  that case, I can detect my velocity inside the box!

  Basically, Maxwell's theory was saying that not all reference frames
  are equally privileged. Maxwell's theory says that there is a baseline
  reference frame in which the luminiferous aether appears stationary
  and in which light really does travel at it's ``correct'' rate of $c$.

  \item So you have these two principles: the principle of relativity
  (which says you can't detect inertial reference frames), and the
  principle of invariant light speed (from Maxwell). And these seem to
  be irreconcilable, because a Galialean transformation ought to be
  detectable.

  At first, they thought that Maxwell must be wrong. But every way they
  could try to fix Maxwell suggested results that were not observed
  experimentally. Presumably these were \emph{variable light speed}
  theories.

  \item Now, around this time Lorentz (and then Poincaré) started to
  reconsider the assumption that our coordinate system $(x, y, z, t)$
  ought to undergo a Galilean transformation in a new inertial reference
  frame.

  Instead, they noticed that Maxwell's laws would remain invariant so
  long as we replaced:

  \begin{nedqn}
    x'
  \eqcol
    \frac{x - ut}{\sqrt{1 - u^2/c^2}}
  \\
    t'
  \eqcol
    \frac{t - ux/c^2}{\sqrt{1 - u^2/c^2}}
  \end{nedqn}

  For simplicity we assume that velocity is only along the $x$ axis.
  This transformation is called a \define{Lorentzian transformation}.

  So here is what they were saying: what if your perception of $x$
  position and the rate of time \emph{changed} depending on your
  velocity? If that were the case, it would ``undo'' your ability to
  detect the ostensible change in the speed of light.

  \item Now we are in a weird scenario. Newton's laws are invariant
  under a Galilean transformation, while Maxwell's laws are invariant
  under a Lorentzian transformation. If a change in inertial reference
  frame results in a Galilean transformation being observed, then
  electrodynamics lets us observe the change. If a change in inertial
  reference frame results in a Lorentzian transformation, then mechanics
  will allow us to detect the change.

  So even though the Lorentzian transformation lets us reconcile the
  principle of relativity and the principle of invariant light speed
  within the context of electrodynamics, it \emph{breaks} the principle
  of relativity in the context of mechanics.

  This is what Einstein resolved. He realized that you could very
  slightly tweak Newton's laws so that they \emph{were} invariant under
  a Lorentzian transformation. All you need to do is assume that
  \emph{mass is observed to change at different velocities}. We shall
  return to this point.

  \item Let's pause a moment. Why didn't they throw out the principle of
  invariant light speed? Presumably because Maxwell's laws were working
  great, and any ``fix'' they could think of predicted effects that were
  not observed when experiments were performed.

  Why didn't they throw out the principle of relativity? Presumably they
  liked it and felt it was intuitive. But more important: the
  Michelson-Morley experiment. This experiment ought to have detected
  the Earth's movement through luminiferous aether, but it failed to do
  so. The experimental result came back just as if the Earth was
  stationary in the luminiferous aether. That result corresponded with
  the principle of relativity, but not what they would have expected
  from their interpretation of Maxwell's theory (under a Galilean
  transformation).

  \item We may further ask: isn't it weird that the Lorentzian transform
  turns out to be the ``right one?'' Why not the Galilean
  transformation? It feels obvious that this should be the right one.

  This relates to our understanding of space and time. I think it is
  Minkowski who introduced the concept of ``space-time.'' When you are
  working in this concept space, the Lorentzian transformation becomes
  the natural one. I think. I don't believe Feynman fully discusses this
  in this chapter yet.

  \item Let's describe the Michelson-Morley experiment. They decided to
  do an experiment that would use the invariant speed of light in aether
  to detect the velocity in an inertial reference frame. They were
  assuming that a moving body experiences a Galilean transformation of
  its coordinate space.

  Here's what they did. They would shine a light on a partially
  reflective mirror angled at 45 degrees. Some light would pass through
  the mirror, and some would reflect off. In both cases, there are
  (fully reflective) mirrors to reflect the light \emph{back at the
  mirror}. The mirrors are set at an equal distance $L$ from the initial
  partially-reflective mirror.

  When the light returns to the partially reflective mirror, it has
  traveled a distance $2L$. It doesn't matter which path. Therefore we
  expect the two beams of light to be ``in phase.'' I think they
  invented the interferometer to detect this.

  \item The key to the experiment comes next. They next put the aparatus
  at a velocity $u$. The velocity is straight at one of the mirrors, but
  orthogonal to the other. The distance traveled by the light is no
  longer $2L$ (for either beam).

  (BTW: they didn't move the aparatus. Rather, they aligned it with the
  Earth's direction of travel about the Sun. They did the experiment and
  recorded the interference pattern. They then rotated the experiment
  90\degree and repeated. They expected to find a change in interference
  pattern. These are historical details; they're irrelevant to the
  thought experiment.)

  The beam reflected orthogonally needs to travel a distance not only
  $L$, but also some transverse distance. Feynman does a little easy
  math to show this is: $\frac{2L}{\sqrt{1 - u^2/c^2}}$.

  On the other hand, the light that travels along the path of velocity
  must travel a distance of $\frac{2L}{1 - u^2/c^2}$. It turns out that
  this distance traversed is greater than the distance traversed by the
  orthogonal beam.

  All of which is to say: we expect the two beams to be \emph{out of
  phase}. \emph{But that is not what was observed.} Instead, the light
  remained in phase. This is a null result: it was \emph{not} what would
  have been expected from Maxwell's laws under a Galilean
  transformation.

  This failure to detect a change in the interference pattern was a
  \emph{preservation} of the principle of relativity. But it didn't make
  much sense with respect to their interpretation of Maxwell.

  \item Instead of throwing out Maxwell's laws, Lorentz proposed a
  solution: that distances were \emph{foreshortened} by a factor of
  $\sqrt{1 - u^2/c^2}$ along the axis of movement.

  You may observe that this is what is expected under the Lorentzian
  transformation. And it would explain the null result: with this factor
  of correction, the distances equalize and thus we don't expect the
  light to go out of phase.

  \item We have shown how distance is transformed when moving in a new
  inertial reference frame. It turns out time must be transformed as
  well. How?

  The thought experiment is this. Imagine a light clock - you fire a
  photon, it travels one meter, hits a reflector, and it returns to a
  photoreceptor. This is one click of the clock.

  Synchronize two clocks, and then send someone off on a spaceship.
  Important: have them mount their clock perpendicular to their
  direction of travel. Otherwise, the clock would appear foreshortened
  to an outside observer.

  Anyway, inside the spaceship the clock appears to be clicking
  normally. Time is passing as usual.

  ``Stationary'' outside the spaceship, the spaceship clock appears to
  be clicking slower than usual. That's because the light needs to
  travel an extra distance because of the lateral movement of the
  spaceship. The light appears to propagate at the same rate as always
  ($c$), so this clock is clicking more slowly according to the outside
  observer. It's not really a proper light-meter clock anymore.

  Presuming the principle of relativity - that inside the spaceship we
  cannot detect its motion - this implies that all processes seem to run
  slower than expected (when viewed from the outside). If they ran at
  the ``stationary'' rate, they would seem to run faster than the clock
  inside the spaceship. And that would be detectable and violate
  relativity.

  Thus we see that time dilation is a necessary consequence.

  \item Our time example assumes that there is no shrinkage of space
  perpindicular to the direction of travel. Why should there not be?

  Imagine two equally sized hoola hoops with paper on the inside. One of
  these is our rocketship, while the other is ``stationary.'' We crash
  them into each other.

  If one of the hoola hoops had shrunk perpindicular to the travel, then
  it would ``fit inside'' the other. This is detectable because the
  paper inside the smaller hoop is fine, while the larger hoop's paper
  is smashed.

  That would violate relativity.

  \item Feynman gives a practical example. Muons live a very short time,
  and therefore they ought not be able to travel very far. But they are
  detected at much greater distances than they should be expected to
  live for. How?

  He suggests: the muon is so fast, that the short time it takes to
  disintegrate actually appears to be quite long from a ``stationary''
  observer. Thus the muon can travel a great distance from the
  perspective of the stationary observer.

  Muons will live for ``longer'' if traveling at higher velocities. But
  if you back that relativistic factor out, you get a stable estimate of
  their ``true'' lifespan. This has been done at various muon
  velocities.

  \item \TODO{Simultaneity}

  \item \TODO{Four vectors}

  \item We now get to Einstein. Consider a force acting on an object.
  We've traditionally assumed that this implies a constant acceleration
  on the object (which has constant mass).

  However, relativity prohibits that we should be able to go faster than
  light. If we emit a photon on the front of our spaceship, both the
  spaceship pilot and a ``stationary'' observer see the photon move away
  from the spaceship at light speed. But if the spaceship were
  \emph{faster} than light speed, the photon would go backward. And that
  would not be compatible.

  In that case, force cannot accelerate us at constant speed. But the
  tweak is quite simple. Instead of talking about constant acceleration
  of a constant mass, let's talk about a constant \emph{change in
  momentum} to a \emph{changing mass}.

  Here we note: mass is really a concept of \emph{inertia}. Inertia is
  how quickly an object accelerates as a force is applied. Basically: as
  we increase the speed, an object's inertia is also increasing.

  He gives an example of how synchrotrons need much stronger
  electromagnetic fields to do deflection of fast moving particles,
  because the inertia of the particles is so much greater than would be
  expected except for relativistic effects.

  \item We do not presently derive the equation (that's the next chapter
  I think), but it is:

  \begin{nedqn}
    m
  \eqcol
    \frac{m_0}{\sqrt{1 - v^2/c^2}}
  \end{nedqn}

  Since mass will become infinite if $v = c$, this shows that no object
  would be able to accelerate past $c$, since its mass (aka inertia)
  would become so great.

  What is $m_0$? It is the \define{rest mass} of the object. The rest
  mass is relative to a frame of reference. $m_0$ will be different in
  different frames, I guess it is smallest if you are standing next to
  the mass and it is not moving...

  \item Feynman gives another example. Consider a gas. Heat the gas. The
  mass is:

  \begin{nedqn}
    m
  \eqcol
    \frac{m_0}{\sqrt{1 - v^2/c^2}}
  \end{nedqn}

  He notes that through an application of the binomial theorem, this is:

  \begin{nedqn}
    m
  \approxcol
    m_0 \parens{
      1 + \frac{1}{2} v^2 / c^2 + \frac{3}{8} v^4/c^4 + \cdots
    }
  \end{nedqn}

  Dropping all the smallest terms, we have:

  \begin{nedqn}
    m
  \approxcol
    m_0 + \half m_0 v^2 / c^2
  \end{nedqn}

  But this is

  \begin{nedqn}
    m
  \approxcol
    m_0 + \mathit{KE} / c^2
  \end{nedqn}

  Which is to say: the change in the mass is equal to the change in
  kinetic energy (divided by $c^2$). Or:

  \begin{nedqn}
    \Delta \mathit{KE}
  \approxcol
    \Delta m c^2
  \end{nedqn}

  \item Instead of speaking in terms of mass, Eistein started talking in
  terms of energy. He said that if $m_0$ is the \define{rest mass}, then
  $m_0 c^2$ is the \define{rest energy}. Likewise we have:

  \begin{nedqn}
    mc^2
  \eqcol
    m_0 c^2 + \half m_0 v^2 + \cdots
  \end{nedqn}

  Here we have the \define{total energy} of the body being equal to its
  (a) rest energy, (b) its kinetic energy, (c) some other junk.

  Feynman does some math. He says: let's say that $mc^2$ is the total
  energy. Then if we do work on the object, the rate of change in energy
  is $\vF \cdot \vv$. So:

  \begin{nedqn}
    \fpartial[E]{t}
  \eqcol
    \vF \cdot \vv
  \nedcomment{definition of work}
  \\
    \fpartial[m c^2]{t}
  \eqcol
    \vv \cdot \fpartial[m \vv]{t}
  \\
  &&
  \nedcomment{assuming that change in energy changes $mc^2$}
  \\
    c^2
    2m
    \fpartial[m]{t}
  \eqcol
    2m
    \vv \cdot \fpartial[m \vv]{t}
  \nedcomment{trick}
  \\
    c^2
    \fpartial[m^2]{t}
  \eqcol
    \fpartial[(mv)^2]{t}
  \nedcomment{de-vectorize}
  \\
    c^2 m^2
  \eqcol
    m^2 v^2 + C
  \nedcomment{integrage}
  \\
    c^2 m^2
  \eqcol
    m^2 v^2 + m^2_0 c^2
  \nedcomment{define $m_0$ appropriately}
  \\
    m
  \eqcol
    \frac{
      m_0
    }{
      \sqrt{1 - v^2 / c^2}
    }
  \end{nedqn}

  What does this show? It says that if the so-called ``total energy''
  $mc^2$ of a body incorporates both its rest energy and its kinetic
  energy, then the mass of the object must evolve (in response to
  chagnges to velocity) in the way that Einstein describes.

  \item This furthermore says: if there is any question of mass being
  ``anhiliated,'' then that must imply that the reduced body will lose
  (aka emit) energy. A lot of energy.

\end{enumerate}

\subsection{Pedagogical Note}

I found this chapter very challenging. A first problem is that the
entire discussion is motivated by a conclusion of Maxwell's laws, and we
have studied no EM at this point.

The only thing we really need to know from Maxwell is that light speed
is constant in the aether. But we have never studied any waves. So we
don't know why propagation through a medium should be constant.

Further, the Michelson-Morley experiment is based on detecting whether
some light waves are in or out of phase. But we don't know anything
about phase.

He not only talks about space/time dilation. He also talks about changes
in relativistic mass. I suppose this is important, because the reason to
bring up relativity in the context of our mechanics discussion must
relate to mass/inertia/force. But the horse is before the cart - he
doesn't derive or really justify the relativistic mass formula.
Presumably that happens in the next chapter.

He does a final confusing thing: he ``derives'' the relativistic mass
formula from the concept of total energy $mc^2$. But this assumes that
total energy ``is a thing'' and incorporates both rest energy and
kinetic energy. Where does total energy come from? The relativistic mass
formula! We literally have no intuition for this concept of total energy
except that it is implied by the relativistic mass formula. So how can
it build our intuition for the relativistic mass formula??

I think the student is justified to be frustrated by this chapter.
