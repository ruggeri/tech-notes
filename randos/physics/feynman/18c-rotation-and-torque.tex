\subsection{Rotational Velocity, Momentum, And Inertia}

\begin{enumerate}
  \item Consider a point-mass moving with velocity $v$ at a distance $r$
  from a reference axis. We define \define{angular velocity} $\omega$ to
  be $\omega = \frac{\vtan}{r}$. Note: the point-mass does not need to
  be moving in a fixed circle about the reference axis for $\omega$ to
  be definable.

  \item Assume $v$ is entirely tangential. Assume $v$ is held constant.
  If so, the point-mass is at \define{perigee}: the closest it will ever
  get to the reference axis. As time goes on, $r$ will increase, and $v$
  will no longer remain purely tangential. The tangential component will
  decrease (and the radial component will increase).

  \item Assume that the point-mass travels to a new position, offset by
  $\theta$ degrees from its original perigee. The new radius is now $r'
  = \frac{r}{\cos \theta}$. With some basic geometry (see diagram), we
  can determine that the new tangential velocity is $v^{\text{TAN}'} = v
  \cos\theta$.

  \item The new angular velocity is now calculated to be:

  \begin{nedqn}
    \omega' = \frac{v \cos \theta}{r'} = v \cos^2 \theta = \omega \cos^2\theta
  \end{nedqn}

  \item No forces are present, so this is \emph{unlike} the
  translational scenario where the velocity of an object does not change
  unless a force is applied. Here, the angular velocity has changed,
  even though no force (translational or angular) is applied.

  \item We can try to save things by introducing an analogue of
  \emph{momentum}. We know that translational force really changes the
  translational momentum of an object, not simply its velocity. At
  normal speeds the distinction is moot, but at relativistic speed it
  matters.

  \item Not that relativistic speeds matter in our examples. But it's a
  good hint.

  \item We'll denote \define{angular momentum} as $L$. If this were
  simply $m\omega$, nothing would be fixed. In our example, we would
  have $L' = L \cos^2 \theta$.

  \item We need to define an analogue of \emph{translational inertia},
  which is simply the mass. In this case, we want our \define{rotational
  inertia} $I$ to be proportional to $r^2$, which will fix-up our
  problems and leave $L' = L$ when there is no force.

  \item We've shown that if an object starts with a purely tangential
  velocity, it will not experience a change in angular momentum as it
  drifts and its velocity deviates from tangential. This implies more
  generally that \emph{any} drifting body with a constant velocity, even
  if it is not tangential, will not experience a change in angular
  momentum. The reason is because we can always draw back to a
  hypothetical ``starting point'' when the velocity \emph{would have}
  been tangential.

  \item This is the ``lever arm'' definition of angular momentum, and it
  is unchanging as a body moves at constant velocity.

  \item Correspondingly, we can simply ignore any radial velocity when
  we calculate $L = I \frac{\vtan}{r}$. That is: the definition holds
  even when the velocity is not perfectly perpendicular to the radius.
\end{enumerate}

\subsection{Torque And Angular Acceleration}

\begin{enumerate}
  \item Since radial velocity is irrelevant to angular momentum, which
  is constant, we know likewise that radial acceleration does not change
  angular momentum. However, we can note that radial acceleration
  \emph{does} change $\fpartial{t} r$ and thus $\fpartial{t} I$. So the
  presence of initial radial velocity or continued radial acceleration
  will affect $\fpartial{t} \omega$.

  \item However, we can note that, if radial acceleration perfectly
  balances to result in $\fpartial{t} r = 0$, then we must have that
  $\fpartial{t} \omega = 0$. This is the \define{centripetal force}
  needed to hold the body in an orbit of radius $r$.

  \item What happens if we begin to apply a tangential force? Let us say
  that the body is made to undergo a tangential acceleration. If the
  centripetal force is held constant, then the body will begin to spiral
  outward.

  \item That would be okay, but let's stay simple for now. Let us say
  that the centripetal force will respond to any increase in tangential
  velocity so that the body continues to rotate at radius $r$. Then,
  force $\Ftan$ implies an instantaneous change in tangential velocity
  $\atan = \frac{\Ftan}{m}$.

  \item We can represent this as an \define{angular acceleration}
  $\alpha = \frac{\Ftan}{mr}$.

  \item This works okay, but if we define our ``angular force'' (which
  wea call \define{torque}) as simply $\tau = \Ftan$, we have a few
  problems. First, we want $\tau = \fpartial{t} L$, and this does not
  hold because $I \ne mr$. To fix this, we would need $\tau = \Ftan r$.
  Then, presuming that the radius $r$ is fixed, we have:

  \begin{nedqn}
    \tau = I \alpha
  \end{nedqn}

  \noindent
  This is the desired analogue to $F = ma$.

  \item Correspondingly, unless we set $\tau = Fr$, note that the work
  done by a rotation of $\theta$ radians is not $F\theta$, but in fact
  $F\theta r$.

  \item This matters if we ever hope to analyze the net effect of
  several torques applied at different points. Imagine equal but
  opposite forces $F, -F$ applied at two different radii $r,
  \frac{r}{2}$. Then a rotation by $\theta$ degrees should do work equal
  to $F \frac{r}{2} \theta$. If the torques are defined as $Fr,
  -F\frac{r}{2}$, then we can simply sum a total torque $F\frac{r}{2}$
  and multiply by $\theta$ to get that $\tau \theta$ work is done.
\end{enumerate}
