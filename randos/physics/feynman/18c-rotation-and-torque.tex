\subsection{Rotational Velocity, Momentum, And Inertia}

\begin{enumerate}
  \item Consider a point-mass moving with velocity $v$ at a distance $r$
  from a reference axis. We define \define{angular velocity} $\omega$ to
  be $\omega = \frac{\vtan}{r}$. Note: the point-mass does not need to
  be moving in a fixed circle about the reference axis for $\omega$ to
  be definable.

  \item Assume $v$ is entirely tangential. Assume $v$ is held constant.
  If so, the point-mass is at \define{perigee}: the closest it will ever
  get to the reference axis. As time goes on, $r$ will increase, and $v$
  will no longer remain purely tangential. The tangential component will
  decrease (and the radial component will increase).

  \item Assume that the point-mass travels to a new position, offset by
  $\theta$ degrees from its original perigee. The new radius is now $r'
  = \frac{r}{\cos \theta}$. With some basic geometry (see diagram), we
  can determine that the new tangential velocity is $v^{\text{TAN}'} = v
  \cos\theta$.

  \item The new angular velocity is now calculated to be:

  \begin{nedqn}
    \omega' = \frac{v \cos \theta}{r'} = v \cos^2 \theta = \omega \cos^2\theta
  \end{nedqn}

  \item No forces are present, so this is \emph{unlike} the
  translational scenario where the velocity of an object does not change
  unless a force is applied. Here, the angular velocity has changed,
  even though no force (translational or angular) is applied.

  \item We can try to save things by introducing an analogue of
  \emph{momentum}. We know that translational force really changes the
  translational momentum of an object, not simply its velocity. At
  normal speeds the distinction is moot, but at relativistic speed it
  matters. We won't be considering relativistic speeds in our discussion
  of angular motion, but this is a good hint.

  \item We'll define \define{angular momentum}, which we denote as $L$.
  If this were simply $m\omega$, nothing would be fixed. In our example,
  we would have $L' = L \cos^2 \theta$.

  \item We need to define an analogue of \emph{translational inertia},
  which is simply the mass. In this case, we want our \define{rotational
  inertia} $I$ to be proportional to $r^2$, which will fix-up our
  problems and leave $L' = L$ when there is no force.

  \item We've shown that if an object starts with a purely tangential
  velocity, it will not experience a change in angular momentum as it
  drifts and its velocity deviates from tangential. This implies more
  generally that \emph{any} drifting body with a constant velocity, even
  if it is not tangential, will not experience a change in angular
  momentum. The reason is because we can always draw back to a
  hypothetical ``starting point'' when the velocity \emph{would have}
  been tangential.

  \item This is the ``lever arm'' definition of angular momentum, and it
  is unchanging as a body moves at constant velocity.
\end{enumerate}

\subsection{Radial Velocity And Acceleration}

\begin{enumerate}
  \item As we've seen, angular momentum is defined as $L = I \omega$.
  $\omega = \frac{\vtan}{r}$ doesn't care about any radial velocity. So
  angular momentum is independent of radial velocity. Or, for that
  matter, radial acceleration.

  \item However, we can note that differences in radial velocity do
  change $\fpartial{t} r$ and thus $\fpartial{t} I$. So the presence of
  initial radial velocity will affect $\fpartial{t} \omega$.

  \item Likewise, radial acceleration can affect $\fpartialsq{t} r,
  \fpartialsq{t} I$.

  \item Let's consider the special case when radial acceleration
  perfectly balances to result in $\fpartial{t} r = 0$. This means that
  a \define{centripetal force} is forcing the object to rotate in a
  circle about the axis.

  \item We do not expect $L$ to change in this scenario, because the
  acceleration is always radial, and we said that radial velocity and
  acceleration do not affect $L$.

  \item We can verify that $\fpartial{t} L = 0$ by examining
  $\fpartial{t} \omega$. This is $\fpartial{t} \frac{\vtan}{r}$. This is
  a derivative of a ratio of two quantities. We already said that
  $\fpartial{t} r = 0$, so we need only concentrate on $\fpartial{t}
  \vtan$. This is of course zero, because the force is directly and
  always oriented at a right angle to $\vtan$.

  \item My goal here has been to build intuition that radial forces do
  not change angular momentum, though we do notice that radial forces
  \emph{do} change the radial distance and correspondingly the angular
  velocity.
\end{enumerate}

\subsection{Torque And Angular Acceleration}

\begin{enumerate}

  \item What happens if we begin to apply a \emph{tangential} force? Let
  us say that the body is made to undergo a tangential acceleration. If
  the centripetal force is held constant, then the body will begin to
  spiral outward.

  \item That would be okay, but let's stay simple for now. Let us say
  that the centripetal force will respond to any increase in tangential
  velocity so that the body continues to rotate at radius $r$. Then,
  force $\Ftan$ implies an instantaneous change in tangential velocity
  $\atan = \frac{\Ftan}{m}$.

  \item We can represent this as an \define{angular acceleration}
  $\alpha = \frac{\Ftan}{mr}$. This angular acceleration is actually
  realized, because $r$ is being held fixed.

  \item We are very close to defining the angular analogue $\tau$ of
  translation force. But simply setting $\tau = F$ will not be very
  nice.

  \item Just as $F = \fpartial{t} p$, it would be nice if analogously we
  have $\tau = \fpartial{t} L$. That is: we want our twisting force to
  relate to angular momentum exactly as translation force related to
  translational momentum.

  \item Secondly, we would like our twisting analogue of force to have
  the same relationship to \emph{work} that force does. We know that $W
  = Fd$. We want $W = \tau \theta$.

  \item The fix in both cases is the same: we want to set $\tau = Fr$.

  \item This is simply proved if we hold $r$ constant. Then:

  \begin{nedqn}
    \tau \theta
  \eqcol
    \Ftan r \theta = \Ftan d = W
  \intertext{Likewise}
    \tau
  \eqcol
    \Ftan r = \parens{m\atan} r = mr \fpartial{t} \vtan = mr \fpartial{t} r\omega
  \\
  \eqcol
    \fpartial{t} m r^2 \omega = \fpartial{t} I \omega = \fpartial{t} L
  \end{nedqn}

  \item This is nice. All the angular quantities have the appropriate
  relationship to the others. Also, note that, if $r$ is held constant,
  we have:

  \begin{nedqn}
    \tau
  \eqcol
    I \fpartial{t} = I \alpha
  \end{nedqn}

  \noindent
  Here we use $\alpha$ to denote \define{angular acceleration}. Note
  that we have achieved the desired analogue of $F=ma$.

  \item This definition is convenient when considering the aggregate
  effect of many torques $\tau_i$. You may ask: how much total work is
  done by the various torques when we perform a rotation by $\theta$
  rad? One way to calculate is $W = \sum_i \tau_i \theta$.

  \item But if we do not know the individual $\tau_i$, but only their
  aggregate, $\tau = \sum_i \tau_i$, we will still be able to calculate
  the total work performed. It is simply $\tau \theta$. Again, this is
  analogous to force.

  \item This matters when the torques are applied at different radii.
  Imagine equal but opposite forces $F, -F$ applied at two different
  radii $r, \frac{r}{2}$. Then a rotation by $\theta$ degrees should do
  work equal to $F \frac{r}{2} \theta$. If the torques are defined as
  $Fr, -F\frac{r}{2}$, then we can simply sum a total torque
  $F\frac{r}{2}$ and multiply by $\theta$ to get that $\tau \theta$ work
  is done. Whereas if we were considering the total force, we see that
  this is zero.

  \item In this section, I have always assumed that $r$ is held constant
  by an always appropriate centripetal force. However, this doesn't need
  to be assumed at all. We know that a centripetal force is irrelevant
  with regard to angular momentum. Thus we will \emph{still} have $\tau
  = \fpartial{t} L$. Note: if $\tau$ is applied for $t$ seconds, while
  $r$ is allowed to vary, we have $\Delta L = \tau t$. But note that
  since $r$ is changing, the tangential force $\Ftan$ may need to vary
  over time to maintain the same $\tau$.
\end{enumerate}
