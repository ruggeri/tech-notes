\subsection{Rotation and Torque}

\begin{enumerate}
  \item Firstly, we relate \define{angular velocity} $\omega$ and
  \define{angular acceleration} $\alpha$ to the tangential velocities
  and accelerations $v, a$. It depends on the distance to the reference
  axis. If the point is at a distance $r$ from the axis, then then
  angular velocity is $\omega = \frac{v}{r}$.

  \item We want to define analogues to $F, M$ that will describe angular
  acceleration. Since $\alpha = \frac{a}{r}$, we \emph{could} define a
  concept of \define{rotational force} also called \define{torque} as
  $\tau = F$. We would have to define the \define{rotational inertia} as
  $I = rm$.

  \item This would be acceptable if we only wanted to consider a point
  mass, but it would not do if we wanted to consider multiple torques
  applied to a composite object. For instance, consider a for $F$
  applied at a distance $r$, and a force $-F$ applied at a distance
  $\frac{r}{2}$. Consider a rotation by $\theta$ radians:

  \begin{nedqn}
    W
  =
    \parens{F r\omega} - \parens{F \frac{r}{2} \omega}
  =
    F \frac{r}{2} \omega
  \end{nedqn}

  \item That is: work is done by a rotation of $\omega$ radians, thus
  the object will want to rotate when subjected to these torques. This
  defies a simple analysis of $F + -F = 0$. So if we want the sum of
  external torque to say something useful, then we better not define
  $\tau = F$.

  \item Of course, we see immediately that the amount of work performed
  is proportional to the sum $\sum_i F_i r_i$. Thus, it will be useful
  to define $\tau_i = F_i r_i$, and $\tau = \sum_i \tau_i$ will be
  meaningful.

  \item This suggests that we should define rotational inertia as $I =
  r^2 m$, so that an extra term of $r$ will cancel out the $r$ in the
  formula $\tau = F r$. By our choice of definitions, for a point mass,
  we have $\tau = I \alpha$. Naturally, if the torque is maintained for
  time $t$, then $\tau t = I \Delta \omega$.

  \item Corresponding to translational momentum, we may of course define
  \define{angular momentum} which is $L = I \omega$. We know that $\tau
  = \fpartial{t} L$.

  \item Note that I haven't really said very much here. I've just been
  defining new angular quantities in terms of old translational ones. I
  made just one clever choice: I defined torque specifically in the
  hopes that I would be able to aggregate torques in a simple but
  meaningful way. We'll show more about why in a moment!
\end{enumerate}
