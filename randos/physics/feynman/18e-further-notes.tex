\subsection{Further Notes}

% Conservation of angular momentum?
% Changing moment of inertia.

\begin{enumerate}
  \item We can consider rotation around axes that are not the center of
  mass. For instance, take a point mass $m$. Let us apply a rotating
  force on it, always perpendicular to a ``lever arm'' $r$ from a axis
  of rotation. Then our equations from before hold: the work performed
  is equal to $Fr\theta$. We might as well write $\tau = Fr$. As before,
  we know that $\tau \theta = \frac{1}{2} mr^2 \omega^2$.

  \item We may still define the rotational inertia relative to a point
  not the center of mass. This is still $I = m r^2$. We may still
  differentiate with respect to time. We will still find that the change
  in angular momentum is $\tau t = I \omega$.

  \item Newton's 3rd Law says that forces always come in equal and
  opposite pairs. Consider when net force on a system is zero. Since a
  force acting on a subobject is the instantaneous change in momentum,
  net change in momentum is zero and stays fixed. This is the
  conservation of linear momentum.

  \item We can say more. Newton's 3rd Law also ways that the opposite
  forces are directed in opposite directions \emph{along the same line
  in space}.

  For instance, imagine that two parts of the system are pushing away
  from each other. Take any reference axis, and let us translate the two
  equal forces (and opposite) forces into torques. The distance $r_1$
  from the axis to subpart 1 may not necessarily equal $r_2$. But,
  insofar as $r_1, r_2$ are different, then this means that $\vF_1,
  \vF_2$ are at different angles to the radii. The torques will come out
  the same (but opposite).

  \item This shows that a system with no net torque applied will
  maintain a constant angular momentum: \emph{about any rotational
  axis}.

  \item We may last note: the mass of an object cannot be changed
  internally, but its moment of inertia may be changed. This is what
  happens with figure skaters (who begin a spin with their legs and arms
  out and then bring them in). The same thing happens if you spin in a
  chair with your legs out and then bring them in.

  \item In these cases, the moment of inertia has been reduced. But the
  angular momentum is in fact unchanged. This is because no torque is
  applied. On the other hand, kinetic energy will go up. This is why you
  must exert force to contract your arms. You must do the work of
  pulling against the centrifugal force.
\end{enumerate}
