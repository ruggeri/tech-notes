\subsection{Conservation of Angular Momentum}

\begin{enumerate}
  \item We now start considering systems rather than just point-masses.

  \item Imagine a system where the parts are allowed to move freely.
  That is, there are no internal forces. Consider that each subpart $i$
  is subjected to an external torque of $\tau_i$. Then we know that each
  subpart $i$ will experience an instantaneous change in angular
  momentum equal to $\fpartial{t} L_i = \tau_i$.

  \item We can talk about the \define{total angular momentum} of the
  system: $L = \sum_i L_i$. From the preceding, we know that
  $\fpartial{t} L = \sum_i \tau_i$. This sum $\sum_i \tau_i$ we call the
  \define{total external torque} $\tau$.

  \item So far, this has just been definitions. But we now proceed to
  say something interesting: even if there are internal forces, the
  instantaneous change in total angular momentum remains equal to the
  net external torque.

  \item The argument is similar to our argument regarding translational
  force. Basically: whether $\tau_i$ is considered to be only the
  external torque, or if it is inclusive of internal torques, the sum
  $\sum_i \tau_i$ will come out the same regardless.

  \item We will prove this. We know that internal forces always come in
  equal and opposite pairs; we used this property when talking about
  conservation of (translational) momentum. But now it is not simply the
  force that matters: it also matters what proportion of the force is
  tangential, and at what distance from the axis of rotation.

  \item So consider an internal force pair. The forces are $\vF, -\vF$,
  but they may be applied at different distances $r_1, r_2$ from the
  axis. Thus it may easily be that $\vF r_1 - \vF r_2 \ne 0$. In that
  case, the forces, weighted by distance from the axis, don't cancel. Uh
  oh?

  \item If we are considering torque, it doesn't just matter how far
  away a force is applied. It also matters how tangential the force is.
  We'll be saved if the difference in radii is exactly compensated by
  the difference in how tangential the forces are.

  \item And this turns out to be exactly the case! Newton didn't exactly
  say, but he appears to have meant to imply that equal and opposite
  forces always act through a common line. That is: an interaction force
  is always either pulling or pushing two subparts directly toward or
  away from each other.

  \item Both forces will share a lever arm. So we know they both impart
  the same torque by our earlier proof. I have included a diagram. It
  shows a force $\vF$ applied at distance $r$ from the axis of rotation.
  The force $\vF$ is not applied tangentially; it is applied at an angle
  of $\theta$ degrees off tangential. Thus the tangential force is
  $\cosf{\theta} \norm{\vF}$ and the torque is $\tau = r \cosf{\theta}
  \norm{\vF}$.

  Feynman says that we can instead look at the ``lever arm.'' We get
  this from extending $\vF$ until it is in fact tangential. The diagram
  shows that the length of the lever arm is $r \cosf{\theta}$. So we can
  say that torque is:

  \begin{nedqn}
    \tau
  \eqcol
    r \cosf{\theta} F
  \\
  \eqcol
    \textrm{lever arm length} \cdot F
  \end{nedqn}

  \item This formulation is very helpful in proving conservation of
  angular momentum. Two equal/opposite paired forces, even if they
  aren't applied at the same location, will both share a lever arm
  (because the force pair acts along the same line). Thus we know that
  $\tau_1 = -\tau_2$.

  \item This proves that angular momentum is conserved. That is: that
  the total instantaneous change in angular momentum is equal to the net
  external torque on the object.
\end{enumerate}
