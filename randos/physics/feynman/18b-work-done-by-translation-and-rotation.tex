\subsection{Work Done By Translation and Rotation}

\begin{enumerate}
  \item Consider a set of forces $\vF_i$ acting on a body at various
  points $\vx_i$. We consider the total work performed on the body by a
  translation of $\Delta \vx$. Note that, because translation moves all
  points equally, $\Delta \vx_i = \Delta \vx$. Thus the work performed
  is:

  \begin{nedqn}
    \sum_i \vF_i \cdot \Delta \vx_i
  =
    \sum_i \vF_i \cdot \Delta \vx
  =
    \parens{\sum_i \vF_i} \cdot \Delta \vx
  =
    \vF \cdot \vx
  \end{nedqn}

  \noindent
  We call $\vF = \sum_i \vF_i$ the \define{total translational force}.

  \item Let's consider a rotation of $\theta$ degrees about a central
  point. If $\vx_i$ is $\vr_i$ distance away from the axis of rotation,
  then the magnitude of the change $\Delta \vx_i$  is equal to $r_i
  \theta$.

  Since there is no radial motion, we may ignore any part of $\vF_i$
  that is parallel to the radius $\vr_i$. For simplicity, let us assume
  that $\vF_i$ is exactly parallel to the rotation. In this case, the
  work performed is

  \begin{nedqn}
    \sum_i \vF_i \cdot \Delta \vx_i
  \eqcol
    \sum_i F_i \parens{r_i \theta}
  \\
  \eqcol
    \parens{\sum_i F_i r_i} \theta
  \\
  \eqcol
    \parens{\sum_i \tau_i} \theta
  \\
  \eqcol
    \tau \theta
  \end{nedqn}

  \noindent
  We've denoted $\tau_i = F_i r_i$ and we call it \define{torque}. The
  \define{total external torque} is $\tau = \sum_i \tau_i$. Torque
  basically just means ``rotational force.''

  \item Note that torque is relative to an axis. When solving a problem,
  you get to choose the axis. No choice of axis is wrong (torque is just
  a formalism), but some choices of axis will help you solve certain
  problems. Choosing a different axis of rotation will give you
  different torque values.
\end{enumerate}
