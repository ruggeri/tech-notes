\subsection{Total Rotational Inertia}

\begin{enumerate}
  \item This section simply puts together what we know about (1) the
  relationship between torque and angular acceleration of a point mass
  when radius $r$ is fixed and (2) the conservation of angular momentum
  in a complex system.

  \item Imagine a body that is constrained to rotate about an axis. This
  means the velocity of every particle in the body must be perpendicular
  to its radial distance from the axis. Additionally, the angular
  velocity of every particle must be an equal $\omega$.

  \item The second property defines \define{rigid bodies}, where the
  distances of the various constituents $i$ are not allowed to vary.

  \item By conservation of angular momentum, we know the total
  instantaneous change in angular momentum is:

  \begin{nedqn}
    \tau
  \eqcol
    \fpartial{t} L
  \\
  \eqcol
    \sum_i \fpartial{t} L_i
    \nedcomment{external only}
  \end{nedqn}

  \item Consider each point $i$ on the object. We've assumed that $r_i$
  does not change: that the object must rotate about this axis. (We of
  course also assume that $m_i$ does not change). And we've even assumed
  that $\omega_i$ is always equal to a common $\omega$.

  \item Then we have:

  \begin{nedqn}
    \tau
  \eqcol
    \fpartial{t} L
  \\
  \eqcol
    \sum_i \fpartial{t} L_i
    \nedcomment{external and internal}
  \\
  \eqcol
    \sum_i m_i r_i^2 \fpartial{t} \omega
  \\
  \eqcol
    \parens{\sum_i m_i r_i^2}
    \fpartial{t} \omega
  \\
  \eqcol
    \parens{\sum_i I_i}
    \fpartial{t} \omega
  \end{nedqn}

  \item Thus, by setting total $I = \sum_i I_i$, we have:

  \begin{nedqn}
    \tau
  \eqcol
    I \alpha
  \end{nedqn}

  \noindent
  for aggregate bodies. This aggregate $I$ is often called the
  \define{moment of inertia}.

  \item Of course, the preceding has assumed that total rotational
  inertia $I$ is constant. If $I$ is allowed to vary, then it may be
  that $\alpha \ne 0$ even when $\tau = 0$. For instance, this would
  happen if $\fpartial{t} I < 0$. Angular momentum would be conserved,
  as the decrease in rotational inertia is exactly offset by an increase
  in angular velocity.

  \item This is what happens when a figure skater moves their leg from
  outside to inside, reducing their rotational inertia. Their angular
  momentum remains constant (net zero torque). But because the inertia
  is decreasing, their angular velocity $\omega$ increases to
  compensate. They experience a positive angular acceleration $\alpha$.

  \item More generally, we may be applying a non-zero torque to a
  system, but not see the expected angular acceleration, provided the
  system is undergoing a simultaneous change in its rotational inertia.
\end{enumerate}

\subsection{Future Explorations}

\begin{enumerate}
  \item We have examined an aggregate body where (1) subparts are
  required to rotate about a fixed axis, and (2) subparts must all
  rotate by an equal amount $\theta$.

  \item The second requirement is tied to the property of rigid bodies:
  that all subparts must stay at a fixed distance to each other.

  \item The first part seems somewhat arbitrary. Why should we not allow
  the object to simply translate away? We will address this in the next
  section by specifying a very particular and special axis of rotation.
\end{enumerate}
