\section{Basic Physics}

\begin{enumerate}
  \item First he goes through a basic overview of the scientific method.

  \item Next he discusses a traditional pictures of physics. Space is
  three dimensional. Things change over time. Everything consists of
  arrangements of particles. Particles have properties like inertia.
  There are electrical forces that describe chemical properties. And
  there is gravitational force.

  \item He notes that the electrical force is very strong even at
  distance, but it is often ``canceled out'' in the sense that if a
  molecule (with net zero charge) is far from another, they won't have
  any real force on each other. But if two \ce{H2O} molecules for
  instance are close to each other, they will want strongly to arrange
  themselves so that the negative/positive ends are aligned
  appropriately.

  \item He makes a note for the comparrison of electrical and
  gravitational forces. Take two grains of sand. Assume that the
  repulsive electrical force was instead attractive. That is, assume
  that electrons and protons all attracted each other equally. Then the
  force of the one grain of sand on the other would be equivalent to
  \emph{three million tons}. So you can see how important the concept of
  \emph{net charge} is.

  \item The old model was that an atom contained light, negatively
  charged electrons on the outside. The nucleus was on the inside, and
  contains typically an equal number of protons. Additionally, there are
  neutrons in the nucleus. The protons and the neutrons both have
  approximately equal mass - much greater than the electrons.

  \item Instead of saying that protons cause a force on electrons, we
  might say that protons cause an \emph{electrical field} in space. When
  a particle like an electron is placed in the field, it will have a
  force acted upon it.

  \item He gives one reason why we might want to describe things this
  way. He says that \emph{oscilating} a charged object, we can affect
  another charged object \emph{much farther away} than we would expect
  if we only spoke about direct interactions. Somehow the oscillation is
  creating a wave in a field, and this can cause some kind of stronger
  effect. I'm not sure I understand at this point, but I don't know if
  this is an essential point yet.

  \item I believe this has to do with \emph{near} and \emph{far} field
  effects of electromagnetism. I don't understand these at all, but of
  course I expect they will be discussed in due course...

  \item He notes that oscilating waves of charge are used for sending
  power, for transmitting radio, and even, if fast enough, we call it
  ``light'' and we can see it. X-rays and gamma rays are even higher
  frequency. Even higher frequency are cosmic rays. But it's all the
  same thing: these are just electromagnetic waves oscillating at
  different frequencies.

  \item He then talks a little about quantum physics. Here, the
  ``stage'' of physics is not space which changes over time, but
  \emph{space-time}; they are actually both related. He mentions that
  many laws of classical physics are not quite correct in quantum
  physics.

  \item He mentions the uncertainty principle: that $\Delta x \Delta p
  \geq \hslash/2$. He notes that this explains (a) why the atom is so
  big (because otherwise we would know the position of its
  constituents), and (b) whhy absolute zero still involves wiggling.

  \item He mentions that quantum theory doesn't precisely specify what
  will be observed, but only specifies probabilities of observations.

  \item He describes that quantum theory explains \emph{both} wave
  behavior and particle behavior. That is: when we do oscilation of
  charge, that seems to induce an electromagnetic wave. But when we
  oscilate really really fast, our equipment gives a result that looks
  more like what we would get if there were particles. In fact, the
  quantum theory explains both kinds of observations. This is why we
  speak of \emph{photons}.

  \item He mentions quantum electrodynamics: this is the theory of
  electromagnetism made ``quantum dynamically correct.''

  \item He then begins talking about what might hold the nucleus
  together. What are the forces there? He then just barfs a bunch of
  random crap about elemetary particles. I think it is impossible for
  anyone to understand this section at this point, but that is fine.
  He's just pointing out that nuclear forces are not that well
  understood I guess.
\end{enumerate}
