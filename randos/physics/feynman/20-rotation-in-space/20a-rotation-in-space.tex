\subsection{Vector definitions}

\begin{enumerate}
  \item He starts out by noting that everything we learned about torque
  in two dimensions still holds in a two-dimensional plane in space. He
  suggests that we can define analogues of angular momentum ($L_{\vi},
  L_{\vj}, L_{\vk}$) and torque ($\tau_{\vi}, \tau_{\vj}, \tau_{\vk}$).

  \item But then he asks: how can the torque in these three particular
  planes tell us about the torque in some other, fourth plane? He lets
  $\vk$ stay fixed, but rotates $\vi, \vj$.

  Now, we know that, 2-dimensional style, angular momentum in the $\vi,
  \vj$ plane is radius times angular speed. But if we rotate between
  $\vi, \vj$ to produce $\vi', \vj'$, we don't change the radius to the
  point mass, when projected into the $\vi, \vj$ plane.

  We can likewise consider torque: (1) project the radius $\vx$ to the
  point mass into the $\vi, \vj$ plane, (2) project the force $\vF$ into
  the $\vi, \vj$ plane, (2) find the amount perpendicular of (2) that is
  perpendicular to (1), and (3) multiply magnitudes of (1) and (2).

  It's clear that a rotation between $\vi, \vj$ does not affect this
  quantity.

  \item That is: angular momentum and torque in a plane defined by
  perpendicular $\vx, \vy$ are invariant under a rotation that does not
  lift $\vx, \vy$ out of the plane defined by the two of them.

  \item I jump ahead for a moment: we may define $\vL = \vr \cross \vp$
  and $\vtau = \vr \cross \vF$. Because the cross product has been
  defined to produce a vector with:

  \begin{nedqn}
    \norm{\vr \cross \vp}
  \eqcol
    \abs{\cos\theta} \norm{\vr} \norm{\vp}
  \end{nedqn}

  \noindent
  where $\cos$ is the angle between the two vectors, we know we have the
  right magnitude. With regard to magnitude, we know that this will
  imply that a force acting purely in the $\vx, \vy$ plane will produce
  a torque about the $\vz$ axis, which is how to interpret the vector
  nature of this definition.

  We know the direction of the angular momentum/torque is always
  ``correct,'' because it is consistently oriented perpendicular to the
  plane under consideration.

  \item If you rotate the basis with a matrix $\mtxR$, then we have:
  $\vr' = \mtxR \vr$, $\vp' = \mtxR \vp$. Then:

  \begin{nedqn}
    \vL'
  \eqcol
    \vr' \cross \vp'
  \\
  \eqcol
    \parens{\mtxR\vr} \cross \parens{\mtxR\vp}
  \\
  \eqcol
    \mtxR \parens{\vr \cross \vp}
  \end{nedqn}

  \noindent
  You could show this with algebra, but it's easier to see that this
  flows from the geometric definition of cross product that the
  algebraic definition was defined to achieve.

  If we allow any orthonormal matrix $\mtxQ$ to be used, we have to be a
  little careful about flips, which should affect the sign of the
  result.

  \item But you could say that this has only defined the angular
  momentum and torque in the plane defined by $\vr$ and $\vp$ (or $\vr,
  \vF$). Okay. Then consider any $\vz$. Consider $\vL \cdot \vz$. I say
  that is equivalent to nulling out any part of $\vr, \vp$ that is
  parallel to $\vz$.

  That's because any portion of $\vr$ parallel to $\vz$ will always
  yield $\parens{\proj_{\vz} \vr} \cross \vp$ which is
  \emph{perpendicular} to $\vz$, by definition of the cross product, no
  matter what $\vp$ is. Likewise we can ignore any portion of $\vp$
  which is parallel to $\vz$.

  Thus, we are saying that $\vL \cdot \vz$ is equivalent to projecting
  into the $\vx, \vy$ plane, where $\vx, \vy$ are orthogonal vectors
  giving $\vz = \vx \cross \vy$.

  \item This means that $\vL$ \emph{does} tell us the angular momentum
  in an arbitrary plane defined by $\vx, \vy$. We simply use $\vL \cdot
  \parens{\vx \cross \vy}$. This even works if $\vx, \vy$ are not
  orthonormal; we simply correct by dividing by $\norm{\vx \cross \vy}$.
\end{enumerate}

\subsection{Vectorized Rotation Laws}

\begin{enumerate}
  \item We've already defined:

  \begin{nedqn}
    \vtau
  \eqcol
    \vr \cross \vF
  \\
    \vL
  \eqcol
    \vr \cross \vp
  \intertext{and we also know}
    \vtau
  \eqcol
    \fpartial{t} \vL
  \end{nedqn}

  \item We previously defined $\omega = \frac{v}{r}$. Now, we may
  generalize and say that

  \begin{nedqn}
    \vomega
  \eqcol
    \frac{\vr \cross \vv}{\norm{\vr}}
  \intertext{equivalently}
    \vomega \cross \vr
  \eqcol
    \frac{\vr \cross \vv}{\norm{\vr}} \cross \vr
  =
    \vv
  \intertext{Thus we may argue:}
    \vL
  \eqcol
    \vr \cross \vp
  \\
  \eqcol
    \vr \cross \parens{m \vv}
  \\
  \eqcol
    m \vr \cross \parens{\vomega \cross \vr}
  \\
  \eqcol
    m \normsq{\vr} \vomega
  \end{nedqn}

  \item We know that, if an appropriate centripetal force $\vF$ is
  directed opposite to the radius $\vr$, we can hold both $\vr$ and
  $\vomega$ constant. In this case, the point-mass will truly be
  rotating about the axis defined by $\vomega$, with angular velocity
  $\norm{\vomega}$, and at distance $\norm{\vr}$ from the origin.

  \item Consider momentarily that $m = 1$, $\vr = \vi + \vk$, while $\vv
  = \vj$. Then $\vL = \vk - \vi$. If $\normsq{\vr}$ is held fixed at
  $\normsq{\vr} = 2$, then $\vomega$ will be held fixed as $\vomega =
  \half \parens{\vk - \vi}$.

  \item But we could also apply a force that rotates the mass parallel
  to the $\vi, \vj$ plane. In that case, the $\vk$ term of $\vr$ is held
  constant. Note that we will \emph{still} have that $\normsq{\vr}$ is
  held constant.

  However, in this case, $\vomega$ will \emph{not} be held constant.
  This will also rotate parallel to the $\vi, \vj$ plane: the $\vk$
  component will always be fixed. This is simply a result of rotational
  symmetry.

  Thus, because symmetry, we will have $\normsq{\vr}$ held constant, as
  well as $\norm{\vomega}$. Thus $\norm{\vL}$ will also be held constant.
  So the direction of $\vL$ will spin about, but its magnitude will not
  change.

  We verify this by noting that we need to apply a centripetal force
  $-\vi$ to spin a unit mass at $\vi + \vk$ about the $\vk$ axis. This
  results in a torque $\parens{\vi + \vk} \cross -\vi = -\vj$. This is,
  unsurprisingly, perpendicular to $\vL$.

  \item This is a new phenomenon in the 3-dimensional case. In the
  two-dimensional world, a particle subjected to a torque must
  experience a magnitude change of $\vomega$. In three dimensions, a
  torque $\vtau$ can be applied to rotate $\vL$ and $\vomega$ without
  changing their magnitude.

  \item Let's now try to break angular momentum up into two terms: (1) a
  term that reflects the position and velocity of a body's center of
  mass and (2) a term that reflects the rotation of the body about its
  center of mass. We did this before, but I want to repeat this more
  carefully:

  \begin{nedqn}
    \vL
  \eqcol
    \sum_i \vx_i \cross \vp_i
  \\
  \eqcol
    \sum_i \parens{\vx_\CoM + \vr_i} \cross \vp_i
  \\
  \eqcol
    \parens{
      \sum_i \vx_\CoM \cross \vp_i
    }
    + \parens{
      \sum_i \vr_i \cross \vp_i
    }
  \\
  \eqcol
    \parens{
      \vx_\CoM \cross \vp
    }
    + \parens{
      \sum_i \vr_i \cross \parens{m_i \vv_i}
    }
  \\
  \eqcol
    \vL_\CoM
    + \parens{
      \sum_i m_i \vr_i \cross \parens{\vomega_i \cross \vr_i}
    }
  \end{nedqn}

  \item As before, the first term represents the angular momentum that
  derives from the total mass of the object and its velocity. The second
  term is the angular momentum derived from the motion of the
  constituents about the center of mass.

  \item We want to consider the case of a rigid body. In this case,
  every point of the body is undergoing a rotation about an axis defined
  by $\vomega$. In the two dimensional case, we were able to assume that
  $\vomega_i$ was everywhere equal to $\vomega$, and orthogonal to
  $\vr_i$. This let us say $\vr_i \cross \parens{\vomega_i \cross \vr_i}
  = \normsq{\vr_i} \vomega$.

  \item But things are just a bit more tricky in three dimensions. If we
  naively set $\vomega_i = \frac{\vr_i \cross \vv_i}{\norm{\vr_i}}$,
  then note that $\vomega_i$ is defined perpendicular to $\vr_i$. But
  with three dimensions, not all $\vr_i$ lie in the same plane, and thus
  not all $\vomega_i$ will be aligned!

  Note that the \emph{sum} will come out right. The equation above is
  correct. But we will not easily simplify it in terms of a $\vomega$
  for the aggregate body.

  \item The point is this: a velocity $\vv_i$ is compatible with a
  rotation about any axis perpendicular to $\vv_i$. Working in three
  dimensions, this leaves a two-dimensional subspace of possibilities
  for the ``true'' $\vomega_i$. The velocity of a single particle cannot
  by itself tell us the angular velocity of the object.

  \item But let us say that a rigid body truly is rotating about an axis
  $\vu$ (with norm 1) with angular speed $\omega$. Or simply: it's
  rotating with angular velocity $\vomega$. Then we can decompose
  $\vr_i$ into parallel and perpendicular components:

  \begin{nedqn}
    \vr_{\parallel, i}
  \eqcol
    \parens{\vr \cdot \vu} \vu
  \\
    \vr_{\perp, i}
  \eqcol
    \vr_i - \vr_{\parallel, i}
  \end{nedqn}

  \item We may also decompose $\vv_i$ into terms of $\vomega$. I claim
  that $\vomega \cross \vr_{\perp, i} = \vv_i$. Since the body is
  undergoing pure rotation about $\vu$, we know that its velocity must
  be perpendicular to both $\vu$ and $\vr_{\perp, i}$, the projection of
  its radius into the plane perpendicular to $\vu$. Thus $\vv_i$ has the
  direction corresponding to $\vu \cross \vr_{\perp, i}$. Additionally,
  $\vv_i$ should correspond to an angular speed of $\omega$, which
  correction is achieved by $\vomega \cross \vr_{\perp, i}$.

  We may also note that $\vomega \cross \vr_{\parallel, i} = \vec{0}$.
  This is because the vectors are parallel. It makes sense that this
  component can be ignored for velocity purposes, because
  $\vr_{\parallel, i}$ is left unchanged by a pure rotation about
  $\vomega$.

  \item Having decomposed both $\vr_i, \vv_i$, we proceed to keep
  hacking:

  \begin{nedqn}
    \vL
  \eqcol
    \sum_i
      m_i
      \parens{\vr_{\perp, i} + \vr_{\parallel, i}}
      \cross
      \parens{\vomega \cross \vr_{\perp, i}}
  \\
  \eqcol
    \sum_i
      m_i
      \normsq{\vr_{\perp, i}}
      \vomega
    +
    \sum_i
      m_i
      \vr_{\parallel, i}
      \cross
      \parens{\vomega \cross \vr_{\perp, i}}
  \end{nedqn}

  \noindent
  We can see that if $\vu$ and $\vomega$ are parallel to $\vL$ (as we
  expect), then the first term is parallel to $\vL$, while the second
  term is \emph{perpendicular} to $\vL$. Since the terms must sum to
  $\vL$, then the second term must be zero and can be ignored.

  \item In fact, we can show the second term must be zero, provided that
  we properly picked $\vomega$ to explain entirely the tangential
  velocity $\vv_i$ at every point. \ldots
\end{enumerate}
