\section{Space-Time}

\newcommand{\ttilde}{\tilde{t}}
\newcommand{\vtilde}{\tilde{v}}

\begin{enumerate}

  \item We call a position in a \define{space-time} $(x, y, z, ct)$ an
  \define{event}. I also call it a \define{four-vector}. It turns out
  that $ct$ is a more convenient set of units to use.

  \item I shall tend to write $\ttilde = ct$ and $\vtilde = v/c$. Of
  course we always have $\vtilde < 1$.

  \item We may now more conveniently write:

  \begin{nedqn}
    x'
  \eqcol
    \frac{x - \vtilde \ttilde}{\sqrt{1 - \vtilde^2}}
  \\
    \ttilde'
  \eqcol
    \frac{
      \ttilde - \vtilde x
    }{
      \sqrt{1 - \vtilde^2}
    }
  \end{nedqn}

  \item This \emph{almost} looks like a rotation. However, it is not
  quite. We shall define a distance metric for space-time. We will
  expect that Euclidean transformations (which effect only space) will
  not affect our metric. That is: Euclidean transformations ought to be
  \define{isometries} of our space.

  The Euclidean transformations are flips, translations, and rotations.
  They don't touch the time dimension.

  \item We also want to change a frame of reference by changing the
  velocity of the observer. We know that we cannot do a Galilean
  transformation and leave things at that. We must do a Lorentzian
  transformation. Only that will ensure that the speed of EMR
  propagation is equal in all frames (as we observe experimentally).

  \item It turns out that there is a single metric with respect to which
  both Euclidean and Lorentzian transformations are isometries:

  \begin{nedqn}
    d(\vmu_1, \vmu_2)
  \eqcol
    \sqrt{
      \parensq{\Delta \ttilde}
      - \parensq{\Delta x}
      - \parensq{\Delta y}
      - \parensq{\Delta z}
    }
  \end{nedqn}

  Note that the distance metric is written entirely in terms of
  $\Delta$s so we can use it as a norm.

  \item We may verify that Lorentzian transformations preserve distance
  between four-vectors. Note:

  \begin{nedqn}
    \ttilde' - x'^2
  \eqcol
    \parensq{
      \frac{
        \ttilde - \vtilde x
      }{
        \sqrt{1 - \vtilde^2}
      }
    }
    -
    \parensq{
      \frac{
        x - \vtilde \ttilde
      }{
        \sqrt{1 - \vtilde^2}
      }
    }
  \\
  \eqcol
    \parensinv{
      1 - \vtilde^2
    }
    \parens{
      \parensq{\ttilde - x \vtilde}
      -
      \parensq{x - \vtilde \ttilde}
    }
  \\
  \eqcol
    \parensinv{
      1 - \vtilde^2
    }
    \parens{
      \ttilde^2 - 2x \ttilde \vtilde + x^2 \vtilde^2
      -
      x^2 + 2x \ttilde \vtilde - \ttilde^2 \vtilde^2
    }
  \\
  \eqcol
    \parensinv{
      1 - \vtilde^2
    }
    \parens{
      \parens{1 - \vtilde^2} \ttilde^2
      -
      \parens{1 - \vtilde^2} x^2
    }
  \\
  \eqcol
    \ttilde^2 - x^2
  \end{nedqn}

  This shows that there will be no change in norm by applying a
  Lorentzian transformation. Of course, a Galilean transformation would
  not respect this metric.

  \item The metric for space-time defines the \emph{geometry} of
  space-time. When we change our velocity, it \emph{looks} like we have
  warped the geometry of space. For instance, we know that distances get
  stretched. But if we think of space-time \emph{as a whole}, and we
  think outside the box of our spatially-focused geometry, we might see
  that there is no \emph{overall warping} of space-time.

  \item There is something weird about this metric. What's
  $\norm{\parens{x, y, z, 0}}$? It's imaginary!

  In general, the norm of a four-vector at time $t$ will be real if $ct
  > \norm{\parens{x, y, z}}$. That is: only if the four-vector lies
  within the \define{light-cone} centered at the origin.

  \item The \define{affective past} consists of four-vectors within the
  light cone where $t < 0$. The \define{affective future} consists of
  four-vectors within the light cone where $t > 0$.

  Note: points in space-time that are outside our present light-cone
  will \emph{eventually} come to be in our light-cone. The light cone
  always grows with time. But points outside our present light-cone are
  not part of our ``now.''

  \item Under rotation in 3D space, the momentum vector will change
  (direction; not magnitude). But the momentum is in a sense invariant
  in that it will undergo the same rotation that was applied to the
  space.

  The three momentum doesn't have enough information to update under a
  change of inertial reference frame. Momentum doesn't tell you mass or
  velocity; it tells you a mix of those things. But without knowing
  \emph{both} those things, you can't know how to update the momentum -
  either under a Galilean or Lorentz transformation! That is: forgetting
  relativity, knowing the current momentum and the change in frame
  velocity doesn't give enough information to recalculate the new
  momentum.

  This is easiest to see if we consider an object with zero momentum.
  Clearly it is at rest, but what is its mass? Without knowing its mass,
  we can't know its momentum when we change the frame by a velocity of
  $v$.

  \item The relativistic version of momentum is called
  \define{four-momentum}. It is: $(p_x, p_y, p_z, m)$.

  They sometimes say that the ``mass'' component of this vector is
  really an ``energy'' component. This follows from $E = mc^2$. So
  really if we correct by a factor of $c^2$ we can consider the mass as
  energy. In which case the vector represents momentum and energy.

  \item No matter the frame, the space-time norm of this vector is
  always invariant:

  \begin{nedqn}
    \normsq{
      \parens{p_x, p_y, p_z, m c^2}
    }
  \eqcol
    m^2 c^4 - \parensq{m \norm{v}}
  \\
  \eqcol
    \parens{c^2 - \normsq{v}}
    m^2 c^2
  \\
  \eqcol
    \parens{c^2 - \normsq{v}}
    \parensq{
      \frac{m_0 c}{\sqrt{1 - \normsq{v} / c^2}}
    }
  \\
  \eqcol
    \parens{c^2 - \normsq{v}}
    \parensq{
      \frac{
        m_0 c^2
      }{
        c\sqrt{1 - \normsq{v} / c^2}
      }
    }
  \\
  \eqcol
    \parens{c^2 - \normsq{v}}
    \parensq{
      \frac{
        m_0 c^2
      }{
        \sqrt{c^2 - \normsq{v}}
      }
    }
  \\
  \eqcol
    \parensq{m_0 c^2}
  \end{nedqn}

  \item This shows that the space-time norm is always the square of the
  rest energy. This is summed up by this equation: $E^2 - p^2 = E_0^2$.

  Note: the momentum or energy of an object is \emph{not} invariant
  under a change of inertial frame. Four-momentum \emph{is} invariant,
  and in that sense itt is more ``real.''

  However, energy and momentum (that is - four-momentum) \emph{is}
  conserved in interactions in a constant reference frame. In a given
  frame mass is never destroyed, and energy is never lost.

  In this sense, four-momentum is more real than momentum or energy
  alone.

  \item \TODO{Show this evolves Lorentz style?}.

  Consider $(0, 0, 0, m_0)$. As you do a Lorentz transform on space, so
  you do the corresponding transform on this vector.

  \item Last, Feynman discusses a little about photons. Photons travel
  at $c$ in every rest frame.

  To prove this, remember our formula for relativistic change in
  velocity:

  \begin{nedqn}
    c'
  \eqcol
    \frac{
      c - u
    }{
      1 - uc / c^2
    }
  \\
  \eqcol
    \frac{
      c - u
    }{
      c \parens{c - u}
    }
  \\
  \eqcol
    c
  \end{nedqn}

  Photons have zero rest mass. Feynman says photons have energy
  \emph{and they have momentum}. In fact, these are equal!

  By possessing momentum, this implies that photons have (relativistic)
  mass. Moreover, by saying energy equals momentum, it implies the rest
  mass of a photon must be zero.

\end{enumerate}
