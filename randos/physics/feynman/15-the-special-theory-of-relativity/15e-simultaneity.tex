\subsection{Simultaneity}

\begin{enumerate}

  \item Looking at the Lorentzian transform, we see that when
  transforming $t$ to $t'$, we must first subtract $ux/c^2$. That means
  that the change in time depends on the $x$ coordinate.

  \item Imagine we put two photo-receptors at equal distances $L$ from a
  central photo-emitter. We fire two photons from the emitter; we expect
  them to arrive at both receptors simultaneously.

  \item But what if the apparatus is moving? Then a ``stationary''
  observe sees the photon arrive at the leading receptor at time
  $\frac{L}{c - u}$. The trailing receptor is observed to receive the
  photon at $\frac{L}{c + u}$ (earlier).

  \item We can compare the times:

  \begin{nedqn}
    \Delta t
  \eqcol
    \frac{L}{c - u}
    -
    \frac{L}{c + u}
  \\
  \eqcol
    \frac{L(c + u) - L(c - u)}{c^2 - u^2}
  \\
  \eqcol
    \frac{2Lu}{c^2 - u^2}
  \\
  \eqcol
    \frac{\Delta x u/c^2}{1 - u^2 / c^2}
  \end{nedqn}

  \item Notice $\Delta x = \Delta x' \sqrt{1 - u^2/c^2}$. So if we work
  that into the equation, we have:

  \begin{nedqn}
    \Delta t
  \eqcol
    \frac{\Delta x u/c^2}{1 - u^2 / c^2}
  \\
  \eqcol
    \frac{\Delta x' u/c^2}{\sqrt{1 - u^2 / c^2}}
  \end{nedqn}

  This agrees with the offset required by a Lorentzian transformation.

  \item Two observers can see events as occuring in an opposite order.
  Imagine that the apparatus is stationary, but two different observers
  are approaching it from opposite directions at equal velocities $u$.
  Then the order of the two events are reversed.

  \item I won't prove, but I believe that if A sends a pulse to B and C,
  which upon receipt sends a pulse to C, then I believe that A's message
  must always arrive to C before B's. This means observed reordering
  cannot violate causal ordering.

\end{enumerate}
