\subsection{Relativistic Energy}

\begin{enumerate}

  \item Feynman asks us to consider a gas. Heat the gas. The mass is:

  \begin{nedqn}
    m
  \eqcol
    \frac{m_0}{\sqrt{1 - v^2/c^2}}
  \end{nedqn}

  Here $v$ is the speed of the gas molecules. He notes that through an
  application of the binomial theorem, this is:

  \begin{nedqn}
    m
  \eqcol
    m_0 \parens{
      1 + \frac{1}{2} v^2 / c^2 + \frac{3}{8} v^4/c^4 + \cdots
    }
  \end{nedqn}

  Dropping all the smallest terms, we have:

  \begin{nedqn}
    m
  \approxcol
    m_0 + \half m_0 v^2 / c^2
  \end{nedqn}

  But this is

  \begin{nedqn}
    m
  \approxcol
    m_0 + \mathit{KE} / c^2
  \end{nedqn}

  Which is to say: the change in the mass is equal to the change in
  kinetic energy (divided by $c^2$). Or:

  \begin{nedqn}
    \Delta \mathit{KE}
  \approxcol
    \Delta m c^2
  \end{nedqn}

  \item Instead of speaking in terms of mass, Eistein started talking in
  terms of energy. He said that if $m_0$ is the \define{rest mass}, then
  $m_0 c^2$ is the \define{rest energy}. Likewise we have:

  \begin{nedqn}
    mc^2
  \eqcol
    m_0 c^2 + \half m_0 v^2 + \cdots
  \end{nedqn}

  Here we have the \define{total energy} of the body being equal to its
  (a) rest energy, (b) its kinetic energy, (c) some other junk.

  Feynman does some math. He says: let's say that $mc^2$ is the total
  energy. Then if we do work on the object, the rate of change in energy
  is $\vF \cdot \vv$. So:

  \begin{nedqn}
    \fpartial[E]{t}
  \eqcol
    \vF \cdot \vv
  \nedcomment{definition of work}
  \\
    \fpartial[m c^2]{t}
  \eqcol
    \vv \cdot \fpartial[m \vv]{t}
  \\
  &&
  \nedcomment{assuming that change in energy changes $mc^2$}
  \\
    c^2
    2m
    \fpartial[m]{t}
  \eqcol
    2m
    \vv \cdot \fpartial[m \vv]{t}
  \nedcomment{trick}
  \\
    c^2
    \fpartial[m^2]{t}
  \eqcol
    \fpartial[(mv)^2]{t}
  \nedcomment{de-vectorize}
  \\
    c^2 m^2
  \eqcol
    m^2 v^2 + C
  \nedcomment{integrage}
  \\
    c^2 m^2
  \eqcol
    m^2 v^2 + m^2_0 c^2
  \nedcomment{define $m_0$ appropriately}
  \\
    m
  \eqcol
    \frac{
      m_0
    }{
      \sqrt{1 - v^2 / c^2}
    }
  \end{nedqn}

  What does this show? It says that if the so-called ``total energy''
  $mc^2$ of a body incorporates both its rest energy and its kinetic
  energy, then the mass of the object must evolve (in response to
  chagnges to velocity) in the way that Einstein describes.

  \item This furthermore says: if there is any question of mass being
  ``anhiliated,'' then that must imply that the reduced body will lose
  (aka emit) energy. A lot of energy.

\end{enumerate}
