\subsection{Michelson-Morley Experiment}

\begin{enumerate}

  \item Let's describe the Michelson-Morley experiment. They decided to
  do an experiment that would use the invariant speed of light in aether
  to detect the velocity in an inertial reference frame. They were
  assuming that a moving body experiences a Galilean transformation of
  its coordinate space.

  Here's what they did. They would shine a light on a partially
  reflective mirror angled at 45 degrees. Some light would pass through
  the mirror, and some would reflect off. In both cases, there are
  (fully reflective) mirrors to reflect the light \emph{back at the
  mirror}. The mirrors are set at an equal distance $L$ from the initial
  partially-reflective mirror.

  When the light returns to the partially reflective mirror, it has
  traveled a distance $2L$. It doesn't matter which path. Therefore we
  expect the two beams of light to be ``in phase.'' I think they
  invented the interferometer to detect this.

  \item The key to the experiment comes next. They next put the aparatus
  at a velocity $u$. The velocity is straight at one of the mirrors, but
  orthogonal to the other. The distance traveled by the light is no
  longer $2L$ (for either beam).

  BTW: they didn't move the aparatus. Rather, they aligned it with the
  Earth's direction of travel about the Sun. They did the experiment and
  recorded the interference pattern. They then rotated the experiment
  90\degree and repeated. They expected to find a change in interference
  pattern. These are historical details; they're irrelevant to the
  thought experiment.

  \item The beam reflected orthogonally needs to travel a distance not
  only $L$, but also some transverse distance. If the hypotenuse of the
  corresponding right triangle is $y$, and the transverse distance is
  $x$, then we have:

  \begin{nedqn}
    2\frac{y}{c}
  \eqcol
    2\frac{x}{u}
  \\
    x
  \eqcol
    \frac{yu}{c}
  \end{nedqn}

  And now we can substitute into the Pythagorean formula:

  \begin{nedqn}
    y^2
  \eqcol
    x^2 + L^2
  \\
  \eqcol
    \parens{\frac{yu}{c}}^2 + L^2
  \\
    \parens{1 - u^2/c^2}y^2
  \eqcol
    L^2
  \\
    2y
  \eqcol
    \frac{2L}{\sqrt{1 - u^2/c^2}}
  \end{nedqn}

  \item On the other hand, consider the light traveling along the path
  of the velocity. The first part of its journey takes time $\frac{L}{c
  - u}$, while the second part takes time $\frac{L}{c + u}$. We find:

  \begin{nedqn}
    c\parens{\frac{L}{c - u} + \frac{L}{c + u}}
  \eqcol
    c\frac{
      L(c + u) + L(c - u)
    }{
      (c + u)(c - u)
    }
  \\
  &&
  \nedcomment{factor of $c$ because we want distance not time}
  \\
  \eqcol
    c\frac{2Lc}{c^2 - u^2}
  \\
  \eqcol
    \frac{2L}{1 - u^2/c^2}
  \end{nedqn}

  \item We now see that the distance traveled by the light is different
  in the two scenarios. Thus we expect that the light will be out of
  phase when it returns to the half-mirrored plate. \emph{But that is
  not what was observed.} Michelson and Morley found that the light
  remained in phase. This is a null result: it was \emph{not} what would
  have been expected from Maxwell's laws under a Galilean
  transformation.

  \item This failure to detect a change in the interference pattern was
  a \emph{preservation} of the principle of relativity. But it didn't
  make much sense with respect to their interpretation of Maxwell.

\end{enumerate}
