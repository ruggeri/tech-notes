\subsection{Time Dilation}

\begin{enumerate}

  \item A Lorentzian transformation also suggests that time is slower in
  reference frames at velocity. Can we see why this correction is
  necessary?

  \item Consider the transverse arm of the Michelson-Morley aparatus.
  The stationary distance from the half-silvered mirror to the full
  mirror and back is $2L$. But if the aparatus is moving at velocity
  $u$, the distance light must travel to return was calculated to be
  $\frac{2L}{\sqrt{1 - u^2/c^2}}$.

  \item If moving along with the aparatus, each photon that travels from
  the half-silvered mirror to the full mirror and back appears to take
  time $\frac{2L}{c}$. But if not moving with the aparatus, that
  round-trip time is shrunk (rather, stretched) by a factor of $\sqrt{1
  - u^2/c^2}$. This is implied if light propagates at the same rate $c$
  in all reference frames, which is necessary if we are trying to
  reconcile Maxwell with relativity.

  \item That implies that processes in a space-ship at velocity $u$
  would appear to be occuring more slowly when observed by a
  ``stationary'' observe.

  \item Our example assumes that there is no shrinkage of space
  perpindicular to the direction of travel. Why should there not be?

  Imagine two equally sized hoola hoops with paper on the inside. One of
  these is our rocketship, while the other is ``stationary.'' We crash
  them into each other.

  If one of the hoola hoops had shrunk perpindicular to the travel, then
  it would ``fit inside'' the other. This is detectable because the
  paper inside the smaller hoop is fine, while the larger hoop's paper
  is smashed. That would violate relativity. We could see who is the
  ``true'' mover.

  \item Feynman gives a practical example. Muons live a very short time,
  and therefore they ought not be able to travel very far. But they are
  detected at much greater distances than they should be expected to
  live for. How?

  He suggests: muons are typically travelling so fast that the short
  time it takes to disintegrate actually appears to be quite long from a
  ``stationary'' observer. Thus the muon can travel a great distance
  from the perspective of the stationary observer.

  Muons will live for ``longer'' if traveling at higher velocities. But
  if you back that relativistic factor out, you get a consistent
  estimate of their ``true'' lifespan. This has been done at various
  muon velocities.

\end{enumerate}
