\section{Energy}

\subsection{Kinetic Energy}

We've already discussed momentum. This talks about how much force an
object can sustain, and for how long, before it comes to rest.

We can also ask for what \emph{distance} will an object be able to
sustain a force before coming to rest. This is called \emph{kinetic
energy}. The measure of kinetic energy is naturally the Newton meter:
\si{Nm}. This unit is also called the \emph{Joule}: \si{J}.

When calculating kinetic energy, the relevant factors are: (1) the mass
of the object, and (2) the velocity of the object. The kinetic energy
ought to be proportional to the mass: if we double the mass of the
object, we must double the force exerted to maintain the same
deceleration.

How do kinetic energy and velocity relate? Well, imagine that it takes a
constant force of $F$ exerted for $d$ meters to stop an object with mass
$m$ and velocity $v$. We now ask, for what distance must $F$ be exerted
to stop the same object with velocity $\alpha v$?

We know that it must now take $\alpha t$ seconds to stop the object. We
know that from our discussion of momentum. But how much distance is
that?

\begin{nedqn}
  \int_0^t v(t') \dtp
\eqcol
  d
\\
  \int_0^t v_0 - \frac{F}{m} t' \dtp
\eqcol
  d
\\
  v_0t - \frac{F}{2m} t^2
\eqcol
  d
\end{nedqn}

But now we're starting with velocity $\alpha v$, and decelerating for
$\alpha t$ seconds:

\begin{nedqn}
\\
  \int_0^{\alpha t} \alpha v_0 - \frac{F}{m} t' \dtp
\eqcol
  \alpha^2 v_0 t - \frac{F}{2m} (\alpha t)^2
\\\eqcol
  \alpha^2
  \left(
    v_0 t - \frac{F}{2m}t^2
  \right)
\\\eqcol
  \alpha^2 d
\end{nedqn}

From this we may write that:

\begin{nedqn}
  % TODO: Find uses of KE and do mathit on them.
  \mathit{KE}
\simcol
  m v^2
\end{nedqn}

\subsection{Kinetic Energy Is Well Defined}

We've shown how kinetic energy changes when the mass or velocity change.
Now we want to give an exact formula. We also want to prove (like we did
for momentum), that you can use any force $F$ and any distance $d$ just
as long as the product works out correctly. More: we want to prove that
for any $F(d)$ and $d_0$, so long as $\int_0^{d_0} F(d') \ddp$ is equal
to the kinetic energy, the force will stop the moving object over that
distance.

The cleanest way to show this is via a series of changes of variables:

\begin{nedqn}
  \int_0^{d_0} F(d) \dd
\eqcol
  \int_0^{d_0} m a(d) \dd
  \nedcomment{definition of force}
\\\eqcol
  m \int_0^{t_0} a(t) d'(t) \dt
  \nedcomment{change variables to time}
\\\eqcol
  m \int_0^{t_0} v'(t) v(t) \dt
  \nedcomment{write in terms of velocity}
\\\eqcol
  m \int_0^v v \dv
  \nedcomment{change of variables to velocity}
\\\eqcol
  \frac{1}{2} m v^2
\end{nedqn}

Note where we presume that the force over the specified distance is
sufficient to stop the velocity: that the bounds $(0, t_0)$ correspond
to velocity $(0, v)$. Note: I've done this derivation using acceleration
rather than deceleration. Because equivalently the kinetic energy is the
force times the distance required to get an object of mass $m$ up to
speed $v^2$.

\subsection{Less Fancy Derivation}

In case that change of variables was too slick, we can do a simpler
version. Here I'll assume two constant forces $F_1$ and $F_2$ that stop
the object within distances of $d_1$ and $d_2$. Because both forces need
to arrest the same momentum, we have:

\begin{nedqn}
  F_1 t_1
\eqcol
  F_2 t_2
\\
  \frac{F_1}{F_2}
\eqcol
  \frac{t_2}{t_1}
\end{nedqn}

Let's now recall the relation between time to stop and stopping distance:

\begin{nedqn}
  \int_0^{t_1} v(t') \dtp
\eqcol
  d_1
\\
  \int_0^{t_1} \left(
    \int_0^{t'} \frac{F_1}{m} \dtpp
  \right) \dtp
\eqcol
  d_1
\\
  \frac{1}{2}
  \frac{F_1}{m}
  t_1^2
\eqcol
  d_1
\end{nedqn}

And of course also:

\begin{nedqn}
  \frac{1}{2}
  \frac{F_2}{m}
  t_2^2
\eqcol
  d_2
\end{nedqn}

We may now note:

\begin{nedqn}
  \frac{d_1}{d_2}
\eqcol
  \frac{
    \frac{1}{2}
    \frac{F_1}{m}
    t_1^2
  }{
    \frac{1}{2}
    \frac{F_2}{m}
    t_2^2
  }
\\\eqcol
  \frac{F_1}{F_2}
  \parensq{
    \frac{t_1}{t_2}
  }
\\\eqcol
  \frac{t_2}{t_1}
  \parensq{
    \frac{t_1}{t_2}
  }
\\\eqcol
  \frac{t_1}{t_2}
\end{nedqn}

Which now leads us to:

\begin{nedqn}
  \frac{F_1 d_1}{F_2 d_2}
\eqcol
  \frac{F_1}{F_2}
  \frac{d_1}{d_2}
\\\eqcol
  \frac{t_2}{t_1}
  \frac{t_1}{t_2}
\\\eqcol
  1
\end{nedqn}

Great! This derivation isn't fancy like the change-of-variables way, and
it doesn't fully prove things for the variable $F(d)$ force possibility,
but it does use less fancy math.

\subsection{Potential Energy}

Wonderful! We've now shown that kinetic energy is a well-defined
quantity! Next question: why should we care about it? It feels like
\emph{momentum} is a much more commonsense way of measuring the
``energy'' in a moving object. When would the concept of kinetic energy
be more relevant than momentum?

I believe the answer relates to \emph{potential energy}. Potential
basically means ``stored.'' \emph{Gravitational} potential energy is one
example. Say that a \SI{1}{N} object A is suspended on a pulley
\SI{1}{m} above the ground. Attached to the other end of the pulley is a
second \SI{1}{N} object B. Object B sits on the ground.

The net force on the two objects are both zero. They can each sustain
\SI{1}{N} of force on the other (via the pulley) indefinitely. But let
us now assume that I give a little impulse upward to object B. It starts
moving upward.

How long can object A sustain the force on object B? In terms of \si{Ns}
this is undefined. Why? Because object A can sustain the force until it
hits the ground. If I imparted only a very small velocity to B (and thus
A), then the force can be sustained for a very long duration of time. If
I imparted a greater velocity, the force cannot be sustained for a very
long time.

Time is not fundamentally the limiting factor. The limiting factor is
the \emph{distance} that object A can travel through. There is only a
finite distance that it can move through. No matter how long the force
is sustained, it can only be sustained through the given distance.

Note: even when object A hits the ground, it is still sustaining the
force on B. It still has more \emph{time} to sustain the force, but no
more \emph{distance}.

\subsection{Gravitational Potential Energy}

Say objects A and B are gravitationally attracted to each other. How
much potential energy is stored gravitationally?

\begin{nedqn}
  \int_0^{d} G \frac{m_A m_B}{d'^2} \ddp
\eqcol
  G m_A m_B \int_0^d {d'}^{-2} \ddp
\\\eqcol
  G m_A m_B
  \left(
    -d'^{-1}
  \right)
  \intevalbar{0}{d}
\end{nedqn}

Oops! This makes it look like an \emph{infinite} amount of gravitational
energy can be harnessed from the gravitational attraction between two
objects. But that is not true. As the objects become very close, we will
have to expend some of the energy we are extracting in overcoming
\emph{other} forces. In particular: the electrical force will want to
keep the objects apart.

For this reason, it's probably meaningless to talk about gravitational
potential energy without reference to a reference distance.

\subsection{Can You Trade Force and Distance?}

A key question to the concept of potential energy would be: can you use
the stored energy to harness any amount of force, if only for the
appropriate distance?

For instance, can you use a \SI{1}{N} object suspended at a height of
\SI{10}{m} to lift a \SI{10}{N} object up \SI{1}{m}?
