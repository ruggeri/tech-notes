\section{Energy}

\subsection{Kinetic Energy}

We've already discussed momentum. This talks about how much force an
object can sustain, and for how long, before it comes to rest.

We can also ask for what \emph{distance} will an object be able to
sustain a force before coming to rest. This is called \emph{kinetic
energy}. The measure of kinetic energy is naturally the Newton meter:
\si{Nm}. This unit is also called the \emph{Joule}: \si{J}.

When calculating kinetic energy, the relevant factors are: (1) the mass
of the object, and (2) the velocity of the object. The kinetic energy
ought to be proportional to the mass: if we double the mass of the
object, we must double the force exerted to maintain the same
deceleration.

How do kinetic energy and velocity relate? Well, imagine that it takes a
constant force of $F$ exerted for $d$ meters to stop an object with mass
$m$ and velocity $v$. We now ask, for what distance must $F$ be exerted
to stop the same object with velocity $\alpha v$?

We know that it must now take $\alpha t$ seconds to stop the object. We
know that from our discussion of momentum. But how much distance is
that?

\begin{nedqn}
  \int_0^t v - \frac{F}{m} t' \dtp
\eqcol
  d
\\
  vt - \frac{F}{2m} t^2
\eqcol
  d
\end{nedqn}

But now we're starting with velocity $\alpha v$, and decelerating for
$\alpha t$ seconds:

\begin{nedqn}
  \int_0^{\alpha t} \alpha v - \frac{F}{m} t' \dtp
\eqcol
  \alpha^2 vt - \frac{F}{2m} (\alpha t)^2
\\\eqcol
  \alpha^2
  \left(
    vt - \frac{F}{2m}t^2
  \right)
\\\eqcol
  \alpha^2 d
\end{nedqn}

From this we may write that:

\begin{nedqn}
  KE
\defeqcol
  \frac{1}{2} m v^2
\end{nedqn}

\subsection{Kinetic Energy Is Well Defined}

We've shown how kinetic energy changes when the mass or velocity change.
But we haven't yet shown that kinetic energy is constant for all choices
of $F$ and $d$ that would stop the object.

We'll start with two constant forces $F_1$ and $F_2$ that stop the
object within distances of $d_1$ and $d_2$. Because both forces need to
arrest the same momentum, we have:

\begin{nedqn}
  F_1 t_1
\eqcol
  F_2 t_2
\\
  \frac{F_1}{F_2}
\eqcol
  \frac{t_2}{t_1}
\end{nedqn}

Let's now recall the relation between time to stop and stopping distance:

\begin{nedqn}
  \int_0^{t_1} \left(
    \int_0^{t'} \frac{F_1}{m} \dtpp
  \right) \dtp
\eqcol
  d_1
\\
  \frac{1}{2}
  \frac{F_1}{m}
  t_1^2
\eqcol
  d_1
\end{nedqn}

And of course also:

\begin{nedqn}
  \frac{1}{2}
  \frac{F_2}{m}
  t_2^2
\eqcol
  d_2
\end{nedqn}

We may now note:

\begin{nedqn}
  \frac{d_1}{d_2}
\eqcol
  \frac{
    \frac{1}{2}
    \frac{F_1}{m}
    t_1^2
  }{
    \frac{1}{2}
    \frac{F_2}{m}
    t_2^2
  }
\\\eqcol
  \frac{F_1}{F_2}
  \parensq{
    \frac{t_1}{t_2}
  }
\\\eqcol
  \frac{t_2}{t_1}
  \parensq{
    \frac{t_1}{t_2}
  }
\\\eqcol
  \frac{t_1}{t_2}
\end{nedqn}

Which now leads us to:

\begin{nedqn}
  \frac{F_1 d_1}{F_2 d_2}
\eqcol
  \frac{F_1}{F_2}
  \frac{d_1}{d_2}
\\\eqcol
  \frac{t_2}{t_1}
  \frac{t_1}{t_2}
\\\eqcol
  1
\end{nedqn}

This shows that kinetic energy is indeed well-defined no matter the
force. We should really extend this to \emph{any} force $F_1(d)$,
$F_2(d)$. I will do this faster, appealing to a series of changes of
variables:

\begin{nedqn}
  \int_0^{d_0} F(d) \dd
\eqcol
  \int_0^{d_0} m a(d) \dd
  \nedcomment{definition of force}
\\\eqcol
  m \int_0^{t_0} a(t) d'(t) \dt
  \nedcomment{change variables to time}
\\\eqcol
  m \int_0^{t_0} v'(t) v(t) \dt
  \nedcomment{write in terms of velocity}
\\\eqcol
  m \int_0^v v \dv
  \nedcomment{change of variables to velocity}
\\\eqcol
  \frac{1}{2} m v^2
\end{nedqn}

Wonderful! We've now shown that kinetic energy is a well-defined
quantity! Next question: why should we care about it?
