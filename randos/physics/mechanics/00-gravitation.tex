\section{Gravitation}

\subsection{Weight}

We observe that objects appear to have \emph{weight}. Heavier objects
are harder to lift. A little more precisely, if you balance two objects
on a scale, objects with the same weight will balance out.

How is this so? We might imagine that objects are pulled toward the
earth. Objects are ``heavier'' because they are pulled harder toward the
earth.

\subsection{Acceleration Due To Gravity}

If an object is not supported - that is, dropped - it will fall toward
the earth. Objects \emph{accelerate} toward the earth; they start out
falling slowly, then gather speed.

It is observed that the acceleration of an object does not depend on its
weight. No matter the weight, we observe a constant acceleration of
\SI{9.8}{\mpss}.

Here we face a little contradiction. If the ``force'' of gravity were
simply acceleration, then we would not expect that objects of different
weights would fail to balance. Gravity \emph{does} appear to ``pull
harder'' on heavier objects (that's what makes them heavier). But the
observed effect of acceleration is always the same...

\subsection{The Concept Of Force}

How does gravitational ``pulling'' relate to acceleration? We can
experiment.

Let's attach a \SI{1}{N} weight to one end of a pulley, and a \SI{2}{N}
weight to the other end. We expect that gravity will ``pull harder'' on
the \SI{2}{N} weight, overcoming the pulling of the \SI{1}{N} weight.

Let's assume that the ``forces'' of pulling are \emph{additive}. That would
kind of make sense: two \SI{1}{N} weights can balance a single \SI{2}{N}
weight. In that case, we expect the overall ``pulling'' on our pulley
apparatus to be equivalent to the gravitational pulling performed on a
\SI{1}{N} weight.

However, more ``stuff'' is being pulled upon. We observe that the pulley
will accelerate at a rate of:

\begin{nedqn}
  a
\eqcol
  \frac{
    9.8 \si{\mpss}
  }{
    3
  }
\end{nedqn}

We may infer that (1) gravity is a ``force'' (not merely acceleration),
(2) the gravitational force is proportional to the \emph{mass} of the
object acted upon, and (3) acceleration of an object (or set of objects)
is proportional to the net force and inversely proportional to the total
\emph{mass}. Which is to say:

\begin{nedqn}
  F
\eqcol
  ma
\end{nedqn}

and

\begin{nedqn}
  F_g
\simcol
  gm
\end{nedqn}

Where $g \defeq \si{9.8}{\mpss}$.

When an object is subject only to the gravitational force, we therefore
observe constant acceleration.

\subsection{Symmetry of Gravitational Force}

We know that the Earth exerts a gravitational force on objects, but is
the Earth special? Shouldn't every object exert a gravitational force on
every other object?

If that is the case, we might ask why objects fall to the ground instead
of into each other. One presumes that there is indeed a force between
each pair of objects, but that the force is proportional not only to the
mass of the attracted object, but also proportional to the mass of the
\emph{attracting} object.

Assuming this is the case, we have:

\begin{nedqn}
  F_g
\simcol
  m_A
  m_B
\end{nedqn}

This suggests that the gravitational force exerted by A upon B should
equal the gravitational force exerted by B upon A. As an intuition
check: if A is the Earth while B is a \SI{10}{kg} object, we expect the
object to accelerate much faster than the Earth. This follows from
forces being equal in magnitude, but the masses being so different.

\subsection{Gravitational Force Inversely Proportional To Squared Distance}

There are many objects that are much more massive than the Earth. Why do
we humans predominately accelerate toward the Earth and not into the
Sun? Presumably \emph{distance} has something to do with it!

We observe that the force of gravity is inversely proportional to the
\emph{square} of the distance between the two objects. Why the
\emph{square}?

I don't have a really good answer, but I have something at least
suggestive. Imagine a substance emanating from a point, radiating out in
all directions. What is the \emph{density} of the substance as it
expands outward?

The density at a distance $r$ would be spread over the surface of a
sphere of radius $r$. The surface area of a sphere is $4\pi r^2$. Thus
the density is dropping at a rate inversely proportional to the square
of the distance.

Why would gravity be like the density of a substance? I don't know. But
at least this gives a thought.

\subsection{Definition Of The Gravitational Force}

We now know that the gravitational force between two objects is given
by:

\begin{nedqn}
  F_g
\simcol
  \frac{
    m_a
    m_b
  }{
    r^2
  }
\\
\eqcol
  G
  \frac{
    m_a
    m_b
  }{
    r^2
  }
\end{nedqn}

Here $G$ is the gravitational constant. By experiment we determine that:

\begin{nedqn}
  G
\eqcol
  \SI{6.674e-11}{m^3.kg.s^{-2}}
\end{nedqn}

Lovely.

\subsection{Acceleration versus Jerk}

Before moving on, we may ask: why does the force of gravity
\emph{accelerate} an object, rather than \emph{accelerate the
acceleration}? The second derivative of velocity is called \emph{jerk}.

I don't exactly know why!

Here is one thought though. There is no way to ``discontinuously''
create a gravitational force. To ``create'' a gravitational force on
object A, you must first build up mass at point B. Building up the mass
at point B will build up the gravitational force continuously, thus
building up the acceleration.

Perhaps what is relevant here is that we see \emph{mass} as a static
property that exists at a point in time, while the process of
accumulation of mass appears to us as having a duration.

Anyway, I think this may simply be an example of ``how the universe
works.'' I'm not sure how to analyze further. Hopefully, though, we have
seen some justification for $F = ma$ in experiment. It's not as if force
is itself visible or tangible; it is \emph{imputed}. But it seems to be
a simple explanation of our observation.
