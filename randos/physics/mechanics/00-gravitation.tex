\section{Gravitation}

\subsection{Weight}

We observe that objects have \emph{weight}. Heavier objects are harder
to lift. A little more precisely, if you balance two objects on a scale,
objects with the same weight will balance out.

How is this so? We might imagine that objects are pulled toward the
earth. Objects are ``heavier'' because they are pulled harder toward the
earth.

\subsection{Acceleration Due To Gravity}

If an object is not supported - that is, dropped - it will fall toward
the earth. Objects \emph{accelerate} toward the earth; they start out
falling slowly, then gather speed.

Why don't they immediately start falling at a given speed? One might
presume: nature wants to vary physical quantities continuously.
Therefore, smooth acceleration seems natural. (Side note: one presumes
that the gravitational ``force'' of the earth on the object has always
existed. ``Dropping'' didn't discontinuously create the force. We simply
removed the support.)

Why does the rate of acceleration not change over the course of the
dropping? Perhaps it might change as the object approaches the earth? In
fact: it does! Objects accelerate faster the closer they are to the
Earth's surface. But this is hard to observe when working close to the
Earth's surface.

Should the rate of acceleration vary given an object's velocity? Maybe
an object should accelerate to a ``terminal velocity?'' Because of
\emph{drag} (later!) this is actually the case. In an ideal world
without drag?

Perhaps because of some relativistic effect acceleration might vary with
velocity. I am unsure. Certainly this is not what we appear to observe.
On the other hand, I can't think of some simple principle that would
suggest that velocity is irrelevant to the ``strength'' of gravitational
acceleration.

Also, it is not clear to me that gravity ought not affect the
\emph{jerk} of an object (the change in the acceleration).

Anyway, we may set these questions aside and move on. Perhaps this is
``just the way the universe works.''

\subsection{Constant Acceleration and Force}

It is observed that the acceleration of an object does not depend on its
weight. No matter the weight, we observe a constant acceleration of
\SI{9.8}{\mpss}.

Here we face a little contradiction. If the ``force'' of gravity were
simply acceleration, then we would not expect that objects of different
weights would fail to balance.

Consider a 1 pound weight tied to one end of a pulley, while a 2 pound
weight is tied to the other end of the pulley. This does not balance, of
course. The weights accelerate, \emph{but not at the rate of
\SI{9.8}{\mpss}!} The weights accelerate at something closer to
\emph{half} the typical rate of acceleration.

(This analysis is not exactly correct. It is not as if there is a
\SI{1}{N} net force on \emph{both} the 2 pound and 1 pound weights. If
that were the case, they would accelerate at different rates.)

What does the experiment suggest? It suggests that the \emph{strength}
of gravitational attraction \emph{is} proportional to an object's mass.
It's simply that the ability of a ``force'' to accelerate an object is
\emph{inversely proportional} to the object's mass.

This suggests to us that we should be talking about a gravitational
\emph{force}, not simply gravitational acceleration. We may write:

\begin{nedqn}
  F
\eqcol
  ma
\end{nedqn}

\subsection{Symmetry of Gravitational Force}

The gravitational force must be proportional to the mass of the object
acted upon. That follows from our observation of constant acceleration.

We know that the Earth exerts a gravitational force on objects, but is
the Earth special? Any object should exert a gravitational force on any
other object. It doesn't make sense that the Earth would be ``special.''

If that is the case, we might ask why objects fall to the ground instead
of into each other. One presumes that there is indeed a force between
each pair of objects, but that the force is proportional not only to the
mass of the attracted object, but also proportional to the mass of the
attracting object.

Assuming this is the case, we have:

\begin{nedqn}
  F_g
\simcol
  m_A
  m_B
\end{nedqn}

Note that we expect the gravitational force exerted by A on B to be
equal but opposite to the gravitational force exerted by B on A. As an
intuition check: if A is the Earth while B is a \SI{10}{kg} object, we
expect the object to accelerate much faster than the Earth. This follows
from forces being equal in magnitude, but the masses being so different.

\subsection{Gravitational Force Inversely Proportional To Squared Distance}

There are many objects that are much more massive than the Earth. Why do
we predominately accelerate toward the Earth and not the Sun? Presumably
\emph{distance} has something to do with it!

We find that the force of gravity is inversely proportional to the
\emph{square} of the distance between the two objects. Why the
\emph{square}?

I don't have a really good answer, but I have something at least
suggestive. Imagine a substance emanating from a point, radiating out in
all directions. What is the \emph{density} of the substance as it
expands outward?

The density at a distance $r$ would be spread over the surface of a
sphere of radius $r$. The surface area of a sphere is $4\pi r^2$. Thus
the density is dropping at a rate inversely proportional to the square
of the distance.

Why would gravity be like the density of a substance? I don't know. But
at least this gives a thought.

\subsection{Definition Of The Gravitational Force}

We now know that the gravitational force between two objects is given
by:

\begin{nedqn}
  F_g
\simcol
  \frac{
    m_a
    m_b
  }{
    r^2
  }
\\
\eqcol
  G
  \frac{
    m_a
    m_b
  }{
    r^2
  }
\end{nedqn}

Here $G$ is the gravitational constant. By experiment we determine that:

\begin{nedqn}
  G
\eqcol
  \SI{6.674e-11}{m^3.kg.s^{-2}}
\end{nedqn}

Lovely.
