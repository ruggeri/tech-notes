\section{Momentum}

\subsection{Definition of Momentum}

If an object is in motion, how much force - and for how long - can the
object experience before it is stopped? This quantity is called
\emph{momentum}.

Does it make sense to talk about momentum? Won't an object take
different amounts of time to stop, if different levels of force are
exerted?

Of course. But note that if an object experiences constant deceleration,
the time it takes to come to rest is \emph{inversely proportional} to
the rate of deceleration. Note also that, for an object of a fixed mass,
the rate of deceleration implied by a force is \emph{proportional} to
the magnitude of the force.

These factors cancel. For any force $F$, if we multiply this force by
the time it takes to stop the object, we will always get the same value.
This is the momentum. You could measure momentum in Newton-seconds:
\si{N.s}.

Momentum is more typically defined as $p \defeq mv$. This makes sense:
a \SI{1}{kg} mass moving at \SI{1}{m/s} will stop if you apply \SI{1}{N}
for \SI{1}{s}. An object will have greater momentum if the mass is
greater (because a \SI{1}{N} force will decelerate the object more
slowly). And of course an object will have greater momentum if the
velocity is greater (you have more velocity to neutralize).

Note that the typically used unit for momentum is \si{kg.m/s} (not the
equivalent unit \si{N.s}).

\subsection{Definition of Impulse}

Momentum is a property of an object. If we perform a change of momentum
on an object, that change is called an \emph{impulse}. The SI units for
impulse are \si{N.s}. I think this is kind of silly; why not the same
units as for momentum?

Of course, impulse is defined by:

\begin{nedqn}
  J
\defeqcol
  \int F \dt
\end{nedqn}

Naturally we use $J$ to stand for impulse.

\subsection{Conservation of Momentum}

Conservation of momentum is obvious if we assume that all forces are
always matched: equal but opposite.

If a force $F$ is exerted by A on B for a time $t$, that means that an
equal but opposite force $F$ is exerted by B on A for the same duration
$t$. The impulses imparted have the same magnitude, but opposite
directions. Therefore the changes in momentum cancel (regardless the
masses of the objects A and B). Thus momentum is \emph{conserved}.

Note that conservation of momentum relies on momentum being a
\emph{vector quantity}. The only reason for the cancelation of the equal
impulses is that they are \emph{directed} in \emph{opposite directions}.
