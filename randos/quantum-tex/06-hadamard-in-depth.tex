\newcommand{\oket}{\braket{1}}
\newcommand{\zket}{\braket{0}}

\section{Hadamard Operation in Depth}

\begin{remark}
  (This reviews in a little more depth what I've already said about
  the Hadamard elsewhere).

  As we've seen, Hadamard on one qubit involves: reflecting the $y$
  coordinate, then rotating 45deg CCW.

  Hadamard moves a basis state to a uniform superposition. But unlike a
  45deg rotation, it is also its own inverse. So it also maps a uniform
  superposition back to a basis state.

  I see this as: it can simultaneously be ``reassembling'' a split-up
  quantum state while "exploding" a basis state. We'll see this help us
  in Grover's algorithm.
\end{remark}

\begin{remark}
  Let's recall think of what $\mtxH^{\otimes{n}}$ does to $\zket_n$ ($n$
  qubits each with value $0$). $\mtxH$ maps this to every state
  $\braket{x}$, where the amplitude is $\frac{1}{\sqrt{2^n}}$.

  Now lets apply the Hadamard to try to go back. Then every input
  $\braket{x}$ value contributes some amplitude to every output
  $\braket{y}$ value. If we were just working with one qubit, then we
  know that both inputs $\zket$ and $\oket$ gives positive amplitude to
  output $\zket$ (constructive interference).

  At the same time, while inputs $\zket$ and $\oket$ both give amplitude
  to output $\oket$, these amplitudes are \emph{out of phase} and
  \emph{destructively interfere}. The $\zket$ input provides
  \emph{positive} amplitude, while the $\oket$ input provides
  \emph{negative} amplitude.
\end{remark}

\begin{remark}
  We can now consider applying the Hadamard operation iteratively to
  each qubit of the $n$ qubits. If the first qubit is $\oket$, this
  means it contributes an amplitude of $\sqtot$ to $\zket$ and of
  $-\sqtot$ to $\oket$.

  If the next qubit is also $\oket$, then we will multiply the first
  amplitudes by $\sqtot$ and $-\sqtot$ again. And this is how we get the
  equation that we get the equation that for each $y$, $\braket{x}
  \mapsto \braket{y}$ with probability amplitude:

  \begin{nedqn}
    \parensexp{\sqtot}{n}
    \sum (-1)^{\text{count of $1$s in common to both $x$ and $y$}} \braket{y}
  \end{nedqn}
\end{remark}
