\section{Quantum Gates}

\begin{remark}
  In theory, any unitary transformation of $\C^{2^k}$ is a valid
  transformation of a $k$-qubit quantum state. But just as boolean
  circuits can be built out of AND, OR, NOT, et cetera gates, we only
  need a limited number of gates in quantum computing. Here are the most
  common:

  \begin{nedqn}
    \mtxI
  \eqcol
    \begin{bmatrix}
      1 & 0 \\
      0 & 1
    \end{bmatrix}
    \nedcomment{Identity}
  \\
    \mtxX
  \eqcol
    \begin{bmatrix}
      0 & 1 \\
      1 & 0
    \end{bmatrix}
    \nedcomment{Pauli X (NOT gate)}
  \\
    \mtxY
  \eqcol
    \begin{bmatrix}
      0 & -i \\
      i & 0
    \end{bmatrix}
    \nedcomment{Pauli Y (haven't used this)}
  \\
    \mtxZ
  \eqcol
    \begin{bmatrix}
      1 & 0 \\
      0 & -1
    \end{bmatrix}
    \nedcomment{Pauli Z (called phase flip gate)}
  \end{nedqn}

  \noindent
  Perhaps the most useful gate of all:

  \begin{nedqn}
    \mtxH
  \eqcol
    \begin{bmatrix}
      \sqtot & \sqtot \\
      \sqtot & -\sqtot
    \end{bmatrix}
    \nedcomment{Hadamard gate}
  \end{nedqn}

  \noindent
  The Hadamard gate flips the Y axis and then does a 45deg rotation.
\end{remark}

\begin{remark}
  What about when we want to apply a single-qubit gate to one qubit of a
  multi-qubit system? For instance, what matrix defines applying the
  Pauli $\mtxX$ on qubit two while leaving qubit one untouched?

  \begin{nedqn}
    \mtxX \otimes \mtxI
  \eqcol
    \begin{bmatrix}
      0 \mtxI & 1 \mtxI \\
      1 \mtxI & 0 \mtxI
    \end{bmatrix}
  \\
  \eqcol
    \begin{bmatrix}
      0 & 0 & 1 & 0 \\
      0 & 0 & 0 & 1 \\
      1 & 0 & 0 & 0 \\
      0 & 1 & 0 & 0
    \end{bmatrix}
  \end{nedqn}

  \noindent
  Just like decomposable qubit states are products of single-qubit
  states, so can matrices that affect only one qubit be decomposed into
  two-by-two matrices.
\end{remark}

\begin{remark}
  A common \define{product gate} is $\mtxH \otimes \mtxH$:

  \begin{nedqn}
    \mtxH \otimes \mtxH
  \eqcol
    \begin{bmatrix}
       1/2 &  1/2 &  1/2 &  1/2 \\
       1/2 & -1/2 &  1/2 & -1/2 \\
       1/2 &  1/2 & -1/2 & -1/2 \\
       1/2 & -1/2 & -1/2 &  1/2 \\
    \end{bmatrix}
  \end{nedqn}

  \noindent
  The Hadamard product moves the basis state $\braket{0}_k$ to a uniform
  superposition over all basis states.

  More generally, any basis state $\braket{i}_k$ gets moved to a uniform
  superposition over all basis states, \emph{except} the polarity for
  result state $j$ depends on the bitwise product of $i \cdot j \bmod
  2$. That is: maintaining each 1 bit flips the polarity.
\end{remark}

\begin{remark}
  Unsurprisingly, applying $\mtxI \otimes \mtxH$ to an unentangled system
  like $\braket{0} \otimes \braket{1}$ is simple. The result is simply
  $\mtxI \braket{0} \otimes \mtxH \braket{1}$.
\end{remark}

\begin{remark}
  We can measure the entire system, but what happens if we measure just
  one qubit of a multiple-qubit system? Then the probability of
  observing a zero is equal to:

  \begin{nedqn}
    \sum_{i\,\text{with bit $x=0$}} \norm{\alpha_i}^2
  \end{nedqn}

  \noindent
  We then zero out all states where bit $x$ was one. We must renormalize
  in the obvious way. Note that there is no difference in measuring 5
  qubits or simply measuring 5 qubits one-by-one.
\end{remark}

\begin{remark}
  Let's start entangling the state of qubits. The first operation we'll
  explore is CNOT. This will \emph{flip} the second bit iff the first
  bit is one.

  \begin{nedqn}
    \mtx{CNOT}
  \eqcol
    \begin{bmatrix}
      1 & 0 & 0 & 0 \\
      0 & 1 & 0 & 0 \\
      0 & 0 & 0 & 1 \\
      0 & 0 & 1 & 0
    \end{bmatrix}
  \end{nedqn}

  \noindent
  Important note: \emph{no} quantum gate could unconditionally set the
  second bit. That would destroy information and thus would not be
  invertible!

  Also note that our CNOT cannot be decomposed into a product of two
  independent operations on different qubits. That's what makes it
  entangling.

  The CNOT is also called the CX gate.
\end{remark}

\begin{remark}
  We can write a SWAP gate in the usual way. We use CX from qubit one to
  qubit two. Then CX back from qubit two to qubit one. This has swapped
  qubit two into qubit one. Last, we do a final CX to write qubit one
  into qubit two.
\end{remark}

\begin{remark}
  The CCNOT or CCX gate is similar, but only does a NOT operation on the
  third gate if the first two are equal

  \begin{nedqn}
    \braket{00x} \mapsto \braket{00x} \\
    \braket{01x} \mapsto \braket{01x} \\
    \braket{10x} \mapsto \braket{10x} \\
    \braket{11x} \mapsto \braket{11(\neg x)}
  \end{nedqn}

  \noindent
  As ever, we cannot simply ``overwrite'' a qubit!
\end{remark}

\begin{remark}
  Using the X (XOR) and CCX (AND) gates, we can build any boolean
  circuit.

  There are an uncountable number of unitary transformations, so no
  finite or even countable basis set could generate all transformations
  in finite circuits. But if you are okay with arbitrarily close
  approximations, then you can approximate any unitary operation well.

  The Solovay-Kitaev theorem says that any generating set is basically
  equally good.

  The CNOT and a rotation are good enough to approximate any unitary
  operation.
\end{remark}
