\section{Multi Qubit Systems}

\begin{remark}
  We've already started to talk about multi-qubit systems. We know that
  a three qubit state is represented as a unit vector in $\C^{2^3} =
  \C^{8}$.

  How do we describe the state of a system of two qubits, one of which
  is in state $\alpha_0 \braket{0} + \alpha_1 \braket{1}$ and the second
  qubit is in state $\beta_0 \braket{0} + \beta_1 \braket{1}$? If the
  states are independent, we write:

  \begin{nedqn}
    \parens{\alpha_0 \braket{0} + \alpha_1 \braket{1}}
    \otimes
    \parens{\beta_0 \braket{0} + \beta_1 \braket{1}}
  \eqcol
    \alpha_0 \beta_0 \braket{00}
    +
    \alpha_0 \beta_1 \braket{01}
    +
    \alpha_1 \beta_0 \braket{10}
    +
    \alpha_1 \beta_1 \braket{11}
  \end{nedqn}
\end{remark}

\begin{remark}
  The multi-qubit system state above is \define{decomposable} or
  \define{separable} because it can be written as a product of smaller
  qubit systems. Specifically: this is a \define{product state} because
  the state can be written as the product of independent one-qubit
  systems.

  An \define{entangled} state is one which cannot be written as a
  product of one-qubit systems. Here is the classic entangled state:

  \begin{nedqn}
    \sqtot \braket{00} + \sqtot \braket{11}
  \end{nedqn}
\end{remark}

\begin{remark}
  When qubits are entangled, observation of one qubit has implications
  for the subsequent observation of the other qubit. For instance, if we
  observe the first qubit of a system in configuration $\sqtot
  \braket{00} + \sqtot \braket{11}$ to be zero, then subsequent
  observation of the second qubit must \emph{also} yield zero.
\end{remark}
