\documentclass[11pt, oneside]{amsart}

\usepackage{geometry}
\geometry{letterpaper}

\usepackage{ned-common}
\usepackage{ned-abstract-algebra}

\begin{document}

\title{Error Correcting Codes}
\maketitle

\section{Parity}

To \emph{detect} errors, you can store \define{parity} bits per block.
These do not allow you to \emph{correct} errors. If this is a
transmission, then you can re-request a block that fails the parity check.

With a single detection bit, you can detect a single bit-flip 100\% of
the time. The same holds for an odd number of bit flips. However, you
cannot detect \emph{any} even number of bit flips.

\subsection{Longitudinal and Transverse Redundancy Checks}

To get more sensitivity to errors, you can store \emph{multiple} parity
bits. One simple approach is to XOR the bytes of the file together. this
gives one final byte to append to the file. This is called a
\emph{longitudinal} redundancy check. The bytewise approach is sometimes
called BIP-8.

You may combine this with a \emph{transverse} redundancy check. This
calculates one bit of parity per byte. In the old days, IBM had \emph{9
track} recording media: eight data tracks and one parity track.

By combining both longitudinal and transverse redundancy checks, you can
\emph{correct} any one-bit flip. From the transverse check, you detect
what byte is in error. From the longitudinal check you know exactly what
\emph{bit} is in error.

\section{Cyclic Redundancy Check}

\subsection{Intro And Parity Example}

Here is another way to look at parity bits. The message bitstring of $n$
bits is actually a polynomial of $n-1$, with one-zero coefficients from
$\Zmod{2}$. The parity result is the \emph{remainder} of dividing the
message polynomial with an implementation specified \define{generator
polynomial}. The significance of the word \emph{generator} will be more
apparent soon.

Let the generator be $g(x) = x + 1$. Consider encoding various two-bit
messages 00, 01, 10, 11:

\begin{nedqn}
  0x + 0
\mapstocol
  0 (x + 1) + 0
\\
  0x + 1
\mapstocol
  0 (x + 1) + 1
\\
  1x + 0
\mapstocol
  1 (x + 1) + 1
\\
  1x + 1
\mapstocol
  1 (x + 1) + 0
\end{nedqn}

\noindent
This is identical to our standard parity calculation. You get the
remainder (i.e., the parity bit) by doing polynomial long-division and
using the remainder. Notice the \emph{third} scenario of 10 represented
as $x + 0$. Note how you will use 1 of $x + 1$ to eliminate the $x$
term, leaving $-1 \equiv +1$ as the remainder.

Note that by using a polynomial of degree 1 as your generator function,
you are left with a remainder that can be represented with 1 bit. To
create a 32 bit CRC \define{checksum}, you use a generator polynomial of
degree 33. There are many possible choices of generator polynomial -
each corresponds to a different implementation of CRC-32. There are a
number of factors to consider when choosing the generator.

\subsection{Calculation}

Because we are working $\Zmod{2}$, subtracting two polynomials is really
like doing an XOR of corresponding coefficients. For instance:

\begin{nedqn}
  &   & \parens{1x^5 + 0x^4 + 1x^3 + 1x^2 + 0x + 0}
\\
  & - & \parens{0x^5 + 1x^4 + 1x^3 + 0x^2 + 0x + 1}
\\
  & = & \parens{1x^5 + 1x^4 + 0x^3 + 1x^2 + 0x + 1}
\end{nedqn}

We can do polynomial long division to calculate a remainder by cycling
through bits of the message and doing an XOR by the generator polynomial
whenever we encounter a 1. Let's divide a polynomial by $x^3 + 1$.

\begin{IEEEeqnarray*}{+CCCCCCCCCCCC+}
         & 1x^5 & + & 0x^4 & + & 1x^3 & + & 1x^2 & + & 0x^1 & + & 0
\\
  \oplus & 1x^5 & + & 0x^4 & + & 0x^3 & + & 1x^2
\\
  \mapsto& 0x^5 & + & 0x^4 & + & 1x^3 & + & 0x^2 & + & 0x^1 & + & 0
\\
  \oplus &      &   &      &   & 1x^3 & + & 0x^2 & + & 0x^1 & + & 1
\\
  \mapsto& 0x^5 & + & 0x^4 & + & 0x^3 & + & 0x^2 & + & 0x^1 & + & 1
\end{IEEEeqnarray*}

\subsection{Error Polynomial}

A message polynomial $m(x)$ and a corrupted version $m'(x)$ are
indistinguishable modulo a generator polynomial $g(x)$ exactly when:

\begin{IEEEeqnarray*}{+rCll+}
  m(x) \equivcol m'(x) & \pmod{g(x)}
\\
  m(x) - m'(x) \equivcol 0 & \pmod{g(x)}
\end{IEEEeqnarray*}

\subsection{Detects All One Bit Errors}

We call $m(x) - m'(x)$ the \define{error polynomial}. We want to analyze
what kinds of $e(x)$ are equivalent to zero modulo $p(x)$. Consider
one-bit errors. In this case $e(x) = x^k$ for some $k$ (depending on the
position of the error in the message). When can $e(x)$ have $g(x)$ as a
factor?

I claim only when $g(x) = x^i$ for $i < k$. In particular, all one-bit
errors are detected if $g(x)$ has at least two terms. For consider any
$g(x)q(x)$. I claim this must always have at least two non-zero terms:
the product of the leading terms in $g(x), q(x)$, and the product of the
trailing terms in $g(x), q(x)$. These are two distinct terms, and thus
$x^k$ can never be a multiple of $g(x)$.

\subsection{Detects All Two Bit Errors Within A Window}

Consider two flipped bit errors. Then $e(x) = x^{k_1} + x^{k_2} =
x^{k_2} \parens{x^{k_1 - k_2} + 1}$. We've already proven that if $g(x)$
is chosen to have at least two terms, it cannot divide $x^{k_2}$. Let's
presume that $x^{k_2}$ and $g(x)$ share no common factors. That is
guaranteed if the constant term of $g(x)$ is 1. Then, if $g(x)$ is to
divide $e(x)$, it must divide $x^{k_1 - k_2} + 1$. Which is equivalent
to saying:

\begin{IEEEeqnarray*}{+rCll}
  x^{k_1 - k_2} + 1 \equivcol 0 \pmod{g(x)}
\\
  x^{k_1 - k_2} \equivcol 1 \pmod{g(x)}
\end{IEEEeqnarray*}

This basically is asking: what is the \define{order} of the polynomial
$x$ in the quotient field of polynomials over $\Zmod{2}$ modulo $g(x)$?
The greater the order of $x$, the greater the distance $k_1 - k_2$ over
which two bit flips can be detected. For good choices of $g(x)$ (called
\define{primitive polynomials}), $x$ will generate the polynomial field
entire multiplicative group. In this case, powers $x^i$ will cycle
through every non-zero polynomial of length $n$: all $2^n - 1$ of them!

Which means that $x^{k_1 - k_2} \equiv 1$ first when $k_1 - k_2 = 2^n -
1$.

Note: all primitive polynomials $g(x)$ must be \define{irreducible}.
Else there are two elements of the quotient ring $a(x), b(x)$ such that
$a(x)b(x) \equiv 0$. But then at least one of these must not be
generated by powers of $x$ else $x$ would eventually map back to zero!

\subsection{Detects All Odd Number Of Errors}

If one is willing to halve the window of two bit error detection, one
can detect all errors affecting an odd number of bits. Here we choose
$g(x)$ that is the product of (1) an irreducible polynomial $g_1(x)$ of
degree $n$ (not $n+1$ as usual) and (2) $g_2(x) = x + 1$.

The important thing is that $g(x)$ ought to have an \emph{even} number
of terms. A fact that I haven't proven are used: all irreducible
polynomials (excepting $x+1$) have an odd number of terms. If we believe
that, it's easy to see that $g(x) = g_1(x) (x+1)$ has an even number of
terms. This says that $g(x)$ is the XOR of two polynomials with an odd
number of terms. But if I flip an odd number of bits and then flip an
odd number again, I must have flipped an even in total (regardless
whether the two sets of flips overlap).

Now, if $g(x)$ has an even number of terms, no matter how many times we
XOR an error polynomial $e(x)$ with an odd number of terms, we'll still
be left with an odd number of remainder terms.

\subsection{Detects All Burst Errors Of Length At Most $n$}

A \define{burst error} is a subsequence of received bits where (1) the
first and last bits are in error, (2) no subsequence of $k$ bits in
between are correct. The $k$ is a parameter called the \define{guard
band} that can determine when you have \emph{one} or \emph{two} burst
errors. For instance, the end of one burst error comes at least $k$ bits
before the start of a next burst error. Note: burst errors can be of any
length, regardless the chosen guard band $k$.

If one chooses $k=1$, then burst errors are subsequences of received
bits that are \emph{all} in error. Anyway, we'll let $k = \infty$, so
that the entire corrupted range is treated as a single burst error.
We'll want to detect the burst error provided it is not too long.

This model of error is highly relevant to many communication and storage
media.

Consider an error polynomial for a burst. Then $e(x) = x^k e'(x)$. The
$x^k$ factor represents the position the burst error, while $e'(x)$ is a
polynomial representing the pattern of the error. We already know that
$g(x)$ divides $e(x)$ precisely if it divides $e'(x)$. But so long as
the degree of $e'(x)$ is less than the degree of $g(x)$, it cannot be
equivalent to zero.

Thus any burst error of at most $n$ symbols is detected!

% ## Protocols

% * Automatic Request Repeat
%     * Just detect errors and ask for retransmit.
% * Forward Error Correction
%     * Fix errors that you detect with what you deem most likely.
% * Hybrid
%     * Fix minor errors you detect, ask for retransmit of major errors
%       you detect.

% ## Error Correcting Codes

% You want to be able to correct codes without re-requesting. This is
% important for disks (since the writer will forget the data), phones
% (where the latency to re-request is too high).

% A simple way is to send three versions of each bit. The user just
% takes the majority vote. But this is pretty ineffecient for fixing a 1
% bit error.

% Hamming gives us a way to correct *any* 1 bit error. You transmit 4
% bits of data, and 3 parity bits. Here's how:

% ```
% P1 = D1 XOR D2 XOR D4
% P2 = D1 XOR D3 XOR D4
% P3 = D2 XOR D3 XOR D4
% ```

% Let's say we know at most one bit is wrong. Let's show how to correct
% this. Let's say we receive:

% * 0000 000
%     * We know immediately that no data bits could be wrong, since then
%       one of the parity bits would have to be wrong, too.
% * 0000 001
%     * Since the parity bit is checked, then either it is wrong, or
%       some data bit (D2, D3, or D4) is wrong.
%     * But that would mean *other* parity bits should have been
%       checked, so that is not possible.
%     * So it's the *parity* bit that's wrong!
%     * This logic holds for any of the other parity bits.
% * 0001 111
%     * We know immediately that none of the data bits can be wrong,
%       since otherwise one of the parity bits is wrong.
% * 0001 110
%     * If the parity is right, then D2 or D3 should be checked, but
%       this is not possible, since then other parity bits are wrong,
%       too.
% * So one bit errors in the parity don't cause mistakes.
% * 0001 000
%     * We can see here that the data bit must be wrong, else *two*
%       parity bits are wrong.

% So what we're seeing here is that all 1 bit errors can be
% corrected. Another way of looking at it: the Hamming distance between
% any valid Hamming codewords is 3. That's why you can detect two
% errors, and correct one.

% Hamming is widely used for ECC memory; Hamming codes are most useful
% when the error rate is low. With high error rates, you're not going to
% do that well.

% Generalizations of Hamming codes can acheive higher data rates. You
% can also extend Hamming to detect two bit errors, too, by including a
% bit that xors all the Hamming bits.

% ## Reed-Solomon

% Reed-Solomon codes are widely used, they are error-correcting, and can
% also detect missing data. Codes like this are called *erasure codes*.

% Reed-Solomon, given `t` check symbols, can:

% * Detect `t` erroneous symbols.
% * Correct `FLOOR(t/2)` erroneous symbols.
% * It can correct up to `t` erasures.

% With larger symbols, you can handle burst errors well, since this
% corrupts a single (or a few) symbols. This makes it useful for CDs,
% DVDs, where burst errors are common (misaligned laser).

% When Reed-Solomon codes were created, an *efficient* decoding scheme
% did not exist. That came almost a decade later, which allowed for
% their practical use.

% The math looks pretty hard on this, and I don't really care that
% much. Let's just say **Mission Accomplished**.

% # TODO

% * Rolling Hash
% * md5sum

% https://en.wikipedia.org/wiki/Convolutional_code
% https://en.wikipedia.org/wiki/Error_correction_code
% https://en.wikipedia.org/wiki/Error_detection_and_correction
% https://en.wikipedia.org/wiki/Hamming_code
% https://en.wikipedia.org/wiki/Hamming_distance
% https://en.wikipedia.org/wiki/Parchive
% https://en.wikipedia.org/wiki/RAID
% https://en.wikipedia.org/wiki/Reed%E2%80%93Solomon_error_correction

\end{document}
